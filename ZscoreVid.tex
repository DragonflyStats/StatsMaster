  \documentclass[a4]{beamer}
\usepackage{amssymb}
\usepackage{graphicx}
\usepackage{subfigure}
\usepackage{newlfont}
\usepackage{amsmath,amsthm,amsfonts}
%\usepackage{beamerthemesplit}
\usepackage{pgf,pgfarrows,pgfnodes,pgfautomata,pgfheaps,pgfshade}
\usepackage{mathptmx} % Font Family
\usepackage{helvet} % Font Family
\usepackage{color}
\mode<presentation> {
\usetheme{Default} % was Frankfurt
\useinnertheme{rounded}
\useoutertheme{infolines}
\usefonttheme{serif}
%\usecolortheme{wolverine}
% \usecolortheme{rose}
\usefonttheme{structurebold}
}
\setbeamercovered{dynamic}
\title[MA4413]{Statistics for Computing \\ {\normalsize MA4413 Lecture 5A}}
\author[Kevin O'Brien]{Kevin O'Brien \\ {\scriptsize kevin.obrien@ul.ie}}
\date{Autumn 2012}
\institute[Maths \& Stats]{Dept. of Mathematics \& Statistics, \\ University \textit{of} Limerick}
\renewcommand{\arraystretch}{1.5}
%------------------------------------------------------------------------%
\begin{document}

\begin{frame}
\huge{
\[\mbox{Probability Distributions}\]
\[\mbox{The Z-score}\]
}

\Large
\[\mbox{www.Stats-Lab.com}\]
\[\mbox{Twitter: StatsLabDublin}\]

\end{frame}

%------------------------------------------------%
\begin{frame}
\frametitle{Standardization formula}
\Large
\vspace{-1cm}
\begin{itemize}
\item All normally distributed random variables have corresponding $Z$ values, called Z-scores.
\item The Z-score is simply the number of standard deviations away from the mean that a particular score is.

\end{itemize}
%Therefore $z_o$ is the z-score that corresponds to $x_o$.
\end{frame}

\begin{frame}
\frametitle{Standardization formula}
\Large
\begin{itemize}
\item For normally distributed random variables, the z-score can be found using the \textbf{\emph{standardization formula}};
\[z_o = { x_o - \mu \over \sigma}\]
where $x_o$ is a observed value from the underlying normal (``X") distribution, $\mu$ is the mean of that distribution, and $\sigma$ is the standard deviation of that distribution.
\end{itemize}
\end{frame}



\begin{frame}
\frametitle{Standardization formula}
\Large
\vspace{-1cm}
\begin{itemize}
\item Z-scores are typically given to 2 decimal places only.
\item Terms with subscripts mean particular observed values, and are not variable names (Not usual, but useful for sake of clarity.)
\item The distribution of Z-values will only be a normal distribution if the original distribution (X) is normal.
\end{itemize}
\end{frame}
%------------------------------------------------------------------------%
\frame{
\frametitle{The Z-score}
\Large
\vspace{-1cm}
\begin{itemize}
\item Suppose that $X$ is a normal distribution with mean $\mu = 80 $ and that standard deviation $\sigma = 8$.
\[ X \sim N(80,8^2)\]
\item What is the Z-score for $x_o = 100$?
\[
z_{100} = {x_0 - \mu \over \sigma} = {100 - 80 \over 8} = {20 \over 8} = 2.5
\]
\item Therefore $z_{100} = 2.5$
\end{itemize}
}
%------------------------------------------------------------------------%
\begin{frame}
\frametitle{The Z-score}
\Large
\vspace{-1cm}
Again suppose $X$ is a normal distribution with mean $\mu = 80 $ and that standard deviation $\sigma = 8$. \\ \vspace{0.5cm} Compute the Z-score for each of the following:
\begin{itemize}
\item[(i)] $x_a = 96$ \vspace{0.2cm}
\item[(ii)] $x_b = 72$ \vspace{0.2cm}
\item[(iii)] $x_c = 86$ 
\end{itemize}
\end{frame}
\begin{frame}
\frametitle{The Z-score}
\Large
\vspace{-1cm}
Again suppose $X$ is a normal distribution with mean $\mu = 80 $ and that standard deviation $\sigma = 8$. \\ \vspace{0.5cm} Compute the Z-score for each of the following:
\begin{itemize}
\item[(i)] $x_a = 96$  \phantom{space} $z_a = 2$ \vspace{0.2cm}
\item[(ii)] $x_b = 72$  %\phantom{space} $z_b = -1$ \vspace{0.2cm}
\item[(iii)] $x_c = 86$  %\phantom{space} $z_c = 0.75$
\end{itemize}
\end{frame}
\begin{frame}
\frametitle{The Z-score}
\Large
\vspace{-1cm}
Again suppose $X$ is a normal distribution with mean $\mu = 80 $ and that standard deviation $\sigma = 8$. \\ \vspace{0.5cm} Compute the Z-score for each of the following:
\begin{itemize}
\item[(i)] $x_a = 96$  \phantom{space} $z_a = 2$ \vspace{0.2cm}
\item[(ii)] $x_b = 72$  \phantom{space} $z_b = -1$ \vspace{0.2cm}
\item[(iii)] $x_c = 86$  %\phantom{space} $z_c = 0.75$
\end{itemize}
\end{frame}
\begin{frame}
\frametitle{The Z-score}
\Large
\vspace{-1cm}
Again suppose $X$ is a normal distribution with mean $\mu = 80 $ and that standard deviation $\sigma = 8$. \\ \vspace{0.5cm} Compute the Z-score for each of the following:
\begin{itemize}
\item[(i)] $x_a = 96$  \phantom{space} $z_a = 2$ \vspace{0.2cm}
\item[(ii)] $x_b = 72$  \phantom{space} $z_b = -1$ \vspace{0.2cm}
\item[(iii)] $x_c = 86$  \phantom{space} $z_c = 0.75$
\end{itemize}
\end{frame}


\end{document}



%------------------------------------------------------------------------%
\frame{
\frametitle{Z scores}
A Z-score always reflects the number of standard deviations above or below the mean a particular score is.
Suppose the scores of a test are normally distributed with a mean of 50 and a standard deviation of 9
For instance, if a person scored a 68 on a test, then they scored 2 standard deviations above the mean.

Converting the test scores to z scores, an X value of 68 would yield:
\[ Z = {68 - 50 \over 9} =2 \]

So, a Z score of 2 means the original score was 2 standard deviations above the mean.
}
\end{document}
%------------------------------------------------------------------------%
%\end{document}
% The standardization formula
% used to find Z values

%

%------------------------------------------------------------------------%
\frame{
\frametitle{The Standard Normal (Z) Distribution Tables}
\begin{itemize}
\item Importantly, probabilities relating to the z distribution are comprehensively tabulated in \textbf{\emph{Murdoch Barnes table 3}}.

\item Given a value of $k$ (with k usually between 0 and 4), the probability of a standard normal "Z" random variable being greater than (or equal to) k $P(Z \geq k)$ is given in Murdoch Barnes table 3 .
\item Other statistical tables can be used, but they may tabulate probabilities in a different way.
\end{itemize}
}

%------------------------------------------------------------------------%
\frame{
\frametitle{An Important Identity}
If two values $z_o$ and $x_o$ are related in the following way, for some values $\mu$ and $\sigma$,
\[
z_{0} = {x_0 - \mu \over \sigma}
\]
Then we can can say

\[ P(X \geq x_o) = P(Z \geq z_o) \]

or alternatively

\[ P(X \leq x_o) = P(Z \leq z_o) \]

This is fundamental to solving problems involving normal distributions.

}

%------------------------------------------------------------------------%
\frame{
\frametitle{Using Murdoch Barnes tables 3}
\begin{itemize}
\item For some value $z_o$, between 0 and 4, the Murdoch Barnes tables set 3 tabulate $P(Z \geq z_o)$
\item Ideally $z_o$ would be specified to 2 decimal places. If it is not, round to the closest value.
\item We call the third digit (i.e. the digit in the second decimal place) the ``second precision".
\end{itemize}
}

%------------------------------------------------------------------------%
\frame{
\frametitle{Using Murdoch Barnes tables 3}
\begin{itemize}
\item To compute the relevant probability we express $z_o$ as the sum of $z_o$ without the second precision, and the second precision.(For example $1.28 = 1.2 + 0.08$.)
\item Select the row that corresponds to $z_o$ without the second precision (e.g. 1.2).
\item Select the column that corresponds to the second precision(e.g. 0.08).
\item The value that contained on the intersection is $P(Z \geq z_o)$
\end{itemize}
}

%------------------------------------------------------------------------%
\frame{
\begin{table}[ht]
\frametitle{Find $ P(Z \geq 1.28)$}
\vspace{-1.5cm}
%\caption{Standard Normal Distribution } % title of Table
\centering % used for centering table
\begin{tabular}{|c|| c c c c c c|} % centered columns (4 columns)
\hline %inserts double horizontal lines
& \ldots & \ldots & 0.006 &0.07&0.08&0.09 \\
%heading
\hline \hline% inserts single horizontal line
\ldots & \ldots & \ldots &\ldots& \ldots &\ldots&\dots \\ % inserting body of the table
1.0 & \ldots & \ldots &0.1446& 0.1423 &0.1401&0.1379 \\ % inserting body of the table
1.1 & \ldots & \ldots&0.1230& 0.1210 &0.1190&0.1170 \\ % inserting body of the table
1.2 & \ldots & \ldots&0.1038 & 0.1020 &\alert{0.1003}&0.0985\\
1.3 & \ldots & \ldots &0.0869& 0.0853 &0.0838&0.0823 \\ % inserting body of the table
\ldots & \ldots &\ldots&\ldots & \ldots &\ldots&\ldots\\
\hline %inserts single line
\end{tabular}
%\label{table:nonlin} % is used to refer this table in the text
\end{table}
}

\end{document}