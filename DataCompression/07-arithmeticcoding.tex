7 Arithmetic coding 47
Essential reading . . . . . . . . . . . . . . . . . . . . . . . . . . . . . . 47
Further reading . . . . . . . . . . . . . . . . . . . . . . . . . . . . . . . 47
Arithmetic coding . . . . . . . . . . . . . . . . . . . . . . . . . . . . . . 47
Model . . . . . . . . . . . . . . . . . . . . . . . . . . . . . . . . . . . . . 47
Coder . . . . . . . . . . . . . . . . . . . . . . . . . . . . . . . . . . . . . 47
Encoding . . . . . . . . . . . . . . . . . . . . . . . . . . . . . . . . . . . 48
Unique-decodability . . . . . . . . . . . . . . . . . . . . . . . . . . 49
Observation . . . . . . . . . . . . . . . . . . . . . . . . . . . . . . 49
Encoding main steps . . . . . . . . . . . . . . . . . . . . . . . . . 49
Defining an interval . . . . . . . . . . . . . . . . . . . . . . . . . . 50
Encoding algorithm . . . . . . . . . . . . . . . . . . . . . . . . . . 50
Observation . . . . . . . . . . . . . . . . . . . . . . . . . . . . . . 51
Decoding . . . . . . . . . . . . . . . . . . . . . . . . . . . . . . . . . . . 51
Renormalisation . . . . . . . . . . . . . . . . . . . . . . . . . . . . . . . 52
Coding a larger alphabet . . . . . . . . . . . . . . . . . . . . . . . . . . 52
Effectiveness . . . . . . . . . . . . . . . . . . . . . . . . . . . . . . . . . 52
Learning outcomes . . . . . . . . . . . . . . . . . . . . . . . . . . . . . 52
Activities . . . . . . . . . . . . . . . . . . . . . . . . . . . . . . . . . . . 53
Laboratory . . . . . . . . . . . . . . . . . . . . . . . . . . . . . . . . . . 53
Sample examination questions . . . . . . . . . . . . . . . . . . . . . . . 53
