Types of Data Compression
============================

There are two main types of data compression: lossy and lossless.


#### Lossy Data Compression
Lossy data compression is named for what it does. After one applies lossy data compression to a message, the message can never be recovered exactly as it was before it was compressed. When the compressed message is decoded it does not give back the original message. Data has been lost.

Because lossy compression can not be decoded to yield the exact original message, it is not a good method of compression for critical data, such as textual data. It is most useful for Digitally Sampled Analog Data (DSAD). DSAD consists mostly of sound, video, graphics, or picture files. Algorithms for lossy compression of DSAD vary, but many use a threshold level truncation. This means that a level is chosen past which all data is truncated. In a sound file, for example, the very high and low frequencies, which the human ear can not hear, may be truncated from the file. Some examples of lossy data compression algorithms are JPEG, MPEG, and Indeo.

#### Lossless Compression
Lossless data compression is also named for what it does. In a lossless data compression file the original message can be exactly decoded. Lossless data compression works by finding repeated patterns in a message and encoding those patterns in an efficient manner. For this reason, lossless data compression is also referred to as redundancy reduction. Becuase redundancy reduction is dependent on patterns in the message, it does not work well on random messages. Lossless data compression is ideal for text. Most of the algorithms for lossless compression are based on the LZ compression method developed by Lempel and Ziv.

One type of text encoding which is very effective for files with long strings of repeating bits is RLE. RLE stands for Run Length Encoding. RLE uses a sliding dictionary method of the LZ algorithm. The sliding dictionary method utilizes pointers within the compressed file that point to previously represented strings of bits within the file. Here is an example of a message which could be effectively encoded with RLE:

*The rain in Spain falls mainly on the plain.*
The string "ain" could be represented only once and could be pointed to by all later calls to that string.
Huffman coding works by analyzing the frequency, F, of elements, e, in a message, M. 
The elements with the highest frequency, F:e, get assigned the shortest encoding (with the fewest bits). 
Elements with lower frequencies get assigned longer encodings (with more bits). 

*http://www.ccs.neu.edu/home/jnl22/oldsite/cshonor/jeff.html*
