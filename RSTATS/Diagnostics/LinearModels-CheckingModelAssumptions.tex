\documentclass[a4paper,12pt]{article}
%%%%%%%%%%%%%%%%%%%%%%%%%%%%%%%%%%%%%%%%%%%%%%%%%%%%%%%%%
\usepackage{eurosym}
\usepackage{vmargin}
\usepackage{amsmath}
\usepackage{graphics}
\usepackage{epsfig}
\usepackage{subfigure}
\usepackage{fancyhdr}
%\usepackage{listings}
\usepackage{framed}
\usepackage{graphicx}

\setcounter{MaxMatrixCols}{10}
%TCIDATA{OutputFilter=LATEX.DLL}
%TCIDATA{Version=5.00.0.2570}
%TCIDATA{<META NAME="SaveForMode" CONTENT="1">}
%TCIDATA{LastRevised=Wednesday, February 23, 2011 13:24:34}
%TCIDATA{<META NAME="GraphicsSave" CONTENT="32">}
%TCIDATA{Language=American English}

\pagestyle{fancy}
\setmarginsrb{20mm}{0mm}{20mm}{25mm}{12mm}{11mm}{0mm}{11mm}
\lhead{MS4024} \rhead{Mr. Kevin O'Brien}
\chead{Numerical Computation}
%\input{tcilatex}

\begin{document}
\Large
		\subsection*{Analysis of residuals}
		\begin{itemize}
		\item	Perform an analysis of regression residuals ( you can pick the best regression model from last section).
		\item	Are the residuals normally distributed?
		\begin{itemize}
			\item Histogram /  Boxplot / QQ plot / Shapiro Wilk Test
			\item Also you can plot the residuals to check that there is constant variance.
		\end{itemize}	
		\end{itemize}
	
	\begin{framed}
		\begin{verbatim}
		y <- rnorm(10)
		x <- rnorm(10)
		fit1 <- lm(y~x)
		res.fit1 <-  resid(fit1)
		plot(res.fit1)
		\end{verbatim}
     \end{framed}		
\newpage
\section{Assessing a Linear Regression :  Model Diagnostics}
\begin{itemize}
\item The ease of implementation fosters the impression that linear regression is easy: just use the \texttt{\textbf{lm()}} function. Yet fitting
this is only the beginning.

\item After a linear regression analysis has been performed. It is good practice to verify the model’s quality
by running diagnostic checks.

\item The approach we will take is to create diagnostic plots, by plotting the model object. Rather than producing a scatterplot, this method will produce several diagnostic plots:
\end{itemize}


\begin{framed}
\begin{verbatim}

Fit1 <- lm(Sepal.length ~ Sepal.Width, data=iris)
plot(Fit1)
\end{verbatim} 
\end{framed}

\begin{itemize}
\item Next, identify possible outliers either by looking at the diagnostic plot of the residuals
%=================================================================================== %
\newpage
\item Another approach is to use the \texttt{\textbf{outlierTest()}} function of the \textit{\textbf{car}} package:

\end{itemize}

\begin{framed}
\begin{verbatim}
#If package not installed, uncomment next line.
#install.packages("car")

library(car)
outlierTest(Fit1)
\end{verbatim} 
\end{framed}
\begin{verbatim}
> outlierTest(Fit1)

No Studentized residuals with Bonferonni p < 0.05
Largest |rstudent|:
    rstudent unadjusted p-value Bonferonni p
132  2.79155          0.0059429      0.89143

\end{verbatim}
%=================================================================================== %
\newpage
\subsection{Influence Measures}
\begin{itemize}
\item Finally, identify any overly influential observations by using the \texttt{\textbf{influence.measures()}} 
function.
\item If an observation is considered influential, it will be indicated with an asterisk on the right hand side. \item Interpretation of the individual statistics, such as \textit{DFFITS} and \textit{DFBETA} are beyond the scope of this workshop.
\end{itemize}


\begin{framed}
\begin{verbatim}
influence.measures(Fit1)
\end{verbatim} 
\end{framed}

\end{document}