\documentclass[a4]{beamer}
\usepackage{amssymb}
\usepackage{graphicx}
\usepackage{subfigure}
\usepackage{framed}
\usepackage{newlfont}
\usepackage{amsmath,amsthm,amsfonts}
%\usepackage{beamerthemesplit}
\usepackage{pgf,pgfarrows,pgfnodes,pgfautomata,pgfheaps,pgfshade}
\usepackage{mathptmx}  % Font Family
\usepackage{helvet}   % Font Family
\usepackage{color}

\mode<presentation> {
 \usetheme{Frankfurt} % was
 \useinnertheme{rounded}
 \useoutertheme{infolines}
 \usefonttheme{serif}
 %\usecolortheme{wolverine}
% \usecolortheme{rose}
\usefonttheme{structurebold}
}

\setbeamercovered{dynamic}

\title[MA4603]{Science Maths 3 \\ {\normalsize MA4603 Lecture 11A}}
\author[Kevin O'Brien]{Kevin O'Brien \\ {\scriptsize Kevin.obrien@ul.ie}}
\date{Autumn Semester 2017}
\institute[Maths \& Stats]{Dept. of Mathematics \& Statistics, \\ University \textit{of} Limerick}

\renewcommand{\arraystretch}{1.5}

\begin{document}


%-------------------------------------------------%


\begin{frame}
\frametitle{Shapiro-Wilk Test(a)}


\begin{itemize}
\item We will often be required to determine whether or not a data set is normally distributed.
\item Again, this assumption underpins many statistical models.
\item The null hypothesis is that the data set is normally distributed.
\item The alternative hypothesis is that the data set is not normally distributed.
\item One procedure for testing these hypotheses is the Shapiro-Wilk test, implemented in \texttt{R} using the command \texttt{shapiro.test()}.
\item (Remark: You will not be required to compute the test statistic for this test.)
\end{itemize}
\end{frame}
%----------------------------------------%
\begin{frame}[fragile]
\frametitle{Shapiro Wilk Test(b)}
For the data set used previously; $x$ and $y$, we use the Shapiro-Wilk test to determine that both data sets are normally distributed.
\begin{verbatim}

> shapiro.test(x)

        Shapiro-Wilk normality test

data:  x
W = 0.9474, p-value = 0.6378

> shapiro.test(y)

        Shapiro-Wilk normality test

data:  y
W = 0.9347, p-value = 0.5273
\end{verbatim}

\end{frame}


%-------------------------------------------------%
\begin{frame}[fragile]
\frametitle{Graphical Procedures for assessing Normality}

\begin{itemize}
\item The normal probability (Q-Q) plot is a very useful tool for determining whether or not a data set is normally distributed.
\item Interpretation is simple. If the points follow the trendline (provided by the second line of \texttt{R} code \texttt{qqline}).
\item One should expect minor deviations. Numerous major deviations would lead the analyst to conclude that the data set is not normally distributed.
\item The Q-Q plot is best used in conjunction with a formal procedure such as the Shapiro-Wilk test.
\end{itemize}

\begin{verbatim}
>qqnorm(CWdiff)
>qqline(CWdiff)
\end{verbatim}

\end{frame}

%-------------------------------------------------%

\begin{frame}
\frametitle{Graphical Procedures for Assessing Normality}

\begin{center}
\includegraphics[scale=0.32]{10AQQplot}
\end{center}
\end{frame}

\end{document}