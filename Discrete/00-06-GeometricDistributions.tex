\documentclass[]{report}

\voffset=-1.5cm
\oddsidemargin=0.0cm
\textwidth = 480pt

\usepackage{framed}
\usepackage{subfiles}
\usepackage{graphics}
\usepackage{newlfont}
\usepackage{eurosym}
\usepackage{amsmath,amsthm,amsfonts}
\usepackage{amsmath}
\usepackage{color}
\usepackage{amssymb}
\usepackage{multicol}
\usepackage[dvipsnames]{xcolor}
\usepackage{graphicx}
\begin{document}

\chapter{Geometric Distribution}
\begin{itemize}
	\item Geometric distributions model (some) discrete random variables. Typically, a Geometric random variable is the number of trials required to obtain the first failure, for example, the number of tosses of a coin until the first 'tail' is obtained, or a process where components from a production line are tested, in turn, until the first defective item is found.
	
	\item A discrete random variable X is said to follow a Geometric distribution with parameter p, written $X \sim Ge(p)$, if it has probability distribution
	\[P(X=x) = p^{x-1}(1-p)^x\]
	where
	\begin{itemize}
		\item $x = 1, 2, 3, \ldots$
		\item p = success probability; $0 < p < 1$
	\end{itemize}
	\item	The trials must meet the following requirements:
	
	\begin{itemize}
		\item[(i)] the total number of trials is potentially infinite;
		there are just two outcomes of each trial; success and failure;
		\item[(ii)] the outcomes of all the trials are statistically independent;
		\item[(iii)] all the trials have the same probability of success.
	\end{itemize}
%	\item The Geometric distribution has expected value and variance  \[E(X)= 1/(1-p)\] \[V(X)=p/{(1-p)^2}\].
	
	\item	The Geometric distribution is related to the Binomial distribution in that both are based on independent trials in which the probability of success is constant and equal to $p$. 
	
	\item However, a Geometric random variable is the number of trials until the first failure, whereas a Binomial random variable is the number of successes in n trials.
\end{itemize}




\subsection{The Geometric Distribution }

	\begin{itemize}
		\item The Geometric distribution is related to the Binomial distribution in that
		both are based on independent trials in which the probability of success
		is constant and equal to p.
		\item However, a Geometric random variable is the number of trials until the
		first failure, whereas a Binomial random variable is the number of
		successes in n trials.
		\item The Geometric distributions is often used in IT security applications.
	\end{itemize}


\begin{itemize}
	\item The geometric distribution is used for Bernouilli Trials, where there outcome are classified as either failures or successes.
\end{itemize} \bigskip

In probability theory, the geometric distribution is either of two discrete probability distributions:
\begin{itemize}
	\item The probability distribution of the number of trials needed to get first success, supported on the set $\{ 1, 2, 3, \ldots\}$
	\item The probability distribution of the number of failures before the first success, supported on the set $\{ 0, 1, 2, 3, \ldots\}$
	
	\item Which of these one calls "the" geometric distribution is a matter of convention and convenience. A solution for one can quickly be surmised from the other.
	\item These two different geometric distributions should not be confused with each other. 
	\item Often, the name shifted geometric distribution is adopted for the former one (distribution of the number X); 
	\item however, to avoid ambiguity, it is considered wise to indicate which is intended, by mentioning the support explicitly.
	
	\item It’s the probability that the first occurrence of success require k number of independent trials, each with success probability p. 
	
	\item If the probability of success on each trial is p, then the probability that the kth trial (out of k trials) is the first success is
	\[  P(X = k) = (1-p)^{k-1}\,p\, \phantom{space} for k = 1, 2, 3, \ldots \]
	
	
	\item The above form of geometric distribution is used for modeling the number of trials until the first success. 
	
	\item By contrast, the following form of geometric distribution is used for modeling number of failures until the first success:
	\[ P(Y=k) = (1 - p)^k\,p\, \phantom{space} for k = 0, 1, 2, 3, \ldots\]
	
	\item
	In either case, the sequence of probabilities is a geometric sequence.
	
	\item For example, suppose an ordinary die is thrown repeatedly until the first time a "1" appears. 
	\item The probability distribution of the number of times it is thrown is supported on the infinite set ${ 1, 2, 3, \ldots }$ and is a geometric distribution with p = 1/6.
	
	\item
	
	The expected value of a geometrically distributed random variable X is 1/p and the variance is $(1 - p)/p^2$:
	\[ \mathrm{E}(X) = \frac{1}{p}, \qquad\mathrm{var}(X) = \frac{1-p}{p^2}. \]
	\item Similarly, the expected value of the geometrically distributed random variable Y (where Y corresponds to the pmf listed in the right column) is (1 - p)/p, 
	and its variance is (1 - p)/p2:
	\[ \mathrm{E}(Y) = \frac{1-p}{p}, \qquad\mathrm{var}(Y) = \frac{1-p}{p^2}.\]
	
\end{itemize}

\subsection{Bernouilli Trials}
\begin{itemize}
	\item Now consider an experiment with only two outcomes. Independent repeated trials of such an experiment are
	called Bernoulli trials, named after the Swiss mathematician Jacob Bernoulli (1654–1705). \item The term \textbf{\emph{independent
			trials}} means that the outcome of any trial does not depend on the previous outcomes (such as tossing a coin).
	\item We will call one of the outcomes the ``success" and the other outcome the ``failure".
\end{itemize}


\begin{itemize}
	\item Suppose that I am at a party and I start asking girls to dance. Let X be the number of girls that I need to ask in order to find a partner. If the first girl accepts, then X=1. If the first girl declines but the next girl accepts, then X=2. And so on. 
	
	\item When X=n, it means that I failed on the first n-1 tries and succeeded on the nth try. My probability of failing on the first try is (1-p). My probabilty of failing on the first two tries is (1-p)(1-p). 
	
	
	\item My probability of failing on the first n-1 tries is (1-p)n-1. Then, my probability of succeeding on the nth try is p. Thus, we have 
	
	\[ P(X = n) = (1-p)^{n-1}p \]
	
	\item This is known as the geometric distribution. When you have a sequence of numbers in which the (n+1)th number is a multiple of the nth number, it is called a geometric sequence. In this case, P(X = n+1) is a multiple of P(X = n). (What is that multiple?) 
	
	\item What is the probability that it will take more than n tries to succeed? We know that if I ask an infinite number of girls to dance, eventually one of them will accept. So, the probability that it will take more than n tries is the same as the probability that I fail n times. That is, 
	
	\[ P(X > n) = (1-p)^n \]
	
	If X is geometric with parameter p, what is E(X)? 
	
	\item We are faced with an infinite sum. Multiplying X times P(X) for X = 1, 2, 3, ... gives 
	
	\[
	[1] S = p + 2p(1-p) + 3p(1-p)2 +...+np(1-p)n-1 
	\]
	\item Multiply both sides by (1-p) and you have
	
	[2] (1-p)S = p(1-p) + 2p(1-p)2 + 3p(1-p)3 +...+np(1-p)n 
	
	%-------------------------------------------------------------------------------%
	
	\item Subtracting [2] from [1] gives 
	
	\[S - (1-p)S = pS = p[1 + (1-p) + (1-p)2 + ...(1-p)n] = p(1/p) = 1
	S = 1/p \]
	
	\item Therefore, the mean of the geometric distribution is equal to 1/p. If we are trying to estimate how many girls I will have to ask to dance until I find a partner, and p, the probability of one girl accepting, is .2, then on average I will have to ask five girls. 
	You will not have to know it, but for the record, the variance of the geometric distribution is (1-p)/p2. 
	
	%-----------------------
	
	The formulae for geometric distribution is
	
	%P(X=k) = (1-p)^{k-1} \times p^k%
	
	% P(X\leq 4 ) = ?
	
	%P(X=k) = (1-0.2)^{4-1} \times 0.2^4%
	
\end{itemize}





%-------------------------------------------------------------%

\noindent \textbf{Geometric Distribution}

\begin{itemize}
	
	\item Geometric distributions model (some) discrete random variables.
	\item  Typically, a Geometric random variable is the number of trials required to obtain the first \textbf{success}.
	\item For example, the number of tosses of a coin untill the first 'tail' is obtained, or a process where components from a production line are tested, in turn, until the first defective item is found.
	
\end{itemize}

\smallskip

\begin{itemize}
	\item A Geometric random variable is the number of trials until the first \textit{\textbf{success}}, whereas a Binomial random variable is the number of successes in $n$ trials.
\end{itemize}


A discrete random variable X is said to follow a Geometric distribution with parameter p, written \[X \sim Geo(p),\] if it has probability distribution
\[P(X=x) = p^{x-1}(1-p)^x\]
where
\begin{itemize}
	\item $x = 0, 1, 2, 3, \ldots$
	\item p = success probability; $0 < p < 1$
\end{itemize}



\[ P(X = n) = (1-p)^{n-1}p \]

\[ P(X > n) = (1-p)^n \]



\begin{itemize}
	\item $ E[X] = 1/p $
	
	\item The variance of the geometric distribution is 
	\[\operatorname{Var}(X) = (1-p)/p^2\].
	
\end{itemize}



The trials must meet the following requirements:

\begin{itemize}
	\item[(i)] the total number of trials is potentially infinite;
	\item[(ii)] there are just two outcomes of each trial; success and failure;
	\item[(iii)] the outcomes of all the trials are statistically independent;
	\item[(iiv)] all the trials have the same probability of success.
\end{itemize}


%-------------------------------------------------------------%


\subsection{The Geometric Distribution: Mean and Variance}

If X has a geometric distribution with parameter p, we write
\[X \sim Geo(p) \]
Expectation and Variance
If $X \sim Geo(p)$, then:

\begin{itemize}
	\item Expected Value of X : E(X) = 1/p
	\item Variance of X : Var(X) = $(1-p)/p^2$.
\end{itemize}



\begin{itemize}
	\item The Type 2 Geometric distribution has expected value and variance  \[E(X)= {1\over(1-p)}\] \[V(X)=\frac{p}{{(1-p)^2}}\].
	
\end{itemize}



	%---------------------------------------------------------------------------%
	{
		\textbf{The Geometric Distribution}
		\begin{itemize}
			\item The Geometric distribution is related to the Binomial distribution in that
			both are based on independent trials in which the probability of success
			is constant and equal to p.
			\item However, a Geometric random variable is the number of trials until the
			first failure, whereas a Binomial random variable is the number of
			successes in n trials.
			\item The Geometric distributions is often used in IT security applications.
		\end{itemize}
		
		
		Suppose that a random experiment has two possible outcomes, success
		with probability p and failure with probability 1-p .
		
		
		The experiment is repeated until a success happens. The number of
		trials before the success is a random variable X computed as follows
		
		\[P(X = k) = (1-p)^{(k-1)}\times p \]
		
		
		(i.e. The probability that first success is on the k-th trial)
		
		
		
\newpage

			\section{The Hypergeometric Distribution }
			\subsection{Definition}
			
			The following conditions characterize the hypergeometric distribution:
			The result of each draw (the elements of the population being sampled) can be classified into one of two mutually exclusive categories (e.g. Pass/Fail or Female/Male or Employed/Unemployed).
			The probability of a success changes on each draw, as each draw decreases the population (sampling without replacement from a finite population).
			
			A random variable X follows the hypergeometric distribution if its probability mass function (pmf) is given by[1]
			\[ P(X = k) = \frac{\binom{K}{k} \binom{N - K}{n-k}}{\binom{N}{n}},\]
			where
			\begin{itemize}
				\item N is the population size,
				\item K is the number of success states in the population,
				\item n is the number of draws,
				\item k is the number of observed successes,
				\item $\textstyle {a \choose b}$ is a binomial coefficient.
			\end{itemize}
			
			
			
			When sampling is done without replacement of each sampled item taken from a finite population of items, the
			Bernoulli process does not apply because there is a systematic change in the probability of success as items are
			removed from the population. 
			
			
			\begin{itemize}
				\item	When sampling without replacement is used in a situation that would otherwise
				qualify as a Bernoulli process, the hypergeometric distribution is the appropriate discrete probability distribution.
				\item	Given that X is the designated number of successes, N is the total number of items in the population, T is
				the total number of successes included in the population, and n is the number of items in the sample, the formula
				for determining hypergeometric probabilities is
			\end{itemize}
			
			
			\begin{itemize}
					\item Two types of groups
		
					\item Select $k$ from Group 1
					\item Select $n-k$ from group 2.
				\end{itemize}
				
				\[ \frac{ {n_1 \choose k_1}\times {n_2 \choose k_2} }{{n_T \choose k_T}}  \]
				
				\begin{itemize}
					\item $k_T = k_1 + k_2$
					\item $n_T = n_1 + n_2$
					
					\item Suppose we have to selected a group of 8 people from 18.
					\item Of these 18 people, 10 are males and 8 are females.
					\item What is the probability that
					the committee contains 5 females
				\end{itemize}	


\newpage
	\begin{itemize}
		\item The Geometric distribution is related to the Binomial distribution in that
		both are based on independent trials in which the probability of success
		is constant and equal to p.
		\item However, a Geometric random variable is the number of trials until the
		first failure, whereas a Binomial random variable is the number of
		successes in n trials.
		\item The Geometric distributions is often used in IT security applications.
	\end{itemize}

	
	Suppose that a random experiment has two possible outcomes, success
	with probability p and failure with probability 1-p .
	
	
	The experiment is repeated until a success happens. The number of
	trials before the success is a random variable X computed as follows
	
	\[P(X = k) = (1-p)^{(k-1)}\times p \]
	
	
	(i.e. The probability that first success is on the k-th trial)
}


	
	
	Suppose that a random experiment has two possible outcomes, success
	with probability p and failure with probability 1-p .
	
	
	The experiment is repeated until a success happens. The number of
	trials before the success is a random variable X computed as follows
	
	\[P(X = k) = (1-p)^{(k-1)}\times p \]
	
	
	(i.e. The probability that first success is on the k-th trial)
	
	
	If X has a geometric distribution with parameter p, we write
	\[X \sim Geo(p) \]
	Expectation and Variance
	If $X \sim Geo(p)$, then:
	
	\begin{itemize}
		\item Expected Value of X : E(X) = 1/p
		\item Variance of X : Var(X) = $(1-p)/p^2$.
	\end{itemize}




	\end{document}
		
