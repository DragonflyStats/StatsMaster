January 2007 Question 4 (half question)
c) The following contingency table illustrates the levels at which males and females smoke cigarettes. The figures in brackets give the expected number of observations in a cell under the hypothesis that smoking habits are independent of sex.

	Non-smoker	Less than 10 per day	10 or more per day	Sum
Males	180 (200)	60 (?)	60 (?)	300
Females	220 (?)	60 (?)	20 (40)	300
Sum	400	120	80	600
i) What is the probability that a randomly chosen person from the sample smokes more than 10 cigarettes a day?                                                                    (1 mark) 

ii) Given that the person chosen is a non-smoker, calculate the probability that this person is a female.                                                                                       (1 mark)

iii) Replace the question marks with the expected number of observations in a cell under the null hypothesis.                                                                          (2 marks) 

iv) Using the chi-squared test for independence, test the hypothesis that smoking habits are independent of sex at a significance level of 1%. Clearly state the null and alternative hypothesis. What is your conclusion?                              (6 marks)

Autumn Repeat 2007
c) The following contingency table illustrates the levels at which males and females smoke cigarettes. The figures in brackets give the expected number of observations in a cell under the hypothesis that smoking habits are independent of sex.

	Non-smoker	Less than 10/day	10 or more/day	Sum
Males	180 (200)	50 (?)	70 (?)	300
Females	220 (?)	50 (?)	30 (50)	300
Sum	400	100	100	

i)	What is the probability that a randomly chosen person from the sample smokes more than 10 cigarettes a day? 
(1 mark)

ii)	Given that the person smokes more than 10 cigarettes a day, calculate the probability that this person is a male.
(1 mark)

iii)	Replace the question marks with the expected number of observations in a cell under the hypothesis that smoking habits do not depend on sex. 
(2 marks)
iv)	Using the chi-squared test for independence, test the hypothesis that smoking habits are independent of sex at a significance level of 1%. Clearly state your conclusion.
(6 marks)


Autumn 2007/08
c) The following contingency table illustrates the levels of interest individuals have in rugby. The figures in brackets denote the expected number of individuals in a cell under the null hypothesis that the level of interest in rugby is independent of sex.

	Not interested	Somewhat interested	Very Interested	Sum
Males	400 (550)	300 (?)	300 (?)	1000
Females	700 (?)	200 (250)	100 (?)	1000
Sum	1100	500	400	

i) What is the probability that a randomly chosen person is very interested in rugby?                                                                  
  (1 mark) 

ii) Given that the person chosen is not interested in rugby, calculate the probability that this person is a female.                                                                                      
 (1 mark)

iii) Replace the question marks with the expected number of observations in a cell under the hypothesis of no association between sex and level of interest.                                                                        
  (2 marks) 
iv) Using the Pearson chi-squared test for independence, test the hypothesis that the level of interest in rugby is independent of sex at a significance level of 5%. State the null and alternative hypothesis clearly. What is your conclusion? 
                             (6 marks)

