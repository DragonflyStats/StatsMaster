


In the UK, all patients surviving a stroke are supposed to have their cholesterol
levels measured soon after their stroke and regularly thereafter. A sample of
medical records of men and women who had suffered a stroke was examined
to determine whether there was a difference between the sexes in the
proportions of stroke survivors who had had a recently recorded cholesterol
measurement. The following data were obtained. 


%============================================ %
Perform a χ2
test of the null hypothesis that there is no association
between an individual's sex and the chance of he or she having a
recently recorded cholesterol measurement.
(9)

(ii) Compute and interpret an approximate 95\% confidence interval for the
difference between the proportions of females and males having a
recently recorded cholesterol measurement.
(8) 

%============================================== %

We have a 2×2 contingency table. The null hypothesis is that there is no
association between an individual's sex and the chance of he or she having a
recently recorded cholesterol measurement. The contingency table is as
follows, with the expected frequencies in brackets in each cell (e.g. 88.48 =
131 × 206 / 305). 

%================================================= %

