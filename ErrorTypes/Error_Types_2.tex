\documentclass[a4]{beamer}
\usepackage{amssymb}
\usepackage{graphicx}
\usepackage{subfigure}
\usepackage{newlfont}
\usepackage{amsmath,amsthm,amsfonts}
%\usepackage{beamerthemesplit}
\usepackage{pgf,pgfarrows,pgfnodes,pgfautomata,pgfheaps,pgfshade}
\usepackage{mathptmx} % Font Family
\usepackage{helvet} % Font Family
\usepackage{color}
\mode<presentation> {
\usetheme{Default} % was Frankfurt
\useinnertheme{rounded}
\useoutertheme{infolines}
\usefonttheme{serif}
%\usecolortheme{wolverine}
% \usecolortheme{rose}
\usefonttheme{structurebold}
}
\setbeamercovered{dynamic}
\title[MA4704]{Technology Maths 4 \\ {\normalsize Lecture 8B/8C}}
\author[Kevin O'Brien]{Kevin O'Brien \\ {\scriptsize kevin.obrien@ul.ie}}
\date{Spring 2013}
\institute[Maths \& Stats]{Dept. of Mathematics \& Statistics, \\ University \textit{of} Limerick}
\renewcommand{\arraystretch}{1.5}
%----------------------------------------------------------------------------------------------------------%
\begin{document}





\begin{frame}
\frametitle{Example 2: Paired Difference (g)}
\textbf{Critical value}:
\begin{itemize}
\item $\alpha = 0.05, k = 2$ \item small sample , so $df = n-1 = 9$
\item As with an earlier example, CV is computed as \texttt{2.262}
\end{itemize}
\bigskip
\textbf{Decision Rule}:\\
Is $|TS| > CV$? No, we fail to reject the null hypothesis. There is not enough evidence to suggest that this additive makes an improvement.
\end{frame}

\end{document}
%--------------------------------------------------------------------------------------%
\begin{frame}
\frametitle{Example 3: Difference in Means (a) }
Two sets of patients are given courses of treatment under two different drugs. The benefits
derived from each drug can be stated numerically in terms of the recovery times; the readings are given below.

\begin{itemize}
\item Drug 1:  $n_1$ = 40 , $\bar{x}_1$ = 3.3 days and $s_1 = 1.524$
\item Drug 2:  $n_2$ = 45 , $\bar{x}_2$ = 4.3 days and $s_2 = 1.951 $
\end{itemize}
\end{frame}

%-------------------------------------------------------------------------------------------%
\begin{frame}
\frametitle{Example 3: Difference in Means (b) }
\begin{itemize}
\item
The first step in hypothesis testing is to specify the null hypothesis and an alternative hypothesis.
\item When testing differences between mean recovery times, the null hypothesis is that the two population means are equal.
\item That is, the null hypothesis is:\\
$H_0: \mu_1 = \mu_2$ ( The population means are equal)\\
$H_1: \mu_1 \neq \mu_2$ (The population means are different)\\
\end{itemize}
(Remark: Two Tailed Test, therefore $k = 2$, and $\alpha = 0.05$)
\end{frame}

%-------------------------------------------------------------------------------------------%
\begin{frame}
\frametitle{Example 3: Difference in Means (c) }
\begin{itemize}
\item The observed difference in means is 1 day.
\item The relevant formula for the standard error is
\[ S.E(\bar{x}_1 - \bar{x}_2) = \sqrt{{s^2_1\over n_1}+{s^2_2 \over n_2}} \]
\item  \[ S.E(\bar{x}_1 - \bar{x}_2) = \sqrt{{(1.524)^2 \over 40}+{(1.951)^2 \over 45}}   \]
\item  \[ S.E(\bar{x}_1 - \bar{x}_2) = 0.377\mbox{ days}\]
\end{itemize}
\end{frame}

%-------------------------------------------------------------------------------------------%
\begin{frame}[fragile]
\frametitle{Example 2: Difference in Means (d) }
\begin{itemize}
\item The Test statistic is therefore
\[ TS = {\mbox{observed} - \mbox{null} \over \mbox{Std. Error}}  = {1 - 0 \over 0.377 } = 2.65 \]
\item Lets compute the p-value of this : \\
p-value = $P(z \geq 2.65) = 0.0040$
\begin{verbatim}
> 1-pnorm(2.65)
[1] 0.004024589
\end{verbatim}

\item What is this value smaller than threshold $\alpha / k$? \\
\item $\alpha / k$ = $0.05/2$ = 0.0250? Yes the p-value is smaller than this.
\item \textbf{Conclusion:} we reject the null hypothesis. There is a significant different between both drugs, in terms of recovery times.

\end{itemize}
\end{frame}


%-------------------------------------------------------------------------------------------%
\begin{frame}[fragile]
\frametitle{Example 3: Difference in Means (a) }
\begin{itemize}
\item We will approach the same problem in example 2 , but this time using the CV approach.
\item The first two steps i.e. formally stating the null and alternative hypothesis, and computing the test statistic are the same, are the same as example 1.
\item As the sample was large, we could use $CV = 1.96$ (as always, two tailed procedure, with $\alpha=0.05$).
\begin{verbatim}
> qnorm(0.975)
[1] 1.959964
\end{verbatim}
\item Is the $TS > CV$ ?   Is $2.65 > 1.96$ ? - Yes , we reject the null hypothesis.
\end{itemize}
\end{frame}




%-------------------------------------------------------------------------------------------%
\begin{frame}
\frametitle{Example 4: Difference in Means (a) }
\begin{itemize}
\item For a random sample of 10 light bulbs, the mean bulb life is 4,000 hr with a standard deviation of 200 hours.
\item For another brand of bulbs whose useful life is also assumed to be normally distributed, a random sample of 8 has a sample mean of 4,300 hours
and a sample standard deviation of 250 hours. \item Test the hypothesis that there is no difference between the
mean operating life of the two brands of bulbs, using the 5 percent level of significance
\end{itemize}
\end{frame}
%-------------------------------------------------------------------------------------------%

\begin{frame}
\frametitle{Example 4: Difference in Means (b) }
\begin{itemize}\item $n_1 = 10$ and $n_2 = 8$.
\item $\bar{x}_1 = 4000$, $\bar{x}_2 = 4,300 $ , therefore  $\bar{x}_2 - \bar{x}_1 = 300$ hours
\item $s_1  = 200$, $s_2 = 250$ hours.
\item Small sample - Degrees of freedom $n_1 + n_2 - 2 = 10 + 8 - 2 = 16$
\end{itemize}\end{frame}
%-------------------------------------------------------------------------------------------%
\begin{frame}
\frametitle{Example 4: Difference in Means (c) }
\textbf{Pooled variance estimate}
\[ s^2_p = {(n_1 - 1)s^2_1  + (n_2 - 1)s^2_ 2\over n_1 + n_2 - 2 } = {(9 \times 200^2 ) +( 7 \times 250^2) \over 16 }  \]
\[ s^2_p  = 49843.75 \]
\end{frame}

%-------------------------------------------------------------------------------------------%
\begin{frame}
\frametitle{Example 4: Difference in Means (d) }
\textbf{Computing the Standard Error}
\[ S.E(x_1 - x_2) = \sqrt{s^2_p \left({1\over n_1}+{1\over n_2} \right)}\]

\[ S.E(x_1 - x_2) = \sqrt{49843.75 \left({1\over 10}+{1\over 9} \right)}\]

\[ S.E(x_1 - x_2) = \sqrt{11214.84} = 105.9\]

\end{frame}

%-------------------------------------------------------------------------------------------%
\begin{frame}
\frametitle{Example 4: Difference in Means (e) }
\textbf{Test Statistic and Critical Value}\\
\begin{itemize}
\item The Test Statistic is \[ TS  = {(-300) - 0 \over 105.9}  = -2.83 \]
\item The Critical Value is determined from R code with $\alpha = 0.05$, $k=2$, $df = 16 $ \texttt{qt(0.975,df=16)=  2.119905}
\item $CV = 2.120$
\item We can now apply the decision rule : Is the absolute value of the Test Statistic greater than the Critical Value?
\item Is $2.83 > 2.12$? Yes We reject $H_0$. There is evidence of a difference in means.
\end{itemize}
\end{frame}



%-------------------------------------------------------------------------------------------%

\begin{frame}
\frametitle{Example 5: Difference in Proportions (a)}
\begin{itemize}
\item An experiment is conducted investigating the long-term effects of early childhood intervention programs (such as head start).
\item In one experiment, the high-school drop out rate of the experimental group (which attended the early childhood program)
 and the control group (which did not) were compared.
\item In the experimental group, 73 of 85 students graduated from high school. \item In the control group, only 43 of 82 students graduated.
Is this difference statistically significant? (Assume that the 0.05 level is chosen.) \end{itemize}
\end{frame}

%-------------------------------------------------------------------------------------------%
\begin{frame}
\frametitle{Example 5: Difference in Proportions (b)}
\begin{itemize}
\item
The first step in hypothesis testing is to specify the null hypothesis and an alternative hypothesis.
\item When testing differences between proportions, the null hypothesis is that the two population proportions are equal.
\item That is, the null hypothesis is:\\
$H_0: \pi_1 = \pi_2$\\
$H_1: \pi_1 \neq \pi_2$\\
\end{itemize}
(Remark: Two Tailed Test k = 2, and $\alpha = 0.05$)
\end{frame}
%-------------------------------------------------------------------------------------------%
\begin{frame}
\frametitle{Example 5: Difference in Proportions (c)}
\begin{itemize}
\item The next step is to compute the difference between the sample proportions.
\item In this example, $\hat{p}_1 - \hat{p}_2$ = $73/85 - 43/82$ = $0.8588 - 0.5244$.
\item $\hat{p}_1 - \hat{p}_2$ = $0.8588 - 0.5244$ = 0.3344.
\item Difference is $33.44\%$
\end{itemize}
\end{frame}



%-------------------------------------------------------------------------------------------%
\begin{frame}
\frametitle{Example 5: Difference in Proportions (d)}
The formula for the estimated standard error is:

\[ S.E (\hat{p}_1 - \hat{p}_2)  = \sqrt{\bar{p}(100- \bar{p} \left( {1 \over n_1} + {1 \over n_2}  \right)} \]


where $\bar{p}$ is a aggregate proportion ( proportion of successes from overall sample, regardless of which group they are in).
\end{frame}

%-------------------------------------------------------------------------------------------%




\begin{frame}
\frametitle{Example 5: Difference in Proportions (d)}
\textbf{Aggregate Proportion}:\\
\[ \bar{p}  = {x_1  + x_2 \over n_1 + n_2} \times 100\% = {73+43 \over 85 + 82} \times 100\% = { 116 \over 167}\times 100\% = 69.5\% \]
\textbf{Standard Error}:\\
\[ S.E (\hat{p}_1 - \hat{p}_2)  =  \sqrt{69.5 \times 30.5 \left( {1 \over 85} + {1 \over 82}  \right)}  = 7.13\% \]
\end{frame}



%-------------------------------------------------------------------------------------------%
\begin{frame}
\frametitle{Example 5: Difference in Proportions (e)}
\textbf{Test Statistic}:
\begin{itemize} \item Observed difference :
85.88\% - 52.44\%  = 33.44\% \item [ i.e (73/85) - (43 /82) ]
\item Under the null hypothesis, the expected difference is zero.
\item Test Statistic is therefore \[T.S. = {33.44\% \over 7.13\%} = 4.69\]
\end{itemize}

\end{frame}
%-------------------------------------------------------------------------------------------%
\begin{frame}
\frametitle{Example 5: Difference in Proportions (e)}
\begin{itemize}
\item The Critical value is 1.96 ( Large sample , $\alpha = 0.05$, k=2).

\item The test statistic TS = 4.69, is greater than the critical value CV = 1.96, so we reject the null hypothesis.
\item The conclusion is that the probability of graduating from high school is greater for students who have participated in the early childhood intervention program than for students who have not.
\end{itemize}

\end{frame}






\end{document}


\end{document}
