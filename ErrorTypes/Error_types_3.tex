



%----------------------------------------------------------------------------------------------------%

\frame{
\frametitle{Conclusions in hypothesis testing}
\begin{itemize}
\item We always test the null hypothesis.
\item We reject the null hypothesis, or
\item We \emph{ fail to reject} the null hypothesis.
\end{itemize}
}
%---------------------------------------------------------------------------%
\begin{frame}
\frametitle{Types of Error}
\begin{itemize}
\item The probability of a Type I error is designated by the Greek letter alpha ( $\alpha$) and is called the Type I error rate.
\item The probability of a Type II error (the Type II error rate) is designated by the Greek letter beta ( $\beta$ ).
\item A Type II error is only an error in the sense that an opportunity to reject the null hypothesis correctly was lost.
\item It is not an error in the sense that an incorrect conclusion was drawn since no conclusion is drawn when the null hypothesis is not rejected.
\end{itemize}
\end{frame}
%---------------------------------------------------------------------------------------------%




\begin{frame}
\frametitle{Type I and Type II Error}
\begin{itemize}
\item
\item
\item A type II error would occur if it was concluded that the two drugs produced the same effect, i.e. there is no difference between the two drugs on average, when in fact they produced different ones.
\item A type II error is frequently due to sample sizes being too small.
\end{itemize}
\end{frame}
%---------------------------------------------------------------------------%
\begin{frame}
\frametitle{Types of Error}
\begin{itemize}
\item
A Type I error, on the other hand, is an error in every sense of the word. A conclusion is drawn that the null hypothesis is false when, in fact, it is true. Therefore, Type I errors are generally considered more serious than Type II errors.
 \item
The probability of a Type I error ( ) is called the significance level and is set by the experimenter. There is a trade-off between Type I and Type II errors. The more an experimenter protects himself or herself against Type I errors by choosing a low level, the greater the chance of a Type II error.
\item
Requiring very strong evidence to reject the null hypothesis makes it very unlikely that a true null hypothesis will be rejected. However, it increases the chance that a false null hypothesis will not be rejected, thus lowering the likelihood of Type II error.
\item
The Type I error rate is almost always set at .05 or at .01, the latter being more conservative since it requires stronger evidence to reject the null hypothesis at the .01 level then at the .05 level.
\end{itemize}
\end{frame}

%---------------------------------------------------------------------------%
\begin{frame}
\frametitle{Type I and II errors}
There are two kinds of errors that can be made in hypothesis testing:
\begin{itemize}
\item[(1)] a true null hypothesis can be incorrectly rejected
\item[(2)] a false null hypothesis can fail to be rejected.
\end{itemize}

The former error is called a Type I error and the latter error is called a Type II error. \\

The probability of Type I error is always equal to the level of significance that is used as the standard for rejecting
the null hypothesis; it is designated by the lowercase Greek $\alpha$ (alpha).

\end{frame}


%---------------------------------------------------------------------------%
\begin{frame}
These two types of errors are defined in the table below.

\begin{center}
\begin{tabular}{|c|c|c|}
\hline
&True State: H0 True	& True State: H0 False\\\hline
Decision: Reject H0	& Type I error&	Correct\\
Decision: Do not Reject H0	& Correct 	&Type II error\\ \hline
\end{tabular}
\end{center}

\end{frame}


%--------------------------------------------------------------------------------------------------------------------------%
\begin{frame}
\frametitle{Hypothesis Testing}
\large
The inferential step to conclude that the null hypothesis is false goes as follows: The data (or data more extreme) are very unlikely given that the null hypothesis is true.
\bigskip
This means that:
\begin{itemize}\item [(1)] a very unlikely event occurred or
\item[(2)] the null hypothesis is false. \end{itemize}
The inference usually made is that the null hypothesis is false. Importantly it doesn�t prove the null hypothesis to be false.
\end{frame}
%-------------------------------------------------------------------------------------------------------------------------%
\begin{frame}
\frametitle{Type I and II errors}
\large
There are two kinds of errors that can be made in hypothesis testing:
\begin{itemize}
\item[(1)] a true null hypothesis can be incorrectly rejected
\item[(2)] a false null hypothesis can fail to be rejected.
\end{itemize}
The former error is called a \textbf{\emph{Type I error}} and the latter error is called a \textbf{\emph{Type II error}}. \\ \bigskip
The probability of Type I error is always equal to the level of significance $\alpha$ (alpha) that is used as the standard for rejecting the null hypothesis .
\end{frame}
%---------------------------------------------------------------------------%
\begin{frame}
\frametitle{Type II Error}
\begin{itemize}

\item The probability of a Type II error is designated by the Greek letter beta ( $\beta$).
\item A Type II error is only an error in the sense that an opportunity to reject the null hypothesis correctly was lost.
\item It is not an error in the sense that an incorrect conclusion was drawn since no conclusion is drawn when the null hypothesis is not rejected.
\end{itemize}
\end{frame}
%---------------------------------------------------------------------------%
\begin{frame}
\frametitle{Types of Error}
\large
\begin{itemize}
\item
A Type I error, on the other hand, is an error in every sense of the word. A conclusion is drawn that the null hypothesis is false when, in fact, it is true. \item Therefore, Type I errors are generally considered more serious than Type II errors.
\item
The probability of a Type I error ($\alpha$ ) is set by the experimenter. \item There is a trade-off between Type I and Type II errors. The more an experimenter protects himself or herself against Type I errors by choosing a low level, the greater the chance of a Type II error.
\end{itemize}
\end{frame}
%---------------------------------------------------------------------------%
\begin{frame}
\frametitle{Types of Error}
\large
\begin{itemize}
\item
Requiring very strong evidence to reject the null hypothesis makes it very unlikely that a true null hypothesis will be rejected. \item However, it increases the chance that a false null hypothesis will not be rejected, thus lowering the likelihood of Type II error.
\item
The Type I error rate is almost always set at .05 or at .01, the latter being more conservative since it requires stronger evidence to reject the null hypothesis at the .01 level then at the .05 level.
\end{itemize}
\end{frame}



%----------------------------------------------------------------------------------------------------%
\begin{frame}
\frametitle{Type I and Type II errors - Die Example}
\begin{itemize}
\item Recall our die throw experiment example.
\item Suppose we perform the experiment twice with two different dice.
\item We don't not know for sure whether or not either of the dice is fair or crooked (favouring high values).
\item Suppose we get a sum of 401 from one die, and 360 from the other.
\end{itemize}
\end{frame}

%----------------------------------------------------------------------------------------------------%
\begin{frame}
\frametitle{Type I and Type II errors - Die Example}
\begin{itemize}
\item For our first dice (sum 401), we feel that it is likely that the die is crooked.
\item A Type I error describes the case when in fact that dice was fair, and what happened was just an unusual result.
\item For our second dice (sum 360), we feel that it is likely that the die is fair.
\item A Type II error describes the case when in fact that dice was crooked , favouring high values, and what happened was ,again, just an unusual result.
\end{itemize}
\end{frame}



\end{document}

