\documentclass[]{report}

\voffset=-1.5cm
\oddsidemargin=0.0cm
\textwidth = 480pt

\usepackage{framed}
\usepackage{subfiles}
\usepackage{graphics}
\usepackage{newlfont}
\usepackage{eurosym}
\usepackage{amsmath,amsthm,amsfonts}
\usepackage{amsmath}
\usepackage{color}
\usepackage{amssymb}
\usepackage{multicol}
\usepackage[dvipsnames]{xcolor}
\usepackage{graphicx}
\begin{document}


%---------------------------------------------------------------------%

{Today's Class}
\begin{itemize}
\item Definition of Cumulative Distribution Function.
\item Binomial Example
\item Using cumulative tables.
\item Poisson distribution - example
%\item Poisson approximation of the binomial Distribution
%\item Poisson approximation - example

\end{itemize}





%--------------------------------------------------------------------------------------%
{
\textbf{Probability Tables}
In the \textbf{Sulis} workspace there are two important tables used for this part of the course.


This class will feature a demonstration on how to read those tables.
\begin{itemize}
\item The Cumulative Binomial Tables (Murdoch Barnes Tables 1)
\item The Cumulative Poisson Tables (Murdoch Barnes Tables 2)
\end{itemize}

Please get a copy of each as soon as possible.

}

%---------------------------------------------------------------------------%
{
\textbf{Probability Tables}
\begin{itemize}
\item For some value $r$ the tables record the probability of $P(X \geq r)$.
\item The Student is required to locate the appropriate column based on the parameter values for the distribution in question.
\item A copy of the Murdoch Barnes Tables will be furnished to the student in the End of Year Exam. The Tables are not required for the first mid-term exam.
\item Knowledge of the sample space, partitioning of the sample points, and the complement rule are advised.
\end{itemize}
}
%---------------------------------------------------------------------------%



%---------------------------------------------------------------------------%

\section{Binomial Distribution}

{
\subsection{Binomial Distribution : Using Tables}
It is estimated by a particular bank that 25\% of credit card customers pay only the minimum amount due on their monthly credit card bill and do not pay the total amount due. 50 credit card customers are randomly selected.
\begin{enumerate}
\item (3 marks)What is the probability that 9 or more of the selected customers pay only the minimum amount due?
\item (3 marks) What is the probability that less than 6 of the selected customers pay only the minimum amount due?
\item (3 marks)What is the probability that more than 5 but less than 10 of the selected customers pay only the minimum amount due?
\end{enumerate}

}

{
\subsection{Binomial Distribution : Using Tables}
Demonstration on Blackboard re: how to use tables in class.
\begin{enumerate}
\item $P(X \geq 9) = 0.9084$
\item $P(X < 6) = 1- P(X \geq 6) =1 - 0.9930 = 0.0070$
\item $P(6 \leq X \leq 9) = P(X \geq 6) - P(X \geq 10) = 0.9930 - 0.8363 = 0.1567$
\end{enumerate}

}

\section{Worked Example With Tables}



\subsection{Binomial Example 4 }Using recent data provided by the low-cost
arriving on time is estimated to be 0.9.

On four different occasions I am taking a flight with Brianair.
\begin{itemize}
\item[(i)] What is the probability that I arrive on time on all four flights?
\item[(ii)] What is the probability that I arrive on time on exactly two occasions?
\end{itemize}
% Page 44
% Section 4.2.1 


\textbf{The Binomial Distribution}


Cumulative Binomial Probability Distribution Tables

The Murdoch Barnes Table set 1 ("cumulative Binomial probabilities") give cumulative values for probabilities.

Refer to tables at back of handbook.

For some value  r These tables give the probability for r or more successes in n trials :  .

[Page 46]


%=================================================%

\textbf{The Binomial Distribution}


\begin{itemize}
\item Consider a binomial experiment with n=20 and p = 0.50.
\item Use the binomial tables to compute the following.
\item First we have to find which section of those tables which tabulate the probabilities for n = 20 and p = 0.50
\item Last column on at the bottom of page 62
\end{itemize}


%=================================================%

\textbf{The Binomial Distribution}



Part 1)

What is the probability that the number of successes is at least 10  ( i.e. 10 or more? ) 



Here our "r" value is 10

By inspecting the tables we see that value is listed as ".5881" 
( i.e "0.5881").

Part 2) 

What is the probability that the number of successes does not exceed 15  ( i.e. 15 or less? )



Here we are asked for the probability of 15 success or less.

%=================================================%

However, the tables does not list such values for such probabilities.

Instead we will find the probability of the complement event, and then use the complement rule.

The complement event is the probability of 16 or more successes.


From the tables we see that  

Therefore  


%=================================================%

\textbf{The Binomial Distribution}




What is the probability that the number of successes is exactly 16?


Again, the tables does not list such values. Also it would be cumbersome to compute using the formula.
In the last question, we found out the probability of 16 or more successes?

What is the difference? The probability of more than 16 successes.
In other words the probability of  17 or more successes?

%=================================================%

\begin{itemize}
\item If we subtract the probability of 17 or more success, we are left with the probability of exactly 16 successes.
\end{itemize}

Answer 0.0046        (0.46%)




%---------------------------------------------------------------------------%



%---------------------------------------------------------------------%

{The Cumulative Distribution Function}
\begin{itemize}
\item The Cumulative Distribution Function, denoted $F(x)$, is a common way that the probabilities
of a random variable (both discrete and continuous) can be summarized.
\item The Cumulative Distribution Function, which also can be
described by a formula or summarized in a table, is defined as:
\[F(x) = P(X \leq x) \]
\item The notation for a cumulative distribution function, F(x), entails using a capital
"F".  (The notation for a probability mass or density function, f(x), i.e. using a lowercase "f". The notation is not interchangeable.
\end{itemize}


For the binomial distribution, if the probability of success is greater than 0.5, instead of
considering the number of successes, to use the table we consider
the number of failures.

%---------------------------------------------------------------------%


%---------------------------------------------------------------------%

{Binomial Example 1}
Suppose a signal of 100 bits is transmitted and the probability of
sending a bit correctly is 0.9. What is the probability of
\begin{enumerate}
\item at least 10 errors
\item exactly 7 errors
\item Between 5 and 15 errors (inclusively).
\end{enumerate}

%---------------------------------------------------------------------%

{Binomial Example 1}
\begin{itemize}
\item Since the probability of success is 0.9. We consider the distribution
of the number of failures (errors).
\item We reverse the definition of `success' and `failure'. Success is now defined as an error.
\item The probability that a bit is sent incorrectly is 0.1.
\item Let X be the total number of errors. $X \sim B(100, 0.1)$.
\item Answer : $P(X \geq 10) = 0.5487$.
\item $P(X = 7)=P(X \geq 7) - P(X \geq 8) =0.8828 - 0.7939 = 0.0889$.
\item $P(5 \leq X  \leq 15) = P(X \geq 5) - P(X \geq 16) =0.9763 - 0.0399 = 0.9364$
\end{itemize}



\bigskip
As the Murdoch Barnes cumulative Poisson Tables (Table 2) use $m$, so shall we. Recall that Tables 2 gives values of the probability $P(X \geq r )$, when X has a Poisson distribution with
parameter $m$.
\subsection{Cumulative Poisson distribution Tables.}
As with the binomial distribution, t table of probabilities also exists for the Poisson distribution. Thye give probabilities for r or more random events (in the case of the Poisson distribution, the random events are occurences).

\noindent \textbf{Example}\\
Phone calls arrive at the rate of 48 per hour at the reservation desk for Regional Airways.
\begin{itemize}
\item[(i)]  Find the probability of receiving  three calls or more in a five minute interval of time.
\item[(i)] Find the probability of getting exactly 10 calls in 15 minutes.
\end{itemize}
Solution to Part a

\begin{itemize}
\item if the poisson mean  is 48 for a 60 minute period, the Poisson mean for a five minute period is going to be 4.

N.B.   5/60=1/12          So we divide 48 by 12 to get 4

\item Go to Page 67  - top right corner - to column "m=4"

\item go to the column value for r = 3 . Which is 0.7619
\end{itemize} 
Therefore the probability of three or more arrivals in a five minute period is 0.7619

P(x3) = 0.7619 

Solution to Part b

if the poisson mean  is 48 for a sixty period, the Poisson mean for a fifteen minute period is going to be 12.

N.B.   15/60=1/4          So we divide 48 by 4 to get 12

\begin{itemize}
\item Go to Page 68  - bottom half - to column "m=12"
\item go to the column value for r = 10 , which is 0.7576
\item Therefore the probability of ten or more calls in a 15 minute period is 0.7576.
\item But we are interested in the probability of exactly ten. So we subtract the probability of 11 or more calls.
\end{itemize}
\[P(x \geq 11) = 0.6528\]

\[p(x =10) = p(x \geq 10) - p( x \geq 11) = 0.7576 -0.6528 = 0.1048\]

The probability of exactly 10 calls is 0.1048 


%---------------------------------------------------------------------%

\subsection{Useful Results}
(Demonstration on the blackboard re: partitioning of the sample space, using examples on next slide)
\begin{itemize}
\item $P(X \leq 1) = P(X=0) + P(X=1)$
\item $P(X \leq r) = P(X=0)+ P(X=1) + \ldots P(X= r)$
\item $P(X \leq 0) = P(X=0)$
\item $P(X = r) = P(X \geq r ) - P(X \geq r + 1)$
\item \textbf{Complement Rule}: $P(X \leq r-1) = P(X < r) = 1 - P(X \geq r)$
\item \textbf{Interval Rule}:$ P(a \leq X \leq  b)= P(X \geq a) - P(X \geq b + 1).$
\end{itemize}
For the binomial distribution, if the probability of success is greater than 0.5, instead of
considering the number of successes, to use the table we consider
the number of failures.

%---------------------------------------------------------------------%
{
	\textbf{Binomial Distribution : Using Tables}
	It is estimated by a particular bank that 25\% of credit card customers pay only the minimum amount due on their monthly credit card bill and do not pay the total amount due. 50 credit card customers are randomly selected.
	\begin{enumerate}
		\item (3 marks)	What is the probability that 9 or more of the selected customers pay only the minimum amount due?
		\item (3 marks) What is the probability that less than 6 of the selected customers pay only the minimum amount due?
		\item (3 marks)	What is the probability that more than 5 but less than 10 of the selected customers pay only the minimum amount due?
	\end{enumerate}
	
	\section{Binomial Distribution : Using Tables}
	Demonstration on Blackboard re: how to use tables in class.
	\begin{enumerate}
		\item $P(X \geq 9) = 0.9084$
		\item $P(X < 6) = 1- P(X \geq 6) =1 - 0.9930 = 0.0070$
		\item $P(6 \leq X \leq 9) = P(X \geq 6) - P(X \geq 10) = 0.9930 - 0.8363 = 0.1567$
	\end{enumerate}
	
	
	
	
}

{
\subsection{Binomial Distribution : Using Tables}
	It is estimated by a particular bank that 25\% of credit card customers pay only the minimum amount due on their monthly credit card bill and do not pay the total amount due. 50 credit card customers are randomly selected.
	\begin{enumerate}
		\item (3 marks)	What is the probability that 9 or more of the selected customers pay only the minimum amount due?
		\item (3 marks) What is the probability that less than 6 of the selected customers pay only the minimum amount due?
		\item (3 marks)	What is the probability that more than 5 but less than 10 of the selected customers pay only the minimum amount due?
	\end{enumerate}
}
%===========================================% 	
\section{Binomial Distribution : Using Tables}
	Demonstration on Blackboard re: how to use tables in class.
	\begin{enumerate}
		\item $P(X \geq 9) = 0.9084$
		\item $P(X < 6) = 1- P(X \geq 6) =1 - 0.9930 = 0.0070$
		\item $P(6 \leq X \leq 9) = P(X \geq 6) - P(X \geq 10) = 0.9930 - 0.8363 = 0.1567$
	\end{enumerate}
	


%==========================================%
% [Page 140] 
The vice-president of a business firm has reviewed the records of the firm?s personnel and has found that 70\% of the employees read The Wall Street Journal.
If the vice-president was to choose 10 employees at random, what is the probability that the number of these employees who do not read The Wall Street Journal is the following?
\begin{itemize} 
	\item[(i)]              At least five.
	\item[(ii)]             Between four and eight, inclusive.
	\item[(iii)]             No more than seven.
	\item[(iv) ]            What are the mean and variance of this distribution?
\end{itemize}

\section{Question 3 : Binomial Distribution}

\begin{itemize}
	\item 10\% of intended passengers dont show up. Each flight holds 50 people.
	
	\item Binomial distribution with parameters n=50 and p = 0.1
	
	\item From Murdoch Barnes Table 1 (third page of tables at back of book )
\end{itemize}
\begin{itemize}
	\item a) What is the probability that six people or more fail to show up.
	
	\[P(X=6) =0.3839\]
	
	\item 	b) Less than three people fail to show up (i.e. X=0,1 or 2)
	
	\[P(X < 3) = 1- P(X3) =    1 - 0.8883  = 0.1117\] [ANS]
	
	
	\item c) More than 2 but less than 8   (i.e. X = 3,4,5,6,7)
	
	This is equal to \[P(X=3) - P(X=8) = 0.8883 -0.1221 =  0.7662\]
	
\end{itemize}



%=================================================%

\subsection{Cumulative Distribution Function}

The cumulative distribution function (c.d.f.) of a discrete random variable $X$ is the function $F(t)$ which tells you the probability that X is less than or equal to t. \\ So if X has p.d.f. P(X = x), we have:

\[ F(t) = P(X \leq t) = \sum_{(i=0)}^{(i=t)} P(X = x) \]

In other words, for each value that X can be which is less than or equal to $t$, work out the probability that X is that value and add up all such results.

%=====================================================================%

\section{Using Tables}

\begin{itemize}
\item As the Murdoch Barnes cumulative Poisson Tables (Table 2) use $m$, so shall we. 
\item Recall that Tables 2 gives values of the probability $P(X \geq r )$, when X has a Poisson distribution with
parameter $m$.

\item For specified period we have an expected number of occurrences (m).

\item Murdoch Barnes Tables 2
Chose the column with the right value of m


\item For some value r the corresponding value in the table gives $P(X \leq r)$

N.B. It does not give $P(X \leq = r)$.
\end{itemize}







\section{Poisson distribution}
For specified period we have an expected number of occurrences (m).

Murdoch Barnes Tables 2
Chose the column with the right value of m


For some value r the corresponding value in the table gives $P(X \leq r)$

N.B. It does not give $P(X \leq = r)$.





\end{document}



%---------------------------------------%

