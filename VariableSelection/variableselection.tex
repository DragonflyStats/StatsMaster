\documentclass[a4paper,12pt]{article}
%%%%%%%%%%%%%%%%%%%%%%%%%%%%%%%%%%%%%%%%%%%%%%%%%%%%%%%%%%%%%%%%%%%%%%%%%%%%%%%%%%%%%%%%%%%%%%%%%%%%%%%%%%%%%%%%%%%%%%%%%%%%%%%%%%%%%%%%%%%%%%%%%%%%%%%%%%%%%%%%%%%%%%%%%%%%%%%%%%%%%%%%%%%%%%%%%%%%%%%%%%%%%%%%%%%%%%%%%%%%%%%%%%%%%%%%%%%%%%%%%%%%%%%%%%%%
\usepackage{eurosym}
\usepackage{vmargin}
\usepackage{amsmath}
\usepackage{graphics}
\usepackage{epsfig}
\usepackage{framed}
\usepackage{subfigure}
\usepackage{fancyhdr}

\setcounter{MaxMatrixCols}{10}
%TCIDATA{OutputFilter=LATEX.DLL}
%TCIDATA{Version=5.00.0.2570}
%TCIDATA{<META NAME="SaveForMode"CONTENT="1">}
%TCIDATA{LastRevised=Wednesday, February 23, 201113:24:34}
%TCIDATA{<META NAME="GraphicsSave" CONTENT="32">}
%TCIDATA{Language=American English}

\pagestyle{fancy}
\setmarginsrb{20mm}{0mm}{20mm}{25mm}{12mm}{11mm}{0mm}{11mm}
\lhead{Linear Models } \rhead{Kevin O'Brien} \chead{Variable Selection} %\input{tcilatex}


\begin{document}
%========================================================================== %
\section{Stepwise Regression (using \texttt{R}}

SPSS can be very opaque in determining how particularly statistical routines are carried out. Conversely the statistical programming language \texttt{R} is usually quite clear, once a familiarity with the language has been developed.

For variable selection procedures, \texttt{R} used the AIC criterion. When comparing multiple candidate models, the candidate model with the lowest AIC value is the best model. We will use \texttt{R} output to revise variable selection procedures. Recall that we used the \textbf{\textit{mtcars}} data set. The data was extracted from the 1974 Motor Trend US magazine, and comprises fuel consumption and 10 aspects of automobile design and performance for 32 automobiles (1973�74 models). For this data, we tried to determine the optimal set of independent variables to predict the dependent variables \textbf{\textit{mpg}} (miles per gallon).

\begin{description}
\item[cyl]  Number of cylinders
\item[disp]	 Displacement (cu.in.)
\item[hp]  Gross horsepower
\item[drat]	 Rear axle ratio
\item[wt] Weight (lb/1000)
\item[qsec]	 1/4 mile time
\item[vs] V/S
\item[am] Transmission (0 = automatic, 1 = manual)
\item[gear]	 Number of forward gears
\item[carb]	  Number of carburetors
\end{description}

\subsection{Backward Elimination}
The initial model contains all of the independent variables. Candidate models, whereby each of the independent variables are individually removed from the model are fitted. The AIC value for each reduced model is computed. The unreduced model is also used for comparison. The AIC values are tabulated to determine which removal results in the lowest AIC value. In this first case, the removal of cyl would reduced the AIC value from 70.898 (see bottom row) to 68.915. Thus the independent variable cyl is removed from the set of IVs.
\begin{framed}
\begin{verbatim}
Start:  AIC=70.9
mpg ~ cyl + disp + hp + drat + wt + qsec + vs + am + gear + carb

       Df Sum of Sq    RSS    AIC
- cyl   1    0.0799 147.57 68.915
- vs    1    0.1601 147.66 68.932
- carb  1    0.4067 147.90 68.986
- gear  1    1.3531 148.85 69.190
- drat  1    1.6270 149.12 69.249
- disp  1    3.9167 151.41 69.736
- hp    1    6.8399 154.33 70.348
- qsec  1    8.8641 156.36 70.765
<none>              147.49 70.898
- am    1   10.5467 158.04 71.108
- wt    1   27.0144 174.51 74.280
\end{verbatim}
\end{framed}
In the second phase, the process is repeated. This time removing vs results in an AIC value of 66.973. It is then removed from the set of IVs. For this phase, the unreduced model is the model fitted by all independent variables except cyl, which was removed in the previous phase.
\begin{framed}
\begin{verbatim}
Step:  AIC=68.92
mpg ~ disp + hp + drat + wt + qsec + vs + am + gear + carb

       Df Sum of Sq    RSS    AIC
- vs    1    0.2685 147.84 66.973
- carb  1    0.5201 148.09 67.028
- gear  1    1.8211 149.40 67.308
- drat  1    1.9826 149.56 67.342
- disp  1    3.9009 151.47 67.750
- hp    1    7.3632 154.94 68.473
<none>              147.57 68.915
- qsec  1   10.0933 157.67 69.032
- am    1   11.8359 159.41 69.384
- wt    1   27.0280 174.60 72.297

\end{verbatim}
\end{framed}
this process continues until the removal of an IV will not results in an improvement in AIC. This is indicated by having the $<none>$ ( i.e unreduced model) having the lowest AIC value. At the end of the output is the optimal model, according to the backward elimination procedure, using the IVs : am , qsec and wt.
\begin{framed}
\begin{verbatim}


Step:  AIC=61.31
mpg ~ wt + qsec + am

       Df Sum of Sq    RSS    AIC
<none>              169.29 61.307
- am    1    26.178 195.46 63.908
- qsec  1   109.034 278.32 75.217
- wt    1   183.347 352.63 82.790

Call:
lm(formula = mpg ~ wt + qsec + am)

Coefficients:
(Intercept)           wt         qsec           am
      9.618       -3.917        1.226        2.936
\end{verbatim}
\end{framed}
\subsection{Stepwise Regression}
Stepwise Regression differs from Backward Elimination, in that it allows IVs to be re-introduced. Hence the $+$ signs from the second phase onwards.
\begin{framed}
\begin{verbatim}
mpg ~ cyl + disp + hp + drat + wt + qsec + vs + am + gear + carb

       Df Sum of Sq    RSS    AIC
- cyl   1    0.0799 147.57 68.915
- vs    1    0.1601 147.66 68.932
- carb  1    0.4067 147.90 68.986
- gear  1    1.3531 148.85 69.190
- drat  1    1.6270 149.12 69.249
- disp  1    3.9167 151.41 69.736
- hp    1    6.8399 154.33 70.348
- qsec  1    8.8641 156.36 70.765
<none>              147.49 70.898
- am    1   10.5467 158.04 71.108
- wt    1   27.0144 174.51 74.280

Step:  AIC=68.92
mpg ~ disp + hp + drat + wt + qsec + vs + am + gear + carb

       Df Sum of Sq    RSS    AIC
- vs    1    0.2685 147.84 66.973
- carb  1    0.5201 148.09 67.028
- gear  1    1.8211 149.40 67.308
- drat  1    1.9826 149.56 67.342
- disp  1    3.9009 151.47 67.750
- hp    1    7.3632 154.94 68.473
<none>              147.57 68.915
- qsec  1   10.0933 157.67 69.032
- am    1   11.8359 159.41 69.384
+ cyl   1    0.0799 147.49 70.898
- wt    1   27.0280 174.60 72.297


\end{verbatim}
\end{framed}
Again, the procedure finishes when it is found that the unchanged model has the lowest of all possible AIC values.
\begin{framed}
\begin{verbatim}
Step:  AIC=61.31
mpg ~ wt + qsec + am

       Df Sum of Sq    RSS    AIC
<none>              169.29 61.307
+ hp    1     9.219 160.07 61.515
+ carb  1     8.036 161.25 61.751
+ disp  1     3.276 166.01 62.682
+ cyl   1     1.501 167.78 63.022
+ drat  1     1.400 167.89 63.042
+ gear  1     0.123 169.16 63.284
+ vs    1     0.000 169.29 63.307
- am    1    26.178 195.46 63.908
- qsec  1   109.034 278.32 75.217
- wt    1   183.347 352.63 82.790

Call:
lm(formula = mpg ~ wt + qsec + am)

Coefficients:
(Intercept)           wt         qsec           am
      9.618       -3.917        1.226        2.936
\end{verbatim}
\end{framed}
\newpage

%--------------------------------------------------------------------2nd Last Chapter : Variable Selection Procedures%

\newpage
\chapter{Variable Selection Procedures}


There are three types of variable selection procedure.
\begin{itemize}
\item Forward Selection
\item Backward Elimination
\item Stepwise selection
\end{itemize}

The \texttt{R} command we use to perform variable selection procedures is \texttt{step()}

direction  - the mode of stepwise search, can be one of ``both", ``backward", or ``forward", with a default of ``both". If the scope argument is missing the default for direction is "backward".

\section{Backward Elimination}
Our initial model includes all the predictor variables.
\footnotesize\begin{verbatim}
> step(fit.all)
Start:  AIC=70.9
mpg ~ cyl + disp + hp + drat + wt + qsec + vs + am + gear + carb

       Df Sum of Sq    RSS    AIC
- cyl   1    0.0799 147.57 68.915
- vs    1    0.1601 147.66 68.932
- carb  1    0.4067 147.90 68.986
- gear  1    1.3531 148.85 69.190
- drat  1    1.6270 149.12 69.249
- disp  1    3.9167 151.41 69.736
- hp    1    6.8399 154.33 70.348
- qsec  1    8.8641 156.36 70.765
<none>              147.49 70.898
- am    1   10.5467 158.04 71.108
- wt    1   27.0144 174.51 74.280
\end{verbatim}\normalsize

This tables tells us the effect or removing each predictor variable individually, in terms of the AIC.
Consider the first row. This tells us the AIC value of a model fitted without the ``cyl" variable would be $68.915$.
Included in the table is effect of not removing any variables. If the ``wt" variable was to be removed, the AIC value would increase to $74.280$.
\footnotesize
\begin{verbatim}
..
- cyl   1    0.0799 147.57 68.915
..
<none>              147.49 70.898
..
- wt    1   27.0144 174.51 74.280
\end{verbatim}
\normalsize

The procedure removes variables as appropriate, until it found that removing anymore variables would increase the AIC.
\footnotesize \begin{verbatim}
Step:  AIC=61.31
mpg ~ wt + qsec + am

       Df Sum of Sq    RSS    AIC
<none>              169.29 61.307
- am    1    26.178 195.46 63.908
- qsec  1   109.034 278.32 75.217
- wt    1   183.347 352.63 82.790
\end{verbatim}\normalsize

The outcome of this procedure is that ``mpg" is best explained as a linear combination of the ``am", ``qsec" and ``wt" variables.

\begin{verbatim}
Coefficients:
(Intercept)           wt         qsec           am
      9.618       -3.917        1.226        2.936
\end{verbatim}\normalsize

\[
\hat{mpg} = 9.618 - 3.917wt + 1.226qsec + 2.936am
\]

\newpage
\section{Forward Selection}

When performing forward selection we have to specify the minimal model ( in our case, an intercept only model)
and a model with all the candidate predictors.
%----------------------------------%
\footnotesize
\begin{verbatim}
step(fit.none,
scope=list(
	lower=~1,
	upper=~cyl+disp+hp+drat+wt+qsec+vs+am+gear+carb
	),
direction="forward")
\end{verbatim}
\normalsize
%----------------------------------%
\subsubsection{Phase 1}
\footnotesize
\begin{verbatim}
Start:  AIC=115.94
mpg ~ 1
       Df Sum of Sq     RSS     AIC
+ wt    1    847.73  278.32  73.217
+ cyl   1    817.71  308.33  76.494
+ disp  1    808.89  317.16  77.397
+ hp    1    678.37  447.67  88.427
+ drat  1    522.48  603.57  97.988
+ vs    1    496.53  629.52  99.335
+ am    1    405.15  720.90 103.672
+ carb  1    341.78  784.27 106.369
+ gear  1    259.75  866.30 109.552
+ qsec  1    197.39  928.66 111.776
<none>              1126.05 115.943
\end{verbatim}
\normalsize
%----------------------------------%
\subsubsection{Phase 2}
\footnotesize
\begin{verbatim}
Step:  AIC=73.22
mpg ~ wt

       Df Sum of Sq    RSS    AIC
+ cyl   1    87.150 191.17 63.198
+ hp    1    83.274 195.05 63.840
+ qsec  1    82.858 195.46 63.908
+ vs    1    54.228 224.09 68.283
+ carb  1    44.602 233.72 69.628
+ disp  1    31.639 246.68 71.356
<none>              278.32 73.217
+ drat  1     9.081 269.24 74.156
+ gear  1     1.137 277.19 75.086
+ am    1     0.002 278.32 75.217
\end{verbatim}
\normalsize
%----------------------------------%
\subsubsection{Phase 3}
\footnotesize
\begin{verbatim}
Step:  AIC=63.2
mpg ~ wt + cyl

       Df Sum of Sq    RSS    AIC
+ hp    1   14.5514 176.62 62.665
+ carb  1   13.7724 177.40 62.805
<none>              191.17 63.198
+ qsec  1   10.5674 180.60 63.378
+ gear  1    3.0281 188.14 64.687
+ disp  1    2.6796 188.49 64.746
+ vs    1    0.7059 190.47 65.080
+ am    1    0.1249 191.05 65.177
+ drat  1    0.0010 191.17 65.198
\end{verbatim}
\normalsize
%----------------------------------%
\subsubsection{Phase 4}
\footnotesize
\begin{verbatim}
Step:  AIC=62.66
mpg ~ wt + cyl + hp

       Df Sum of Sq    RSS    AIC
<none>              176.62 62.665
+ am    1    6.6228 170.00 63.442
+ disp  1    6.1762 170.44 63.526
+ carb  1    2.5187 174.10 64.205
+ drat  1    2.2453 174.38 64.255
+ qsec  1    1.4010 175.22 64.410
+ gear  1    0.8558 175.76 64.509
+ vs    1    0.0599 176.56 64.654
\end{verbatim}
\normalsize
%----------------------------------%
\newpage
\section{Stepwise Regression}

%----------------------------------%
\subsubsection{Phase 1}
The first phase commences with all the predictor variables being used in the model. The AIC is computed for each exclusion, as well as no exclusion. The candidates models are ranked accordingly. The tables shows that the greatest improvement would be the exclusion of the `cyl' variable.
\footnotesize
\begin{verbatim}
Start:  AIC=70.9
mpg ~ cyl + disp + hp + drat + wt + qsec + vs + am + gear + carb

       Df Sum of Sq    RSS    AIC
- cyl   1    0.0799 147.57 68.915
- vs    1    0.1601 147.66 68.932
- carb  1    0.4067 147.90 68.986
- gear  1    1.3531 148.85 69.190
- drat  1    1.6270 149.12 69.249
- disp  1    3.9167 151.41 69.736
- hp    1    6.8399 154.33 70.348
- qsec  1    8.8641 156.36 70.765
<none>              147.49 70.898
- am    1   10.5467 158.04 71.108
- wt    1   27.0144 174.51 74.280
\end{verbatim}
\normalsize
%----------------------------------%
\subsubsection{Phase 2}
The second phase commences with all the predictor variables being used in the model. The AIC is computed for each exclusion, as well as no exclusion. Also candidate models using previously excluded variables are ranked. The candidates models are ranked accordingly.
The results of this step show that the model would be best improved by removing the variable "vs" from the model.
\footnotesize
\begin{verbatim}
Step:  AIC=68.92
mpg ~ disp + hp + drat + wt + qsec + vs + am + gear + carb

       Df Sum of Sq    RSS    AIC
- vs    1    0.2685 147.84 66.973
- carb  1    0.5201 148.09 67.028
- gear  1    1.8211 149.40 67.308
- drat  1    1.9826 149.56 67.342
- disp  1    3.9009 151.47 67.750
- hp    1    7.3632 154.94 68.473
<none>              147.57 68.915
- qsec  1   10.0933 157.67 69.032
- am    1   11.8359 159.41 69.384
+ cyl   1    0.0799 147.49 70.898
- wt    1   27.0280 174.60 72.297
\end{verbatim}
\normalsize
%----------------------------------%
\newpage
\subsubsection{Phase 3}
The results of the third step show that the model would be best improved by removing the variable "carb" from the model.
\footnotesize
\begin{verbatim}
Step:  AIC=66.97
mpg ~ disp + hp + drat + wt + qsec + am + gear + carb

       Df Sum of Sq    RSS    AIC
- carb  1    0.6855 148.53 65.121
- gear  1    2.1437 149.99 65.434
- drat  1    2.2139 150.06 65.449
- disp  1    3.6467 151.49 65.753
- hp    1    7.1060 154.95 66.475
<none>              147.84 66.973
- am    1   11.5694 159.41 67.384
- qsec  1   15.6830 163.53 68.200
+ vs    1    0.2685 147.57 68.915
+ cyl   1    0.1883 147.66 68.932
- wt    1   27.3799 175.22 70.410
\end{verbatim}
\normalsize
%----------------------------------%
\subsubsection{Phase 4}
The results of the fourth step show that the model would be best improved by removing the variable "gear" from the model.
\footnotesize
\begin{verbatim}
Step:  AIC=65.12
mpg ~ disp + hp + drat + wt + qsec + am + gear

       Df Sum of Sq    RSS    AIC
- gear  1     1.565 150.09 63.457
- drat  1     1.932 150.46 63.535
<none>              148.53 65.121
- disp  1    10.110 158.64 65.229
- am    1    12.323 160.85 65.672
- hp    1    14.826 163.35 66.166
+ carb  1     0.685 147.84 66.973
+ vs    1     0.434 148.09 67.028
+ cyl   1     0.414 148.11 67.032
- qsec  1    26.408 174.94 68.358
- wt    1    69.127 217.66 75.350
\end{verbatim}
\normalsize
%----------------------------------%
\newpage
\subsubsection{Phase 5}
The results of the fifth step show that the model would be best improved by removing the variable "drat" from the model.
\footnotesize
\begin{verbatim}
Step:  AIC=63.46
mpg ~ disp + hp + drat + wt + qsec + am

       Df Sum of Sq    RSS    AIC
- drat  1     3.345 153.44 62.162
- disp  1     8.545 158.64 63.229
<none>              150.09 63.457
- hp    1    13.285 163.38 64.171
+ gear  1     1.565 148.53 65.121
+ cyl   1     1.003 149.09 65.242
+ vs    1     0.645 149.45 65.319
+ carb  1     0.107 149.99 65.434
- am    1    20.036 170.13 65.466
- qsec  1    25.574 175.67 66.491
- wt    1    67.572 217.66 73.351
\end{verbatim}
\normalsize
%----------------------------------%
\subsubsection{Phase 6}
The results of the sixth step show that the model would be best improved by removing the variable "disp" from the model.
\footnotesize
\begin{verbatim}
Step:  AIC=62.16
mpg ~ disp + hp + wt + qsec + am

       Df Sum of Sq    RSS    AIC
- disp  1     6.629 160.07 61.515
<none>              153.44 62.162
- hp    1    12.572 166.01 62.682
+ drat  1     3.345 150.09 63.457
+ gear  1     2.977 150.46 63.535
+ cyl   1     2.447 150.99 63.648
+ vs    1     1.121 152.32 63.927
+ carb  1     0.011 153.43 64.160
- qsec  1    26.470 179.91 65.255
- am    1    32.198 185.63 66.258
- wt    1    69.043 222.48 72.051
\end{verbatim}
\normalsize
%----------------------------------%
\newpage
\subsubsection{Phase 7}
The results of the third step show that the model would be best improved by removing the variable "hp" from the model.
\footnotesize
\begin{verbatim}
Step:  AIC=61.52
mpg ~ hp + wt + qsec + am

       Df Sum of Sq    RSS    AIC
- hp    1     9.219 169.29 61.307
<none>              160.07 61.515
+ disp  1     6.629 153.44 62.162
+ carb  1     3.227 156.84 62.864
+ drat  1     1.428 158.64 63.229
- qsec  1    20.225 180.29 63.323
+ cyl   1     0.249 159.82 63.465
+ vs    1     0.249 159.82 63.466
+ gear  1     0.171 159.90 63.481
- am    1    25.993 186.06 64.331
- wt    1    78.494 238.56 72.284
\end{verbatim}
\normalsize
%----------------------------------%
\subsubsection{Phase 8}
The results of the third step show that the model would not be improved by removing or re-adding any variable to the model.
\footnotesize
\begin{verbatim}
Step:  AIC=61.31
mpg ~ wt + qsec + am

       Df Sum of Sq    RSS    AIC
<none>              169.29 61.307
+ hp    1     9.219 160.07 61.515
+ carb  1     8.036 161.25 61.751
+ disp  1     3.276 166.01 62.682
+ cyl   1     1.501 167.78 63.022
+ drat  1     1.400 167.89 63.042
+ gear  1     0.123 169.16 63.284
+ vs    1     0.000 169.29 63.307
- am    1    26.178 195.46 63.908
- qsec  1   109.034 278.32 75.217
- wt    1   183.347 352.63 82.790
\end{verbatim}
\normalsize
\footnotesize
\begin{verbatim}
Call:
lm(formula = mpg ~ wt + qsec + am, data = mtcars)

Coefficients:
(Intercept)           wt         qsec           am
      9.618       -3.917        1.226        2.936
\end{verbatim}
\normalsize
Note that this model is the same as the outcome of the backward elimination procedure.

