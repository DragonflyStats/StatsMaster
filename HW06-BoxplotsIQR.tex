%=================================================%
\begin{frame}
\frametitle{Boxplots}
 
\noindent \textbf{3.   Percentiles}\\

A percentile is defined as a point below which a certain per cent of the observations lie e.g. the 50th percentile is the point below which half the observations lie. The percentiles that divide the data into four quarters are called: 
\begin{description}
\item[Q1]       - 25th percentile or lower quartile
\item[Q2]       - 50th percentile or median
\item[Q3]        - 75th percentile or upper quartile
\end{description}


\frametitle{The Binomial Distribution}
\Large
\noindent \textbf{Interquartile Range (IQR)}
The Interquartile Range (Q3 – Q1) is a  measure of variability commonly used for skewed data.
The IQR  the difference between the point below which 25% of your data lie and the point below which 75% of your data lie i.e. Q3  - Q1. 
To find the value of the quartiles, think of Q1 as the middle of the data less than or equal to the median, and of Q3 as the middle of the data greater than or equal to the median.
Use the same technique for calculating the mean to find these values.

\frametitle{The Binomial Distribution}
\Large
Example

\begin{itemize}
\item Find the three quartiles and the IQR of the following data

\[ 15  34  7  12  18  9  1  42  56  28  13  24  35  \]

\item First sort the data set into ascending order

\[1 7  9 12 13 15 18  24 28 34 35 42 56]\
\end{itemize}


\[1 7  9 12 13 15 18  24 28 34 35 42 56\]

•	Count how many items are in the data set ( answer 13 items)

•	Which value is the second quartile, which is the median (answer: the 7th item, which is 18)

•	Q1  median of data less than or equal to median ( 7 items)


\frametitle{Boxplots}

1	7  9 12 13 15 18 
 Answer: the 4th item, which is 12
•	Q3  median of data greater than or equal to median ( also 7 items)
\end{frame}
%=================================================%
\begin{frame}
\frametitle{Boxplots}

18  24 28 34 35 42 56 
Answer: the 4th item of this 7, which is 34

The three quartiles are therefore Q1 = 12, Q2 = 18, Q3 = 34
The interquartile range is therefore Q3 – Q1 = 22

\frametitle{Boxplots}
\Large
A graphical representation of the quartiles is called a Box plot (Figure 1.3). 
It displays 
\begin{itemize}
\item[(a)] lower quartile 
\item[(b)] median 
\item[(c)] upper quartile  
\item[(d)] interquartile range (IQR)  
\item[(e)] whiskers of length = 1.5 IQR   
\item[(f)] outlying observations
\end{itemize}

\frametitle{The Coefficient of Variation }
\Large
The Coefficient of Variation 	[page 26]

What happens if you have two sets of data with two different means and two different standard deviations? How do you decide which set is more spread out? Remember the size of the standard deviation is relative to the mean it is associated with.

The coefficient of variation (cv) is often used to compare the relative dispersion between two or more sets of data. It is formed by dividing the standard deviation by the mean and is usually expressed as a percentage i.e. (multiplied by 100). Again we distinguish between the population and sample coefficient of variation.
