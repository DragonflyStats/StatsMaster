
\documentclass[a4]{beamer}
\usepackage{amssymb}
\usepackage{graphicx}
\usepackage{subfigure}
\usepackage{newlfont}
\usepackage{amsmath,amsthm,amsfonts}
%\usepackage{beamerthemesplit}
\usepackage{pgf,pgfarrows,pgfnodes,pgfautomata,pgfheaps,pgfshade}
\usepackage{mathptmx}  % Font Family
\usepackage{helvet}   % Font Family
\usepackage{color}

\mode<presentation> {
	\usetheme{Default} % was Frankfurt
	\useinnertheme{rounded}
	\useoutertheme{infolines}
	\usefonttheme{serif}
	%\usecolortheme{wolverine}
	% \usecolortheme{rose}
	\usefonttheme{structurebold}
}

\setbeamercovered{dynamic}

\title[MA4413]{Statistics for Computing \\ {\normalsize Lecture 10B}}
\author[Kevin O'Brien]{Kevin O'Brien \\ {\scriptsize kevin.obrien@ul.ie}}
\date{Autumn 2012}
\institute[Maths \& Stats]{Dept. of Mathematics \& Statistics, \\ University \textit{of} Limerick}

\renewcommand{\arraystretch}{1.5}


\begin{document}
%-----------------------------------------------------------------------------------------------------------------------------------------------------------%

\begin{frame}
\frametitle{Information Rate}
If the time rate at which source X emits symbols is $r$ (symbols/second), the information rate R of the
source is given by

\[R = rH(X) \mbox{      (b/second)} \]

\end{frame}


%-----------------------------------------------------------------------------------------------------------------------------------------------------------%
\begin{frame}
\frametitle{Information Rate : Example}
\begin{itemize}

\item A high-resolution TV picture consists of about $2 \times 10^6$ picture elements (symbols) and 16
different brightness levels. \item Pictures are repeated at a rate of 32 per second. \item All picture elements
are assumed to be independent, and all levels have equal likelihood of occurrence. \item Calculate the
average rate of information conveyed by this TV picture source.

\end{itemize}
\end{frame}

%-----------------------------------------------------------------------------------------------------------------------------------------------------------%
\begin{frame}
\frametitle{Information Rate : Example}
\begin{itemize}
\item $H(X) = - \sum \limits^{16}_{i=1} {1\over16} \mbox{log}_2{1\over16}$ \bigskip

\item i.e. $H(X) = [ -{1\over16} \mbox{log}_2{1\over16} ] + [- {1\over16} \mbox{log}_2{1\over16} ] \ldots [ -{1\over16} \mbox{log}_2{1\over16}) ] $ \bigskip
\item Sixteen identical terms. Compute one and multiply by 16.

\[ H(X) = 16 \times [ -{1\over16} \mbox{log}_2{1\over16} ]  = -\mbox{log}_2{1\over16} = -(-4) = 4\] \bigskip
\item $H(X)= 4$ b
\item $r =  2(10^6)(32) = 64(10^6)$ elements/sec \bigskip

\item $R = rH(X) = 64(10^6)(4) = 256(l0^6) \mbox{ b/sec } = 256 \mbox{ Mb/sec }$ \bigskip
\end{itemize}
\end{frame}
%\end{document}

\end{document}