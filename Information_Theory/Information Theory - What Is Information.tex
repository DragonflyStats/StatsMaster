\documentclass[a4]{beamer}
\usepackage{amssymb}
\usepackage{graphicx}
\usepackage{subfigure}
\usepackage{newlfont}
\usepackage{amsmath,amsthm,amsfonts}
%\usepackage{beamerthemesplit}
\usepackage{pgf,pgfarrows,pgfnodes,pgfautomata,pgfheaps,pgfshade}
\usepackage{mathptmx}  % Font Family
\usepackage{helvet}   % Font Family
\usepackage{color}

\mode<presentation> {
 \usetheme{Default} % was Frankfurt
 \useinnertheme{rounded}
 \useoutertheme{infolines}
 \usefonttheme{serif}
 %\usecolortheme{wolverine}
% \usecolortheme{rose}
\usefonttheme{structurebold}
}

\setbeamercovered{dynamic}

\title[MA4413t]{Statistics for Computing \\ {\normalsize Lecture 10B}}
\author[Kevin O'Brien]{Kevin O'Brien \\ {\scriptsize kevin.obrien@ul.ie}}
\date{Summer 2011}
\institute[Maths \& Stats]{Dept. of Mathematics \& Statistics, \\ University \textit{of} Limerick}

\renewcommand{\arraystretch}{1.5}


\begin{document}


%-------------------------------------------------------------------------------------------------------------------------------------------------------------------------------------------%


% page 247 

\begin{frame}
\frametitle{Information Rate}
If the time rate at which source X emits symbols is r (symbols/s), the lnformation rate R of the
source is given by

%\[R ; r11(X) b/s (10.10)\]

\end{frame}


%---------------------------------------------------------------------------------------------------------------------------------------%
\begin{frame}
\frametitle{ MUTUAL INFORMATION}
\textbf{A. Conditional and Joint Entropies:}\\
Using the input probabilities P(x,), output probabilities $P(y_i)$, transition probabilities P(yJ|>r,),
and joint probabilities P(x,, yy), we can define the following various entropy functions for a channel
with m inputs and n outputs:
 
\begin{itemize}
\item H(X) = - X P(xi) log %; P(x;) (10.21)
\item H(Y) = -2P(yj)]%<>gz Pty;) UU-22)
\item H<X I Y) = - X X %P<>¤r,yy)1<¤gz Ptmlyj) <10.23)
\item H = -2 ZP%(>¢..y,) 1022 P(y,|X.) (10-24)
\item H(X, Y) - -2 Z F%<><..y,)1¤zz P(><r.y;) <10·25)
\end{itemize}
\end{frame}

%---------------------------------------------------------------------------------------------------------------------------------------%
\begin{frame}
\frametitle{Conditional and Joint Entropy}
These entropies can be interpreted as follows: H(X) is the average uncertainty of the channel input,
and H(Y) is the average uncertainty of the channel output. The conditional entropy H(X]Y) is a
measure of the average uncertainty remaining about the channel input after the channel output has
been observed. And H(X] Y) is sometimes called the equivncation of X with respect to Y. \begin{itemize} \item The
conditional entropy H(Y|X) is the average uncertainty of the channel output given that X was
transmitted.\item  The joint entropy H(X, Y) is the average uncertainty of the communication channel as a
whole.\end{itemize}
\end{frame}
%-----------------------------------------------------------------------------------------------%
\begin{frame}
Two useful relationships among the above various entropies are
\begin{itemize} \item
$H(X, Y)=H(X|Y)+H(Y) (10,26)$
$H(X,Y)=H(Y|X)+H(X) (10.27)$
\end{itemize}
B. Mutual Information:
The mutual information 1(X; Y) of a channel is deiined by
I(X; Y) = H(X)— H(X|Y) b/symbol (10.28)
\end{frame}

%-----------------------------------------------------------------------------------------------%
\begin{frame} % ULCIS 
\frametitle{Self Information}Self-information
This is defined by the following mathematical formula:$I(A) = −logb P(A)$

The self-information of an event measures the amount of one’s surprise
evoked by the event. The negative logarithm $−logb P(A)$ can also be written as \[
log_b  {1 \over P(A)} \]
Note that log(1) = 0, and that $| − log(P(A))|$ increases as P(A) decreases
from 1 to 0. This supports our intuition from daily experience. For example,
a low-probability event tends to cause more ``surprise".
\end{frame}
%---------------------------------------------------------------------------------------------------------------------------------------%



\begin{frame}
\frametitle{Code efficiency and Code redundancy}
% Pg 253/254
The parameter $L$ represents the average number of bits per source symbol used in the source coding process.
The code efficiency is defined as \[\nu = {L_{min} \over L} \]where $L_{min}$ is the minimum possiblve value of $L$. When $\nu$ approaches unity, the codes is said to be efficient. 
The code redundancy $\gamma$ is defined as $\gamma = 1- \nu$.
\end{frame}


%---------------------------------------------------------------------------------------------------------------------------------------%
\begin{frame}
%page 254
\frametitle{Source Coding Theorem}
The source coding theorem states that for zi DMS X with entropy $H(X)$, the average code word length $L$ per symbol is bounded as
L 2 H(X) (10.52)

and further, L can bc made as close to H(X) as dcsircd for some suitably chosen code.
Thus, with$ L_{min} \geq H(X)$.

The code efficiency can be rewritten as
\[\nu = {H(X) \over L} \]
\end{frame}



%-----------------------------------------------------------------------------------------------------------------------------------------------------------%
\begin{frame}
\frametitle{ Kraft inequality}
\begin{itemize}
\item Let X be a DMS with alphabet ($x _i = \{1, 2, . . . ,m\}$). Assume that the length of the assigned binary
code word corresponding to x, is n,.
\item A necessary and sufficient condition for the existence of an instantaneous binary code is
 
 \[ K = \sum^{m}_{i=1}2^{-n_i} \leq 1 \]
which is known as the \textbf{Kraft inequality}.
\item Note that the Kraft inequality assures us of the existence of an instantaneously decodable code
with code word lengths that satisfy the inequality. But it does not show us how to obtain these code
words, nor does it say that any code that satisfies the inequality is automatically uniquely decodable
\end{itemize}
\end{frame}


%-----------------------------------------------------------------------------------------------------------------------------------------------------------%
\begin{frame}
10.5. A high-resolution black—and-white TV picture consists of about $2 \times 10^6$  picture elements and 16
different brightness levels. Pictures are repeated at a rate of 32 per second. All picture elements
are assumed to be independent, and all levels have equal likelihood of occurrence. Calculate the
average rate of information conveyed by this TV picture source.\\ \bigskip

%Hm — -§ img L — 4 11/111e111e111
%P, 16 Z is
%1 : 2(l0§)(32) : 64(l0°) elements/s
%Hence, by Eq. (10.10)
%R = rH(X) : 64(l0°)(4) = 256(l0°) b/s Z 256 Mb/s

\end{frame}

\end{document}