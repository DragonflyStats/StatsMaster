\documentclass[a4]{beamer}
\usepackage{amssymb}
\usepackage{graphicx}
\usepackage{subfigure}
\usepackage{newlfont}
\usepackage{amsmath,amsthm,amsfonts}
%\usepackage{beamerthemesplit}
\usepackage{pgf,pgfarrows,pgfnodes,pgfautomata,pgfheaps,pgfshade}
\usepackage{mathptmx} % Font Family
\usepackage{helvet} % Font Family
\usepackage{color}
\mode<presentation> {
	\usetheme{Default} % was Frankfurt
	\useinnertheme{rounded}
	\useoutertheme{infolines}
	\usefonttheme{serif}
	%\usecolortheme{wolverine}
	% \usecolortheme{rose}
	\usefonttheme{structurebold}
}
\setbeamercovered{dynamic}
\title[MA4413]{Statistics for Computing \\ {\normalsize MA4413 Lecture 11A}}
\author[Kevin O'Brien]{Kevin O'Brien \\ {\scriptsize kevin.obrien@ul.ie}}
\date{Autumn 2011}
%\institute[Maths \& Stats]{Dept. of Mathematics \& Statistics, \\ University \textit{of} Limerick}
\renewcommand{\arraystretch}{1.5}
%------------------------------------------------------------------------%

\begin{document}
	\begin{frame}
	\frametitle{ Kraft inequality}
	\begin{itemize}
		\item Let X be a DMS with alphabet ($x _i = \{1, 2, . . . ,m\}$). Assume that the length of the assigned binary
		code word corresponding to x, is n.
		\item A necessary and sufficient condition for the existence of an instantaneous binary code is
		
		\[ K = \sum^{m}_{i=1}2^{-n_i} \leq 1 \]
		which is known as the \textbf{Kraft inequality}.
		\item Note that the Kraft inequality assures us of the existence of an instantaneously decodable code
		with code word lengths that satisfy the inequality. But it does not show us how to obtain these code
		words, nor does it say that any code that satisfies the inequality is automatically uniquely decodable
	\end{itemize}
\end{frame}


%-----------------------------------------------------------------------------------------------------------------------------------------------------------%
\begin{frame}
\frametitle{Kraft inequality}
\begin{itemize}
	\item \textbf{\emph{Kraft's inequality}} gives a necessary and sufficient condition for the existence of a uniquely decodable code for a given set of codeword lengths (more so variable length codes)
	
	\item More specifically, Kraft's inequality limits the lengths of codewords in a prefix code, and can be thought of in terms of a constrained budget to be spent on codewords, with shorter codewords being more expensive.
	
	
\end{itemize}
\end{frame}
%-----------------------------------------------------------------------------------------------------------------------------------------------------------%
\begin{frame}
\frametitle{Kraft inequality}
\begin{itemize}
\item Let X be a DMS with alphabet ($x _i = \{1, 2, \ldots ,m\}$). \item Assume that the length of the assigned binary
code word corresponding to $x_i$ is $n_i$.
\item Kraft inequality is given as

\[ K = \sum^{m}_{i=1}2^{-n_i} \leq 1 \]

\end{itemize}
\end{frame}
%-----------------------------------------------------------------------------------------------------------------------------------------------------------%
\begin{frame}
\frametitle{Kraft inequality}
\begin{itemize}
\item If Kraft's inequality holds with strict inequality (i.e. $K < 1$), the code has some redundancy.
\item If Kraft's inequality holds with strict equality (i.e. $K = 1$), the code in question is a complete code.
\item The closer K is to 1, the more efficient the code is.
\item If Kraft's inequality does not hold, the code is not uniquely decodable.
\end{itemize}
\end{frame}
%-----------------------------------------------------------------------------------------------------------------------------------------------------------%
\begin{frame}
\frametitle{Kraft inequality}
\begin{itemize}
\item Note that the Kraft inequality assures us of the existence of an instantaneously decodable code
with code word lengths that satisfy the inequality. \item However it does not show us how to obtain these code
words, nor does it say that any code that satisfies the inequality is automatically uniquely decodable
\end{itemize}
\end{frame}




%---------------------------------------------------------------------------------------------------------------------------------------%
\begin{frame}
\frametitle{Kraft inequality}
Compute the value for K in each case, and determine whether Kraft's Inequality is observed.
\begin{itemize}
\item \textbf{code 1 and 2} K = 1 so Kraft's inequality is obeyed.  \[ K = \sum^{m}_{i=1}2^{-n_i} = (2^{-2} \times 4) = 1 \]
\item \textbf{code 3} K = 1.5 so Kraft's inequality is not obeyed.   \[ K = \sum^{m}_{i=1}2^{-n_i} = (2^{-1} + 2^{-1} + 2^{-2} + 2^{-2}) = 1.5 \]
\item \textbf{code 4 } K = 1 so Kraft's inequality is obeyed.   \[ K = \sum^{m}_{i=1}2^{-n_i} = (2^{-1} + 2^{-2} + 2^{-3} + 2^{-3}) = 1 \]

\item \textbf{code 5 and 6 } K =  0.9375 so Kraft's inequality is obeyed.   \[ K = \sum^{m}_{i=1}2^{-n_i} = (2^{-1} + 2^{-2} + 2^{-3} + 2^{-4}) =  0.9375 \]
\end{itemize}
\end{frame}

\begin{frame}
\frametitle{Kraft inequality: Fixed length codes.}
\begin{itemize}
\item Consider a 6 symbol alphabet: $\{x_1, x_2, x_3, x_4, x_5,x_6\}$
\item If a fixed length code is to be used, how many digits are required for each symbol.
\item Each codeword must be distinct.
\item Answer: We must have 3 digits in every codeword.
\item Compute K for this alphabet, and use Kraft's Inequality to appraise this code.
\item $K = 6 \times 2^{-3} = 0.750$. This code is uniquely decodable, but not very efficient.
\end{itemize}
\end{frame}


% - Lossless data compression.
% - Huffman Coding
% - Inverse Mapping

\end{document}