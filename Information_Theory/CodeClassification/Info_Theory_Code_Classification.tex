\documentclass[a4]{beamer}
\usepackage{amssymb}
\usepackage{graphicx}
\usepackage{subfigure}
\usepackage{newlfont}
\usepackage{amsmath,amsthm,amsfonts}
%\usepackage{beamerthemesplit}
\usepackage{pgf,pgfarrows,pgfnodes,pgfautomata,pgfheaps,pgfshade}
\usepackage{mathptmx} % Font Family
\usepackage{helvet} % Font Family
\usepackage{color}
\mode<presentation> {
	\usetheme{Default} % was Frankfurt
	\useinnertheme{rounded}
	\useoutertheme{infolines}
	\usefonttheme{serif}
	%\usecolortheme{wolverine}
	% \usecolortheme{rose}
	\usefonttheme{structurebold}
}
\setbeamercovered{dynamic}
\title[MA4413]{Statistics for Computing \\ {\normalsize MA4413 Lecture 11A}}
\author[Kevin O'Brien]{Kevin O'Brien \\ {\scriptsize kevin.obrien@ul.ie}}
\date{Autumn 2011}
\institute[Maths \& Stats]{Dept. of Mathematics \& Statistics, \\ University \textit{of} Limerick}
\renewcommand{\arraystretch}{1.5}
%------------------------------------------------------------------------%
\begin{document}
%---------------------------------------------------------------------------------------------------------------------------------------%
\begin{frame}
\frametitle{Classification of Codes}
In this section we look at to classify codes according to the following categories.
\begin{enumerate}
\item Fixed Length Codes
\item Variable Length Codes
\item Distinct Codes
\item Prefix-Free Codes
\item Uniquely decodable codes
\item Instantaneous Codes
\item Optimal Codes
\end{enumerate}
\end{frame}
%---------------------------------------------------------------------------------------------------------------------------------------%
\begin{frame}
\frametitle{Classification of Codes}
Classification of codes is best illustrated by an example. Consider the table below where a source of
size 4 has been encoded in binary codes with symbol 0 and 1.\\ \bigskip
% Table 10-I Binary Codes
\begin{center}
\begin{tabular}{|c| c| c| c| c| c| c|}
\hline
X& Code l& Code 2& Code 3 &Code 4& Code 5& Code 6\\\hline
$x_1$& 00& 00 &0 &0 &0 &1\\
$x_2$& 01& 01 &1 &10 &01 &01\\
$x_3$ &00 &10& 00& 110& 011 &001\\
$x_4$ &11& 11& 11& 111 &0111 &0001\\\hline
\end{tabular}
\end{center}
\end{frame}


%---------------------------------------------------------------------------------------------------------------------------------------%
\begin{frame}
\begin{itemize}
\item[1.] Fixed-Length Codes: A fixed-length code is one whose code word length is fixed. Code 1 and code 2are
fixed-length codes with length 2.
\item[2.] Variable-Length Codes: A variable-length code is one whose code word length is not fixed. All codes except codes 1 and 2 are variable-length codes.
\item[3.] Distinct Codes:
A code is distinct if each code word is distinguishable from other code words. All codes except code 1 are distinct codes. Notice the codes for $x_l$ and $x_3$.
\item[4.] Prefix-Free Codes:
A code in which no code word can be formed by adding code symbols to another code word is
called a prefix-free code. Thus, in a prefix-free code no code word is a prefix of another. Codes 2, 4,
and 6 are prefix-free codes.
\end{itemize}
\end{frame}

%-----------------------------------------------------------------------------------------------------------------------------------------------------------%
\begin{frame}
5. Uniquely Decodable Codes
\begin{itemize}
\item A distinct code is uniquely decodable if the original source sequence can be reconstructed perfectly
from the encoded binary sequence. \item Note that code 3 is not a uniquely decodable code. \item
For example, the binary sequence 1001 may correspond to the source sequences $x_2x_3x_2$ or $x_2x_1x_1x_2$.
\item
A sufficient condition to ensure that a code is uniquely decodable is that no code word is a prefix of
another. \item Thus, the prefix-free codes 2, 4, and 6 are uniquely decodable codes. Note that the prefix -free
condition is not a necessary condition for unique decodability. \item For example, code 5 does
not satisfy the prefix-free condition, and yet it is uniquely decodable since the bit 0 indicates the
beginning of each code word of the code.
\end{itemize}
\end{frame}
%-----------------------------------------------------------------------------------------------------------------------------------------------------------%
\begin{frame}
6. Instantaneous Codes
\begin{itemize} \item A uniquely decodable code is called an instantaneous code if the end of any code word is
recognizable without examining subsequent code symbols. \item The instantaneous codes have the property
previously mentioned that no code word is a prefix of another code word.  \end{itemize}
7. Optimal Codes
\begin{itemize}
\item A code is said to be optimal if it is instantaneous and has minimum average length $L$ for a given
source with a given probability assignment for the source symbols.
\end{itemize}
\end{frame}
%-----------------------------------------------------------------------------------------------------------------------------------------------------------%
\end{document}