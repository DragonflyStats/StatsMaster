\documentclass[a4]{beamer}
\usepackage{amssymb}
\usepackage{graphicx}
\usepackage{subfigure}
\usepackage{newlfont}
\usepackage{amsmath,amsthm,amsfonts}
%\usepackage{beamerthemesplit}
\usepackage{pgf,pgfarrows,pgfnodes,pgfautomata,pgfheaps,pgfshade}
\usepackage{mathptmx}  % Font Family
\usepackage{helvet}   % Font Family
\usepackage{color}

\mode<presentation> {
 \usetheme{Default} % was Frankfurt
 \useinnertheme{rounded}
 \useoutertheme{infolines}
 \usefonttheme{serif}
 %\usecolortheme{wolverine}
% \usecolortheme{rose}
\usefonttheme{structurebold}
}

\setbeamercovered{dynamic}

\title[MathsCast]{MathsCast Presentations \\ {\normalsize Data Compression}}
\author[Kevin O'Brien]{Kevin O'Brien \\ {\scriptsize kevin.obrien@ul.ie}}
\date{Summer 2011}
\institute[Maths \& Stats]{Dept. of Mathematics \& Statistics, \\ University \textit{of} Limerick}

\renewcommand{\arraystretch}{1.5}


%------------------------------------------------------------------------%
\begin{document}
%----------------------------------------------------------------------------------------------------%
\frame{
\frametitle{Data Compression}

}




%-----------------------------------------------------------------------------------%
\begin{frame}
\frame{Past Paper Example}
\begin{itemize}
\item The frequency of a `0' signal to a binary channel is 0.6. If the `0' is the input, then `0' is the output with probability 0.8.

\item If `1' is the input, then output is `1' with probability 0.9.

\item Calculate the probability that the input is 1, given that the output is 1.
\end{itemize}

\end{frame}
%-----------------------------------------------------------------------------------%
\begin{frame}
\frame{Past Paper Example}
\begin{itemize}
\item Let $IU$ be the event that the input is `1' (i.e. $I$ for Input, $U$ rather than $O$ for `1'.)
\item Let $OU$ be the event that the output is `1' (i.e. $O$ for output.)
\item Let $IZ$ be the event that the input is `0' (i.e. $Z$ for Zero.)
\item Let $OZ$ be the event that the output is `0'
\end{itemize}

\begin{itemize}
\item $P(IZ) = 0.6$  Necessarily $P(IU) = 0.4$
\item $P(OZ|IZ) = 0.8$  Necessarily $P(OU|IZ) = 0.2$
\item $P(OU|IU) = 0.9$  Necessarily $P(OZ|IU) = 0.1$
\end{itemize}
\end{frame}
%-----------------------------------------------------------------------------------%
\begin{frame}
\frametitle{Example: Part 1}
Calculate the probability that the output is `1'.
\begin{itemize}
\item $P(OU) = P(OU \ cap IU) + P(OU \ cap IZ)$

\item $P(OU) = [P(OU|IU) \times P(IU) ] + [P(OU|IZ) \times P(IZ) ]$

\item $P(OU) = [0.9 \times 0.4 ] + [0.2 \times 0.6 ]$

\item $P(OU) = [0.36] + [0.12] = \boldsymbol{0.48}$
\end{itemize}
Necessarily $P(OZ) = 0.52$.
\end{frame}
%-----------------------------------------------------------------------------------%
\begin{frame}
\frametitle{Example: Part 2}
Calculate the probability that the output is `0' ( from first principles ).
\begin{itemize}
\item $P(OZ) = P(OZ \ cap IU) + P(OZ \ cap IZ)$

\item $P(OZ) = [P(OZ|IU) \times P(IU) ] + [P(OZ|IZ) \times P(IZ) ]$

\item $P(OZ) = [0.1 \times 0.4 ] + [0.8 \times 0.6 ]$

\item $P(OZ) = [0.04] + [0.48] = \boldsymbol{0.52}$
\end{itemize}

\end{frame}
%-----------------------------------------------------------------------------------%
\begin{frame}
\frametitle{Example: Part 3}
Compute $P(IU|OU)$ 

\begin{itemize}
\item Use Bayes's Theorem to solve this.

\[P(IU|OU) = { P(OU|IU) \times P(IU)\over P(OU) } \]

\item $P(OU|IU)\times P(IU)$ = $0.9 \times 0.4 = 0.36$
\item From before $P(OU) = 0.48$
\item Therefore $P(IU|OU) = 0.36 / 0.48 = \boldsymbol{0.75}$
\item Necessarily $P(IZ|OU)  = 0.25$
\end{itemize}

\end{frame}

%-----------------------------------------------------------------------------------%
\begin{frame}
\frametitle{Example: Part 4}
Compute $P(IZ|OZ)$ 
\begin{itemize}
\item Use Bayes's Theorem to solve this.

\[P(IZ|OZ) = {P(OZ|IZ)\times P(IZ)\over P(OZ)} \]

\item $P(OZ|IZ)\times P(IZ)$ = $0.8 \times 0.6 = 0.48$

\item From before $P(OZ) = 0.52$

\item Therefore $P(IZ|OZ) = 0.48 / 0.52 = \boldsymbol{0.923}$

\item Necessarily $P(IU|OZ)  = 0.077$
\end{itemize}
\end{frame}
%--------------------------------------------------------------------------------%
\begin{frame}
\begin{itemize}
\item The specifications for the length of a component is a minimum of 9.90 mm  and a maximum of 10.44mm.

\item A batch of parts is produced that is normally distributed with a a mean of 10.20mm and a standard deviation of 0.20mm.

\item Each part costs $\$10$ to produce. Those that are too short or too long have to be scrapped, or shortened, at a further cost of $\$8$.

\item Compute the percentage of the parts which are (i) Undersize (ii) Oversize.

\item Compute the expecte cost of producing 10000 parts.

\item Suppose we are able to adjust the processing method such that the mean is halfway between the upper and lower specification.
\end{itemize}
\end{frame}


%--------------------------------------------------------------------------------%
\begin{frame}
\begin{itemize}
\item The number of undersize parts is therefore 670.
\item The number of oversize parts is therefore 1150.
\item The number of parts that don't meet specification is therefore 1820.
\item The additional cost is $1820 \times 8 = 14,560$
\end{itemize}
\end{frame}


%--------------------------------------------------------------------------------%
\begin{frame}
If the mean is changed to 10.17mm, then the symmetry of the normal distribution will mean that the same number of items are too short as too long.

To compute this proportion

\[ z = {10.44 - 10.17 \over 0.2} = 1.35 \]

\begin{itemize}
\item From the tables $P(Z \geq 1.35) = 0.0885$

\item 885 items will be too short, 885 items will be too long.
\end{itemize}
\end{frame}



%-------------------------------------------------------------%


\begin{frame}
Information entropy is often used as a preliminary test for randomness. 


Generally speaking, random data will have a high level of information entropy, and a low level of information entropy is a 
good indicator that the data isn't random. 
(A low level of entropy isn't definitive proof that the data isn't random, but it means you should be suspcious and submit the generator to further tests.)


However, the converse relation doesn't hold, meaning a high degree of 
information entropy is no guarantee of randomness. 
For example, a compressed file (e.g., a ZIP file) has a high level of information entropy, but is 
in fact highly structured, and it will fail many other tests for randomness. 
Hence, you have to be a little careful using information entropy as a metric for randomness. 
To get meaningful results, you really need to combine it with other tests.
\end{frame}


%-------------------------------------------------------------%




\begin{frame}
In information theory, entropy is a measure of the uncertainty associated with a random variable. 
The term by itself in this context usually refers to the Shannon entropy, which quantifies, in the 
sense of an expected value, the information contained in a message, usually in units such as bits. 
Equivalently, the Shannon entropy is a measure of the average information content one is missing 
when one does not know the value of the random variable. 
The concept was introduced by Claude E. Shannon in his 1948 paper "A Mathematical Theory of Communication".


A Source X delivers a message among a set of M possible messages $x_i$ , $i = 1, \ldots ,M$
with fixed probabilities $p_i = p(x_i)$,  $i = 1, \ldots ,M$ (pre-defined probabilities).
Messages can be seen as ”events”.
\end{frame}


%-------------------------------------------------------------%


\begin{frame}


Example 1:\\ Binary source with probabilities p and 1 − p\\
H(X) = −p log p − (1 − p) log(1 − p)


Example 2:\\ M equiprobable messages: \\
H(x) = logM


\end{frame}


%-------------------------------------------------------------%


\begin{frame}
\frametitle{Conditional Entropy}
Is always lower than a priori uncertainty on X
$H(X|Y ) < H(X)$
\end{frame}




%-------------------------------------------------------------%
\begin{frame}
\frametitle{Mutual Information}
Mutual information defined as
I (X;Y ) = H(X) − H(X|Y )
\begin{itemize}


\item Represents by how much uncertainty decreases on average by knowledge of Y
\item Represents information gained on X by observing Y
\end{itemize}
\end{frame}




%-----------------------------------------------------------%
\begin{frame}


The $t$ distribution is the appropriate basis for
determining the standardized test statistic when the sampling
distribution of the mean is normally distributed but $s$ is not
known. The sampling distribution can be assumed to be normal
either because the population is normal or because the sample is
large enough to invoke the central limit theorem. \\ The $t$
distribution is required when the sample is small ($n < 30$). For
larger samples, normal approximation can be used. For the critical
value approach, the procedure is identical to that described in
Section 10.3 for the normal distribution, except for the use of $t$
instead of z as the test statistic.
\end{frame}
%-----------------------------------------------------------%
\begin{frame}
The intelligence quotient (IQ) of 36 randomly chosen students was measured.
Their average IQ was 109.9 with a variance of 324.
The average IQ of the population as a whole is 100.
\begin{enumerate}
\item Calculate the p-value for the test of the hypothesis that on average
students are as intelligent as the population as a whole against the alternative that on average students are more intelligent.
\item Can we conclude at a significance level of $1\%$ that students are on average more intelligent than the population as a whole?
\item Calculate a $95\%$ confidence interval for the mean IQ of all students.
\end{enumerate}
\[Z_{Test} = \frac{X- \mu}{\frac{\sigma}{\sqrt{n}}} = \frac{109.9 - 100}{\frac{18}{\sqrt{36}}} = \frac{9.9}{3} = 3.3\]
\[p.value = P(Z \geq Z_{Test}) = P(Z \geq 3.3) = 0.00048\]
\end{frame}
%--------------------------------------------------------------%
\begin{frame}
\begin{itemize}
\item $\bar{X} \pm t_{1-\alpha/2,\nu}S.E.(\bar{X})$
\item $\nu = 1.96$
\item $t_{1-\alpha/2,\nu} = 1.96$
\item $109.9 \pm (1.96 \times 3) = [104.02, 115.79]$
\end{itemize}
\end{frame}
%--------------------------------------------------------------%
\begin{frame}
For a particular Java assembler interface, the operand stack size has the
following probabilities:
\begin{tabular}{|c||c|c|c|c|c|}
  \hline
  % after \\: \hline or \cline{col1-col2} \cline{col3-col4} ...
  Stack Size  & 0 & 1 & 2 & 3 & 4 \\ \hline
 Probability & .15 & .05 & .10 &.20 &.50\\
  \hline
\end{tabular}
\begin{itemize}
\item Calculate the expected stack size.
\item Calculate the variance of the stack size.
\end{itemize}
\end{frame}\end{document}

