\documentclass[]{report}

\voffset=-1.5cm
\oddsidemargin=0.0cm
\textwidth = 480pt

\usepackage{framed}
\usepackage{subfiles}
\usepackage{graphics}
\usepackage{newlfont}
\usepackage{eurosym}
\usepackage{amsmath,amsthm,amsfonts}
\usepackage{amsmath}
\usepackage{color}
\usepackage{amssymb}
\usepackage{multicol}
\usepackage[dvipsnames]{xcolor}
\usepackage{graphicx}
\begin{document}




Define quota sampling. In what circumstances would you use it?
In what circumstances would you use stratified random sampling?
Give two ways in which stratified random sampling differs from quota sampling.

\subsection{Basic Probability Questions}
Q1a.  Two fair dice are thrown. What is the probability of at least one odd number?
Q1b.  What is the probability of at least one odd number if four fair dice are thrown?

%
%http://stattrek.com/Lesson2/Binomial.aspx
%http://stattrek.com/Lesson2/Normal.aspx
%http://www.intmath.com/counting-probability/12-binomial-probability-distributions.php
%http://www.elderlab.yorku.ca/~aaron/Stats2022/BinomialDistribution.htm
%http://www.mathsisfun.com/combinatorics/combinations-permutations.html

%-----------------------------------------------------------------%
\section{Continuous Joint Probability Distribution}
Continuous Joint Probability Distributions arise from groups of continuous random variables.
Continuous joint probability distributions are characterized by the Joint Density Function, which is similar to that of a single variable case, except that this is in two dimensions.
The joint density function $f(x,y)$ is characterized by the following:

%See Fin Stats






%---------------------------------------------------------------------%

\subsection*{Section 3 : Probability}

How to Compute Probability: Equally Likely Outcomes
Sometimes, a statistical experiment can have n possible outcomes, each of which is equally likely. Suppose a subset of r outcomes are classified as "successful" outcomes.

The probability that the experiment results in a successful outcome (S) is:

P(S) = ( Number of successful outcomes ) / ( Total number of equally likely outcomes ) = r / n

Consider the following experiment. An urn has 10 marbles. Two marbles are red, three are green, and five are blue. If an experimenter randomly selects 1 marble from the urn, what is the probability that it will be green?

In this experiment, there are 10 equally likely outcomes, three of which are green marbles. Therefore, the probability of choosing a green marble is 3/10 or 0.30.
\begin{itemize}
	\item Conditional probability
	\item Independent events
	\item Repeated independent events
\end{itemize}

%---------------------------------------------------------------------%
\newpage

\subsection{ 4 : Probability Distributions}

\subsubsection{part 1 : Introduction to random variables}
Expected Values

\subsubsection{part 2 : Discrete Probability distributions}




\begin{enumerate}
	\item Binomial dstribution
	\item Poisson distribution
	\item Geometric distribution
\end{enumerate}

part 3 :  Continuous Probability distributions



\begin{itemize}
	\item Normal distribution
	\item Uniform distribution
	\item Exponential distribution
\end{itemize}

Additionally, every normal curve (regardless of its mean or standard deviation) conforms to the following "rule".

\begin{itemize}
	\item About $68\%$ of the area under the curve falls within 1 standard deviation of the mean.
	\item About $95\%$ of the area under the curve falls within 2 standard deviations of the mean.
	\item About $99.7\%$ of the area under the curve falls within 3 standard deviations of the mean.
\end{itemize}

Collectively, these points are known as the empirical rule or the 68-95-99.7 rule. Clearly, given a normal distribution, most outcomes will be within 3 standard deviations of the mean.


%%			\frametitle{Introduction to Statistics}
			\Large  
			\begin{itemize}
				\item Mean
				\item Median
				\item Mode  ( useful for categorical data)
			\end{itemize}

%%			\frametitle{Introduction to Statistics}
			\Large 
			
			\textbf{Sample Mean}
			\[x=xin    \mbox{for} i = \{1.....n\} \] 
			

%%			\frametitle{Introduction to Statistics}
			\Large 
			\textbf{Population mean}
			
			
			When population has a finite quantity N, the population mean can be calculated as follows,
			
			=xiNfori = 1.....N.
			

%%			\frametitle{Introduction to Statistics}
			\Large
			
			The Median
			
			The Median is denoted as x.
			
			It can also be consider as the second quartile Q3
			
			The Sample variance and sample standard deviation
			


			
			The Interquartile Range (IQR)
			
			IQR =Q3-Q1


	%%	\frametitle{Measures of Centrality}

		\begin{center}
			
			
			\begin{verbatim}
			4.10, 4.10, 4.25, 4.25, 4.25,  
			4.35, 4.40, 4.53, 4.90, 5.20, 
			5.26, 5.35, 5.45, 5.71, 6.09, 
			6.10, 6.30, 6.50, 6.80, 7.11.
			\end{verbatim}
			
			
		\end{center}
		\begin{itemize}
			\item There are 20 values in this data set.
			\item The sum of the values is 105.
		\end{itemize}

	

\end{document}