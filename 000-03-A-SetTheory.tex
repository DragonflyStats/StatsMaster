\documentclass[]{report}

\voffset=-1.5cm
\oddsidemargin=0.0cm
\textwidth = 480pt

\usepackage{framed}
\usepackage{subfiles}
\usepackage{graphics}
\usepackage{newlfont}
\usepackage{eurosym}
\usepackage{amsmath,amsthm,amsfonts}
\usepackage{amsmath}
\usepackage{color}
\usepackage{amssymb}
\usepackage{multicol}
\usepackage[dvipsnames]{xcolor}
\usepackage{graphicx}
\begin{document}

\section{Set Theory}
\section{Set Theory Revision}
		In this set theory formulation of probability the sample space for a problem corresponds to an important set. Since the sample space contains every outcome that is possible, it forms a setting of everything that we can consider. So the sample space becomes the universal set in use for a particular probability experiment.
		

			\subsection{Set Theory : Union and Intersection}
			

		
		Set theory is used to represent relationships among events.\\ \bigskip
		
		\noindent \textbf{Union of two events:}
		\begin{itemize}
			\item The union of events A and B is the event containing all the sample points
			belonging to A or B or both. 
			\item This is denoted $A\cup B$, (pronounce as ``A union
			B").\end{itemize} \bigskip
		\noindent \textbf{Intersection of two events:}\\
		\begin{itemize}
			\item 
			The intersection of events A and B is the event containing all the sample
			points common to both A and B.
			\item This is denoted $A\cap B$, (pronounce as ``A intersection
			B").
		\end{itemize}
	
	


\end{document}
