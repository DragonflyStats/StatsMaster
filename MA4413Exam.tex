\documentclass[]{article}

%opening
\title{Overview of MA4413 2013 Paper}
\author{UL Mathematics \& Statistics}

\begin{document}

\maketitle

\begin{itemize}
\item[(1a)] (6 Marks) Basic Probability
\item[(1b)] (6 Marks) Descriptive Statistics
\item[(1c)] (8 Marks) Discrete RVs
%------------------------ %
\item[(2a)] (12 Marks)Poisson and Exponential
\item[(2b)] (8 Marks) Normal Distribution
%------------------------ %
\item[(3a)] (12 Marks) Confidence interval and Theory Questions
\item[(3b)] (8 Marks) Hypothesis Testing ( with \texttt{R})
%------------------------ %
\item[(4a)] (12 Marks)Hypothesis Testing (2 Sample)
\item[(4b)] (8 Marks) Accuracy, Precision, Recall and F measure
%------------------------ %
\item[(5a)] (6 Marks) Huffman Coding
\item[(5b)] (8 Marks) Entropy Information / Calculations
\item[(5c)] (6 Marks) Binary Channels
\end{itemize}
\newpage
\section*{Question 2B Discrete Random Variable [6 Marks]}



\newpage
\section{Probability Questions}
%--------------------------------------------------------------%
\subsection*{Part 2A}

(a)										       

\newpage
\section*{Question 5A - 20 marks}
%Question 5. (20 marks)
(a) Consider the binary channel in the figure below.

\begin{itemize}
\item[(i)] (1 Mark) Determine the channel matrix of the channel
\item[(i)] (1 Mark)  Find P(Y1) and P(Y2) when P(X1) = 0:7 and P(X2) = 0:3
\item[(i)] (1 Mark)  Find the joint probabilities P(X1; Y1) and P(X2; Y2).
\end{itemize}
%----------------------------------------------------- %
(b) A discrete memoryless source X has five symbols $\{x_1; x_2; x_3; x_4; x_5\}$ with prob-
abilities P(x1) = 0:40 , P(x2) = 0:25, P(x3) = 0:15, P(x4) = 0:12 and
P(x5) = 0:08.
\begin{verbatim}
i. (4 marks) Construct a Huffman code for X.
ii. (4 marks) Calculate the efficiency of the code.
iii. (2 marks) Calculate the redundancy of the code.
\end{verbatim}
\end{document}
