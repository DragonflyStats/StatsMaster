\documentclass[]{report}

\voffset=-1.5cm
\oddsidemargin=0.0cm
\textwidth = 480pt

\usepackage{framed}
\usepackage{subfiles}
\usepackage{graphics}
\usepackage{newlfont}
\usepackage{eurosym}
\usepackage{amsmath,amsthm,amsfonts}
\usepackage{amsmath}
\usepackage{color}
\usepackage{amssymb}
\usepackage{multicol}
\usepackage[dvipsnames]{xcolor}
\usepackage{graphicx}
\begin{document}
\chapter{Conditional Probability}


\section{Conditional Probability}
Suppose $B$ is an event in a sample space $S$ with $P(B) > 0$.
The probability that an event $A$ occurs once $B$has occurred or, specifically, the
conditional probability of A given $B$ (written $P(A|B)$), is defined as follows:
\[ P(A|B) = \frac{P(A\cap B)}{P(B)}\]

\begin{itemize}
\item This can be expressed as a multiplication theorem
\[ P(A\cap B) = P(A|B)\times P(B) \]
\item The symbol $|$ is a vertical line and does not imply division.
\item Also $P(A|B)$ is not the same as $P(B|A)$.
\end{itemize}
Remark: The Prosecutor's Fallacy , with reference to the O.J. Simpson trial.


What is the probability of one event given that another event occurs? For example, what is the probability of a mouse finding the end of the maze, given that it finds the room before the end of the maze?

This is represented as:


\[P[A | B]\]
or "the probability of A given B."

\[P(A | B) = \frac{P(A\cap B)}{P(B)}\]

If A and B are independent of one another, such as with coin tosses or child births, then:


\[P[A | B] = P[ A ]\]
Thus, "what is the probability that the next child a family bears will be a boy, given that the last child is a boy."

This can also be stacked where the probability of A with several "givens."


\[P[A | B_1, B_2, B_3 ]\]
or "the probability of A given that B1, B2, and B3 are true?"



\begin{itemize}
\item pairwise disjoint sets
\item The addition principle
\end{itemize}


\section{Conditional Probability}

\[ P(X|Y)  = \frac{P(X \mbox{ and } Y)}{P(Y)} \]

Re-arranging
\[ P(X \mbox{ and } Y) =  P(X|Y)\times P(Y) \]
Therefore we can say
\[ P(F \mbox{ and } A) =  P(F|A)\times P(A) \]\\
\[ P(F \mbox{ and } B) =  P(F|B)\times P(B) \]





\[ P(F \mbox{ and } A) =  P(F|A)\times P(A) \]

\[ P(F \mbox{ and } B) =  P(F|B)\times P(B) \]




\[ P(F \mbox{ and } A) =  P(F|A)\times P(A) = 0.80 \times 0.01\]\\
\[ P(F \mbox{ and } A) = 0.008\]

\[ P(F \mbox{ and } B) =  P(F|B)\times P(B) = 0.20 \times 0.03\]\\
\[ P(F \mbox{ and } B) = 0.006\]

%=================================================================================================$
\textbf{Recall:}
\[ P(F) = P(F \mbox{ and } A) + P(F \mbox{ and } B) \]








\[P[A | B]\]
or "the probability of A given B."

\[P(A | B) = \frac{P(A\cap B)}{P(B)}\]

If A and B are independent of one another, such as with coin tosses or child births, then:


\[P[A | B] = P[ A ]\]
Thus, "what is the probability that the next child a family bears will be a boy, given that the last child is a boy."

This can also be stacked where the probability of A with several "givens."


\[P[A | B_1, B_2, B_3 ]\]
or "the probability of A given that B1, B2, and B3 are true?"



%==================================================================%
\section{Algebaic Results}


\subsection{Multiplication Rule}
The multiplication rule is a result used to determine the probability that two events, $A$ and $B$, both occur.
The multiplication rule follows from the definition of conditional probability.\\ \bigskip

The result is often written as follows, using set notation:
\[ P(A|B)\times P(B) = P(B|A)\times P(A) \qquad \left( = P(A \cap B) \right) \]

Recall that for independent events, that is events which have no influence on one another, the rule simplifies to:
\[P(A \cap B)  = P(A)\times P(B) \]


From the first year intake example, check that
\[ P(E|F)\times P(F) = P(F|E)\times P(E)\]
\begin{itemize}
\item $P(E|F)\times P(F) = 0.58 \times 0.38  = 0.22$
\item $P(F|E)\times P(E) = 0.55 \times 0.40  = 0.22$
\end{itemize}

\subsection{Law of Total Probability}
The law of total probability is a fundamental rule relating marginal probabilities to conditional probabilities. The result is often written as follows, using set notation:
\[ P(A)  = P(A \cap B) + P(A \cap B^c) \]

where $P(A \cap B^c)$ is probability that event $A$ occurs and $B$ does not.\\ \bigskip


Using the multiplication rule, this can be expressed as
\[ P(A) = P(A | B)\times P(B) + P(A | B^{c})\times P(B^{c}) \]


From the first year intake example , check that
\[ P(E)  = P(E \cap M) + P(E \cap F) \]
with $ P(E) = 0.40$, $ P(E \cap M) = 0.18$ and  $ P(E \cap F) = 0.22$
\[ 0.40  = 0.18 + 0.22 \]
\textbf{Remark:}  $M$ and $F$ are complement events.








\section{Conditional Probabilities}
Bayes Theorem


\[ P(A|B)  = \frac{P(A \mbox{and} B)}{P(B)} \]









\end{document}
