

\section*{Expected Values}

\begin{itemize}
\item If some variable X has its values specfied with associated probabilities, then the expected value $E(x)$is
\item Expected value ( or expectation ) is the arithmetic mean of the a given sum of values.
\item It is calculated using probabilities insteead of frequencues.
\end{itemize}

\[
E(X) = \sum x p(x)
\]

When a die is thrown, each of the possible faces 1, 2, 3, 4, 5, 6 (the xi's) has a 
probability of 1/6 (the p(xi)'s) of showing.\\ The expected value of the face showing is therefore: 

\[\mu = E(X) = (1 x 1/6) + (2 x 1/6) + (3 x 1/6) + (4 x 1/6) + (5 x 1/6) + (6 x 1/6) = 3.5 \]

Notice that, in this case, E(X) is 3.5, which is not a possible value of X. 

%=============================================================%


\section*{Discrete Random Variable}
\begin{itemize}

\item Examples of discrete random variables include the number of children in a family, the Friday night attendance at a cinema, the number of patients in a doctor's surgery, the number of defective light bulbs in a box of ten.

\item For a discrete random variable observed values can occur only at isolated points along a scale of values. In other words, observed values must be integers.
\item Consider a six sided die: the only possible observed values are 1, 2, 3, 4, 5 and 6. 
\item It is not possible to observe values that are real numbers, such as 2.091.

\item \textit{(Remark: it is possible for the average of a discrete random variable to be a real number.)}

\item For a discrete random variable observed values can occur only at isolated points along a scale of values. \item For a six sided dice, the only possible observed values are 1, 2, 3, 4, 5 and 6. It is not possible to observe values such as 5.732.

\item For a continuous random variable all possible fractional values of the variable cannot be listed, and
therefore the probabilities that are determined by a mathematical function are portrayed graphically by a
probability density function, or probability curve.
\end{itemize}





\section{Discrete Random Variable}
\begin{itemize}

\item Examples of discrete random variables include the number of children in a family, the Friday night attendance at a cinema, the number of patients in a doctor's surgery, the number of defective light bulbs in a box of ten.
\end{itemize}

\textbf{Discrete Random Variables}
\begin{itemize}
\item For a discrete random variable observed values can occur only at isolated points along a scale of values. In other words, observed values must be integers.
\item Consider a six sided die: the only possible observed values are 1, 2, 3, 4, 5 and 6. 
\item It is not possible to observe values that are real numbers, such as 2.091.

\item \textit{(Remark: it is possible for the average of a discrete random variable to be a real number.)}

\end{itemize}


\section{Expected Value for Discrete Random Variables}
The expected value of a random variable X is symbolised by E(X) or $mu$.


If X is a discrete random variable with possible values $\{ x1, x2, x3,\ldots , xn\}$, and$ p(x_i)$ denotes P(X = xi), then the expected value of X is defined by: 

\[E(X) = \sum x_i \times P(x_i) \]

where the elements are summed over all values of the random variable X. 


\subsection{Discrete Random Variables}

A random variable is a numerical description of the outcome of an experiment.

Random variables can be classified as discrete or continuous, depending on the numerical values they may take.

A ranom variable that may assume any numerical value in an interval or collection of intervals is called a continuous random variable.

\subsection{Question 3}

Suppose a fair coin is tossed six times. The number of heads which can occur with their respective
probabilities are as follows:

xi0123456
p(xi)1/646/6415/6420/6415/646/641/64

a)Compute the expected value (i.e. expected number of heads).
b)Compute the variance of the number of heads.




\end{document}
