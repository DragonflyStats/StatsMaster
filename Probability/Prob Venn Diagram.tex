\documentclass[IntroMain.tex]{subfiles} 
\begin{document}
	%------------------------------------------------------------%

%http://www.mathsireland.com/LCHGeneralNotes/PermCombProb/5_5_Prob_MultAnd/Q_5_5_Prob_MultAnd.html
\begin{frame}
\Huge
\[\mbox{Introduction to Probability}\]
\LARGE
\[\mbox{Calculations using the Choose Operator}\]

\Large
\[\mbox{kobriendublin.wordpress.com}\]
\[\mbox{Twitter: @StatsLabDublin}\]

\end{frame}


\begin{frame}
\frametitle{Counting Sets with Venn Diagrams}
\Large
\vspace{-2cm}
%----------------------------------------%
\begin{itemize}
\item The Venn Diagram shows the number of elements in each subset of set $S$.
\item If $P(A) = 3/10$ and $P(B) = 1/2$, find the values of $x$ and $y$
\end{itemize}
%----------------------------------------%
%Venn Diagrams
\end{frame}



\begin{frame}
\frametitle{Counting Sets with Venn Diagrams}
\Large
\vspace{-1cm}
%----------------------------------------%
\begin{itemize}
\item The total number of items in the data set is $x+y+5$
\item There are $x+1$ items in Area $A$
\item There are $x+y$ items in Area $B$
\item We can say
\[ P(A) = \frac{3}{10} = \frac{x+1}{x+y+5}\]
\[ P(B) = \frac{1}{2} = \frac{x+y}{x+y+5} \]
\end{itemize}
%----------------------------------------%
\end{frame}



\begin{frame}
\frametitle{Counting Sets with Venn Diagrams}
\Large
\vspace{-2.5cm}
\textbf{Cross Multiplication}
\[ P(A) = \frac{3}{10} = \frac{x+1}{x+y+5}\]

\end{frame}


\begin{frame}
\frametitle{Counting Sets with Venn Diagrams}
\Large
\vspace{-2.5cm}
\textbf{Cross Multiplication}
\[ P(B) = \frac{1}{2} = \frac{x+y}{x+y+5} \]

\end{frame}


\begin{frame}
\frametitle{Counting Sets with Venn Diagrams}
\Large
\vspace{-2.8cm}
\textbf{Simultaneous Equations}
\begin{itemize}
\item[1)] $7x-3y=5$
\item[2)] $x+y=5$
\end{itemize}
\end{frame}

\begin{frame}
\frametitle{Counting Sets with Venn Diagrams}
\Large
\vspace{-2.8cm}
\textbf{Simultaneous Equations}
\begin{itemize}
\item $7x-3y=5$
\item $x+y=5$
\end{itemize}
\end{frame}
\end{document}
