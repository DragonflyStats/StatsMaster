\documentclass{beamer}

\usepackage{graphics}

\begin{document}

\begin{frame}
\begin{centering}
{\Large
\\

\vspace{1.3cm}
Probability Trees
\\
\vspace{1.3cm}
{\large kobriendublin.wordpress.com

}
}
\end{centering}

\end{frame}

%--------------------------------------------------------- %
\begin{frame}
\frametitle{Probability Trees}
{
\large
\vspace{-0.3cm}
\begin{itemize}
\item Two gamblers, A and B, are playing each other in a tournament to win a jackpot $\$6,000$. 
\item The first gambler to win 5 games, wins the tournament, and the jackpot outright.
\item Each player has an equal chance of winning each game. Also, a tie is not possible.
\item The tournament is suspended after the seventh game. At this point A has won 3 games, while B has won 4.
\item They agree to finish then and divide up the jackpot, according to how likely an outright victory would have been for both.
\end{itemize}
How much money did A end up with?
}
\end{frame}

%---------------------------------------------- %
\begin{frame}
\frametitle{Probability Trees}
{  \large 
\vspace{-0.4cm}
\begin{itemize}
\item Consider that A needed to win two more game, while B only need to win one more?
\item One could suppose that B was twice a likely as A to win the jackpot.
\item That would mean that the shares of the jackpot would be $\$2,000$ for A and 
$\$4,000$ for B.
\end{itemize}
}
\end{frame}
\begin{frame}
\frametitle{Probability Trees}
{  \large 
\vspace{-0.4cm}
\begin{itemize}
\item Consider that A needed to win two more game, while B only need to win one more?
\item One could suppose that B was twice a likely as A to win the jackpot.
\item That would mean that the shares of the jackpot would be $\$2,000$ for A and 
$\$4,000$ for B.
\item \alert{WRONG!}
\end{itemize}
}
\end{frame}
\begin{frame}
\frametitle{Probability Trees}
{  \large 
\vspace{-0.4cm}
\begin{itemize}
\item At the end of the seventh game, A had a $25\%$ chance of winning the jackpot.
\item A's share of the jackpot is  the $1,500$.
\item B had a $75\%$ chance of winning, so gets $4,500$.
\end{itemize}
}
\end{frame}




\begin{frame}
\[ \mbox{Introduction to Probability}\]


\LARGE
\[ \mbox{Introduction to Probability}\]
\[ \mbox{Twitter: @StatsLabDublin}\]
\end{frame}
%----------------------------------------------------------%
%Page 29
\begin{frame}
In how many ways can a group of four people be selected from three men and four women?
In how many of these groups are there more women than men?
\end{frame}
%----------------------------------------------------------%
\begin{frame}
\frametitle{Selecting A Committee}
\begin{itemize}
\item Firstly - we know that there are 7 people to choose from altogether.
\item From these seven people, we have to choose four people.
\end{itemize}
\[ {7 \choose 4} = \frac{7!}{4! \times (7-4)!} \]
\end{frame}


%----------------------------------------------------------%
\end{document}

Cx=	75	;	Cy=	40
Ax=	25	;	Ay=	100
Bx=	75	;	By=	160
Dx=	125	;	Dy=	210
Bx=	75	;	By=	160
Ex=	125	;	Ey=	110

Xs=c(Cx,Ax,Bx,Dx,Bx,Ex)
Ys=c(Cy,Ay,By,Dy,By,Ey)
plot(Xs,Ys,pch=18,col="red",ylim=c(0,240),xlim=c(0,200))

lines(Xs,Ys,col="red",lwd=2)

text(38,70,"B wins",col="purple",font=2)
text(38,134,"A wins",col="purple",font=2)

text(88,190,"A wins",col="purple",font=2)
text(88,134,"B wins",col="purple",font=2)

#############################################

text(14,108,"A at 3",col="tomato",font=2)
text(14,92,"B at 4",col="tomato",font=2)


text(Cx+17,Cy,"P = 0.50",col="violetred2",font=2)
text(Dx+17,Dy,"P = 0.25",col="violetred2",font=2)
text(Ex+17,Ey,"P = 0.25",col="violetred2",font=2)
