
\documentclass[]{article}
\voffset=-1.5cm
\oddsidemargin=0.0cm
\textwidth = 490pt



\usepackage{amsmath}
\usepackage{graphicx}
\usepackage{multicol}
\usepackage{amssymb}
\usepackage{framed}

%\usepackage[paperwidth=21cm, paperheight=29.8cm]{geometry}
%\usepackage[angle=0,scale=1,color=black,hshift=-0.4cm,vshift=15cm]{background}
%\usepackage{multirow}
\usepackage{enumerate}

%\SetBgScale{1}
%\SetBgAngle{0}
%\SetBgColor{black}
%\SetBgContents{\rule{1pt}{30cm}}
%\SetBgHshift{-8.4cm}
%
%\backgroundsetup{contents={
%\begin{tabular}{c|c}
%\hspace{2cm} & \\[0.7cm]
%& {\bf Statistics for Computing ------ Lecture 1 ------ Solutions} \\[0.3cm]
%%\hline
%\hspace{2cm} & \hspace{18.5cm} \\ [28cm]
%\end{tabular}}}





\begin{document}
	
	%----------------------------------------------%
	
	% PMS SPring 2006 Question 5
	
	\subsection*{Probability : Medical Diagnosis Worked Example}
	
	\begin{itemize}
		\item A new test has been developed to diagnose a particular disease. If a person has the disease, the test has a 95\% chance of identifying them as having the disease. 
		
		\item If a person does not have the disease, the test has a 1\% chance of identifying them as having the disease. 
		\item Suppose that 5\% of the population have this disease. Suppose we select a person at random from the population.
	\end{itemize}
	
	
	%--------------------------------------------------------------------------------------%
	
	\noindent \textbf{Questions}
	\begin{description}
		\item[Q1] - What is the probability that the test will identify them as having the disease?
		
		\item[Q2] - What is the probability that the person has the disease given that the test identifies them as having the disease?
	\end{description}
	
	
	\noindent \textbf{Solution: State Each of the Events}
	
	\begin{itemize}
		\item Let \textbf{P} signify that a test will give a “positive” result.
		\item Let \textbf{N} signify that a test will give a “negative” result.	
		\item Let \textbf{D} signify that the person in question has the disease.
		\item Let \textbf{H} signify that the person doesn’t have the disease ( or in other words , is healthy) .
	\end{itemize}
	\smallskip
	
	\noindent We are asked to determine the following 
	
	\begin{itemize}
		\item[Q1] The probability of a positive test - $\Pr(P)$
		\item[Q2] The probability that they have the disease given that they have tested positive – $\Pr(D|P)$
	\end{itemize}
	
	%================================%
	\noindent \textbf{Solution: What Information are we given?}\\
	
	\noindent We are told that 5\% of the population have this disease 
	We know that D and H are complementary events, so we can work out the probabilities of both. 
	\[\Pr(D) = 1-\Pr(H)\]
	
	
	\[  \Pr(D) = 0.05 \;\;\; \;\;\;\;  \therefore\;\;\ \Pr(H) = 0.95\]
	(P and N are complements also)\\ \smallskip
	
	
	\noindent People who test positive are made up of two groups
	\begin{itemize}
		\item People who test positive and who do have the disease  ($P \mbox{ and } D$)
		\item People who test positive and who don’t have the disease  ($P \mbox{ and } H$)
		\[\Pr(P) \; =  \;\Pr(P \mbox{ and } D) \; +  \;\Pr(P \mbox{ and } H)\]
		
	\end{itemize}
	\smallskip
	
	\begin{itemize}
		\item A new test has been developed to diagnose a particular disease. If a person has the disease, the test has a 95\% chance of identifying them as having the disease. 
		\[ \Pr(P|D) = 0.95\]
		
		\item If a person does not have the disease, the test has a 1\% chance of identifying them as having the disease. 
		\[ \Pr(P|H) = 0.01\]
	\end{itemize}	
	
	\begin{framed}
		\noindent The conditional probability is useful here		 
		{
			\large
			\[ \Pr(A | B) = \frac{\Pr(A \mbox{ and } B)}{\Pr(B)} \]
		}
		\noindent We can rearrange it as follows 		 
		{
			\large
			\[ \Pr(A \mbox{ and } B) =  \Pr(A|B) \times \Pr(B) \]
		}
		%For the second part, we simply use Bayes Rule , 
		%using information we have determined previously
		%
		%Recall:
		%
		%\[ P(A | B) = \frac{ p(B|A)\times p(A)}{p(B)} \] 
	\end{framed}
	\medskip
	\noindent We can now write our equation in terms of all the information we have :
	
	\begin{itemize}
		\item $\Pr(P \mbox{ and } D) = \Pr(P|D) \times \Pr(D)$
		
		\[\Pr(P \mbox{ and } D) = 0.95 \times 0.05 = 0.0475\]
		
		\item $\Pr(P \mbox{ and } H) = \Pr(P|H) \times \Pr(H)$
		\[\Pr(P \mbox{ and } H) = 0.01 \times 0.95 = 0.0095\]
	\end{itemize}
	{
		\large
		\[\Pr(P)  =  \Pr(P \mbox{ and } D)  +  \Pr(P \mbox{ and } H)\]
		\[\Pr(P)  =  0.0475  +  0.0095 =  \boldsymbol{0.057}\]
	}
	%===============================================%
	\bigskip
	
	\noindent \textbf{Solution: Answer to Question 2}\\
	\noindent The answer to the first question is $\Pr(P) =  0.057$. We still have to compute $\Pr(D|P)$. Now that we have all the information we need, we simply use the Conditional Probability Formula again.
	{
		\large
		\[\Pr(D|P) = \frac{\Pr(P \mbox{ and } D)}{\Pr(P)} = \frac{0.0475}{0.057} = \textbf{0.8333}\]
		
	}
	
	
\end{document}