


Question 1
Events  (Important the event names are explained)

\begin{description}
\item[F:] Student plays footballP(F) = 50%
\item[S:] Student does SwimmingP(S) = 20%
\item[F and S:]            Student takes part in both swimming and football  
P(S and F) = 15%
\end{description}
Find P (F or S)

Use addition rule

\[P (F or S)  =  P(F) + P(S) – P( Fand S)\]
\[=  50\% + 20 \% - 15\% =    55\%\]


(We subtract 15\% to stop the “boths” getting counted twice)

Probability of playing neither
This is the complement event of playing one or both sports.
\[P(Neither) = 1 –  P( F or S)  = 45\%\]

\begin{enumerate}
\item 
2 components A and B.
\begin{description}
\item[P(A)] = event that A is workingP(A) = 0.98
\item[P(B)] = event that B is workingP(B) = 0.95
\item[P(A and B)] = event that both A and B are working = $P(A) \times P(B) = 0.98 \times 0.95 = 0.931$
\end{description}

\item 
Lots of useless information.
Complement event of at least one working is that they are both broken.

Answer  100 – 4\% = 96%

\item 


<pre><code>
Events 
A components from supplier A  P (A) = 0.8 
B components from supplier B  P (B) = 0.2 
F = resistor fails test

1\% Probability of a Failure given that component is from Supplier A.      P(F|A) = 0.01
3\% Probability of a Failure given that component is from Supplier B.      P(F|B) = 0.03

</code></pre>
Probability of flaw : P(F)
Failed resistors either come from A or B
\[P( F) =  P ( F \mbox{ and } A)  +  P( F \mbox{ and } B)\]
Use conditional Probability  rule
\[P(F) = P(F|A)\times P(A)  + P(F|B)\times P(B)\]

\[P(F)  =  ( 0.01 x 0.8 ) + ( 0.03 x 0.2) = (0.008) + (0.06) = 0.014\]
Answer 1.4%


Part ii
Given that a component failed, what was the probability of coming from A
P(A|F) 

P(A|F) = P(A and F)  / P(F)We found P(A and F) earlier ; 0.008

P(A|F) = 0.008/0.014 =  0.57[answer : 57%]




\end{enumerate}


% Question 5 in lectures
% Answer P(D) =0.0395 (3.95%)
% P(C|D) = 0.0506(5.06%)


%
%
% x=seq(2,18,length=1600)
% y=dnorm(x,10,2)
% plot(x,y,type="l",  col="black")
% x=seq(10,12.4,length=240)
% y=dnorm(x,10,2)
% polygon(c(10,x,12.4),c(0,y,0),col="wheat")
% abline(h=0)
% text(14,0.15,"P(10 < X < 12.4)")
%
%
%
%
% x=seq(2,18,length=1600)
% y=dnorm(x,10,2)
% plot(x,y,type="l",  col="black")
% L=2
% U=9.7
% x=seq(L,U,length=((U-L)*100) )
% y=dnorm(x,10,2)
% polygon(c(L,x,U),c(0,y,0),col="wheat")
% abline(h=0)
% text(14,0.15,"P(10 < X < 12.4)")
%
%
%abline(v=10)








\section*{Probability}
\subsection*{Question 1 Part A}

Consider the experiment where three coins are flipped. [-0.2cm]
\begin{enumerate}[(i)]
\item List all possible outcomes. \item  Calculate $\Pr(\text{more heads than tails})$? \item Calculate $\Pr(\text{two tails})$?
\end{enumerate}
Remark: draw out a probability tree


























%=====================================%
\subsection*{Question 3 Part A - Simple Addition Rule for Probability}

\begin{itemize}
\item P(M) = 0.60
\item P(CS) = 0.40
\item P(M and CS) = 0.20

\[  P(M \cup CS) = P(M) + P(CS) - P(M \cap CS) 

= 0.60 + 0.40 -0.20 
= 0.80
\] 














\textbf{Combining Probabilities - Worked Example}

Bayes Rule is given in the Formulae  

We can rearrange it as follows  


%--------------------------------------------------------------------------------------%
%--------------------------------------------------------------------------------------%

\section{Combining Probabilities - Worked Example}
We can now write our equation in terms of all the information we have :

ANS

For the second part, we simply use Bayes Rule again, using information we have determined previously


The Addition Rule for Probability

$P(A \cup B ) = P(A) + P(B) - P(A \cap B)$

\[{8 \choose 2} =\frac{8!}{2!(8-2)!} \quad = \frac{8\times7\times6!}{2!\times 6!} = \frac{8\times7}{2\times 1} \quad = \frac{56}{2} \quad = 28\]

%---------------------------------------------------------------%

\section{Probability Rules}

There are two rules which are very important.
\begin{itemize}
\item All probabilities are between 0 and 1 inclusive
\[0 \leq P(E) \leq 1\]
\item The sum of all the probabilities in the sample space is 1


\section{Addition Rule}
The addition rule is a result used to determine the probability that event $A$ or
event $B$ occurs or both occur. The result is often written as follows, using set
notation:
\[ P(A\cup B) = P(A) + P(B)- P(A \cap B)\] 
\begin{itemize}
\item $P(A)$ = probability that event $A$ occurs.
\item $P(B)$ = probability that event $B$ occurs.
\item $P(A\cup B)$ = probability that either event $A$ or event $B$ occurs, or both
occur.
\item $P(A\cap B)$ = probability that event $A$ and event $B$ both occur.
\bigskip

\noindent \textbf{Remark:} $P(A\cap B)$ is subtracted to prevent the relevant outcomes being
counted twice.





