\documentclass[compress]{beamer}        % [compress] (written before {beamer} <=> navigation bar one line, all subsections in 1 line instead of 2

% Setup appearance:
\usetheme{CambridgeUS}
%	AnnArbor | Antibes | Bergen |
%	Berkeley | Berlin | Boadilla |
%	boxes | CambridgeUS | Copenhagen |
%	Darmstadt | default | Dresden |
%	Frankfurt | Goettingen |Hannover |
%	Ilmenau | JuanLesPins | Luebeck |
%	Madrid | Malmoe | Marburg |
%	Montpellier | PaloAlto | Pittsburgh |
%	Rochester | Singapore | Szeged |
%	Warsaw
%

\useoutertheme[footline=authorinstitute,subsection=false]{miniframes}
\usecolortheme{whale}

%	albatross | beaver | beetle |
%	crane | default | dolphin |
%	dove | fly | lily | orchid |
%	rose |seagull | seahorse |
%	sidebartab | structure |
%	whale | wolverine


\setbeamertemplate{footline}
{
  \hbox{%
  \begin{beamercolorbox}[wd=.25\paperwidth,ht=2.25ex,dp=1ex,center]{title in head/foot}%
    \usebeamerfont{date in head/foot}\insertshortauthor
  \end{beamercolorbox}%
  \begin{beamercolorbox}[wd=.5\paperwidth,ht=2.25ex,dp=1ex,center]{date in head/foot}%
    \usebeamerfont{title in head/foot}\insertshortinstitute
  \end{beamercolorbox}%
  \begin{beamercolorbox}[wd=.25\paperwidth,ht=2.25ex,dp=1ex,center]{title in head/foot}%
    \usebeamerfont{date in head/foot}
    \insertframenumber{} / \inserttotalframenumber
    %\insertframenumber{} / \insertpresentationendpage
  \end{beamercolorbox}}%
  \vskip0pt%
}

%\setbeamercolor{titlelike}{parent=structure}
%\setbeamercolor{structure}{fg=beamer@blendedblue}
%% \useinnertheme{rounded}
%\setbeamerfont{block title}{size={}}
%\usefonttheme[onlylarge]{structurebold}   % title and words in the table of contents bold
%\setbeamerfont*{frametitle}{size=\normalsize,series=\bfseries}
\setbeamertemplate{navigation symbols}{}
\setbeamercolor{frametitle}{parent=boxes, bg=white}
{ % only on titlepage


\usepackage{times}
\usepackage{amsmath,amssymb,amsthm}
\usepackage{color}
\usepackage{changepage}
\usepackage{multirow}
\usepackage[absolute,overlay]{textpos}
\usepackage{enumerate}
%\usepackage{pgfpages}
\usepackage[all]{xy}
\usepackage{textcomp}
\usepackage{etex}
\usepackage{tikz}
\usetikzlibrary{shapes}
%\usepackage{handoutWithNotes}
%\pgfpagesuselayout{4 on 1}[border shrink=1mm]




\definecolor{camblue}{RGB}{26,26,89}
\definecolor{Rblue}{RGB}{0,255,255}
\definecolor{Rdarkblue}{RGB}{0,0,255}
\definecolor{Rgreen}{RGB}{0,205,0}
\definecolor{green2}{RGB}{51,204,51}
\newcommand{\tcb}{\textcolor{beamer@blendedblue}}
\newcommand{\tcbb}{\textcolor{camblue}}
\newcommand{\tcr}{\textcolor{red}}
\newcommand{\tcg}{\textcolor{gray}}
\newcommand{\tcgr}{\textcolor{green2}}
\newcommand{\tcblk}{\textcolor{black}}
\newcommand{\tcRg}{\textcolor{Rgreen}}
\newcommand{\tcRdb}{\textcolor{Rdarkblue}}
\newcommand{\tcRb}{\textcolor{Rblue}}
\newcommand{\tcw}{\textcolor{white}}
\newcommand{\m}{\phantom{-}}
\newcommand{\bp}{\tcbb{$\bullet$}\:}


\title{{\huge Statistics for Computing\\[0.1cm]MA4413}}
\author[Kevin Burke]{{\bf\\[0.5cm]{\huge Lecture 8}\\[0.2cm]\emph{The Poisson Distribution}\\[1.4cm]Kevin Burke}\\[0.3cm]\tcb{kevin.burke@ul.ie}}

\institute[University of Limerick, Maths \& Stats Dept]{}
\date{}

%\TPGrid[5mm,5mm]{1}{1}

\begin{document}


\begin{frame}[t]
\titlepage
\end{frame}




\section{Poisson Approximation}
\subsection{Poisson Approximation}
\begin{frame}{\bf \tcb{Poisson Approximation}}
We saw in the previous lecture that the binomial probability function is $\Pr(X=x) = \binom{n}{x}\,p^x\,(1-p)^{n-x}$.\\[0.4cm]

However, {\bf when {\boldmath$n$} is large and {\boldmath$p$} is small}, this formula can present computational difficulties.\\[0.4cm]

In this case we can use the {\bf Poisson approximation}. Letting $\lambda = n\,p$:
\begin{align*}
\boxed{\Pr(X=x) = \binom{n}{x}\,p^x\,(1-p)^{n-x} \approx \frac{\lambda^x}{x\,!}\, e^{-\lambda}}.
\end{align*}
This approximation works well when $n > 20$ and $p < 0.1$.\\[0.5cm]

Note: $\lambda$ is the Greek letter ``lambda''.

\end{frame}


\subsection{Example: Rare Disease}
\begin{frame}{\bf \tcb{Example: Rare Disease}}
Let's assume that a rare disease affects 0.1\% of all individuals. What is the probability that 10 individuals in a group of 5000 have this disease?\\[0.4cm]

Let $X =$ ``the number of individuals who have the disease''. Clearly this has a binomial distribution: $X \sim \text{Binomial}(n=5000, p=0.001)$
\begin{align*}
\Rightarrow \Pr(X = 10) = \binom{5000}{10} \, 0.001^{10} \, 0.999^{4990} = 0.018.\\
\end{align*}


Since $n > 20$ and $p < 0.1$, we can use the \emph{Poisson approximation} with $\lambda = n\,p = 5000(0.001) = 5$
\begin{align*}
\Rightarrow \Pr(X = 10) \approx \frac{5^{10}}{10\,!}\, e^{-5} = 0.018.
\end{align*}

\end{frame}


\subsection{Example: Rolling Two Dice}
\begin{frame}{\bf \tcb{Example: Rolling Two Dice}}
Consider the experiment of rolling two dice and adding the numbers showing. If repeated 30 times, what is the probability that on 2 occasions the sum is equal to 3?\\[0.4cm]

You should be able to calculate the probability of getting a sum of 3 in one trial: $p = \frac{2}{36} = \frac{1}{18}$.\\[0.4cm]

Repeating 30 times leads to $X \sim \text{Binomial}(n=30, p=\frac{1}{18})$
\begin{align*}
\Rightarrow \Pr(X = 2) = \binom{30}{2} \, \left(\frac{1}{18}\right)^{2} \, \left(\frac{17}{18}\right)^{28} = 0.2709.
\end{align*}

Using the \emph{Poisson approximation} with $\lambda = n\,p = 30\times\frac{1}{18} = \frac{30}{18} = \frac{5}{3}$
\begin{align*}
\Rightarrow \Pr(X = 2) \approx \frac{(\frac{5}{3})^{2}}{2\,!}\, e^{-\frac{5}{3}} = 0.2623.
\end{align*}

\end{frame}



\subsection{Question 1}
\begin{frame}{\bf \tcb{Question 1}}
Using both the binomial probability function, $p(x) = \binom{n}{x} \, p^x\,(1-p)^{n-x}$, and the Poisson approximation, $p(x) \approx \frac{\lambda^x}{x\,!} e^{-\lambda}$ where $\lambda = n\,p$, evaluate each of the following:\\[0.3cm]
\begin{enumerate}[a)]\itemsep0.5cm
\item $\Pr(X=2)$ when $X \sim \text{Binomial}(n=20,p=0.1)$.
\item $\Pr(X=5)$ when $X \sim \text{Binomial}(n=100,p=0.02)$.
\item $\Pr(X=3)$ when $X \sim \text{Binomial}(n=1000,p=0.005)$.
\item $\Pr(X=1)$ when $X \sim \text{Binomial}(n=10000,p=0.0001)$.
\end{enumerate}

\end{frame}




\section{Poisson Distribution}
\subsection{Poisson Distribution}
\begin{frame}{\bf \tcb{Poisson Distribution}}

The Poisson approximation is useful. However, the {\bf Poisson distribution} is an important probability distribution in its own right.\\[0.3cm]

It is the probability distribution for \emph{the number of events occurring within an interval} of time / area / volume etc.\\[0.3cm]

For example, the number of:\\[0.0cm]
\begin{itemize}\itemsep0.2cm
\item System crashes per year.
\item Text messages received per hour.
\item Tasks processed by a CPU per hour.
\item Flaws in a sheet of metal per m$^2$.
\item Typographical errors per page.\\[0.3cm]
\end{itemize}
Note: the events \emph{must occur independently within the interval}, i.e., the occurrence of one event has no effect on another.


\end{frame}



\subsection{Why the Poisson Distribution Arises}
\begin{frame}{\bf \tcb{Why the Poisson Distribution Arises}}
Assume that a system crashes on average $\lambda$ times per year.\\[0.8cm]
Think about the \emph{precise moment in time} of one such crash.\\[0.0cm]
\begin{itemize}
\item This is the result of a Bernoulli trial with $\{1= \text{crash}, 0= \text{work}\}$ which has generated the outcome $1=$ crash.\\[0.8cm]
\end{itemize}

Now think of the \emph{whole year}.\\
\begin{itemize}
\item Throughout the year we observe the results of a \emph{sequence of identical Bernoulli trials} where $p =\Pr(\text{crash})$.\\[0.4cm]
\item Let $n$ be the total number of Bernoulli trials during this period, i.e., there is a trial for \emph{every single} precise moment in time.
\end{itemize}

\end{frame}


\subsection{Why the Poisson Distribution Arises}
\begin{frame}{\bf \tcb{Why the Poisson Distribution Arises}}
Assuming that these Bernoulli trials are \emph{independent}, we know that the number of crashes, $X$, has a binomial distribution (see Lecture7):\\
\begin{align*}
X \sim \text{Binomial}(n,p).\\
\end{align*}

What can we say about the value of $n$? Think - how many precise moments in time are there in the year (or any period of time)?\\[0.8cm]

What about $p$? Think - if we pick some moment in time, what is the likelihood that the system crashes at \emph{exactly} that moment in time?

\end{frame}



\subsection{Why the Poisson Distribution Arises}
\begin{frame}{\bf \tcb{Why the Poisson Distribution Arises}}

Since time is a \emph{continuous} quantity, there are an \emph{infinite} number of possible times that the system can crash, i.e., $n = \infty$.\\[0.6cm]
For any moment in time, the probability that the system crashes at \emph{exactly} that moment is very low, i.e., $p \approx 0$.\\[0.6cm]

Since {\bf{\boldmath$n$} is large and {\boldmath$p$} is small} we know from earlier that
\begin{align*}
\text{Binomial}(n,p) \rightarrow \text{Poisson}(\lambda).
\end{align*}
Here the \emph{average number of events per interval} is $\lambda = n\,p$.\\[0.6cm]

The same arguments hold for intervals of area / volume etc. Thus, the Poisson distribution arises naturally in a variety of situations.
\end{frame}

\subsection{Poisson Distribution}
\begin{frame}{\bf \tcb{Poisson Distribution}}
The {\bf Poisson distribution} is used for calculating the probability of $x$ events within an interval of time / area / volume etc.:
\begin{align*}
\boxed{X \sim \text{Poisson}(\lambda)}
\end{align*}
\begin{align*}
\boxed{\Pr(X = x) = \frac{\lambda^x}{x\,!}\,e^{-\lambda}}\\[-1cm]
\end{align*}
\begin{align*}
\text{where } \boxed{x \in \{0,1,2,\ldots,\infty\}}
\end{align*}
\begin{align*}
\boxed{E(X) = \lambda}
\end{align*}
\begin{align*}
\boxed{Var(X) = \lambda}\\[-0.5cm]
\end{align*}

\end{frame}


\subsection{Example: System Crashes}
\begin{frame}{\bf \tcb{Example: System Crashes}}\label{poisexample}
Let's assume that a system crashes on average three times per year $\Rightarrow$ $\lambda = 3$ and $\Pr(X = x) = \frac{\lambda^x}{x\,!}\,e^{-\lambda} = \frac{3^x}{x\,!}\,e^{-3}$.\\[0.4cm]

What is the probability that:\\[0.6cm]

\ldots there are no crashes in a year?
\begin{align*}
\Pr(X=0) = \frac{3^0}{0\,!}\,e^{-3} = \frac{1}{1}\,e^{-3} = 0.0498.
\end{align*}

\ldots at least one crash in a year?
\begin{align*}
\Pr(X\ge1) = 1 - \Pr(X=0) = 1-0.0498 = 0.9502.
\end{align*}


\end{frame}




\subsection{Example: System Crashes}
\begin{frame}{\bf \tcb{Example: System Crashes}}\label{poisranexample}

\ldots less than 2 crashes in a year? {\footnotesize(since $X$ is discrete ``$< 2\,$'' means ``$\le 1\,$'')}
\begin{align*}
\Pr(X<2) = \Pr(X\le1) &= p(0) + p(1) \\[0.2cm]
&= \frac{3^0}{0\,!}\,e^{-3} + \frac{3^1}{1\,!}\,e^{-3} \\[0.2cm]
& = 0.0498 + 0.1494 = 0.1992.\\
\end{align*}


\ldots between 4 and 6 crashes in a year?
\begin{align*}
\Pr(4\le X \le 6) &= p(4) + p(5) + p(6) \\[0.2cm]
&= \frac{3^4}{4\,!}\,e^{-3} + \frac{3^5}{5\,!}\,e^{-3} + \frac{3^6}{6\,!}\,e^{-3} \\[0.2cm]
&= 0.1680 +0.1008 + 0.0504 = 0.3192.
\end{align*}

\end{frame}

\subsection{Poisson vs Binomial}
\begin{frame}{\bf \tcb{Poisson vs Binomial}}
\begin{itemize}\itemsep0.4cm
\item Binomial$(n, p)$
\begin{itemize}\itemsep0.2cm
\item You have the total number of Bernoulli trials, $n$, and the probability of an event in each trial, $p$.
\item $X \in \{0,1,2,\ldots, n\}$, i.e., the maximum number of events is $n$ since there are $n$ trials.
\item Usage: probability of 1 event in 4 trials, less than 2 events in 6 trials, no events in 3 trials etc.
\end{itemize}
\item Poisson$(\lambda)$
\begin{itemize}\itemsep0.2cm
\item You have the average number of events, $\lambda$, within a fixed interval of time / area / volume etc.
\item $X \in \{0,1,2,\ldots, \infty\}$, i.e., there is no upper limit for $X$ since there are an infinite number of Bernoulli trials throughout the interval.
\item Usage: probability of 1 event per interval, less than 2 events per interval, no events per interval etc.
\end{itemize}
\end{itemize}

\end{frame}




\subsection{Different Time-frame}
\begin{frame}{\bf \tcb{Different Time-frame}}
Note the following:\\
\begin{itemize}\itemsep0.2cm
\item $\lambda$ is the average number of events per 1 interval.
\item $\lambda \times 2$ is the average number of events per 2 intervals.
\item $\lambda \times 3$ is the average number of events per 3 intervals.
\item $\lambda \times 0.25$ is the average number of events per 0.25 intervals.
\item \ldots etc.\\[0.5cm]
\end{itemize}

In general:
\begin{itemize}
\item $\boxed{\lambda t \text{ is the average number of events per } t \text{ intervals}}$.\\[0.5cm]
\end{itemize}

$\Rightarrow$ {\bf the number of events per {\boldmath$t$} intervals has a Poisson{\boldmath$(\lambda t)$} distribution.}

\end{frame}




\subsection{Example: System Crashes}
\begin{frame}{\bf \tcb{Example: System Crashes}}
We said that there are $\lambda = 3$ crashes per year.\\[0.4cm]

What is the probability of no crashes in 2 years? $\Rightarrow$ $\lambda t = 3(2) = 6$ crashes on average per 2 years.
\begin{align*}
\Pr(X=0) = \frac{6^0}{0\,!}\,e^{-6} = \frac{1}{1}\,e^{-6} = 0.0025.\\[-0.2cm]
\end{align*}

What is the probability of more than 2 crashes in 6 months (i.e., 0.5 years)? $\Rightarrow$ $\lambda t = 3(0.5) = 1.5$ crashes on average per 0.5 years.\vspace{-0.3cm}
\begin{adjustwidth}{-0.1cm}{0cm}
\begin{align*}
\Pr(X > 2) = 1 - \Pr(X\le2) &= 1-[p(0) + p(1) + p(2)]\\
&=  1 - \left(\frac{1.5^0}{0\,!}\,e^{-1.5} + \frac{1.5^1}{1\,!}\,e^{-1.5} + \frac{1.5^2}{2\,!}\,e^{-1.5}\right)\\
&= 1 - (0.2231 + 0.3347 + 0.2510) \\
&= 1 - 0.8088 = 0.1912.
\end{align*}
\end{adjustwidth}

\end{frame}


\subsection{Question 2}
\begin{frame}{\bf \tcb{Question 2}}

You receive emails at an average rate of 2 per hour. What is the probability of receiving:\\[0.2cm]

\begin{enumerate}[a)]\itemsep0.2cm
\item Six emails in one hour.
\item Less than three emails in an hour.
\item No emails in two hours.
\item More than four emails in two hours.
\item At least one email in half an hour.
\item What is the value of the mean number of emails received in one hour? What is the corresponding standard deviation?
\end{enumerate}


\end{frame}



\section{Poisson Tables}
\subsection{Poisson Tables}
\begin{frame}{\bf \tcb{Poisson Tables}}

The {\bf Poisson tables} are used in the same way as the binomial tables.\\[1cm]

In particular, {\bf ``greater than or equal to''} probabilities are tabulated:
\begin{align*}
\boxed{\Pr(X \ge r)}
\end{align*}
where $r$ is the value in question.\\[1cm]

We select the appropriate Poisson distribution by finding the $\lambda$ value in the column headings (note: the tables use the symbol $m$ rather than $\lambda$).
\end{frame}


\subsection{Example: System Crashes}
\begin{frame}{\bf \tcb{Example: System Crashes}}\label{poisexampletab}
With $X \sim \text{Poisson}(\lambda=3 \text{ / year})$, what is the probability that:\\[0.4cm]

\ldots there are no crashes in a year?
\begin{align*}
\Pr(X=0) = \Pr(X \ge 0) - \Pr(X \ge 1) = 1.0000 - 0.9502 = 0.0498.\\[-0.1cm]
\end{align*}

\ldots at least one crash in a year?
\begin{align*}
\Pr(X\ge1) = 0.9502.\\[-0.1cm]
\end{align*}

\ldots less than 2 crashes in a year?
\begin{align*}
\Pr(X<2) = 1 - \Pr(X\ge2) &= 1 - 0.8009 = 0.1991.\\[-0.1cm]
\end{align*}

\ldots between 4 and 6 crashes in a year?
\begin{align*}
\Pr(4\le X \le 6) &= \Pr(X \ge 4) - \Pr(X \ge 7) = 0.3528 - 0.0335 = 0.3193.
\end{align*}

\end{frame}



\subsection{Example: System Crashes}
\begin{frame}{\bf \tcb{Example: System Crashes}}

What is the probability of no crashes in 2 years? $\Rightarrow$ $\lambda t = 3(2) = 6 = m$ in the tables.
\begin{align*}
\Pr(X=0) = \Pr(X \ge 0) - \Pr(X \ge 1) = 1.0000 - 0.9975 = 0.0025.\\[0.2cm]
\end{align*}

What is the probability of more than 2 crashes in 6 months (i.e., 0.5 years)? $\Rightarrow$ $\lambda t = 3(0.5) = 1.5 = m$ in the tables.
\begin{align*}
\Pr(X > 2) = \Pr(X\ge3) &= 0.1912.
\end{align*}

\end{frame}


\subsection{Question 3}
\begin{frame}{\bf \tcb{Question 3}}


You receive emails at an average rate of 2 per hour. What is the probability of receiving:\\[0.2cm]

\begin{enumerate}[a)]\itemsep0.2cm
\item Six emails in one hour.
\item Less than three emails in an hour.
\item No emails in two hours.
\item More than four emails in two hours.
\item At least one email in half an hour.\\[0.8cm]
\end{enumerate}

Note: you calculated these in Question 1 using the \emph{formula} for the probability function.

\end{frame}









\section{R Code}
\subsection{R Code}
\begin{frame}{\bf \tcb{R Code}}
As with the binomial distribution, we can calculate probabilities for the Poisson distribution also:\\[0.3cm]

\begin{tabular}{|l|}
\hline
Examples: \\[0.2cm]
\texttt{dpois(0,lambda=3)} \\
gives \texttt{0.04978707}, \\[0.2cm]
\texttt{dpois(4:6,lambda=3)} \\
gives \texttt{0.16803136 0.10081881 0.05040941}\\[0.2cm]
and\\[0.2cm]
\texttt{sum(dpois(4:6,lambda=3))} \\
gives \texttt{0.3192596}\\
\hline
\multicolumn{1}{c}{}\\[0.1cm]
\end{tabular}

Compare the above with slides \pageref{poisexample} and \pageref{poisranexample}
\end{frame}



\subsection{R Code}
\begin{frame}{\bf \tcb{R Code}}
\emph{Greater than} probabilities, i.e., $\Pr(X > x)$, can also be calculated.\\[0.4cm]
It is important to note that this differs from the Poisson tables which (as we saw) provide \emph{greater than or equal to} probabilities.\\[0.4cm]

\begin{tabular}{|l|}
\hline
Examples: \\[0.2cm]
\texttt{ppois(0,lambda=3,lower=F)} \\
gives \texttt{0.9502129} which is $\Pr(X > 0) = \Pr(X \ge 1)$.\\[0.2cm]
\texttt{ppois(3,lambda=3,lower=F)} \\
gives \texttt{0.3527681} which is $\Pr(X > 3) = \Pr(X \ge 4)$.\\[0.2cm]
\texttt{ppois(6,lambda=3,lower=F)} \\
gives \texttt{0.03350854} which is $\Pr(X > 6) = \Pr(X \ge 7)$.\\
\hline
\multicolumn{1}{c}{}\\[0.0cm]
\end{tabular}

Compare this with slide \pageref{poisexampletab}.
\end{frame}



\subsection{R Code}
\begin{frame}{\bf \tcb{R Code}}
We can \emph{generate} Poisson random variables as follows:\\[0.5cm]

\begin{tabular}{|l|}
\hline
Example: \\[0.2cm]
\texttt{rpois(100,lambda=3)} \\
generates 100 Poisson$(\lambda=3)$ variables.\\
\hline
\multicolumn{1}{c}{}\\[0.2cm]
\end{tabular}


\end{frame}






\end{document} 