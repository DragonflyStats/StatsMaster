\documentclass[a4paper,12pt]{article}
%%%%%%%%%%%%%%%%%%%%%%%%%%%%%%%%%%%%%%%%%%%%%%%%%%%%%%%%%%%%%%%%%%%%%%%%%%%%%%%%%%%%%%%%%%%%%%%%%%%%%%%%%%%%%%%%%%%%%%%%%%%%%%%%%%%%%%%%%%%%%%%%%%%%%%%%%%%%%%%%%%%%%%%%%%%%%%%%%%%%%%%%%%%%%%%%%%%%%%%%%%%%%%%%%%%%%%%%%%%%%%%%%%%%%%%%%%%%%%%%%%%%%%%%%%%%
\usepackage{eurosym}
\usepackage{vmargin}
\usepackage{amsmath}
\usepackage{graphics}
\usepackage{epsfig}
\usepackage{framed}
\usepackage{subfigure}
\usepackage{fancyhdr}

\setcounter{MaxMatrixCols}{10}
%TCIDATA{OutputFilter=LATEX.DLL}
%TCIDATA{Version=5.00.0.2570}
%TCIDATA{<META NAME="SaveForMode"CONTENT="1">}
%TCIDATA{LastRevised=Wednesday, February 23, 201113:24:34}
%TCIDATA{<META NAME="GraphicsSave" CONTENT="32">}
%TCIDATA{Language=American English}

\pagestyle{fancy}
\setmarginsrb{20mm}{0mm}{20mm}{25mm}{12mm}{11mm}{0mm}{11mm}
\lhead{MA4128} \rhead{Kevin O'Brien} \chead{Week 8 Part B} %\input{tcilatex}

%http://www.electronics.dit.ie/staff/ysemenova/Opto2/CO_IntroLab.pdf
\begin{document}


\subsection{Akaike Information Criterion}


Akaike's information criterionis a measure of the goodness of fit of
an estimated statistical model. The AIC was developed by Hirotsugu Akaike under the name of ``an information criterion" in 1971. The AIC is a \textbf{\textit{model selection}} tool i.e. a method of comparing two
or more candidate regression models. The AIC methodology attempts to find the model that best explains the data with a minimum of parameters. (i.e. in keeping with the law of parsimony)

The AIC is calculated using the "likelihood function" and the number of parameters ( Likelihood function : not on course). The likelihood value is generally given in code output, as a complement to the AIC.
Given a data set, several competing models may be ranked according to their AIC, with the one having the lowest AIC being the best. (Although, a difference in AIC values of less than two is considered negligible).

The Akaike information criterion is a measure of the relative goodness of fit of a statistical model. It was developed by Hirotsugu Akaike, under the name of "an information criterion" (AIC), and was first published by Akaike in 1974.
\bigskip
%AIC provides a means for comparison among models—a tool for model selection.
%\bigskip
%AIC is good for prediction.\\

\[\mbox{AIC} = 2p - 2\ln(L)\]

\begin{itemize}
	\item $p$ is the number of free model parameters.
	\item $L$ is the value of the Likelihood function for the model in question.
	\item For AIC to be optimal, $n$ must be large compared to $p$.\\
\end{itemize}
\subsubsection{Schwarz's Bayesian Information Criterion}
An alternative to the AIC is the Schwarz BIC, which additionally takes into account the sample size $n$.

\[\mbox{BIC} = p\ln{n} - 2\ln(L)\]



\section{Information Criterions}


We define two types of information criterion: the Bayesian Information
Criterion (BIC) and the Akaike Information Criterion (AIC). In AIC and BIC, we choose the model that
has the minimum value of:
\[AIC = −2log(L)+2m,\]
\[BIC = −2log(L)+mlogn\]

where
\begin{itemize}
\item L is the likelihood of the data with a certain model,
\item n is the number of observations and
\item m is the number of parameters in the model.
\end{itemize}
\subsection{AIC}
The Akaike information criterion is a measure of the relative \textbf{goodness of fit} of a statistical model.

When using the AIC for selecting the parametric model class, choose
the model for which the AIC value is lowest.


\end{document}
