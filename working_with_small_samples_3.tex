\documentclass[a4]{beamer}
\usepackage{amssymb}
\usepackage{graphicx}
\usepackage{subfigure}
\usepackage{framed}
\usepackage{newlfont}
\usepackage{amsmath,amsthm,amsfonts}
%\usepackage{beamerthemesplit}
\usepackage{pgf,pgfarrows,pgfnodes,pgfautomata,pgfheaps,pgfshade}
\usepackage{mathptmx}  % Font Family
\usepackage{helvet}   % Font Family
\usepackage{color}

\mode<presentation> {
 \usetheme{Frankfurt} % was
 \useinnertheme{rounded}
 \useoutertheme{infolines}
 \usefonttheme{serif}
 %\usecolortheme{wolverine}
% \usecolortheme{rose}
\usefonttheme{structurebold}
}

\setbeamercovered{dynamic}

\title[MA4603]{Science Maths 3 \\ {\normalsize MA4603 Lecture 11A}}
\author[Kevin O'Brien]{Kevin O'Brien \\ {\scriptsize Kevin.obrien@ul.ie}}
\date{Autumn Semester 2017}
\institute[Maths \& Stats]{Dept. of Mathematics \& Statistics, \\ University \textit{of} Limerick}

\renewcommand{\arraystretch}{1.5}

\begin{document}

%------------------------------------------------------------------------------%
\begin{frame}
\frametitle{Small samples}
\begin{itemize} \item We indicated that use of the normal distribution in estimating a population mean is warranted
for any large sample ($n > 30$). \item For a small sample ($n \leq 30$) only if the population is normally distributed
\textbf{and} $\sigma$ is known, the standard normal distribution can be used compute quantiles. In practice,
this case is unusual.
\item Now we consider the situation in which the sample is small and the population is normally distributed,
but $\sigma$ is not known.
\end{itemize}
\end{frame}
%------------------------------------------------------------------------------%
\begin{frame}
\frametitle{Student's $t-$distribution (1)}
\begin{itemize}
\item Student's $t-$distribution is a variation of the normal distribution, designed to factor in the increased uncertainty resulting from smaller samples.
\item The distribution is really a family of distributions, with
a somewhat different distribution associated with the degrees of freedom ($df$). For a confidence interval for the
population mean based on a sample of size n, $df = n - 1$.
\end{itemize}
\end{frame}

%------------------------------------------------------------------------------%
\begin{frame}
\frametitle{Student's $t-$distribution (2)}
\begin{itemize}
\item With increasing
sample size, the $t-$distribution approaches the form of the standard normal (`Z') distribution.
\item In fact the standard normal distribution can be thought of as the $t-$distribution with $\infty$ degrees of freedom.
\item For computing quantiles, we will consider the `Z' distribution in this way.
\item For values of $n$ greater then 30, the difference between using $df = n-1$ and $df = \infty$ is negligible.

\item As this will be relevant later, remember that a confidence interval is a \textbf{two-tailed} procedure, i.e. $k=2$.
\end{itemize}
\end{frame}

%-----------------------------------------------------------%

\begin{frame}[fragile]
\frametitle{Student's $t-$distribution (3)}

\begin{itemize}
\item Student's t- values are determined using the \texttt{t} family of commands (e.g. \texttt{qt, pt, dt}).
\item To compute quantiles, use the code below.
\item The degrees of freedom must be additionally be specified. Degrees of freedom are computed as sample size minus one ($n-1$)
\item As the degrees of freedom gets larger and larger, the student t distribution converges to the Z distribution.

\end{itemize}
\begin{verbatim}

> qt(0.975, df=4:28)
 [1] 2.776445 2.570582 2.446912 2.364624 2.306004
 [6] 2.262157 2.228139 2.200985 2.178813 2.160369
[11] 2.144787 2.131450 2.119905 2.109816 2.100922
[16] 2.093024 2.085963 2.079614 2.073873 2.068658
[21] 2.063899 2.059539 2.055529 2.051831 2.048407
\end{verbatim}
\end{frame}
%------------------------------------------------------------------------------%
\begin{frame}
\frametitle{Confidence Interval for a Mean (Small Sample)}
\begin{itemize}
\item The mean operating life for a random sample of $n = 10$ light bulbs is $\bar{x} = 4,000$ hours, with the sample
standard deviation $s = 200$ hours. \item The operating life of bulbs in general is assumed to be approximately normally distributed.\item
We estimate the mean operating life for the population of bulbs from which this sample was taken, using a 95 percent
confidence interval as follows:

\[4,000\pm(2.262)(63.3)  = (3857,4143)\]

\item The point estimate is 4,000 hours. The sample standard deviation is 200 hours, and the sample size is 10. Hence \[S.E(\bar{x} ) = { 200 \over \sqrt{10}} = 63.3\]

\item From last slide, the t quantile with $df=9$ is 2.262.
\end{itemize}
\end{frame}

%------------------------------------------------------------------------%


% -- Lecture 8B
% -- Revise the Tables
% -- Sample Size Estimation for mean
% -- Example SSE for mean
% -- SSE for Proportion
% -- Example SSE for proportion
% -- Paired Test

\end{document}




