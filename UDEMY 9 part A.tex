\documentclass{beamer}

\usepackage{amsmath}

\begin{document}
\begin{frame}
\tableofcontents
\end{frame}
\section{Counting methods and probability}

%Chapter 4

Essential reading
Epp $9CilOflS 6.1, 6.2, 6.3, 6.4 or M&B Sections 3.2, 3.3, 3.4, 3.5, 4.1
Keywords
Counting in which order is important, the Multiplication Principle, the Addition Principle,
permutations and the factorial notation, counting in which order is not important, the
Principle of Inclusion-Exclusion, counting using sets, probability.
Many of the problems that arise in discrete mathematics, probability
and their applications involve counting the number of ways in which
some operation can be performed or counting the number of items
satisfying some given specifications. We shall consider two types of
counting problems depending on whether or not the items we are
counting are affected by order.
\end{frame}
%=============================================================================================================== %
\begin{frame}
The basic counting methods
Learning objectives for this section
I Illustrating a counting problem by a rooted tree
I Using the Multiplication Principle
I How to distinguish between combinations and permutations
I The number of permutations of length r chosen from a set 01 n objects
I The number of combinations of r objects chosen from a set of n distinct objects
I Factorial notation
I Deciding whether an (ordered) sequence or an (unordered) set is the correct type
of solution to a given counting problem
I Counting problems involving permutations and combinations
Counting in which order is important
Consider the following two related examples.

\begin{frame}

\end{frame}
\begin{frame}
Example 4.1 

From the menu for a school meal, you may select one of
three starters, one of five main dishes and one of two desserts. How
many different meals consisting of a starter, a main course and a
dessert is it possible to order?

Example 4.2 You toss a coin six times in succession; when it comes
down showing “heads”, you record a 1 and when it comes down
showing “tails”, you record a 0. How many different sequences of 0’s
and 1‘s is it possible to obtain?

\end{frame}
\begin{frame}
You can easily solve these small problems by enumerating the
possibilities (i.e. making a list and checking that you haven't left any
out). But we would really like a general method for solving
problems of this type, larger as well as smaller ones.
\end{frame}
\begin{frame}
The problem described in Example 4.1 can be divided into stages.
\begin{description}
\item[Stage 1] Select one of the three starters;
\item[Stage 2] Select one of the five main dishes.
\end{description}
Each of the 3 choices at Stage 1 is followed by one of 5 choices
at Stage 2, giving 3 >< 5 = 15 different combinations of starter
and main dish.
\end{frame}
\begin{frame}
\begin{description}
\item[Stage 1] Select one of the two desserts.
\end{description}
Each of these 15 combinations of starter and main dish can be
followed by any one of 2 choices for dessert at Stage 3, giving a
total of3 >< 5 >< 2 = 30 different meals.
l 1.starter l 2. main course l 3. dessert l
3 choices >< 5 choices >< 2 choices = E
This approach gives the following useful counting principle.
\end{frame}
\begin{frame}
\frametitle{The Multiplication Principle}
Suppose that a process can be completed in Ic stages. If there are m1
choices at Stage 1 and each of these can be followed by m2 choices at
Stage 2, . . ., andfinally each of the m;€_1 choices at Stage (lc — 1) can
be followed by nit, choices at Stage la, then the number of different
outcomes of the process is given by the product mun; . . . mk.
We can illustrate the above process by means of a rooted tree, as
described in Chapter 3.
Starters
m1 = 3
so root has deg. 3
Main courses
m2 = 5
DES5EIts
ITL3 = 2
The rooted tree has 30 leaves
corresponding io the 771117121113 = 3 >< 5 >< 2 : 30 meal choices
\end{frame}
%=============================================================================================================== %
\begin{frame}

Figure 4.1 : There are 30 possible school meals
The root of the tree corresponds to the beginning of the process, the
edges leaving the root correspond to the m1 choices of Stage 1, the
vertices at the first level to the ml different outcomes from Stage 1,
and so on. The resulting tree will have mung . . . mk external
vertices, all belonging to level k, corresponding to the different
outcomes from the whole process. Figure 4.1 shows the rooted tree
depicting the 30 possible school meals of Example 4.1.
\end{frame}
%=============================================================================================================== %
\begin{frame}

Now let us turn to the solution of Example 4.2. We need to calculate
the number of possible sequences of six symbols, each of which is
either a 1 or a 0. We can regard this as a 6-stage problem in which
we have just two choices at stage 1 and each of these is followed by
two choices at stage 2, and so on. Thus we may apply the
\end{frame}
%=============================================================================================================== %
\begin{frame}
Multiplication Principle with ml = m2 = . . . = TTL6 = 2.
Hence the number of ways of completing the 6 stages is
(2 >< 2 >< 2 >< 2 >< 2 >< 2) : 2“, which gives the total number of
different sequences as 64.
1 2 3 4 5 6
Record: 2 >< 2 >< 2 >< 2 >< 2 >< 2 = 64 outcomes
choices at stage 3
Figure 4.2: There are 64 possible outcomes from tossing a coin 6 times.
Example 4.3 We now draw the tree diagram to illustrate Example 4.2.
(a) The full tree has 7 levels (height 6).
(b) It is a binary tree.
(c) The first three levels of the tree are shown in Figure 4.3.
1st toss: 1 0
2nd toss: 1 Ll
Figure 4.3: The start of the binary tree illustrating Example 4.2.
(d) Due to space restrictions we leave it to you to draw the
remaining four levels of the tree. There will be 64 external
vertices at level 7.
\end{frame}
\begin{frame}
\frametitle{The number of binary strings}
A finite sequence of zeros and ones is known as a binary string.
Each digit (O or 1) in the string is called a bit and a binary string of n
The basic counting methods
51



ClS102 Mathematics ior computing volume 2
bits is called an n-bit binary string. Thus O1 is a 2»bit binary string
and 10110 is a 5-bit binary string. The complete list of solutions to
\end{frame}
\begin{frame}
Example 4.2 gives all the 6-bit binary strings. Generalizing the
method of Example 4.2 by which we showed that there are exactly
2° 6-bit binary strings, we can prove the following result.
\end{frame}
\begin{frame}
Theorem 4.1 There are 2" different n.-bit binary strings.
Proof. We can construct any n-bit binary string in n stages, where for
i = 1, 2, . . . ,n, stage 1'. consists of choosing the ith digit in the
sequence. At each stage, we have exactly 2 choices, regardless of
which choice we made at the previous stage. Thus we may apply the
Multiplication Principle with ml = in; = . . . = mu = 2, giving 2”
11.-bit binary strings altogether;
\end{frame}
\section{Permutations and the factorial notation}
\begin{frame}
\frametitle{Permutations and the factorial notation}
Example 4.4 Four playing cards, the Ace, King, Queen and Jack of
hearts, are shuffled and then dealt, face up, in a row. In how many
different orders can these cards be laid down?
\bigskip
This is a 4-stage problem; there are 4 possibilities for the card that is
turned up first; when this has been decided, it leaves just 3
possibilities for the card that is turned up next; this in turn leaves 2
possibilities for the card that is turned up third and just 1 possibility
for the final card. Thus the total number of different arrangements
is \[4 \times 3 \times 2 \times 1 =24\] by the Multiplication Principle. 
\end{frame}
\begin{frame}
Here is the
corresponding tree diagram:
Q
Q
i
Any arrangement of a set of 11, distinct items that puts these items in
an order is called a permutation of the set. Thus we have shown
above that there are 24 different permutations of a set containing 4
elements.
\end{frame}
%=============================================================================================================== %
\begin{frame}
Example 4.5 Suppose that we repeat Example 4.4 using all the 13
cards in one suit. We now have a 13-stage problem and the total
number of permutations of the cards is 13 >< 12 >< . .. x 3 >< 2 >< 1. We
denote this number by the symbol 13!, read “13 factorial”.
In general, for any n 3 1, the notation 11! , read “vi factorial”, stands
for the product n >< (n — 1) >< (n — 2) >< ... >< 3 >< 2 >< 1. For algebraic
reasons, it is convenient to define
O! = 1.
We shall explain later why this is so.
The algorithm we considered in Example 2.2 computes n! by simply
52



iterating a while-loop which multiplies together all the positive
integers between n and 1.

\end{frame}
\begin{frame}
\frametitle{Factorials}
Notice that for n Z 1, factorials are defined recursively, using the
recurrence relation
\[n! = n \times (n-1)! \] 
By using this recurrence relation it is possible to make a recursive
function to calculate nl.

Generalizing Examples 4.4 and 4.5, we have the following result.
\end{frame}
\begin{frame}
Theorem 4.2 Suppose that we have a set of n distinct items. The number
of ordered lists without repetition that we can make from these is nl.
This Theorem follows from the Multiplication Principle as Figure 4.4
indicates. An alternative statement of Theorem 4.2 is:
Theorem 4.2‘ There are nl permutations of a set with n distinct items.
List: i Pos 1 i Pos 2 i Pos 3 i Pos n — 1 | Pos n i
i i i l l
2) >< >< 2 >< 1
n ><(n—1)><(1i— =
when Pos 1 has been chosen,
there are only n 7 1 items when Pos 1 2 have been chosen
left to choose from ’ . ’
there are only n ~ 2 items
left to choose from
Figure 4.4: There are nl ways of making an ordered list of n distinct objects
\end{frame}
\begin{frame}
Example 4.6 The number of ordered lists ot 1" elements chosen from TL
Suppose that in Example 4.5, we lay down just four of the 13 cards
in a row. Using the Multiplication Principle we see that the number
of different arrangements possible is
13!
>< >< >< 9!
13 12 11 10= .
The following result generalizes Example 4.6 above.
Theorem 4.3 Suppose that n and /r are positive integers, with r § n.
Then the number of ordered lists ofr distinct items chosen from a set of
n distinct elements is
nl
n(n— 1)('7L —2)...(n — (r — 1)) = (n7T)!.
Proof. Choosing r objects from n is an r»stage process where we
have n choices at stage 1, n — 1 choices at stage 2, n — 2 choices at
stage 3, and so on up to stage 1' where we have n — (vi — 1) choices.
Hence the required number of lists is
(n—1)(n~2)...(n~('r~1)),
The basic counting methods
I
53



4.1.2
C|S102 Mathematics ior computing volume 2
which is the left hand side. To see that the right hand side is also
equal to this is a straightforward calculation:
n! _ n(11.—1)(n—2)...(n—1"+1) (n—r)(n—'r—1)...
I\-7
>~
(n—r)! _ (n.—r)(n—r—1)...
= n(n—l)(n—2)...(n—r+1)
= n(n—1)(n—2)...(n—(r—1)).I
Counting in which order is not important
We next consider the problem of counting the number of ways of
choosing a subset from a given set. Recall that sets are unordered,
so that for example
{a, n, if} = {t, (1, 
This means that when we are selecting subsets, the order in which
the items are selected is not important.
\end{frame}
\begin{frame}
Example 4.7 We consider two fundamentally different problems:
(a) A President, Secretary and Treasurer for the student Computing
Society are elected from among 45 students. Given that no
student can fill more than one of these posts, how many
different selections are possible?
\end{frame}
\begin{frame}
(b) A set of three students is to be chosen from a class of 45
students to represent the class on the Staff-Student Consultative
Committee. How many different selections are possible?
In (a), the order of selection is important, because it is not
unimportant which of the elected persons gets the post of President,
who gets the post of Secretary and who gets the post of Treasurer.
Hence by the Multiplication Principle, the number of different
selections is 45 >< 44 x 43.
In (b), there are 45 ways of choosing the first student, then 44 ways
of choosing the second and finally 43 ways of choosing the third.
This gives us 45 >< 44 x 43 ways of choosing an ordered list of three
students, just as for (a) above. But this time we are not interested in
the order in which the students are chosen; all that interests us is
the set of students selected. Suppose that A, B , C, is an ordered list
of three students. Then we would get the some set {A B, C} of
students from any of the following six ordered lists:
A,B,C; A,C.B; B,A,C; B,C,A; C,A,B; C,B,A.
You will recognize that any of the six possible permutations of a set
of three items will give rise to the same unordered set. Thus among
the 45 >< 44 >< 43 different ordered lists of three students, each
possible set of three students occurs exactly 6 = 3! times. Thus the
number of different sets of three students is
45 >< 44 >< 43 45!
3! 42! 3!
In general, we call a subset of r items chosen from a set of n distinct
items a combination of 1- items chosen from n items. The
54
I\)
>-1



following result generalizes Example 4.7.
Theorem 4.4 The number of different combinations of r items chosen
. . .
from n items is 
Proof. There are  different ordered lists that we could make of
T items chosen from 77 items. But given any subset of 7‘ items, we can
form 1"! different ordered lists of its elements. Thus the elements of
each subset of r items occur on 1-! different lists. Hence the number
of different subsets that can be chosen is  I
We denote the number of combinations of 1" items chosen from n
items by the binomial coefficient  read “n choose '1'". Thus
(7)  <4-H
Other books may use different notations for  Among these are
C (71. r), CI‘ and "CT, but as the  is the most universally
recognized, we shall not be using other symbols on this course. You
are also encouraged not to use these other notations for binomial
coefficients.
\end{frame}
%=============================================================================================================== %
\begin{frame}
Example 4.8 Bill knows 7 children.
(a) He has 4 cinema tickets and thus wants to take 3 of the children
to the cinema. In how many possible ways can he select 3
children to take to the cinema?
Answer: 
This selection is unordered as we are selecting a set of 3
children to accompany Bill. The order in which the children are
chosen is not important as the end result is the same: they all
end up with one cinema ticket each.
(b) The next day Bill has three tasks to do: (1) he has to do the
dishes; (2) he has to eat lunch at a nice restaurant and (3) he
has to solve his maths assignment. ln how many possible ways
can he select 3 children to help him with his tasks?
Answer: 7 >< 6 >< 5.
\end{frame}
\begin{frame}
This selection is ordered because the three tasks are very
different, some are much more pleasant than others (e.g.
solving the maths assignment is obviously everybody’ s favourite,
while nobody wants to do dishes!). This is thus a 3-stage
problem where stage i, for i = 1, 2, 3, is choosing somebody to
help with task (i). The answer thus follows from Theorem 4.3.
\end{frame}
\begin{frame}
We can immediately deduce the following identity for binomial
coefficients from equation (4.1).

Lemma 4.5 Let n and r be integers with 0 g r § n. Then
n _ vi _ n!
n—v' _ r _(7l*’f)l7‘l‘

Proof. Computing (TIL) and  using the equation (4.1) both yield
the fraction on the right hand side. I

\end{frame}
\begin{frame}
Although Lemma 4.5 follows immediately from equation (4.1), we
would expect that there should also be an underlying logical reason
why Lemma 4.5 holds. This logical reason is illustrated by the
following example.
Example 4.9 It is required to make cloth covers for the notice board
for each of the 17 different sports clubs in the Athletic Union. There
is only sufficient green cloth for 13 boards and the remainder are to
be covered in blue.
\end{frame}
\begin{frame}
There are  ways of choosing 13 boards to be covered in green,
(leaving 4 boards to be covered blue); or, looked at the other way
round, there are  ways of choosing 4 boards to be covered in blue
(leaving 13 boards to be covered in green). This is just two different
ways of counting all ways of getting 13 green boards and 4 blue, so
it is clear that both expressions should give the same answer. Thus,
17 17 17><16><15><14
(13)'<4>' 4><3><2><1 "2380
\end{frame}
\begin{frame}
We can now give a reason why we should put 0! = 1. There is only
one way of selecting a subset of size 0 from a set of n items (we
choose none of them); similarly there is only one way of choosing a
subset of size n from a set of n items (we choose all of the items).
Thus we would expect  : 1 and  : 1. You will see this holds in
equation (4.1) if we define 0! to be equal to 1. With this definition,
which must have seemed rather mysterious when it was first given,
the expression  =  holds for all values of r where
O § 'r § 7L.
\end{frame}
\begin{frame}
Summary of strategies for counting problems
When you are faced with a counting problem, you need to analyse it
very carefully before applying one of the formulae we have derived
in this section. If you rely solely on memorising the formulae you
may use the wrong one! We have encountered three main types of
counting problems:

\end{frame}
\section{Summary of counting strategies}
\begin{frame}
\frametitle{Sequences with repetition allowed}

In this type, we are required to find the number of possible sequences where
each term is drawn from a given set and repetitions are allowed.
Examples of this type: Binary strings; codes, such as telephone
numbers, car numbers, codewords, etc. ; integers which can be
formed from a given set of digits.
Solution to this type: If the number of choices for each term of the
sequence is a constant number n, say, then the solution is n",
where r is the length of the sequence.
\end{frame}
\begin{frame}
\frametitle{Sequences with no repetitions allowed}
In this type, we are required to find the number of possible sequences where
each term is drawn from a given set and repetitions are not allowed.
Examples of this type: Permutations of a set; lists; arrangements of
objects in a row; integers with distinct digits; teams in which
each person is chosen to perform a distinct task, etc.
Solution to this type: The number of different arrangements of r
distinct objects chosen from a set of n distinct objects is L!
(n—r)!
56

\end{frame}
\begin{frame}

Choosing subsets ot a given set
In this type, we are interested only in choosing the subset and the order in which
the elements are selected is not important.
Examples of this type: Teams, groups and subsets where the
members are not chosen to fulfill different functions; hands of
cards, etc.
Solution to this type: The number of ways of selecting a subset of
size r from a set of size n is  = 
\end{frame}
%=============================================================================================================== %
\begin{frame}
4.2 Counting using sets
Learning objectives for this section
I Definition of what it means for sets A1, A2, . . . ,A,, to be pairwise disjoint
I Definition of what it means for sets A1, A2. . . . .An to partition a set S
I Statement and use of the Addition Principle
I Statement and use of the Principle of Inclusion-Exclusion of Theorem 4.6 to count
the number of elements in the union of two arbitrary sets A and B
I Using Venn diagrams (or the Principle oi Inclusion-Exclusion extended to three
sets) to find the size of a union of three finite sets
I Using the Addition Principle and Theorem 4.6 in coniunction with the basic
counting methods for permutations and combinations we learnt in the previous
section
In Volume 1, Chapter 2, we defined the size or cardinality of a
finite set S to be the number of elements in S and denoted this by
lSl. In this section, we shall find a way of counting the number of
elements in the union of two or more sets.
Example 4.10 Finding the cardinality of a union of sets
Let A = {1,2,3} , B = {O,4.5} and C = {3,4,5,6}.
(a) A U B = {O,1,2,3,4,5}, so that lALJ Bl = 6 = |Al + |B|.
(b) A U C = {1,2.3.4.,5,6}, so that lA U Cl = 6.
But in this case, lA| + lC| = 3 + 4 = 7.
(c) BLJC = {O,3,4,5,6}, so that |BLJCl = 5.
But in this case, lBl + lCl = 3 +4 = 7.
Then:
Counting using sets
57



