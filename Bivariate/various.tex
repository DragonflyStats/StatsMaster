\documentclass[a4paper,12pt]{article}
%%%%%%%%%%%%%%%%%%%%%%%%%%%%%%%%%%%%%%%%%%%%%%%%%%%%%%%%%%%%%%%%%%%%%%%%%%%%%%%%%%%%%%%%%%%%%%%%%%%%%%%%%%%%%%%%%%%%%%%%%%%%%%%%%%%%%%%%%%%%%%%%%%%%%%%%%%%%%%%%%%%%%%%%%%%%%%%%%%%%%%%%%%%%%%%%%%%%%%%%%%%%%%%%%%%%%%%%%%%%%%%%%%%%%%%%%%%%%%%%%%%%%%%%%%%%
\usepackage{eurosym}
\usepackage{vmargin}
\usepackage{amsmath}
\usepackage{graphics}
\usepackage{epsfig}
\usepackage{framed}
\usepackage{subfigure}
\usepackage{fancyhdr}

\setcounter{MaxMatrixCols}{10}
%TCIDATA{OutputFilter=LATEX.DLL}
%TCIDATA{Version=5.00.0.2570}
%TCIDATA{<META NAME="SaveForMode"CONTENT="1">}
%TCIDATA{LastRevised=Wednesday, February 23, 201113:24:34}
%TCIDATA{<META NAME="GraphicsSave" CONTENT="32">}
%TCIDATA{Language=American English}

\pagestyle{fancy}
\setmarginsrb{20mm}{0mm}{20mm}{25mm}{12mm}{11mm}{0mm}{11mm}
\lhead{MA4128} \rhead{Kevin O'Brien} \chead{Week 8} %\input{tcilatex}

%http://www.electronics.dit.ie/staff/ysemenova/Opto2/CO_IntroLab.pdf
\begin{document}

% \tableofcontents


%-----------------------------------------------------------------------------------------%
\section{Multiple Linear Regression}
Multiple regression: To quantify the relationship between several independent (predictor) variables and a dependent (response) variable. The coefficients ($a, b_{1} to b_{i}$) are estimated by the least squares method, which is equivalent to maximum likelihood estimation. A multiple regression model is built upon three major assumptions:

\begin{enumerate}
\item The response variable is normally distributed,
\item The residual variance does not vary for small and large fitted values (constant variance),
\item The observations (explanatory variables) are independent.
\end{enumerate}


\subsection{Dummy Variables}
A dummy variable is a numerical variable used in regression analysis to represent subgroups of the sample in your study. In research design, a dummy variable is often used to distinguish different treatment groups. In the simplest case, we would use a 0,1 dummy variable where a person is given a value of 0 if they are in the control group or a 1 if they are in the treated group. Dummy variables are useful because they enable us to use a single regression equation to represent multiple groups. This means that we don't need to write out separate equation models for each subgroup.





\section{Training and validation}
%http://www.jmp.com/support/help/Validation_2.shtml
Using Validation and Test Data

%When you have sufficient data, you can subdivide your data into three parts called the training, validation, and test data. During the selection process, models are fit on the training data, and the prediction error for the models so obtained is found by using the validation data. This prediction error on the validation data can be used to decide when to terminate the selection process or to decide what effects to include as the selection process proceeds. Finally, once a selected model has been obtained, the test set can be used to assess how the selected model generalizes on data that played no role in selecting the model.

In some cases you might want to use only training and test data. For example, you might decide to use an information criterion to decide what effects to include and when to terminate the selection process. In this case no validation data are required, but test data can still be useful in assessing the predictive performance of the selected model. In other cases you might decide to use validation data during the selection process but forgo assessing the selected model on test data. 
%Hastie, Tibshirani, and Friedman (2001) note that it is difficult to give a general rule on how many observations you should assign to each role. They note that a typical split might be 50\% for training and 25% each for validation and testing.











\end{document}
