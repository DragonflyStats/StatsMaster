\documentclass[a4paper,12pt]{article}
%%%%%%%%%%%%%%%%%%%%%%%%%%%%%%%%%%%%%%%%%%%%%%%%%%%%%%%%%%%%%%%%%%%%%%%%%%%%%%%%%%%%%%%%%%%%%%%%%%%%%%%%%%%%%%%%%%%%%%%%%%%%%%%%%%%%%%%%%%%%%%%%%%%%%%%%%%%%%%%%%%%%%%%%%%%%%%%%%%%%%%%%%%%%%%%%%%%%%%%%%%%%%%%%%%%%%%%%%%%%%%%%%%%%%%%%%%%%%%%%%%%%%%%%%%%%
\usepackage{eurosym}
\usepackage{vmargin}
\usepackage{amsmath}
\usepackage{framed}
\usepackage{graphics}
\usepackage{epsfig}
\usepackage{subfigure}
\usepackage{enumerate}
\usepackage{fancyhdr}

\setcounter{MaxMatrixCols}{10}
%TCIDATA{OutputFilter=LATEX.DLL}
%TCIDATA{Version=5.00.0.2570}
%TCIDATA{<META NAME="SaveForMode"CONTENT="1">}
%TCIDATA{LastRevised=Wednesday, February 23, 201113:24:34}
%TCIDATA{<META NAME="GraphicsSave" CONTENT="32">}
%TCIDATA{Language=American English}

\pagestyle{fancy}
\setmarginsrb{20mm}{0mm}{20mm}{25mm}{12mm}{11mm}{0mm}{11mm}
\lhead{Statistics with \texttt{r}} \rhead{Kevin O'Brien} \chead{Analysis of Two-factor Designs} %\input{tcilatex}

\begin{document}

\section{Analysis of Two-factor Designs}

A two-factor analysis of variance consists of three significance tests: a test of each of the two main effects and a test of the interaction of the variables. An analysis of variance summary table is a convenient way to display the results of the significance tests. A summary table for the hypothetical experiment described in the section on factorial designs and a graph of the means for the experiment are shown below.

\begin{verbatim}
                 Sum of        Mean
SOURCE   df      Squares      Square      F       p
     T    1    47125.3333  47125.3333  384.174   0.000
     D    2       42.6667     21.3333    0.174   0.841
    TD    2     1418.6667    709.3333    5.783   0.006
 ERROR   42     5152.0000    122.6667
 TOTAL   47    53738.6667
\end{verbatim}

\subsection{Sources of Variation}

The summary table shows four sources of variation: (1) Task, (2) Drug dosage, (3) the Task x Drug dosage interaction, and (4) Error.

\subsection{Degrees of Freedom}

\begin{itemize}
\item The degrees of freedom total is always equal to the total number of numbers in the analysis minus one. The experiment on task and drug dosage had eight subjects in each of the six groups resulting in a total of 48 subjects. Therefore, df total = 48 - 1 = 47.

\item The degrees of freedom for the main effect of a factor is always equal to the number of levels of the factor minus one. Therefore, df task = 2 - 1 = 1 since there were two levels of task (simple and complex). Similarly, df dosage = 3 - 1 = 2 since there were three levels of drug dosage (0 mg, 100 mg, and 200 mg).

\item The degrees of freedom for an interaction is equal to the product of the degrees of freedom of the variables in the interaction. Thus, the degrees of freedom for the Task x Dosage interaction is the product of the degrees of freedom for task (1) and the degrees of freedom for dosage (2). Therefore, df Task x Dosage = 1 x 2 = 2.

\item The degrees of freedom error is equal to the degrees of freedom total minus the degrees of freedom for all the effects. Therefore, df error = 47 - 1 - 2 - 2 = 42.
\end{itemize}

\subsection{Mean Squares}
As in the case of a one-factor design, each mean square is equal to the sum of squares divided by the degrees of freedom. For instance, Mean square dosage = 42.6667/2 = 21.333 where the sum of squares dosage is 42.6667 and the degrees of freedom dosage is 2.


\subsection{F Ratios}
The F ratio for an effect is computed by dividing the mean square for the effect by the mean square error. For example, the F ratio for the Task x Dosage interaction is computed by dividing the mean square for the interaction ( 709.3333) by the mean square error (122.6667). The resulting F ratio is: F = 709.3333/122.6667 = 5.783


\subsection{Probability Values}
\begin{itemize}
    \item To compute a probability value for an F ratio, you must know the degrees of freedom for the F ratio. The degrees of freedom numerator is equal to the degrees of freedom for the effect. 
    \item The degrees of freedom denominator is equal to the degrees of freedom error. Therefore, the degrees of freedom for the F ratio for the main effect of task are 1 and 42, the degrees of freedom for the F ratio for the main effect of drug dosage are 2 and 42, and the degrees of freedom for the F for the Task x Dosage interaction are 2 and 42.

\item An F distribution calculator can be used to find the probability values. For the interaction, the probability value associated with an F of 5.783 with 2 and 42 df is 0.006.
\end{itemize}


\subsection{Drawing Conclusions}

When a main effect is significant, the null hypothesis that there is no main effect in the population can be rejected. In this example, the effect of task was significant. Therefore it can be concluded that, in the population, the mean time to complete the complex task is greater than the mean time to complete the simple task (hardly surprising). The effect of dosage was not significant. Therefore, there is no convincing evidence that the mean time to complete a task (in the population) is different for the three dosage levels

The significant Task x Dosage interaction indicates that the effect of dosage (in the population) differs depending on the level of task. Specifically, increasing the dosage slows down performance on the complex task and speeds up performance on the simple task. The effect of increasing the dosage therefore depends on whether the task is complex of simple.

There will always be some interaction in the sample data. The significance test of the interaction lets you know whether you can infer that there is an interaction in the population.

\end{document}