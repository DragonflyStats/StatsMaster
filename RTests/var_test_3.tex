\documentclass[a4]{beamer}
\usepackage{amssymb}
\usepackage{graphicx}
\usepackage{subfigure}
\usepackage{framed}
\usepackage{newlfont}
\usepackage{amsmath,amsthm,amsfonts}
%\usepackage{beamerthemesplit}
\usepackage{pgf,pgfarrows,pgfnodes,pgfautomata,pgfheaps,pgfshade}
\usepackage{mathptmx}  % Font Family
\usepackage{helvet}   % Font Family
\usepackage{color}

\mode<presentation> {
 \usetheme{Frankfurt} % was
 \useinnertheme{rounded}
 \useoutertheme{infolines}
 \usefonttheme{serif}
 %\usecolortheme{wolverine}
% \usecolortheme{rose}
\usefonttheme{structurebold}
}

\setbeamercovered{dynamic}

\title[MA4603]{Science Maths 3 \\ {\normalsize MA4603 Lecture 11A}}
\author[Kevin O'Brien]{Kevin O'Brien \\ {\scriptsize Kevin.obrien@ul.ie}}
\date{Autumn Semester 2017}
\institute[Maths \& Stats]{Dept. of Mathematics \& Statistics, \\ University \textit{of} Limerick}

\renewcommand{\arraystretch}{1.5}

\begin{document}

%----------------------------------------%

\begin{frame}

\frametitle{Test for Equality of Variance (a)}
\begin{itemize}
\item In this procedure, we determine whether or not two data sets have the same variance.
\item The assumption of equal variance underpins several inference procedures.
\item We will not get into too much detail about that, but concentrate on how to assess equality of variance.
\item The null and alternative hypotheses are as follows:
\[ H_0: \sigma^2_1 = \sigma^2_2 \]
\[ H_1: \sigma^2_1 \neq \sigma^2_2 \]
\end{itemize}

\end{frame}
%----------------------------------------%
\begin{frame}
\frametitle{Test for Equality of Variance (b)}
\begin{itemize}
\item When using \texttt{R} it would be convenient to consider the null and alternative in terms of variance ratios.
\item Two data sets have equal variance if the variance ration is 1.
\end{itemize}

\[ H_0: \sigma^2_1 / \sigma^2_2 = 1 \]
\[ H_1: \sigma^2_1 / \sigma^2_2 \neq 1 \]
\end{frame}
%----------------------------------------%
% - x=c(14,13,16,20,12,18,11,09,13,11)
% - y=c(15,13,18,20,10,17,23,11,10)
%----------------------------------------%
\begin{frame}[fragile]
\frametitle{Test for Equality of Variance(c)}
You would be required to compute the test statistic for this procedure.
The test statistic is the ratio of the variances for both data sets.
\[ TS = \frac{s^2_x}{s^2_y} \]
The standard deviations would be provided in the \texttt{R} code. \begin{verbatim}
> sd(x)
[1] 3.40098
> sd(y)
[1] 4.630815
\end{verbatim}
To compute the test statistic.
\[ TS = \frac{3.40^2}{4.63^2} = \frac{11.56}{21.43} = 0.5394 \]

\end{frame}
%----------------------------------------%
\begin{frame}[fragile]
\frametitle{Variance Test (d)}
\begin{verbatim}
> var.test(x,y)

        F test to compare two variances

data:  x and y
F = 0.5394, num df = 9, denom df = 8, p-value = 0.3764
alternative hypothesis: true ratio of variances is not equal to 1
95 percent confidence interval:
 0.1237892 2.2125056
sample estimates:
ratio of variances
         0.5393782
\end{verbatim}

\end{frame}
%----------------------------------------%

\begin{frame}
\frametitle{Variance Test (e)}
\begin{itemize}
\item The p-value is 0.3764, above the threshold level of 0.0250.
\item We fail to reject the null hypothesis.
\item We can assume that there is no significant difference in sample size.
\item Additionally the $95\%$ confidence interval (0.1237, 2.2125) contains the null values i.e. 1.
\end{itemize}
\end{frame}

\end{document}