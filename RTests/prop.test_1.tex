\documentclass[a4]{beamer}
\usepackage{amssymb}
\usepackage{graphicx}
\usepackage{subfigure}
\usepackage{framed}
\usepackage{newlfont}
\usepackage{amsmath,amsthm,amsfonts}
%\usepackage{beamerthemesplit}
\usepackage{pgf,pgfarrows,pgfnodes,pgfautomata,pgfheaps,pgfshade}
\usepackage{mathptmx}  % Font Family
\usepackage{helvet}   % Font Family
\usepackage{color}

\mode<presentation> {
 \usetheme{Frankfurt} % was
 \useinnertheme{rounded}
 \useoutertheme{infolines}
 \usefonttheme{serif}
 %\usecolortheme{wolverine}
% \usecolortheme{rose}
\usefonttheme{structurebold}
}

\setbeamercovered{dynamic}

\title[MA4603]{Science Maths 3 \\ {\normalsize MA4603 Lecture 11A}}
\author[Kevin O'Brien]{Kevin O'Brien \\ {\scriptsize Kevin.obrien@ul.ie}}
\date{Autumn Semester 2017}
\institute[Maths \& Stats]{Dept. of Mathematics \& Statistics, \\ University \textit{of} Limerick}

\renewcommand{\arraystretch}{1.5}

\begin{document}


%-------------------------------------------------%
%-------------------------------------------------%
\begin{frame}[fragile]
\frametitle{Single Sample Proportion Test (a)}

\begin{itemize}
\item In this procedure, we determine whether or not we are are justified in assuming that the population proportion takes a certain value.
\item For example, suppose we believed that the population proportion of students with iphones or androids was $80\%$.
\item We would write the null and alternative accordingly.
\[H_0 : \pi = 80\% \]
\[H_1 : \pi \neq 80\% \]
\item The  appropriate \texttt{R} command is \texttt{prop.test(x,n,p)}
\item $x$ is the number of successes, $n$ is the sample size and $p$ is the population proportion assumed under the null hypothesis.
\item Suppose we survey 65 students, with 50 replying that they had an iphone or android.
\end{itemize}

\end{frame}
%-------------------------------------------------%

\begin{frame}[fragile]
\frametitle{Single Sample Proportion Test (b)}

\begin{verbatim}
> prop.test(50,65,0.80)

        1-sample proportions test

data:  50 out of 65, null probability 0.8
X-squared = 0.2163, df = 1, p-value = 0.6418
alternative hypothesis: true p is not equal to 0.8
95 percent confidence interval:
 0.6452269 0.8610191
sample estimates:
        p
0.7692308
\end{verbatim}

\end{frame}

\begin{frame}[fragile]
\frametitle{Single Sample Proportion Test (c)}

\begin{itemize}
\item The p-value is above the threshold. Therefore we fail to reject the null hypothesis that the population proportion ($\pi$) is $80\%$.

\item The observed proportion is a very straightforward calculation:

\[ \hat{p} = \frac{50}{65} = 0.76923= 76.92\%\]
\item Nonetheless, you would be required to show how it was calculated.
\end{itemize}

\end{frame}

%-------------------------------------------------%
\begin{frame}[fragile]
\frametitle{Test of Equality for Two Sample Proportions (a)}
The null hypothesis is that two populations have the same proportions for a particular characteristic.
\[H_0 : \pi_1 = \pi_2 \]
\[H_1 : \pi_1 \neq \pi_2 \]
\begin{itemize}
\item The command is \texttt{prop.test(c(x1,x2),c(n1,n2))}
\item $x1$ and $x2$ are the number of successes from both samples.
\item $n2$ and $n2$ are the sample sizes for both groups.
\item The difference in population proportions assumed under the null hypothesis is zero.
\item (It is possible to specify a different null value, but we will not consider this in this module.)
\end{itemize}
\end{frame}

%-------------------------------------------------%
\begin{frame}
\frametitle{Test of Equality for Two Sample Proportions (b)}
\begin{itemize}
\item Consider a study where the proportion of Irish students who owned mobile devices, such as iphones and androids was compared to the corresponding proportion of French student.
\item As before, $65$ Irish students were interviewed, with $50$ replying that they owned mobile devices.
\item $90$ french students were interview, with 60 responding that they owned mobile devices.
\item The test of equality of proportions is implemented on the next slide.
\end{itemize}
\end{frame}

%-------------------------------------------------%


\begin{frame}[fragile]
\frametitle{Test of Equality for Two Sample Proportions (c)}
Based on the p-value, we fail to reject the null hypothesis. There is not enough evidence to assume a difference in proportions. Also the expected difference assumed under the null hypothesis, i.e. 0, is contained in the confidence interval.
\begin{verbatim}
> prop.test(c(50,60),c(65,90))

        2-sample test for equality of proportions

data:  c(50, 60) out of c(65, 90)
X-squared = 1.4613, df = 1, p-value = 0.2267
alternative hypothesis: two.sided
95 percent confidence interval:
 -0.05202058  0.25714878
sample estimates:
   prop 1    prop 2
0.7692308 0.6666667
\end{verbatim}
\end{frame}
\begin{frame}[fragile]
\frametitle{Test of Equality for Two Sample Proportions (d)}
\begin{itemize}
\item You would be required to compute the differences in observed proportions.
\item Additionally you will given the \texttt{R} code for one sample procedures. This may or may not be relevant for answering the question.
\end{itemize}
\begin{verbatim}
> prop.test(60,90,0.80)
...
...
X-squared = 9.184, df = 1, p-value = 0.002441
alternative hypothesis: true p is not equal to 0.8
95 percent confidence interval:
 0.5585219 0.7604058
sample estimates:
        p
0.6666667
\end{verbatim}

\end{frame}
%-------------------------------------------------%

\end{document}