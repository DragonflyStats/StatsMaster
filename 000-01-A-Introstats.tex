
%--------------------------------------------------------------------------------------%

\textbf{Population}

As is so often the case in statistics, some words have technical
meanings that overlap with their common use but are not the same.
‘Population’ is one such word. It is often difficult to decide
which population should be sampled. For instance, if we wished to
sample 500 listeners to an FM radio station specializing in music
should the population be of listeners to that radio station in
general, or of listeners to that station’s classical music
programme, or perhaps just the regular listeners, or any one of
many other possible populations that you can construct for
yourself? In practice the population is often chosen by finding
one that is easy to sample from, and that may not be the
population of first choice.



\noindent \textbf{Bivariate Data}
\begin{itemize}
	\item Univariate statistics describes statistics related to one variables.
	\item Bivariate statistics describes statistics related to two variables $X$ and $Y$.
	\item Multivariate statistics describes statistics related to multiple variables (not part of course).
\end{itemize}


%-------------------------------------------------%
% R Code
%
% X = c(10, 15, 20, 25, 30, 35, 40)
% Y = c(11, 19, 34, 52, 58, 81, 109)
% plot(X,Y,pch=18,col="red",font.lab=2,main="Scatter Plot of X and Y")
% cor(X,Y) =0.9830478

%-------------------------------------------------%
\noindent \textbf{Covariance}
Covariance is a strength of the measure of the linear relationship between two variables.
\[ cov(x,Y) = \]


%---------------------------------------------------------------------------- %

\noindent \textbf{Example}
Given that $p_1= 1/4, p_2= 1/8, p_3= 1/8,p_4= 1/3, p_5 = 1/6$ find:

\begin{itemize}
	\item $\displaystyle\sum_{i=1}^{i=n} p_{i} \times x_{i}$
	\item $\displaystyle\sum_{i=1}^{i=n} p_{1} \times x_{i}^2$
\end{itemize}


\end{document}
