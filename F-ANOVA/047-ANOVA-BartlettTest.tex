	\documentclass[a4paper,12pt]{article}
%%%%%%%%%%%%%%%%%%%%%%%%%%%%%%%%%%%%%%%%%%%%%%%%%%%%%%%%%%%%%%%%%%%%%%%%%%%%%%%%%%%%%%%%%%%%%%%%%%%%%%%%%%%%%%%%%%%%%%%%%%%%%%%%%%%%%%%%%%%%%%%%%%%%%%%%%%%%%%%%%%%%%%%%%%%%%%%%%%%%%%%%%%%%%%%%%%%%%%%%%%%%%%%%%%%%%%%%%%%%%%%%%%%%%%%%%%%%%%%%%%%%%%%%%%%%
\usepackage{eurosym}
\usepackage{vmargin}
\usepackage{amsmath}
\usepackage{framed}
\usepackage{graphics}
\usepackage{epsfig}
\usepackage{subfigure}
\usepackage{enumerate}
\usepackage{fancyhdr}

\setcounter{MaxMatrixCols}{10}
%TCIDATA{OutputFilter=LATEX.DLL}
%TCIDATA{Version=5.00.0.2570}
%TCIDATA{<META NAME="SaveForMode"CONTENT="1">}
%TCIDATA{LastRevised=Wednesday, February 23, 201113:24:34}
%TCIDATA{<META NAME="GraphicsSave" CONTENT="32">}
%TCIDATA{Language=American English}

\pagestyle{fancy}
\setmarginsrb{20mm}{0mm}{20mm}{25mm}{12mm}{11mm}{0mm}{11mm}
\lhead{Statistics with \texttt{r}} \rhead{Kevin O'Brien} \chead{Bartlett's Test} %\input{tcilatex}

\begin{document}

\subsection{Bartlett's test for Homogeneity of Variances}
 

Equal variances across samples is called homogeneity of variances. Bartlett's test is used to test if multiple samples have equal variances. 

Some statistical tests, such as the analysis of variance, assume that variances are equal across groups or samples.  The Bartlett test can be used to verify that assumption.

\begin{itemize}
\item The null hypothesis is that each of the samples have equal variance.
\item The alternative hypothesis states that at least one sample has a significantly different variance.
\end{itemize}

%----------------------------------------------------------------------------------------------------------------- %
\end{document}
