\documentclass[]{report}

\voffset=-1.5cm
\oddsidemargin=0.0cm
\textwidth = 480pt

\usepackage{framed}
\usepackage{subfiles}
\usepackage{graphics}
\usepackage{newlfont}
\usepackage{eurosym}
\usepackage{amsmath,amsthm,amsfonts}
\usepackage{amsmath}
\usepackage{color}
\usepackage{enumerate}
\usepackage{amssymb}
\usepackage{multicol}
\usepackage[dvipsnames]{xcolor}
\usepackage{graphicx}
\begin{document}
\section{Probability Distributions: 	Tutorial Sheet }
	\begin{enumerate}	
		\item During the day, cars pass along a point on a remote road at an average rate of one per 20 minutes. Calculate the probability that 
		\begin{enumerate}[(i)]
			\item 		The time between the arrival of 2 cars is greater than 1 hour.
		\item The time between the arrival of 2 cars is less than 10 minutes 
		\item	The time between the arrival of 2 cars is greater than 20 minutes, but less than 40 minutes. 
			
		\end{enumerate}

		\item 	Suppose X has an Exp(λ) distribution.
		\begin{enumerate}[(i)]
			\item 	Derive E(X) and Var(X). [Use the fact that 
			%limx->∞ xk e-λx = 0, for any positive integer k].
			\item 	Using induction, show that the k-th moment of X is given by %k!/λk .
			\item 	      Show that X has the memoryless property.
			\end{enumerate}
		
		
		\item	A die is thrown until either a six is obtained or five rolls are done (a truncated geometric distribution). Let X be the number of rolls.
			\begin{enumerate}[(i)]
				\item 		Define the cumulative distribution function of X.
		\item 	Find the median of X.
	\end{enumerate}
		
		
		
	\end{enumerate}	
	\end{document}