Tutorial 2 Question 5

b) Ignore this question

c) mean

	Midpoint	Frequency	fi.xi
0 to 5	2.5	39	97.5
5 to 10	7.5	67	502.5
10 to 15	12.5	38	475
15 to 20	17.5	26	455
20 to 25	22.5	13	292.5
25 to 30	27.5	8	220
30 to 35	32.5	5	162.5
35 to 40	37.5	4	150
		200	2355
			
		mean 	11.775


Cumulative  	Freq. 	Table	
Less than 5	39	 
Less than 10	106	Median is in here 
Less than 15	144	 
Less than 20	170	 
Less than 25	183	 
Less than 30	191	 
Less than 35	196	 
Less than 40	200	 

Median is in class “5 and under 10”
Width of each class = 5		W
Lower bound of median class = 5		L
Cumulative frequency to class before median class  = 39
Sampel size (n)  = 200			therefore  n/2 = 100

Median
  


Classes	 	 	 	 	  
0 to 5	2.5	39	-9.27	85.9329	3351.3831
5 to 10	7.5	67	-4.27	18.2329	1221.6043
10 to 15	12.5	38	0.73	0.5329	20.2502
15 to 20	17.5	26	5.73	32.8329	853.6554
20 to 25	22.5	13	10.73	115.1329	1496.7277
25 to 30	27.5	8	15.73	247.4329	1979.4632
30 to 35	32.5	5	20.73	429.7329	2148.6645
35 to 40	37.5	4	25.73	662.0329	2648.1316
					13719.88


 
 
 
 
 
 
Variance	 


Tutorial 2 Question 6

Both the mean and median time for supplier 1 is better (as in lower)
Also the measures of variability are lower. This means that the delivery times are more consistent.
We go with supplier 1








Tutorial 2 - Question 7

Firstly , sort all the data into ascending order

1  	1 	2 	2 	2 	2 	3 	3 	3	 3
3  	3 	4 	4 	5 	5 	5	 5 	5 	5



  	6 	6 	7 	7 	8 	8 	8 	8 	8
9 	10 	11 	15 	15 	16 	17 	19 	20 	40

Next , get all the quartiles

Q1 :  average of 10th and 11th elements		(3 +3)/2 	=	3
Q2 :  average of 20th and 21st elements		(5 +6)/2 	=	5.5
Q3 :  average of 30th and 31st elements		(8 +9)/2 	=	8.5

The IQR = Q3 – Q1 = 8.5 – 3 = 5.5

Now lets find the upper and lower fences

Lower fence  = Q1 – 1.5xIQR	 = 3 – 8.25 = -5.25		
(makes no sense to use negative number , so lets use 0)

Upper fence = Q3 + 1.5x IQR = 8.5 + 8.25 = 16.75

Now let’s find the outliers
Any values less than the lower fence? 	No
Any value higher than the upper fence? 	Yes -  four 17, 19 , 20 and 40

Where do we draw the “whiskers” : lowest value not an outlier (1) and highest value not an outlier  (16)






Our boxplot is therefore

 

Report
Mention all the important sample values (including range)
Mention Skew / Symmetry
Discuss the outliers – any idea where they come from?

