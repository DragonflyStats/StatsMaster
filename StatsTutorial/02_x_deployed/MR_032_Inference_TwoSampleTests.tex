\documentclass[a4paper,12pt]{article}
%%%%%%%%%%%%%%%%%%%%%%%%%%%%%%%%%%%%%%%%%%%%%%%%%%%%%%%%%%%%%%%%%%%%%%%%%%%%%%%%%%%%%%%%%%%%%%%%%%%%%%%%%%%%%%%%%%%%%%%%%%%%%%%%%%%%%%%%%%%%%%%%%%%%%%%%%%%%%%%%%%%%%%%%%%%%%%%%%%%%%%%%%%%%%%%%%%%%%%%%%%%%%%%%%%%%%%%%%%%%%%%%%%%%%%%%%%%%%%%%%%%%%%%%%%%%
\usepackage{eurosym}
\usepackage{vmargin}
\usepackage{amsmath}
\usepackage{framed}
\usepackage{multicol}
\usepackage{graphics}
\usepackage{epsfig}
\usepackage{subfigure}
\usepackage{enumerate}
\usepackage{fancyhdr}

\setcounter{MaxMatrixCols}{10}
%TCIDATA{OutputFilter=LATEX.DLL}
%TCIDATA{Version=5.00.0.2570}
%TCIDATA{<META NAME="SaveForMode"CONTENT="1">}
%TCIDATA{LastRevised=Wednesday, February 23, 201113:24:34}
%TCIDATA{<META NAME="GraphicsSave" CONTENT="32">}
%TCIDATA{Language=American English}

\pagestyle{fancy}
\setmarginsrb{20mm}{0mm}{20mm}{25mm}{12mm}{11mm}{0mm}{11mm}
\lhead{MathsResource} \chead{Two Sample Inference Procedures} \rhead{Tutorial Sheet} %\input{tcilatex}
\begin{document}

\begin{enumerate}
\item
An exercise physiologist wants to determine if several short bouts of exercise provide the same benefit for cardiovascular fitness as one long bout of exercise. 

60 volunteers are randomly assigned to group 1 and do standardized aerobic exercise on a stationary bicycle for 30 minutes once a day, 5 days a week. 40 volunteers are randomly assigned to group 2 and do the same exercise for 10 minutes, 3 times a day, 5 days a week. Cardiovascular fitness was measured by VO2 max (maximum oxygen consumption while exercising). 

\begin{description}
	\item[Group 1] The mean change in VO2 after 12 weeks of exercise was 2.1 for group 1 with a standard deviation of 1.7.
	\item[Group 2] The mean change in VO2 after 12 weeks of exercise was 0.7 for group 2 with a standard deviation of 1.4. 
\end{description}

\noindent Test the hypothesis that there is no significant difference between two groups are the same. (You may assume that both populations have equal variance.)

\begin{enumerate}[(a)]
	\item Formally state your null and alternative hypotheses.
	\item Compute the test statistic.
	\item Discuss your conclusion to this test, supporting your statement with reference to appropriate values.
\end{enumerate}

\item To learn about the feeding habits of bats, 22 were tagged and tracked by
radio. The distances (in metres) flown between feedings were noted, and the
following summary statistics were obtained:


\begin{center}
\begin{tabular}{|c|c|c|} \hline 
Sample & Female bats & Male bats \\ \hline
Sample Size & 12 & 10 \\ \hline
Mean & 180 & 140 \\ \hline
Standard deviation & 54 & 46 \\ \hline
\end{tabular}
\end{center}

You may assume that the variances of the distances flown between feeding is the same for the populations of both males and female bats.


\begin{enumerate}[(a)]
\item Test the Hypothesis that the variances of the distances flown between feeding is the same for the populations of both males and female bats.
\item Construct a 95\% confidence interval for the difference in distances flown between feeding is the same for the populations of both males and female bats. Interpret your findings. 
\item Using 5\% significance, test the Hypothesis that the mean distance flown between feedings is the same for the populations of both males and female bats.
\end{enumerate}
%====================================================%
 \item  A study was made of children who were hospitalized as a result of a car accident. 280 of the children were not wearing seat belts and 98 of these were seriously injured. 130 children wore seat belts and 26 were seriously injured. 
\begin{enumerate}[(a)]
\item Test the hypothesis that the rate of serious injury is the same for children who wear a seat belt or not. Clearly state your null and alternative hypotheses and your conclusion. Use a significance level of 5\%.
\end{enumerate}
%%%%%%%%%%%%%%%%%%%%%%%%%%%%%%%%%%%%%%%%%%%5
\item A fruit grower wishes to test a new spray that a manufacturer claims will
reduce the amount of fruit lost due to damage by a certain insect. To test
the claim, the grower sprays 150 trees with the new spray and 120 other
trees with the standard spray. The yield of fruit was measured, in kg, for
each tree. Summary statistics were as follows. You may assume equal variance for the populations.

\begin{center}
\begin{tabular}{|c|c|c|} \hline 
                       & New spray & Standard spray\\ \hline 
Sample yield per tree  & 280       & 255\\ \hline 
Sample variance        & 650      & 640\\ \hline 
\end{tabular}
\end{center}
\begin{enumerate}[(a)]
\item  Construct a 95\% confidence interval for the difference between the mean yields for the two sprays and interpret your findings. 
\item  Test the hypothesis that the new spray is effective in reducing damage. Clearly state your null and alternative hypotheses and your conclusion. Use a significance level of 5\%.
\end{enumerate}

\item Two samples of students are randomly selected from two IT training companies; Echelon and Deltatech. The mean and the standard deviation of the number of marks obtained in a well known IT competency exam by both sets of students are described below:\\

\begin{center}
\begin{tabular}{|c|c|c|c|}
  \hline
	& Group Size &	Sample Mean &	Std. Deviation\\ \hline
DeltaTech	& 15	& 242	& 40 \\
Echelon	& 14	& 225	& 28 \\
  \hline
\end{tabular}
\end{center}

%Calculate a 95\% confidence interval for the difference between the mean number of marks obtained by males and females in the population of school leavers as a whole.
%(7 marks)

Test the hypothesis that Echelon students and DeltaTech students, on average, obtain the same mark in the IT certification exam. Use a significance level of $5\%$. You may assume that any required assumptions have been validated.
% State your hypotheses clearly. What is the significance level of this test?

\smallskip 
\begin{enumerate}[(a)]

\item Formally state the null and alternative hypotheses.
\item Compute the Test Statistic.
\item State the appropriate Critical Value for this hypothesis test.
\item Discuss your conclusion to this test, supporting your statement with reference to appropriate values.
\item Construct a 95\% confidence interval for the difference between the mean exam scores and interpret your findings. 
\end{enumerate}
\end{enumerate}
\end{document}
