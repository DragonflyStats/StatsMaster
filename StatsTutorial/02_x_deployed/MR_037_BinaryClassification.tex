
\documentclass[a4paper,12pt]{article}
%%%%%%%%%%%%%%%%%%%%%%%%%%%%%%%%%%%%%%%%%%%%%%%%%%%%%%%%%%%%%%%%%%%%%%%%%%%%%%%%%%%%%%%%%%%%%%%%%%%%%%%%%%%%%%%%%%%%%%%%%%%%%%%%%%%%%%%%%%%%%%%%%%%%%%%%%%%%%%%%%%%%%%%%%%%%%%%%%%%%%%%%%%%%%%%%%%%%%%%%%%%%%%%%%%%%%%%%%%%%%%%%%%%%%%%%%%%%%%%%%%%%%%%%%%%%
\usepackage{eurosym}
\usepackage{vmargin}
\usepackage{amsmath}
\usepackage{framed}
\usepackage{multicol}
\usepackage{graphics}
\usepackage{epsfig}
\usepackage{subfigure}
\usepackage{enumerate}
\usepackage{fancyhdr}

\setcounter{MaxMatrixCols}{10}
%TCIDATA{OutputFilter=LATEX.DLL}
%TCIDATA{Version=5.00.0.2570}
%TCIDATA{<META NAME="SaveForMode"CONTENT="1">}
%TCIDATA{LastRevised=Wednesday, February 23, 201113:24:34}
%TCIDATA{<META NAME="GraphicsSave" CONTENT="32">}
%TCIDATA{Language=American English}

\pagestyle{fancy}
\setmarginsrb{20mm}{0mm}{20mm}{25mm}{12mm}{11mm}{0mm}{11mm}
\lhead{MathsResource} \chead{Binary Classification} \rhead{ Tutorial Sheet} %\input{tcilatex}
\begin{document}%--------------------%
\begin{enumerate}

\item % \textbf{Binary Classification Question (Decision Theory)}\\
Consider the following confusion matrix.(The total number of experiments is 10,000)
\begin{center}
	\begin{tabular}{|c|c|c|}
		\hline 
		& 	Predict Negative & Predict Positive \\ 
		\hline 
		Observed Negative	& 9700 & 80 \\ 
		\hline 
		Observed Positive	& 100  &  120 \\ 
		\hline 
	\end{tabular} 
\end{center}
Compute the following measurements
\begin{multicols}{2}
	\begin{enumerate}
		\item Accuracy \item Recall \item Precision \item The F-measure
	\end{enumerate}
\end{multicols}
\item Explain the class imbalance problem in binary 
classification procedures. Explain how this would advserely affect some
performance measures for binary classification procedures.
\item %  Binary Classification (4 Marks)
For following binary classification outcome table (i.e. confusion matrices), calculate the following appraisal metrics.

\begin{multicols}{2}
	\begin{enumerate}[(a)]
		\item Accuracy;
		\item Recall
		\item Precision;
		\item F-measure.
	\end{enumerate}
\end{multicols}
\noindent \textbf{Confusion Matrix 1} \smallskip
\begin{center}
	
	\begin{tabular}{|c||c|c|}
		\hline 
		& Predict Negative & Predict Positive \\ \hline  \hline 
		Observed Negative & 9560 &  100 \\ \hline 
		Observed Positive & 270 & 70 \\ \hline 
	\end{tabular} 
\end{center}
\noindent \textbf{Confusion Matrix 2}
\begin{center}
	\begin{tabular}{|c||c|c|}
		\hline 
		& Predict Negative & Predict Positive \\ \hline  \hline 
		Observed Negative & 9500 &  320 \\ \hline 
		Observed Positive & 20 & 160 \\ \hline 
	\end{tabular} 
\end{center}


\item A confusion matrix is a table that is often used to describe the performance of a binary classification procedure on a set of test data for which the true values are known.
\begin{center}
\begin{tabular}{|c|c|c|}
\hline  & Predicted Negative & Predicted Positive \\ 
\hline Observed Negative & True Negative & False Positive \\ 
\hline Observed Positive & False Negative & True Positive \\ 
\hline 
\end{tabular} 
\end{center}
\begin{enumerate}[(a)]
\item  What is the F-measure? Explain its function and how it is computed.\\ 
\noindent \textit{ Hint: F-score = $(2 \times P \times R) / (P + R)$.}

\item Explain the Class Imbalance problem for binary classification procedures. Explain how this would adversely affect some performance measures. You may refer to the confusion matrix below in your answer.



 \item %  Binary Classification (4 Marks)
 For the confusion matrix below, calculate the following appraisal metrics.

	\begin{enumerate}[(i)]
		\item (1 Mark) Accuracy,
		\item (1 Mark) Recall,
		\item (1 Mark) Precision,
		\item (1 Mark) F-measure.
	\end{enumerate}


 \begin{center}
	
	\begin{tabular}{|c||c|c|}
		\hline 
		& Predict Negative & Predict Positive \\ \hline  \hline 
		Observed Negative & 9260 &  220 \\ \hline 
		Observed Positive & 180 & 340 \\ \hline 
 	\end{tabular} 
 \end{center}

\item (2 Marks) Define Specificity and Sensitivity. You make reference to previous answers.
\item  What is a ROC curve? Explain its function, how it is determined, and the means of interpreting the curve. Support your answer with a sketch.
\end{enumerate}

\newpage

\item A confusion matrix is a table that is often used to describe the performance of a binary classification procedure on a set of test data for which the true values are known.
\begin{center}
\begin{tabular}{|c|c|c|}
\hline  & Predicted Negative & Predicted Positive \\ 
\hline Observed Negative & True Negative & False Positive \\ 
\hline Observed Positive & False Negative & True Positive \\ 
\hline 
\end{tabular} 
\end{center}
\begin{enumerate}[(i)]



\item Define each of the following appraisal metrics for binary classification:
\begin{enumerate}[(i)]
    \item (1 Mark) Accuracy,
    \item (1 Mark) Precision,
    \item (1 Mark) Recall.
    \end{enumerate}

\item (2 Marks) What is the F-measure? Explain its function and how it is computed.\\ 
\noindent \textit{ Hint: F-score = $(2 \times P \times R) / (P + R)$.}
%\item[iii.](2 Marks) Define Specificity and Sensitivity. You make reference to previous answers.
%\item[iv.](3 Marks) What is a ROC curve? Explain its function, how it is determined, and the means of interpreting the curve. Support your answer with a sketch.
\end{enumerate}


% \item %  Binary Classification (4 Marks)
% For the confusion matrix below, calculate the following appraisal metrics.
%
%	\begin{enumerate}[(i)]
%		\item (1 Mark) Accuracy,
%		\item (1 Mark) Recall,
%		\item (1 Mark) Precision,
%		\item (1 Mark) F-score.
%	\end{enumerate}
%
%
% \begin{center}
%	
%	\begin{tabular}{|c||c|c|}
%		\hline 
%		& Predict Negative & Predict Positive \\ \hline  \hline 
%		Observed Negative & 9300 &  100 \\ \hline 
%		Observed Positive & 200 & 400 \\ \hline 
% 	\end{tabular} 
% \end{center}

\medskip


\end{enumerate}
\end{document}
