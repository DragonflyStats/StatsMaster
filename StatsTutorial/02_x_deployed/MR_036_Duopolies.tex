\documentclass[a4paper,12pt]{article}
%%%%%%%%%%%%%%%%%%%%%%%%%%%%%%%%%%%%%%%%%%%%%%%%%%%%%%%%%%%%%%%%%%%%%%%%%%%%%%%%%%%%%%%%%%%%%%%%%%%%%%%%%%%%%%%%%%%%%%%%%%%%%%%%%%%%%%%%%%%%%%%%%%%%%%%%%%%%%%%%%%%%%%%%%%%%%%%%%%%%%%%%%%%%%%%%%%%%%%%%%%%%%%%%%%%%%%%%%%%%%%%%%%%%%%%%%%%%%%%%%%%%%%%%%%%%
\usepackage{eurosym}
\usepackage{vmargin}
\usepackage{amsmath}
\usepackage{framed}
\usepackage{multicol}
\usepackage{graphics}
\usepackage{epsfig}
\usepackage{subfigure}
\usepackage{enumerate}
\usepackage{fancyhdr}

\setcounter{MaxMatrixCols}{10}
%TCIDATA{OutputFilter=LATEX.DLL}
%TCIDATA{Version=5.00.0.2570}
%TCIDATA{<META NAME="SaveForMode"CONTENT="1">}
%TCIDATA{LastRevised=Wednesday, February 23, 201113:24:34}
%TCIDATA{<META NAME="GraphicsSave" CONTENT="32">}
%TCIDATA{Language=American English}

\pagestyle{fancy}
\setmarginsrb{20mm}{0mm}{20mm}{25mm}{12mm}{11mm}{0mm}{11mm}
\lhead{MathsResource} \chead{Operations Research} \rhead{ Tutorial Sheet} %\input{tcilatex}
\begin{document}%--------------------%
\begin{enumerate}

%-----------------%
\item %\textbf{Monopoly - Introduction to Duopolies}\\% Bonanno
Let the inverse demand function and the cost function be given by
$P = 50 − 2Q$ and $C = 10 + 2q$
respectively, where Q is total industry output and q is the firm’s output. 
Cosnider the case of a monopoly ($Q=q$). Determine the optimal output level, market price, and profit for the firm.
%-----------------%
\item % \textbf{Monopoly Question (MS4315 Mark Burke)}\\
The costs incurred by a firm in a production period are
$c = 100 + 2x$
where x is the number of items produced in that period. The items each sell
at a price of
\[P(x) = 10 - \frac{x}{50}. \]
Find the level of production that maximises the firm’s profits
when the firm has a monopoly.

%=============================%
\item \textbf{Cartel Question}\\
Consider an industry with two firms. Firms are identical and produce an
homogenous product. Firms have to select outputs (capacity) in order to maximize
profits. Each firm knows its own total cost of production, the total cost of production of
the competitor and the industry demand.
\\
The following data are known by both firms and describe the industry
situation:
\begin{itemize}
\item P = 140 - (Q1+Q2) \textit{(industry demand)}
\item TC1 = 20Q1 \textit{(total cost of firm 1),}
\item TC2 = 20Q2 \textit{(total cost of firm 2).}
\end{itemize}
Suppose that both firms agree to form a cartel. The goal of the cartel is to set the
industry output at a level that maximizes industry profits. A rule governing the cartel
behavior specifies how the industry output and profits must be shared among the cartel
members.


\item Consider the asymmetric duopoly game: Firm i (where $i = 1, 2$) produces $x_i$
        items
        at a cost of
        \[C(x_i) = x_i + 100i.\]
        The items each sell at a price of
        \[p(x_1, x_2) = 5 - \frac{x_1 + x_2}{500}.\]

        \begin{enumerate}[(a)]
        \item Find the equilibrium of the game if it is played as a Cournot game.
        \item Find the equilibrium if it played as a Stackelberg game with Firm 1 as
        leader. 
        \item Contrast and comment on the two solutions. 
        \end{enumerate}
        \medskip
%--------------------%
%===============================================%


\item Consider the asymmetric duopoly game: Firm $i$ (where $i = 1, 2$) produces $x_i$
        items
        at a cost of
        \[C(x_i) = x_i + 100i.\]
        The items each sell at a price of
        \[p(x_1, x_2) = 5 - \frac{x_1 + x_2}{500}.\]

        \begin{enumerate}[(a)]
        \item Find the equilibrium of the game if it is played as a Cournot game.
        \item Find the equilibrium if it played as a Stackelberg game with Firm 1 as
        leader. \end{enumerate}

\end{enumerate}
\end{document}
