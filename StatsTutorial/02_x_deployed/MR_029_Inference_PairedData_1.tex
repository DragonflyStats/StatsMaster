\documentclass[a4paper,12pt]{article}
%%%%%%%%%%%%%%%%%%%%%%%%%%%%%%%%%%%%%%%%%%%%%%%%%%%%%%%%%%%%%%%%%%%%%%%%%%%%%%%%%%%%%%%%%%%%%%%%%%%%%%%%%%%%%%%%%%%%%%%%%%%%%%%%%%%%%%%%%%%%%%%%%%%%%%%%%%%%%%%%%%%%%%%%%%%%%%%%%%%%%%%%%%%%%%%%%%%%%%%%%%%%%%%%%%%%%%%%%%%%%%%%%%%%%%%%%%%%%%%%%%%%%%%%%%%%
\usepackage{eurosym}
\usepackage{vmargin}
\usepackage{amsmath}
\usepackage{framed}
\usepackage{multicol}
\usepackage{graphics}
\usepackage{epsfig}
\usepackage{subfigure}
\usepackage{enumerate}
\usepackage{fancyhdr}

\setcounter{MaxMatrixCols}{10}
%TCIDATA{OutputFilter=LATEX.DLL}
%TCIDATA{Version=5.00.0.2570}
%TCIDATA{<META NAME="SaveForMode"CONTENT="1">}
%TCIDATA{LastRevised=Wednesday, February 23, 201113:24:34}
%TCIDATA{<META NAME="GraphicsSave" CONTENT="32">}
%TCIDATA{Language=American English}

\pagestyle{fancy}
\setmarginsrb{20mm}{0mm}{20mm}{25mm}{12mm}{11mm}{0mm}{11mm}
\lhead{MathsResource} \chead{Inference: Paired Data Procedures} \rhead{Tutorial Sheet} %\input{tcilatex}
\begin{document}




\begin{enumerate}

 
\item A 12-week plyometric-training programme was devised to improve standing long jump performance for track and field athletes.

A group of ten athletes are selected to take part. To test whether this training improves performance, the athletes are tested for their long jump performance
 before they undertake a plyometric-training programme and then again at the end of the programme 
(i.e., the dependent variable is ``standing long jump performance", and the two related groups are the standing long jump values ``before" and ``after" the 12-week plyometric-training programme).

% latex table generated in R 3.4.2 by xtable 1.8-2 package
% Thu Nov 02 12:44:35 2017
\begin{table}[ht]
\centering
\begin{tabular}{|r|r|r|}
  \hline
 Athlete & Before & After \\ 
  \hline
1 & 225 & 239 \\ 
  2 & 214 & 223 \\ 
  3 & 211 & 221 \\ 
  4 & 210 & 221 \\ 
  5 & 203 & 217 \\ 
  6 & 219 & 224 \\ 
  7 & 200 & 214 \\ 
  8 & 202 & 211 \\ 
  9 & 224 & 226 \\ 
  10 & 209 & 221 \\ 
   \hline
\end{tabular}
\end{table}

\begin{enumerate}[(a)]
\item  Briefly explain the difference between paired samples and independent samples.
\item Compute the mean of the case-wise differences.
\item Compute the standard deviation of the case-wise differences.
\item By performing a suitable hypothesis test, what is your conclusion about the effectiveness of this training program. 
\item Calculate a 95\% confidence interval for the mean of the case-wise differences.

\end{enumerate}

\item A microbiologist measures the total growth in 24 hours of two strains of a germ culture in the same petri dish. Nine identical specimens are prepared. The growth rate for both each specimen, with the growth rate for both specimens, is tabulated below.
	
	\begin{center}
		\begin{tabular}{|c|c|c|} \hline 
			Specimen &	Strain 1	&	Strain 2	\\ \hline \hline
			1 & 212 & 224 \\ \hline
			2 & 234 & 231 \\ \hline
			3 & 214 & 209 \\ \hline
			4 & 236 & 243 \\ \hline
			5 & 221 & 231 \\ \hline 
			6 & 212 & 216 \\ \hline
			7 & 202 & 213 \\ \hline 
			8 & 210 & 216 \\ \hline
			9 & 248 & 242 \\ \hline
		\end{tabular} 
	\end{center}
	\noindent At a significance level of 5\%, is there sufficient evidence to state that there is any difference in growth rates between the two strains.
	
	
	% State your hypotheses clearly. What is the significance level of this test?

	
	\begin{enumerate}[(a)]
		\item Formally state the null and alternative hypotheses.
		\item Compute the mean and standard deviation of the case-wise differences.
		\item Compute the test statistic.
		\item State the appropriate critical value for this hypothesis test. 
		\item  Discuss your conclusion to this test, supporting your statement with reference to appropriate values.
	\end{enumerate}
\end{enumerate}


\medskip

\end{document}
