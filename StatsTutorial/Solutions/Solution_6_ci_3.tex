

% section 4
In an election campaign, a campaign manager requests that a sample of votes be polled to determine public support for a candidate. In a sample of 150 votes 72 expressed plans to support the candidate.

What is the point estimate of the proportion of the voters who will support the candidate in the election?

contruct and interpret a 95\% confidence interval for the proportion of votes in the population that support the candidate.

given the confidennce interval, is the campaign manager justified in feeling confident that the candidate has at least 50% support

\[S.E. (\hat{P}) = \sqrt{{\hat{p}(1-\hat{p} \over n}}\]
\newpage

\section{CI for Proportion: Example (1)}

\begin{itemize}
\item $\hat{p}  = 0.62$
\item Sample Size $n=250$
\item Confidence level $1-\alpha$ is $95\%$
\end{itemize}



\section{CI for Proportion: Example (2)}

\begin{itemize}
\item First, lets determine the quantile.
\item The sample size is large, so we will use the Z distribution.
\item (Alternatively we can uses the $t-$ distribution with $\infty$ degrees of freedom.
\end{itemize}

