\subsection{Computing a Confidence Interval - Worked Example}
\begin{itemize}
\item Suppose that 9 bags of salt granules are selected from the supermarket
shelf at random and weighed. The weights in grams are \[812.0, 786.7, 794.1,
791.6, 811.1, 797.4, 797.8, 800.8, 793.2.\] 
\item Give a 95\% confidence interval for the
mean of all the bags on the shelf. Assume the population is normal.


Here we have a random sample of size n = 9. The mean is 798.30. The sample
variance is $s^2 = 72.76$, which gives a sample standard deviation $s = 8.53$.

\item The upper 2.5\% point of the Student's $t$ distribution with n-1 (= 9-1 = 8) degrees of freedom is 2.306.

\item The 95\% confidence interval is therefore from 
\[(798.30 - 2.306 \times (8.53/\sqrt{9}),\;\; 798.30 + 2.306 \times (8.53/\sqrt{9})\]
which is\\
\[(798.30 - 6.56, 798.30 + 6.56) = (791.74, 804.86)\]
\item \textit{It is sometimes more useful to write this as $798.30 \pm 6.56$.}

\item Note that even if we do not assume the population is normal (that assumption is
never really true) the Central Limit Theorem might suggest that the confidence interval
is nearly right. 
\item A larger confidence would increase the length of the interval, so we
trade off increased certainty of coverage against a longer interval
\end{itemize}

