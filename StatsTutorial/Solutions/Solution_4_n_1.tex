\documentclass[]{report}

\voffset=-1.5cm
\oddsidemargin=0.0cm
\textwidth = 480pt

\usepackage{framed}
\usepackage{subfiles}
\usepackage{graphics}
\usepackage{newlfont}
\usepackage{eurosym}
\usepackage{enumerate}
\usepackage{amsmath,amsthm,amsfonts}
\usepackage{amsmath}
\usepackage{color}
\usepackage{amssymb}
\usepackage{multicol}
\usepackage[dvipsnames]{xcolor}
\usepackage{graphicx}
\begin{document}
%%- May 2013 Question 3 
\section*{Normal Distribution Question}

\noindent \textbf{Important Information from the Question}
\begin{itemize}
	\item Normal Mean $\mu$ = 1000 units
	\item Normal Standard Deviation $\sigma$ = 200 units 
\end{itemize}
\smallskip
\noindent \textbf{Objectives}
Compute the following : 
\begin{itemize}
	\item $P(X \geq 1400 )$ More than 1400
	\item $P(X \leq 500)$ Less than 500
\end{itemize}
\smallskip

\noindent \textbf{Part 1 -  More than 1400}

Firstly compute the Z-score for 1400.

\[ Z_{1400} =  \frac{X - \mu}{\sigma} = \frac{1400 - 1000}{200} = \frac{400}{200} = 2  \]

So the \textbf{Z-score} in this case is 2.

This much we can say
\[P(X \geq 1400) = P(Z \geq 2)\]

$P(Z \geq 2)$ can be determined using statistical tables.
Depending on which statistical tables you are using, you will get one of the following answers. (Note the 
second and third statements are examples of complementary probabilities.)
\begin{itemize}
	\item $P (0 \leq Z \leq 2)$ = 0.4775
	\item $P ( Z \leq 2)$ = 0.9775
	\item $P ( Z \geq 2)$ = 0.0225
\end{itemize}
The last expression is useful here. Recall that $P(X \geq 1400) = P(Z \geq 2)$. Therefore

\[P(X \geq 1400) = 0.0225\]
\end{document}
