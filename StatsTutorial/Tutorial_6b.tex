\documentclass[]{report}

\voffset=-1.5cm
\oddsidemargin=0.0cm
\textwidth = 480pt

\usepackage{framed}
\usepackage{subfiles}
\usepackage{graphics}
\usepackage{newlfont}
\usepackage{eurosym}
\usepackage{amsmath,amsthm,amsfonts}
\usepackage{amsmath}
\usepackage{multirow}
\usepackage{enumerate}
\usepackage{color}
\usepackage{amssymb}
\usepackage{multicol}
\usepackage[dvipsnames]{xcolor}
\usepackage{graphicx}
\begin{document}

\subsection{CI for Difference in Two Means}
A research company is comparing computers from two different companies, X-Cel and Yellow, on the basis of energy consumption per hour. Given the following data, compute a $95\%$ confidence interval for the difference in energy consumption.
\begin{center}
\begin{tabular}{|c|c|c|c|}
\hline
Type & sample size & mean & variance \\ \hline
X-cel & 17 & 5.353 & 2.743 \\ \hline
Yellow & 17 & 3.882 & 2.985 \\ \hline
\end{tabular}
\end{center}
Remark: It is reasonable to believe that the variances of both groups is the same. Be mindful of this.

%---------------------------------------------------------%

\subsection{CI for Difference in Two Means (Example)}
From the previous example (comparing X-cel and Yellow) lets compute a 95\% confidence interval when the sample sizes are $n_x=10$ and $n_y=12$ respectively. (Lets assume the other values remain as they are.)
\begin{center}
\begin{tabular}{|c|c|c|c|}
\hline
Type & sample size & mean & variance \\ \hline
X-cel & 10 & 5.353 & 2.743 \\ \hline
Yellow & 12 & 3.882 & 2.985 \\ \hline
\end{tabular}
\end{center}
The point estimate $\bar{x} - \bar{y}$ remains as 1.469. Also we require that both samples have equal variance. As both $X$ and $Y$ have variances at a similar level, we will assume equal variance.

%---------------------------------------------------------%


\subsection{Computing the Confidence Interval}
\begin{itemize} \item Pooled variance $s^2_p$ is computed as:

\[ s^2_p = \frac{(10-1)2.743 + (12-1)2.985}{(10-1) + (12-1)}  = \frac{57.52}{20} = 2.87\]

\item Standard error for difference of two means is therefore

\[ S.E.(\bar{x}-\bar{y}) = \sqrt{  2.87 \left({1\over 10}+{1\over 12} \right)} = 0.726 \]

\item The aggregate sample size is small i.e. 22. The degrees of freedom is $n_x+n_y-2 = 20$.
From Murdoch Barnes tables 7, the quantile for a $95\%$ confidence interval is 2.086.

\item The confidence interval is therefore
\[ 1.469  \pm (2.086 \times 0.726) = 1.4699 \pm 1.514 =  (-0.044, 2.984 )  \]
\end{itemize}

\end{document}
