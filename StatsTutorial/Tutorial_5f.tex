\documentclass[]{report}

\voffset=-1.5cm
\oddsidemargin=0.0cm
\textwidth = 480pt

\usepackage{framed}
\usepackage{subfiles}
\usepackage{graphics}
\usepackage{newlfont}
\usepackage{eurosym}
\usepackage{amsmath,amsthm,amsfonts}
\usepackage{amsmath}
\usepackage{color}
\usepackage{amssymb}
\usepackage{multicol}
\usepackage[dvipsnames]{xcolor}
\usepackage{graphicx}
\begin{document}



\section{Two Sample Mean - Inference Worked Example}
%%Question 3 Part b : Confidence interval for the difference in means of two samples.
%% b) 
The mean and standard deviation of the weights of a sample of Irish students according to sex are given below

\begin{center}
\begin{tabular}{|c|c|c|c|}\hline
& Number & Mean & Std. Dev. \\ \hline
Male &100&75&10 \\ \hline
Female&110& 66&8 \\  \hline

\end{tabular}
\end{center}


Calculate a 95\% confidence interval for the difference between the mean weight of all male
students and the mean weight of all female students.(7 males)

%%General Structure of a Confidence Interval



%------------------------------------------------------------- %
\noindent \textbf{Observed difference}

\begin{itemize}
\item let X denote the weights of male students    X= 75
\item let Y denote the weights of female students  Y= 66
\item The difference in the mean of weights X-Y= 9
\end{itemize}
%------------------------------------------------------------- %
\noindent \textbf{Quantile}

\begin{itemize}
\item Large sample (both groups are greater than 30).
\item Population variance is unknown.
\item Use t distribution with  degrees of Freedom.
\item Confidence level is 95\%. Therefore significance levels is 5%.
\item Confidence intervals are always two tailed procedures.
\item Column = k = $\frac{0.01}{2}= 0.005$
\end{itemize}

\begin{framed}
\noindent \textbf{Murdoch Barnes table 7}

\begin{itemize}
\item Row: df =  
\item Column = 0.005

\end{itemize}

Therefore the quantile is =  2.576 
\end{framed}

Stardard Error

%------------------------------------------------------------- %





Confidence Interval is therefore

\[99\% CI = 9 \pm (2.576 \times 1.256) \]


Part (ii)
Based on this confidence interval, test the hypothesis that on average male students are 6kgs heavier than female students.

State your hypotheses clearly. What is the significance level of this test?   (3 marks)

H0:X-y= 6




Since  we do not reject the null hypothesis at a significance level of 1\%.





Part (ii)
Based on this confidence interval, test the hypothesis that on average male students are 6kgs heavier than female students.
State your hypotheses clearly. What is the significance level of this test?   (3 marks)

\begin{description}

\item[Null :]                 True difference between means is zero

\item[Alternative:]         True difference between means is not zero
\end{description}
Or alternatively

\begin{description}
\item[Null :]              H0:X-Y= 0   True difference between means is zero

\item[Alternative:]         True difference between means is not zero
\end{description}



\begin{itemize}
\item Since the null value, 0 is inside the confidence interval  .

\item This means that the true difference of means could be zero.

\item We do not reject the null hypothesis at a significance level of 1\%.

\end{itemize}


%----------------------------------------------------------------------------------------------------------------------SLR%
\noindent \textbf{Test Statistic}

Remember the general structure of a test statistic

\begin{framed}
\[TS =  \frac{\mbox{Observed Value-Null Value}}{\mbox{Std. Error}} \]
\end{framed}


\begin{itemize}
\item Standard Error\[S.E.(D) = \frac{S_D}{n}=2.1378= 0.7555\]

\item Test statistis is a t random variable

\itemTest Statistic\[x-0S.E.(D)=2 - 00.755= 2.649\]
\end{itemize}

\noindent \textbf{Critical values}

\begin{itemize}
\item The sample size (n=8) is small ($n \leq 30$). Use t distribution with n-1 degrees of Freedom.
\item The test is one tailed.  k= 1  ( why?  ">" symbol in the alternative hypothesis).
\item Murdoch Barnes table 7
\item Column:  $\alpha/k = 0.05/1= 0.05$
\item Row: df = 7
\item Critical value =  1.895    
\end{itemize}
\noindent \textbf{Decision rule}

\begin{framed}
\noindent Is the absolute value of the test statistic value greater than the critical value?
\begin{itemize}
\item If Yes: we reject the null hypothesis

\item If No: We fail to reject the null hypothesis. (not enough evidence)
\end{itemize}
\end{framed}

Here TS = 2.64  is greater than CV = 1.895.

\noindent \textbf{Decision rule}
We reject the null hypothesis. Students do put on weight during college. 




\section{Worked Example}
Calculate a 99\% confidence interval for the difference between the proportion of all Irish having access to the
Internet and the proportion of all Spaniards having access to the internet.  (4 marks)



\noindent \textbf{Standard Error for confidence interval}

\[\frac{p1(1 -p1)}{n1}+ \frac{p2(1 -p2)}{n2}\]
\[=\frac{0.750.25}{1000}+ \frac{0.700.30}{2000}     =  0.017103\]

\noindent \textbf{Quantile for a 99\% confidence interval}
\begin{itemize}
\item significance level  =1\%
\itemnumber of tails = 2
\itemdegrees of freedom = 
\itemquantile = 2.576 
\end{itemize}



99\% Confidence Interval for difference of two proportions

%================================================================= %


Useful pieces of information


Sample size  n=100


\begin{description}
\item[Part i]
Calculate the equation of the least square regression line and interpret the value of the slope.

\item[part ii]

Using this regression model, estimate the mean weight of individuals who are 3 metres tall.
\item[part iii]

Is such an estimate reliable?  briefly explain why.

No it is not reliable. Consider the range of values of the x predcitro variable
\end{description}



\begin{itemize}
\item SE = $\sqrt{ [p_1 \times (1 - p_1) / n_1] + [p_2 \times (1 - p_2) / n_2] } $
\item SE = $\sqrt{ [0.40 \times 0.60 / 400] + [0.30 \times 0.70 / 300] } $
\item SE  = $\sqrt{[ (0.24 / 400) + (0.21 / 300) ]}$ = $\sqrt{(0.0006 + 0.0007)}$ = $\sqrt{0.0013} = 0.036$
\end{itemize}

\section{Worked Example : Two Sample Test} Deltatech software has 350 programmers divided into two groups with 200 in Group A and 150 in Group B. In order to compare the efficiencies of the two groups, the programmers are observed for  1 day.

\begin{itemize}
\item  The 200 programmers in Group A averaged 45.2 lines of code with a standard deviation of 8.4.

\item The 150 programmers in Group B averaged 42.7 lines of code with a standard deviation of 5.2.

\end{itemize}


Let $x_a$ denote the average number of lines of code per day produced by programmers in Group A and
let $x_b$ be the corresponding quantity for Group B.

Provide an estimate of a — b and calculate an approximate 95\% confidence interval for a — b .

%==============================================================%

\subsubsection{Worked Examples} 
 Using the data in Q1 (Deltatech), test the claim that Group A are more efficient than Group B by
Interpreting the 95\% confidence interval.
Computing the appropriate test statistic.
Computing the appropriate p-value.



%==============================================================%
\subsubsection{Worked Examples} 

The strength of concrete depends, to some extent, on the method used for drying. Two different methods showed the following results for independently tested specimens.  ( You may assume that there are equal variances).

\[ WHERE IS DATA \]


\begin{itemize}
\item[(i)] Does Method 1 appear to produce concrete with a greater mean strength? State your conclusions clearly.
\item[(ii)]Construct a 95\% confidence interval for the difference between the two means. Interpret this interval.

\end{itemize}








\subsubsection{Worked Example}
A survey of 1000 Irish indicates that 750 have access to the Internet. A survey of 2000 Spaniards indicates that 1400 have access to the Internet.

\noindent By calculating the appropriate p-value, test the hypothesis that the proportion of all Irish having access to the Internet
is equal to the proportion of all Spaniards having access to the internet at a significance level of 5\%. % (8 marks)


%================================================================= %
\begin{description}
\item[Step 1] : Formally state the null and alternative hypotheses
\item[Step 2] : Determine the test statistic
\item[Step 3a] : Determine the p.value
\item[Step 4a] : Decision Rule for p-values.
\end{description}

%================================================================= %

\noindent \textbf{Step 1 : Formally state the null and alternative hypotheses}\\ 
Proportion of people having internet access is the same in both Ireland and Spain
Proportion of people having internet access differs in Ireland and Spain


Alternatively we write the hypotheses as follows (the null value is more evident).
%================================================================= %
\noindent \textbf{Step 2 Compute the Test Statistic}\\

\[p_{Irl}= \frac{750}{1000}= 0.75\] 
\[p_{Esp}= \frac{1400}{2000}= 0.70\]

Observed Difference = 0.75 - 0.70 = 0.05  

Now lets compute the standard error (from Formulae)







\end{document}
