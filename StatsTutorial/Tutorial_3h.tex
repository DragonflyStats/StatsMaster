










%================================================================================%

\subsubsection{Question 3.6}

A dice is thrown 5 times. Calculate the probability of

i.
Obtaining exactly one six


ii)    Obtaining at least one six
bi.
Calculate the (theoretical) mean and variance of the number of sixes obtained?



P(x=k)=nkpk(1-p)n-k






Part 1 


P(x=1)=51(16)1(56)4


\[ \choose{5,1}= \frac{5!1!}{4!}= 5\]


P(x=1)= 5(16)1(56)4= 0.401 


%================================================================================%


Part 2


obtaining at least one head is complement of obtaining zero heads


P(x1)= 1 - p(x=0)



P(x=0)=50(16)0(56)5


50=5!0!5!= 1


P(x1)= 1- 0.401 =0.599 

%================================================================================%





\section*{Question 3}


Which are the following pairs of events are mutually exclusive?

i.
Two dice are thrown: A is the event the sum is 10, B is the event the sum is 11


ai.
A hand of two cards is dealt: A is the event that the hand includes at least one red card, B is the event that the hand includes at least one black card.


bi.
student is chosen from the class at random: A is the event that the student is female, B is the event that a student is left-handed.









\subsection*{Solution}

\begin{itemize}
\item[(i)] is mutually exclusive. cant throw 10 and 11 in same throw of two dice.


\item[(ii)] not mutually exlusive: can have one red card and one black card.


\item[(iii)] not mutually exclusine: can have a lefthanded female
\end{itemize}

%================================================================================%
\end{document}
