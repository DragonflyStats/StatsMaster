\documentclass[a4paper,12pt]{article}
%%%%%%%%%%%%%%%%%%%%%%%%%%%%%%%%%%%%%%%%%%%%%%%%%%%%%%%%%%%%%%%%%%%%%%%%%%%%%%%%%%%%%%%%%%%%%%%%%%%%%%%%%%%%%%%%%%%%%%%%%%%%%%%%%%%%%%%%%%%%%%%%%%%%%%%%%%%%%%%%%%%%%%%%%%%%%%%%%%%%%%%%%%%%%%%%%%%%%%%%%%%%%%%%%%%%%%%%%%%%%%%%%%%%%%%%%%%%%%%%%%%%%%%%%%%%
\usepackage{eurosym}
\usepackage{vmargin}
\usepackage{amsmath}
\usepackage{framed}
\usepackage{graphics}
\usepackage{epsfig}
\usepackage{subfigure}
\usepackage{enumerate}
\usepackage{fancyhdr}

\setcounter{MaxMatrixCols}{10}
%TCIDATA{OutputFilter=LATEX.DLL}
%TCIDATA{Version=5.00.0.2570}
%TCIDATA{<META NAME="SaveForMode"CONTENT="1">}
%TCIDATA{LastRevised=Wednesday, February 23, 201113:24:34}
%TCIDATA{<META NAME="GraphicsSave" CONTENT="32">}
%TCIDATA{Language=American English}

\pagestyle{fancy}
\setmarginsrb{20mm}{0mm}{20mm}{25mm}{12mm}{11mm}{0mm}{11mm}
\lhead{Maths Resource} \chead{Probability Theory} \rhead{Poisson Distribution} %\input{tcilatex}
\begin{document}
\begin{enumerate}
%====================================================================================%
\item The Poisson random variable X with parameter $\lambda > 0$ has probability mass function

\[{\displaystyle p_X(x) =e^{-\lambda }{\frac {\lambda ^{x}}{x!}}}\] where
$ x = \{0, 1, 2, \ldots\}$ and $\lambda>0$ .

\begin{enumerate}[(a)]
\item Show that, for integer $x \geq 0$,

\[{\displaystyle p_X(x) =e^{-\lambda }{\frac {\lambda ^{x}}{x!}}}\]

and outline how this could be used to calculate the probability mass function for a given value of $\lambda$.



\item Show that $E(X) = Var(X) = \lambda$.



\item Suppose that Y has a Poisson distribution with mean $\mu$ independently of X, and that $W = X +Y$. Use the relation

\[ P(W=w) =  \sum^{n}_{x=0} P(X=x) P(Y=w-x) \]

to show that W has a Poisson distribution, and write down its mean.
\end{enumerate}

\item An office has two computer systems, one of Type A and one of Type B. The numbers of breakdowns per day on these systems, X and Y say, have independent Poisson distributions with respective means 2 and 0.5.
\begin{enumerate}[(i)]
\item Find the conditional probability that if there is exactly one breakdown on a given day then it is the Type A system that fails.
\item Find the probability that on a given day there are more than 2 breakdowns.
\item The office is one of 50 run by a large company. The offices are each equipped with one Type A system and one Type B system, which function independently in the way described above. Write down the distribution of T, the total number of breakdowns occurring in the 50 offices on any given day. Use a suitable approximation to estimate $T<0.95$, the number of breakdowns per day which will be exceeded on at most 5\% of days.


\end{enumerate}

%%%%%%%%%%%%%%%%%%%%%%%%%%%%%%%%%%%%%%%%%
%% RSS- HC2- 2009 Question 
\item The discrete random variable X has the Poisson distribution with parameter $\lambda$, so that
its probability mass function is given by

\[ {\displaystyle  p_X(x) = e^{-\lambda }{\frac {\lambda ^{x}}{x!}}}  \]

\begin{enumerate}[(a)]
\item Given that $E(X)$ = $\lambda$, show that $E(X^2)$ = $\lambda$($\lambda$ + 1) and deduce Var(X).

\item The random variables Y and Z have Poisson distributions with respective
parameters $\mu$ and $\lambda$, and X, Y and Z are independent. Use the relation

\[ P(W=w) =  \sum^{n}_{x=0} P(X=x) P(Y=w-x) \]
to show that the random variable $W = X + Y$ has a Poisson distribution, and
write down its mean and variance. 
\item What is the distribution of the random
variable $V = Y + Z$ ?

\item  The random variables T and U are defined by $T = W - Z$, $U = V - Z$. Find
$E(T)$, $Var(T)$, $E(U)$ and $Var(U)$. 
\item Explain why $P(U < 0) = 0$ but $P(T < 0) > 0$.
\end{enumerate}

%%%%%%%%%%%%%%%%%%%%%%%%%%%%%%%%%%%%%%%%%%%%%%%%%%%%
 % RSS HC2 2007 Question 4


\item The number of flaws, $X$, in a standard length of yarn is assumed to be Poisson distributed with probability mass function
\[ {\displaystyle  p_X(x) = e^{-\lambda }{\frac {\lambda ^{x}}{x!}}}  \]
where $\lambda$ is a positive parameter. A textile manufacturer buys yarn from suppliers P, Q and R in the long-run proportions 1:2:3. It is known from experience that the numbers of flaws in lengths of yarn from these suppliers are independently Poisson distributed with respective parameter values $\lambda_P=3$,
$\lambda_Q=2$ and $\lambda_R=1$.

\begin{enumerate}[(a)]
\item An unlabelled length of yarn is found to have 2 flaws. Is it more likely to have come from supplier $Q$ or supplier $R$?

\item A second unlabelled length of yarn, known to be from the same supplier as the first, is also found to have 2 flaws. Are the two lengths of yarn more likely to have come from supplier $Q$ or supplier $R$? Comment briefly on this result in comparison with that of part (a).
\end{enumerate}
%%%%%%%%%%%%%%%%%%%%%%%%%

% RSS- HC2 - 2012
\item A statistics lecturer holds a weekly 'surgery' of 2 hours during which students of his
course can visit him to ask for help with their work. It is assumed that individual
students arrive randomly and independently during the surgery period (but not before)
at a rate of 5 per hour, so that the number of students Nt arriving in any interval of
t hours during the surgery period can be taken as having the Poisson distribution

\[ P(N_t=n) = exp(-5t) \frac{(5t)^n}{n!}, \]
where $n=\{0,1,2,3 \ldots \}$.

\begin{enumerate}[(a)]
\item The lecturer makes a cup of coffee immediately before his surgery period
begins and will take 10 minutes to drink it. Find the probability that he
finishes his coffee before any student arrives.
\item  A student has just arrived and asked for help, and his query takes 15 minutes.
Find the probability that at least 2 more students arrive while this student is
being dealt with.
\item The course runs for a term of 10 weeks, each surgery lasts for 2 hours and
surgery visits in different weeks are independent. Use a suitable approximation
to calculate the probability that the total number of surgery visits is more than
110. Explain why the exact probability might be anticipated to be more than
the calculated value.
\item Suggest reasons why a Poisson model may not be a good assumption in
practice.
\end{enumerate}




%%%%%%%%%%%%%%%%%%%%%%%%%%%%%%%%%%%%%%%%%%%%%%%%55

% RSS HC2 2010 Question 2

\item Flaws in lengths of rope made by Company A occur in a Poisson process at rate $\lambda_A$ per metre length, so that the number of flaws X in a length of l metres of rope has the Poisson probability mass function


\[ {\displaystyle  p_X(x) = e^{-(\lambda_A l) }{\frac {(\lambda_A l) ^{x}}{x!}}}  \]
where $x = \{0, 1, 2, \ldots \}$ and  $\lambda_A > 0$.

\begin{enumerate}[(a)]
\item Find the probability that there are 
\begin{enumerate}[(i)]
    \item no flaws,
    \item more than 2 flaws,
\end{enumerate}  in a 1000-metre length of rope made by company A, given that $\lambda_A$ = 0.002.

\item Company B makes similar rope, indistinguishable in appearance from that made by Company A, in which flaws occur in a Poisson process at rate $\lambda_B$ = 0.003 per metre. A boat is rigged with 100 metres of rope from Company A and 100 metres of rope from Company B. Assuming that the lengths of rope supplied by A and B are independent, find the probability that
\begin{enumerate}[(i)]
    \item there are no flaws,
    \item there is exactly one flaw, in the rigging of this boat.
\end{enumerate}

\item A manufacturer of rigging for sailing boats buys 75\% of his rope from Company A and 25\% from Company B. The supplier's label has become detached from a drum of rope of length 2 km which is found to have 7 flaws. Find the probability that this drum was supplied by Company A.

\item Suppose, instead, that the rope in this drum had been found to have 8 flaws. Find the probability that this drum was supplied by Company A. Compare this probability with your answer to part (c) and comment.

\end{enumerate}

\end{enumerate}



\end{document}
