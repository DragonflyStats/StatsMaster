
\documentclass[]{article}
\usepackage{framed}
\usepackage{amsmath}
\usepackage{amssymb}
\usepackage{multicol}
%opening

\begin{document}
%---------------------------------------------------------------------------------------
\section*{Question 2 (Sample Variant 1)[25 marks]}
\begin{itemize}

\item[(a)] \textbf{\textit{Probability Distribution (6 Marks)}}\\ % Exponential %6 MARKS
Suppose that a student is taking a multiple-choice exam in which each question has four choices.
Suppose that she has no knowledge of the correct answer to any of the questions. Furthermore suppose that she selects one of the possible choices at random as her answer.
\begin{itemize}
\item [i.](2 Marks) If there are five multiple-choice questions on the exam, what is the probability that she will answer four questions correctly.
\item [ii.](2 Marks) What is the probability that she will answer none of the questions correctly?
\item [iii.](2 Marks) What is the probability that she will answer at least two questions correctly?
\end{itemize}

\item[(b)] \textbf{\textit{Exponential Distribution (8 Marks)}}\\ % Exponential %6 MARKS
A power supply unit for a computer component is assumed to follow an exponential distribution with a mean life of 1,400 hours.  What is the probability that the component will:
\begin{itemize}
\item [i.](2 Marks)	fail in the first 700 hours?
\item [ii.](3 Marks) survive more than 1,750 hours?
\item [iii.](3 Marks) last between 1,050 hours and 1,750 hours?
\end{itemize}


\item[(c)] \textbf{\textit{Normal Distribution (8 Marks)}}\\
%\subsubsection*{Part C} % Normal %6 MARKS
Assume that the diameter of a critical component is normally distributed with a Mean of 50mm and a Standard Deviation of 2mm.

NB 	You must draw a rough sketch of the normal curve and estimate the approximate probability of the following measurements occurring on an individual component.
\begin{itemize}
\item [i.](3 Marks)	Between 50 and 51.2mm
\item [ii.](3 Marks) Less than 48.5 mm
\item [iii.](2 Marks) Between 48.2 and 51.2 mm
\end{itemize}

Use the normal tables to get the exact probabilities for the above.
						
	
\item[(d)] \textbf{\textit{Poisson Approximation of the Binomial Distribution (3 Marks)}}
\begin{itemize}
\item[(i)] (2 Marks) Describe how the Poisson distribution can be used to approximate the binomial distribution.
\item[(ii)] (1 Mark) Explain the circumstances in which this approximation may be used in preference to the binomial distribution.
\end{itemize}	
	
\end{itemize}						
\end{document}
