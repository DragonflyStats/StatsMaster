\documentclass[]{article}
\usepackage{framed}
\usepackage{amsmath}
\usepackage{amssymb}
\usepackage{multicol}
%opening

\begin{document}

%---------------------------------------%
\section*{Question 1 (Sample Variant 1)[25 marks]}

\begin{itemize}
\item[(a)] \textbf{\textit{Probability (6 Marks)}}\\ An IT consultant is responsible for three software engineering projects X, Y and Z.
He knows that the probability of completing project X in time is 0.99, for project Y this probability is 0.95
and for project Z it is 0.80.

\begin{itemize}
\item[(i)] (1 marks) What assumption do you need to make in order to calculate the probability
of completing all three projects in time, from the information given?
\item[(ii)] (3 marks) Calculate the probability of completing all three projects in time.
\item[(iii)] (2 marks) Calculate the probability that only projects X and Y will be completed on time.
\end{itemize}

\item[(b)] \textbf{\textit{Probability (8 Marks)}}\\ The following contingency table illustrates the number of 400 students in different
departments according to gender.

\begin{center}
\begin{tabular}{|c|c|c|c|}
  \hline
  % after \\: \hline or \cline{col1-col2} \cline{col3-col4} ...
   & Computer Science & Statistics & Equine Science \\\hline
  Males & 140 & 100 & 20  \\  \hline
  Females & 30 & 80 & 30  \\ \hline

  \hline
\end{tabular}
\end{center}

\begin{itemize}
\item[(i)] (2 marks) What is the probability that a randomly chosen person from the sample is a
computer science student?
\item[(ii)] (2 marks) What is the probability that a randomly chosen person from the sample is both female and studying statistics?
\item[(iii)] (2 marks) What is the probability that a randomly chosen person from the sample is male?
\item[(iv)] (2 marks) Given that a student studies statistics, what is the probability that the student is female?
%\item[v] (2 marks) What is the probability that a randomly chosen person from the sample is a
%male or a statistics student?
%\item[vi] (2 marks) Given that the student is female, what is the probability that she is an
%equine science student?
\end{itemize}
\item[(c)] \textbf{\textit{Discrete Random Variables (6 Marks)}}\\The probability distribution of discrete random variable $X$ is tabulated below. There are 6 possible outcome of $X$, i.e. 1, 2, 3, 4 ,5 and 6.
\begin{center}
\begin{tabular}{|c||c|c|c|c|c|c|}
\hline
$x_i$  & 1 & 2 & 3 & 4 & 5 & 6 \\\hline
$P(x_i)$ & 0.16 & 0.14 & \mbox{   k   } & 0.17 & 0.21 & 0.19\\
\hline
\end{tabular}
\end{center}

\begin{itemize}
\item[i] (1 marks) Compute the value for $k$.
\item[ii] (2 marks) Determine the expected value $E(X)$.
\item[iii] (2 marks) Evaluate $E(X^2)$.
\item[iv] (1 marks) Compute the variance of random variable $X$.
\end{itemize}\item[(d)] \textbf{\textit{Descriptive Statistics (5 Marks)}}\\
Consider the following data set of seven numbers:

\begin{center}
\textbf{\texttt{4, 18,  2,  7, 18,  3,  4 }}
\end{center}
% 4 Marks

\noindent For this sample, compute the following descriptive statistics:
\begin{itemize}
%\item[a.] (1 Mark) The median,
\item[(i)] (1 Mark) The mean,
\item[(ii)] (1 Mark) The median,
\item[(iii)] (2 Marks) The variance,
\item[(iv)] (1 Mark) The standard deviation.
\end{itemize}
\end{itemize}

\end{document}