
\documentclass[]{article}
\usepackage{framed}
\usepackage{amsmath}
\usepackage{amssymb}
\usepackage{multicol}
\usepackage{graphicx}
%opening

\begin{document}

%---------------------------------------%
\section*{Question 2 (Sample Variant 3)[25 marks]}
\begin{itemize}
%\subsection*{Part 2A : Poisson Distribution }

\item[(a)] \textbf{\textit{Probability Distributions (9 Marks)}}\\
Telephone calls arrive at a switchboard at the rate of 20 per hour.  Assume that the telecentre operators take 4 minutes to deal with a customer query.  Calculate the probability of :
\begin{itemize} 
\item[(i)] (3 Marks)                 3 or more calls arriving in any 4 minute period.
\item[(ii)] (2 Marks)               No phone calls arriving in a 4 minute period,
\item[(iii)] (3 Marks)              Exactly one phone call arriving in any 4 minute period,
\item[(iv)] (1 Marks)             What is the average and standard deviation of the number of phone calls arriving in a 4 minute period.
\end{itemize}
%\noindent (When answering, justify your answer with workings, or by reference to an axiom, theorem or rule.)


\bigskip
%-----------------------------------%
%\subsection*{Question 2B Binomial Distribution [3 Marks] } % 12 Marks
% New Question On Binomial
\item[(b)] \textbf{\textit{Probability Distributions (7 Marks)}}\\ On average, six people per hour use an electronic teller machine during the prime shopping hours in a department store. Therefore it is assumed that the expected time until the next customer will arrive will be 10 minutes. You may assume that the distributions of waiting times can be described by the exponential probability distribution.

\begin{itemize}
\item[(i)] (3 Marks) What is the probability that at least 10 minutes will pass between the arrival of two customers?
\item[(ii)] (2 Marks) What is the probability that after a customer leaves, another customer does not arrive for at least 20 minutes?
\item[(iii)] (2 Marks) What is the probability that a second customer arrives within 1 min after a first customer begins a banking transaction?
\end{itemize}
\bigskip
%-----------------------------------%
\item[(c)] \textbf{\textit{Probability Distributions (9 Marks)}}\\
For a digital communication channel, the probability of a bit being received in error is $5\%$. Consider the case where 100 bits are transmitted. Answer the following questions.

\begin{itemize}
\item[(i)] (3 marks)	What is the probability that the number of bits received in error is 5?
\item[(ii)] (3 marks) What is the probability that the number of bits received in error is greater than 10?
\item[(iii)] (3 marks) What is the expected value for the number of bit will be error. What is the variance for this value?
\end{itemize}

%\noindent(When answering, justify your answer with workings, or by reference to an axiom, theorem or rule.)




%\item[(d)] \textbf{\textit{Poisson Approximation of the Binomial Distribution (3 Marks)}}
%\begin{itemize}
%\item[(i)] (2 Marks) Describe how the Poisson distribution can be used to approximate the binomial distribution.
%\item[(ii)] (1 Mark) Explain the circumstances in which this approximation may be used in preference to the binomial distribution.
%\end{itemize}
\end{itemize}
\end{document}