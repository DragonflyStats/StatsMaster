\documentclass[a4paper,12pt]{article}
%%%%%%%%%%%%%%%%%%%%%%%%%%%%%%%%%%%%%%%%%%%%%%%%%%%%%%%%%%%%%%%%%%%%%%%%%%%%%%%%%%%%%%%%%%%%%%%%%%%%%%%%%%%%%%%%%%%%%%%%%%%%%%%%%%%%%%%%%%%%%%%%%%%%%%%%%%%%%%%%%%%%%%%%%%%%%%%%%%%%%%%%%%%%%%%%%%%%%%%%%%%%%%%%%%%%%%%%%%%%%%%%%%%%%%%%%%%%%%%%%%%%%%%%%%%%
\usepackage{eurosym}
\usepackage{vmargin}
\usepackage{amsmath}
\usepackage{framed}
\usepackage{graphics}
\usepackage{epsfig}
\usepackage{subfigure}
\usepackage{enumerate}
\usepackage{fancyhdr}

\setcounter{MaxMatrixCols}{10}
%TCIDATA{OutputFilter=LATEX.DLL}
%TCIDATA{Version=5.00.0.2570}
%TCIDATA{<META NAME="SaveForMode"CONTENT="1">}
%TCIDATA{LastRevised=Wednesday, February 23, 201113:24:34}
%TCIDATA{<META NAME="GraphicsSave" CONTENT="32">}
%TCIDATA{Language=American English}

\pagestyle{fancy}
\setmarginsrb{20mm}{0mm}{20mm}{25mm}{12mm}{11mm}{0mm}{11mm}
\lhead{MathsResource} \chead{Probability} \rhead{Tutorial Sheets} %\input{tcilatex}
\begin{document}
%================================================================================%


\begin{enumerate}
%================================================================%
%%- RSS- HC2- 2010 - Question 3
\item The random variable T has the exponential distribution with rate parameter $\lambda$, so that the probability density function (pdf) of T is
$$ {\displaystyle f_{T}(t)={\begin{cases}\lambda e^{-\lambda t}&t\geq 0,\\0&t<0.\end{cases}}} $$
\begin{enumerate}[(a)]
\item Obtain the cumulative distribution function (cdf) $F_T(t)$ of T, and draw the graph of $F_T(t)$. 
\item Show that $P(a < T \leq b) = e^{-\lambda a} - e^{-\lambda b}$. 
\item Given that $P(0 < T \leq 1) = 2P(1 < T \leq 2)$, find the value of $\lambda$ to three significant figures. 
\item For any choice of $c$ and $t$ such that $t > c > 0$, find $P(T > t | T > c)$. 
Deduce the conditional pdf of T given that $T > c$. In a similar way, find the conditional pdf of $T – c$ given that $T > c$, and comment briefly on your results. 
\end{enumerate}

%====================================================================%
\item 
The random variable X has the exponential distribution with probability density function

$$ f_{X}(x) = \lambda e^{-\lambda x}, \qquad \mbox{where x } >0 \;\; \& \;\; \lambda >0 $$
\begin{enumerate}
\item Show that $E(X) = 1/\lambda$.

\item Show that $P(X>a) = e^{-\lambda x}$ for any $a > 0$. Deduce the median of $X$.

\item For any $b > 0$, find $P(X>a+b|X>a)$ and comment on this result.

\end{enumerate}

Now consider the case where $\lambda = 1$.
\begin{enumerate}
\item Sketch the graph of $f(x)$.

\item State with a reason whether the distribution of $X$ is positively or negatively skew.

\item Write down the mode of the distribution of X, and find the value of k such that Mean – Mode = k(Mean – Median).

\item A student has read that, for many distributions,
\begin{itemize}
\item the skewness is positive if the mean is greater than the median
\item the value of k is about 3.
\end{itemize}

Comment on the truth of each of these statements for the distribution of X.
\end{enumerate}




%===========================================================================%
%%- RSS - HC2 - 2007 - Question 2
\item The random variable X has the exponential probability density function (pdf) given by



$$ f_{X}(x) = \lambda e^{-\lambda x}, \qquad \mbox{where x } >0 \;\; \& \;\; \lambda >0 $$



\begin{enumerate}[(a)]

\item  Show that $ E(X) = 1/\lambda$ and find the standard deviation of $X$.

\item  Show that, for any $c > 0$, $P(X > c) = exp(−\lambda c)$.

Hence show that, for any $x > c$, $P(X > x | X > c) = exp(−\lambda(x−c))$. Deduce the conditional pdf of $X$ given that $X > c$, and comment briefly.

\item A random sample has been selected from a distribution that is thought to be exponential. The values obtained, arranged in ascending order, are $$\{0.1, 0.1, 0.2, 0.4, 1.1, 2.3, 2.5, 3.4, 4.3, 5.6\}.$$ [You are given that the sum and sum of squares of these values are 20.0 and 74.38 respectively.] 

Calculate the sample mean and the sample standard deviation and say with a reason whether you think the exponential model is suitable for the distribution underlying this sample.

\end{enumerate}

%==================================================================================================%
\item
%% RSS HC2 2012 Question 4
\begin{enumerate}[(a)] 
\item The continuous random variable X has the exponential distribution with
probability density function (pdf) $f(x)$ given by
$$ f_{X}(x) = \lambda e^{-\lambda x}, \qquad \mbox{where x } >0 \;\; \& \;\; \lambda >0. $$
\begin{enumerate}[(i)]
\item Find the cumulative distribution function (cdf) F(x) of X, and sketch the
graph of $F(x)$ for the case $\lambda = 1/2$. Mark on your graph the median of X.

\item  The continuous random variable Y is independent of $X$ and has a
distribution with pdf $g(y)$ given by

$$ g_{y}(y) = \mu e^{-\mu y}, \qquad \mbox{where y } >0 \;\; \& \;\; \mu >0 $$

Write down the cdf of Y.
\end{enumerate}
%-------------------------------%
\item Striplights A and B, from two different suppliers, have lifetimes respectively
distributed as X and Y in part (a), where $\lambda = 1/2$ and $\mu = 1/3$, for lifetimes
measured in units of 1000 hours. Two new striplights, one from each supplier,
are installed at the same time. Their lifetimes may be assumed to be
independent.
\begin{enumerate}[(i)]
\item Find the values of $P(X \leq 2)$ and $P(Y \leq 2)$.
\item Find the probability that both striplights last at least 2000 hours, i.e. that
$X \geq 2$ and Y $\geq 2$.
\item Find the probability that exactly one striplight lasts at least 2000 hours.
\item Given that exactly one striplight lasts at least 2000 hours, find the
probability that it is A.
\end{enumerate}
\end{enumerate}

%%%%%%%%%%%%%%%%%%%%%%%%%%%%%%%%
\end{enumerate}
\end{document}
