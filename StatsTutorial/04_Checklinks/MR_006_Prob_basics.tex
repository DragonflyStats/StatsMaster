\documentclass[a4paper,12pt]{article}
%%%%%%%%%%%%%%%%%%%%%%%%%%%%%%%%%%%%%%%%%%%%%%%%%%%%%%%%%%%%%%%%%%%%%%%%%%%%%%%%%%%%%%%%%%%%%%%%%%%%%%%%%%%%%%%%%%%%%%%%%%%%%%%%%%%%%%%%%%%%%%%%%%%%%%%%%%%%%%%%%%%%%%%%%%%%%%%%%%%%%%%%%%%%%%%%%%%%%%%%%%%%%%%%%%%%%%%%%%%%%%%%%%%%%%%%%%%%%%%%%%%%%%%%%%%%
\usepackage{eurosym}
\usepackage{vmargin}
\usepackage{amsmath}
\usepackage{framed}
\usepackage{graphics}
\usepackage{epsfig}
\usepackage{subfigure}
\usepackage{enumerate}
\usepackage{fancyhdr}

\setcounter{MaxMatrixCols}{10}
%TCIDATA{OutputFilter=LATEX.DLL}
%TCIDATA{Version=5.00.0.2570}
%TCIDATA{<META NAME="SaveForMode"CONTENT="1">}
%TCIDATA{LastRevised=Wednesday, February 23, 201113:24:34}
%TCIDATA{<META NAME="GraphicsSave" CONTENT="32">}
%TCIDATA{Language=American English}

\pagestyle{fancy}
\setmarginsrb{20mm}{0mm}{20mm}{25mm}{12mm}{11mm}{0mm}{11mm}
\lhead{MathsResource} \chead{Probability} \rhead{Tutorial Sheets} %\input{tcilatex}
\begin{document}
%%%%%%%%%%%%%%%%%%%%%%%%%%%%%%%%%%%%%%%%%%%%%%%%%%%%%%%%%%%%%%%%%%%%

\begin{enumerate}
%===================================================================%
%%- RSS - HC2 - 2009 or 2010 - Question 1
\item A standard pack of 52 playing cards consists of 4 suits (clubs, diamonds, hearts and
spades), each consisting of 13 cards numbered $2, 3, 4, \ldots, 10$, Jack, Queen, King, Ace
(their face values). In the game of poker, a hand of 5 cards is drawn without
replacement from a well-shuffled pack.
\begin{enumerate}[(a)]
\item How many different poker hands are possible?

\item A poker hand consisting of a pair of cards with the same face value and three
other cards with the same face value (different from that of the pair) is called a
full house. Find the probability that a poker hand drawn from a well-shuffled
pack is a full house. Express your answer either as a fraction in lowest terms
or as a decimal correct to 3 significant figures.

\item A poker hand consisting of a pair of cards with the same face value and three
other cards with face values different from each other and from that of the pair
is called a pair. Find the probability that a poker hand drawn from a wellshuffled
pack is a pair. Express your answer either as a fraction in lowest
terms or as a decimal correct to 3 significant figures.
\end{enumerate}
%%%%%%%%%%%%%%%%%%%%%%%%%%%%%%%%%%%%%%%%%%%%%%%%%%%%%%%%%%%%%%%%%%%%
%%- RSS HC2 2011 or 2012 Question 1
\item Clay pots made at a pottery are subject to three types of defect. It is found that 10\% of pots show brittle fracture (B), 4\% have cracked glazing (C) and 10\% are discoloured (D). 

\begin{enumerate}[(a)]
\item Assuming that all three types of defect occur independently, what is the probability that a randomly chosen pot has no defects? 
\item  Experience has shown that the three types of defect do not all occur independently. 20\% of pots with brittle fracture also have cracked glazing, but 
both of these defects occur independently of discoloration. 
\begin{enumerate}[(i)]
\item  Find the probability that a pot has both brittle fracture and cracked glazing, and hence 
find the probability that a pot has either or both of these defects. 
\item What is now the probability that a randomly chosen pot has no defects? (5) 
\item Suppose that a pot does not have cracked glazing and is not discoloured. Find the probability that it has brittle fracture. (6) 
\end{enumerate}
\end{enumerate}


%==================================================================================================%
\item  In Newtopia, the weather on any day is dry with probability 23 and wet with probability 13, the weather on different days being independent.
\begin{enumerate}[(a)]

\item Find the probability that the next three days are dry.
\item Find the probability that exactly two of the next three days are wet.
\item A Newtopian resident walks his dog with probability 0.9 when it is dry but with probability 0.6 when it is wet. If it is known that he walked his dog last Tuesday, what is the probability that last Tuesday was dry in Newtopia?
\end{enumerate}

\item  Suppose now that the weather on different days is not independent but that
$$P(\mbox{next day is dry} | \mbox{today is dry}) = 0.8$$
and $$P(\mbox{next day is wet} | \mbox{today is wet}) = 0.6$$.
\begin{enumerate}[(a)]

\item  Given that today is dry, what is the probability that the next three days are dry?

\item Given that today is wet, what is the probability that exactly two of the next three days are wet?
\item  Let $p$ denote the overall probability that any day is dry. Explain clearly why p must satisfy the equation
$$0.8p + 0.4(1 – p) = p,$$
and deduce the value of $p$.
\end{enumerate}


%%%%%%%%%%%%%%%%%%%%%%%%%%%%%%%%%%%%%%%%%%%%%%%%%%%%%%%%%%%%%%%%%%%%%

\item  Let A and B be two events with $P(B) > 0$.
\begin{enumerate}
\item  Write down an expression for the conditional probability of A given B.
\item Determine the conditional probability of A given B in the following
cases.
\begin{enumerate}[(i)]
\item $A$ and $B$ are independent.
\item $A$ and $B$ are mutually exclusive.
\item $B \subset A$.
\end{enumerate}

\end{enumerate}
%%%%%%%%%%%%%%%%%%%%%%%%%%%%%%%%%%%%%%%%%%%%%%%%%%%%%%%%%%%%%%%%%%%%

%% RSS HC2 2015 Question 1

\item A diagnostic test for a disease gives a positive result with probability 0.98 for
people who have the disease, and a negative result with probability 0.99 for
people who do not have the disease. Suppose 3\% of the population have the
disease.
\begin{enumerate}[(a)]
\item A person is selected at random from the population and given the test.
If the result is positive, what is the probability that this person has the
disease?
\item Suppose a person, initially selected at random from the population, is
given the test once and the result is positive. This person is then given
the test, independently, a second time and the result is again positive.
What is the probability that this person has the disease?
\end{enumerate}
%%%%%%%%%%%%%%%%%%%%%%%%%%%%5
\item Two football teams M and C each have one game left to play (not against each
other) in the season. If M wins and C does not win, or if M draws and C loses,
then M wins the championship. Otherwise C wins the championship. The
probabilities that M wins, draws or loses the last game are $1/2$ ,$1/6$  and $1/3$ 
respectively. The probabilities that C wins, draws or loses the last game are $2/3$, $1/6$ and $1/6$ respectively.
\begin{enumerate}[(a)]
\item  What is the probability that M wins the championship?
\item  What is the probability that C has drawn the last game given that M has
won the championship?
\end{enumerate}


\end{enumerate}
%%%%%%%%%%%%%%%%%%%%%%%%%%%%%%%%%%%%%%%%%%%%%%%%%%%%%%%%%%%%%%%%%%%%%5555
\end{document}
