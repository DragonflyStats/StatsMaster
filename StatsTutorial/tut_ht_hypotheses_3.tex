\documentclass[a4paper,12pt]{article}
%%%%%%%%%%%%%%%%%%%%%%%%%%%%%%%%%%%%%%%%%%%%%%%%%%%%%%%%%%%%%%%%%%%%%%%%%%%%%%%%%%%%%%%%%%%%%%%%%%%%%%%%%%%%%%%%%%%%%%%%%%%%%%%%%%%%%%%%%%%%%%%%%%%%%%%%%%%%%%%%%%%%%%%%%%%%%%%%%%%%%%%%%%%%%%%%%%%%%%%%%%%%%%%%%%%%%%%%%%%%%%%%%%%%%%%%%%%%%%%%%%%%%%%%%%%%
\usepackage{eurosym}
\usepackage{vmargin}
\usepackage{amsmath}
\usepackage{framed}
\usepackage{graphics}
\usepackage{epsfig}
\usepackage{subfigure}
\usepackage{enumerate}
\usepackage{fancyhdr}

\setcounter{MaxMatrixCols}{10}
%TCIDATA{OutputFilter=LATEX.DLL}
%TCIDATA{Version=5.00.0.2570}
%TCIDATA{<META NAME="SaveForMode"CONTENT="1">}
%TCIDATA{LastRevised=Wednesday, February 23, 201113:24:34}
%TCIDATA{<META NAME="GraphicsSave" CONTENT="32">}
%TCIDATA{Language=American English}

\pagestyle{fancy}
\setmarginsrb{20mm}{0mm}{20mm}{25mm}{12mm}{11mm}{0mm}{11mm}
\lhead{Maths Resource} \rhead{Tutorial Sheet B} \chead{Hypothesis Testing} %\input{tcilatex}
\begin{document}
\section*{Hypothesis Testing : Tutorial Sheet C}
\begin{enumerate}

\item 
An IT competency test, used for staff recruitment, is devised so as to give a normal distribution of scores with a mean of 100. A random sample of 49 experienced IT users  who are given the test achieve a mean score of 121 with a standard deviation of 14. 

\begin{enumerate}[(a)]
\item	Perform a hypothesis test to assess whether this group of IT Users is unusual (i.e. have a different mean from the general population).
\item	Compute a 95\% confidence interval for the group.
\end{enumerate}
%%%%%%%%%%%%%%%%%%%%%%%%%%%%%%%%
\item  
A claim has been made that the mean body temperature of healthy adults is equal to 98.6 degrees. 
A sample of 106 people has produced a mean body temperature of 98.2 degrees and a standard deviation of 0.62. Test the claim using a 5\% significance.

\item 
A manufacturer of computer monitors  has for many years used a process giving a mean life of 4700 hours and a standard deviation of 1460 hours. 
A new process is tried to see if it will increase the life significantly. A sample of 100 monitors gave a mean life of 5000 hours.  
Does the new process make a difference at the 5\% level of significance?

\item In a study of store checkout scanners, 1234 items were checked and 20 of them were overcharges.
Use a 5\% significance level to test the claim that with scanners, 1\% of sales are overcharges.




\item The average height of a sample of 16 students was 173cm with a variance of 144cm$^2$. The average height of the Irish population is 169cm. 
\begin{enumerate}[(a)]
\item Can it be stated at a significance level of 5\% that students are on average taller than the population as a whole? 
\item What assumption is used to carry out this test? Is this assumption reasonable?
\end{enumerate}


%----------------------------------------------------------------------- %
\item 
The average mass of a sample of 64 Irish teenagers (Let say - 18 year old males) was 73.5kg with a variance of 100kg$^2$. 
The average mass of an equivalent sample of 81 Japanese teenagers was 68.5kg with a variance of 81 kg$^2$. 
\begin{enumerate}[(a)]
\item Test the hypothesis that Irish students are larger (in terms of mass) than Japanese teenagers.
\item By calculating the appropriate $p-$value, test the null hypothesis that the mean mass of all Irish students is 70kg at significance levels of 5\%. 
\item Using the appropriate confidence interval, test the hypotheses that the average mass of all Irish students is a) 70kg, b) 72kg, c) 75kg at a significance level of 5\%.
\end{enumerate}


% \textbf{\textit{Standard Error Formula}}
% \[ S.E.(\bar{x}_1 - \bar{x}_2)  = \sqrt{\frac{s_1^2}{n_1} + \frac{s_2^2}{n_2}} \]
% \[ S.E.(\bar{x}_1 - \bar{x}_2)  = \sqrt{\frac{10^2}{64} + \frac{9^2}{81}}  = \sqrt{2.56} = 1.6\]
% \newpage


\item A study was carried out in which researchers collected crime data. Of those convicted of
arson, 50 were drinkers and 43 abstained. Of those convicted of fraud, 63 were drinkers and 144
abstained. Use a 0.01 level of significance to test the claim that the proportion of drinkers among
convicted arsonists is greater than the proportion of drinkers convicted of fraud.




%-------------------------------------%
\item 
A coal-fired power plant is considering two different systems for pollution abatement.
The first system has reduced the emission of pollutants to acceptable levels 68\% of the time,
as determined from 200 air samples. The second, more expensive system has reduced the
emission of
pollutants to acceptable levels 70\% of the time, as determined from 250 air samples. lf the
expen sive system is significantly more eilective than the inexpensive system in reducing the
pollutants to acceptable levels, then the management of the power plant will install the
expensive system.

\begin{enumerate}[(a)]
\item Which system will be installed if management uses a significance level of 0.05 in making
its decision?
\item Construct a 95\% confidence interval for the difference in the two proportions. Interpret
this interval.
\end{enumerate}
%------------------%


\item  
The quality control manager at the Telektronic Company considers the production of telephone answering machines to be ’out of control’ when the overall rate of defects exceeds 4\%. 
Testing of a random sample of 150 machines revealed that 9 are defective. The production manager claims that production is not out of control and no corrective action is necessary. Use a 0.05 significance level to test the production manager’s claim.

\item 
It is generally assumed that older people are more likely to vote for the Conservatives than younger people. In a survey, 160 of 400 people over 40 and 120 of 400 people under 40 stated they would vote Conservative. 
\begin{enumerate}[(a)]
\item Do the data support this hypothesis at a significance level of 5\%?
\item Calculate a 95\% confidence interval for the difference between the proportion of people over 40 voting Conservative and the proportion of people below 40 voting Conservative. 
\end{enumerate}
\item The following are measurements (in mm) of a critical
dimension on a sample of engine crankshafts:
\begin{center}
\begin{tabular}{|c|c|c|c|} \hline
224.120 & 224.017 & 223.976 & 223.961 \\ \hline
224.089 & 223.982 & 223.980 & 223.989  \\ \hline
223.960 & 223.902  & 223.987 & 224.001  \\ \hline
\end{tabular}
\end{center}

\begin{enumerate}[(a)]
\item Calculate the mean and standard deviation for these data.
\item The process mean is supposed to be $\mu$ = 224mm. Is this the
case? Give reasons for your answer.\\
\item Construct a 95\% confidence interval for these data and
interpret.\\
\item Check that the normality assumption is valid using 2 suitable
plots.
\end{enumerate}


\item Answer the following theory questions on hypothesis testing.
\begin{enumerate}[(a)]
\item In the context of hypothesis testing, explain what a p-value is, and how it is used. Support your answer with a simple example.
\item What is meant by Type I error and Type II error?
\end{enumerate}



%-------------------------------------%
% \section*{Question 4 (Two Sample Means, small samples, one tailed)}
\item The working lifetimes of 100 of both of two different types of batteries were observed. The mean lifetime for the sample of type 1 batteries was 25 hrs with a standard deviation of 4hrs. The mean lifetime for the sample of type 1 batteries was 23 hrs with a standard deviation of 3hrs. 
\begin{center}
\begin{tabular}{|c||c|}
\hline 
Type 1 & Type 2 \\ \hline \hline
$n_1$ = 100 & $n_2$ = 100 \\ \hline
$x_1$ = 25 hours & $x_1$ = 23 hours \\ \hline
$s_1$ = 4 hours & $s_1$ = 3 hours \\ \hline
\end{tabular} 
\end{center}
\begin{itemize}
\item[(a)] Test the hypothesis that the mean working lifetimes of these batteries do not differ at a significance level of 5\% .

\item[(b)] Calculate a 95\% confidence interval for the difference between the average working lifetimes of these batteries. 
\item[(c)] Using this confidence interval, test the hypothesis that battery 1 on average works for 3 hours longer than battery 2.
\end{itemize}

%------------------------------------- %

%%%%%%%%%%%%%%%%%%%%%%%%%%%%%%%%%%%%%%%%
\end{enumerate}



%%%%%%%%%%%%%%%%%%%%%%%%%%%%%%%%%%%%%%%%%%%%%%%%%%%%%%%%
\end{document} 
