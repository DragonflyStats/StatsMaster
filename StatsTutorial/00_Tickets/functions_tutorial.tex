 \documentclass[a4paper,12pt]{article}
%%%%%%%%%%%%%%%%%%%%%%%%%%%%%%%%%%%%%%%%%%%%%%%%%%%%%%%%%%%%%%%%%%%%%%%%%%%%%%%%%%%%%%%%%%%%%%%%%%%%%%%%%%%%%%%%%%%%%%%%%%%%%%%%%%%%%%%%%%%%%%%%%%%%%%%%%%%%%%%%%%%%%%%%%%%%%%%%%%%%%%%%%%%%%%%%%%%%%%%%%%%%%%%%%%%%%%%%%%%%%%%%%%%%%%%%%%%%%%%%%%%%%%%%%%%%
\usepackage{eurosym}
\usepackage{vmargin}
\usepackage{amsmath}
\usepackage[thinlines]{easytable}

\usepackage{enumerate}
\usepackage{multicol}
\usepackage{graphics}
\usepackage{epsfig}
\usepackage{framed}
\usepackage{subfigure}
\usepackage{fancyhdr}

\setcounter{MaxMatrixCols}{10}
%TCIDATA{OutputFilter=LATEX.DLL}
%TCIDATA{Version=5.00.0.2570}
%TCIDATA{<META NAME="SaveForMode" CONTENT="1">}
%TCIDATA{LastRevised=Wednesday, February 23, 2011 13:24:34}
%TCIDATA{<META NAME="GraphicsSave" CONTENT="32">}
%TCIDATA{Language=American English}

%\pagestyle{fancy}
\setmarginsrb{20mm}{0mm}{20mm}{25mm}{12mm}{11mm}{0mm}{11mm}
%\lhead{MA4413 2013} \rhead{Mr. Kevin O'Brien}
%\chead{Midterm Assessment 1 }
%\input{tcilatex}

\begin{document}

Introduction to Financial Mathematics Worked Examples FUNCTIONS 
Produced by the Maths Learning Centre, The University of Adelaide. May 3, 2013 
The questions on this page have worked solutions and links to videos on the following pages. 
Click on the link with each question to go straight to the relevant page. You will need to have the question handy to refer to while watching the videos. 
Questions 
\begin{enumerate}
%===============%
\item  See Page 3 for worked solutions. Write the following sets in interval notation: 
\begin{enumerate}[(a)]
    \item $\{x \in RI Cx<6 \} U \{a2+1 la \in R\} $
    \item $\{x \in R I 2x < 5\} \{y \in R I y2 = 9\} $
    \item $\{x \in R J x2 1\} n \{x \in R I X2 <4\} $
\end{enumerate}


%===============%
\item See Page 5 for worked solutions. 
For each of the following quadratic equations find the y-intercept, the axis of symmetry, the vertex and the zeros 
and sketch the function, showing each of these features on your graph. 
\begin{enumerate}[(a)]
    \item $y = 3x^2 + 2x — 8$ 
    \item $y = 2x^2 + 4x + 5$ 
\end{enumerate}
%===============%
\item 
See Page 7 for worked solutions. Find the y-intercept, symmetry point and zeros of the cubic equation $y = X3 — X2 + x - 1 = (x - 1)(x2 + 1)$. 
Sketch the graph of this function. 
%===============%
\item  See Page 8 for worked solutions. Give the domain and range 
of the rational functions 1 1 $f(x) = — x + 2$ and $g(x) = x^2 +1$. 

%===============%
\item See Page 9 for worked solutions. 

A government assesses income tax T on each citizen's income Y as follows: for income up to \$10,000, no tax; a tax rate of 2\% on 
income from \$10,000 to \$50,000 and 4\% on income above \$50,000 up to a maximum tax of \$2,800. Sketch T as a function of Y. 


\item See Page 10 for worked solutions. 
{-1 — x (a) Let f (x) = 0 x2 — 4 
if x < —1 if — 1 < x < 0. if 1 < x < 2 
Sketch a graph of f and write down the domain and range of f . 
(b) Consider the real-valued function g(x) =  ar; F  (x2 + 1)(x2 1). 
Find the largest possible domain for g. 

\item See Page 12 for worked solutions. 
Let f (x) = x2 - 4, g(x) = VT, and h(x) - 1° x 2 ifxx> 2 if  < 2 Find expressions for the following functions and give the domain in each case. 
\begin{enumerate}[(a)]
    \item  f + g \item  9/f \item  fog \item  g f \item  f• g \item h 9 \item  / o h \item h o f. 
\end{enumerate}
\item See Page 16 for worked solutions. 
Sketch the functions g(t) = 2t + 2-1 and h(t) - Zt - 2-t. 
\item See Page 17 for worked solutions. 
Consider the function f(x) = 2x -F 1 x - 1 
\begin{enumerate}[(a)]
    \item State the domain and range of f . 
\item  Show that f is a 1-1 function. 
\item Find the inverse function f 
\end{enumerate}

\item See Page 19 for worked solutions. 

Find the inverse of the function g(x) - x3 + 2, and plot both g and g-1 on the same axes. 
\item See Page 20 for worked solutions. 
Evaluate the following limits (if possible): (a) lim -K1  ((x3 + 2) sin(r/2 - x)) (b) to  iim 2 2 ) 
(c) lim  Nci (d) X2 - 3x 4 li .-•m rg-N2 - a 2 x + 2 
\\
(Note that the function sin is not part of the IFM course, so you can ignore part (a).) 
\end{enumerate}
%===============%
\end{document}
