% - http://www.analyzemath.com/high_school_math/grade_12/complex_numbers.html
\begin{document}

Grade 12 Problems on Complex Numbers with Solutions and Answers
Complex numbers are important in applied mathematics. Problems and questions on complex numbers with detailed solutions are presented. 


%============%

Evaluate the following expressions 
a) (3 + 2i) - (8 - 5i) 
b) (4 - 2i)*(1 - 5i) 
c) (- 2 - 4i) / i 
d) (- 3 + 2i) / (3 - 6i) 


If (x + yi) / i = ( 7 + 9i ) , where x and y are real, what is the value of (x + yi)(x - yi)? 


Determine all complex number z that satisfy the equation 
z + 3 z' = 5 - 6i

where z' is the complex conjugate of z. 

Find all complex numbers of the form z = a + bi , where a and b are real numbers such that z z' = 25 and a + b = 7 
where z' is the complex conjugate of z. 


The complex number 2 + 4i is one of the root to the quadratic equation x2 + bx + c = 0, where b and c are real numbers. 
a) Find b and c 
b) Write down the second root and check it. 

Find all complex numbers z such that z2 = -1 + 2 sqrt(6) i. 

Find all complex numbers z such that (4 + 2i)z + (8 - 2i)z' = -2 + 10i, where z' is the complex conjugate of z. 


Given that the complex number z = -2 + 7i is a root to the equation: 
z3 + 6 z2 + 61 z + 106 = 0 

find the real root to the equation. 

a) Show that the complex number 2i is a root of the equation 
z4 + z3 + 2 z2 + 4 z - 8 = 0 

b) Find all the roots root of this equation. 

P(z) = z4 + a z3 + b z2 + c z + d is a polynomial where a, b, c and d are real numbers. Find a, b, c and d if two zeros of polynomial P are the following complex numbers: 2 - i and 1 - i. 

%=============%

Solutions to the Above Questions
a) -5 + 7i 
b) -6 - 22i 
c) -4 + 2i 
d) -7/15 - 4i/15 
Sponsored Content


Sleeep® Earplugs For A Longer Sleep
flareaudio.com


How good of a driver are you? Test your driving knowledge
Regit


[Gallery] Princes William And Harry Have A Secret Sister That No One Talks About
Absolutehistory


[Gallery] He Was Unforgettable On 'Crossroads,' But What He's Up To Now Is Quite Surprising
BrainSharper


Explore Hungary's Thriving Capital in 24 Hours
Departures International


[Gallery] The Mask Slips. The Truth Behind Jim Carrey's Absence From Hollywood Pictures
Absolutehistory
Recommended by 


(x + yi) / i = ( 7 + 9i ) 
(x + yi) = i(7 + 9i) = -9 + 7i 
(x + yi)(x - yi) = (-9 + 7i)(-9 - 7i) = 81 + 49 = 130 

Let z = a + bi , z' = a - bi ; a and b real numbers. 
Substituting z and z' in the given equation obtain 
a + bi + 3*(a - bi) = 5 - 6i 
a + 3a + (b - 3b) i = 5 - 6i 
4a = 5 and -2b = -6 
a = 5/4 and b = 3 
z = 5/4 + 3i 

z z' = (a + bi)(a - bi) 
= a2 + b2 = 25 
a + b = 7 gives b = 7 - a 
Substitute above in the equation a2 + b2 = 25 
a2 + (7 - a)2 = 25 
Solve the above quadratic function for a and use b = 7 - a to find b. 
a = 4 and b = 3 or a = 3 and b = 4 
z = 4 + 3i and z = 3 + 4i have the property z z' = 25. 

a) Substitute solution in equation: (2 + 4i)2 + b(2 + 4i) + c = 0 
Expand terms in equation and rewrite as: (-12 + 2b + c) + (16 + 4b)i = 0 
Real part and imaginary part equal zero. 
-12 + 2b + c = 0 and 16 + 4b = 0 
Solve for b: b = -4 , substitute and solve for c: c = 20 
b) Since the given equation has real numbers, the second root is the complex conjugate of the given root: 2 - 4i is the second solution. 
Check: (2 - 4i)2 - 4 (2 - 4i) + 20 
(Expand) = 4 - 16 - 16i - 8 + 16i + 20 
= (4 - 16 - 8 + 20) + (-16 + 16)i = 0 
 
Let z = a + bi 
Substitute into given equation: (a + bi)2 = -1 + 2 sqrt(6) i 
Expand: a2 - b2 + 2 ab i = - 1 + 2 sqrt(6) i 
Real part and imaginary parts must be equal. 
a2 - b2 = - 1 and 2 ab = 2 sqrt(6) 
Equation 2 ab = 2 sqrt(6) gives: b = sqrt(6) / a 
Substitute: a2 - ( sqrt(6) / a )2) = - 1 
a4 - 6 = - a2
Solve above equation and select only real roots: a = sqrt(2) and a = - sqrt(2) 
Substitute to find b and write the two complex numbers that satisfies the given equation. 
z1 = sqrt(2) + sqrt(3) i , z2 = - sqrt(2) - sqrt(3) i 

Let z = a + bi where a and b are real numbers. The complex conjugate z' is written in terms of a and b as follows: z'= a - bi. Substitute z and z' in the given equation 
(4 + 2i)(a + bi) + (8 - 2i)(a - bi) = -2 + 10i 
Expand and separate real and imaginary parts. 
(4a - 2b + 8a - 2b) + (4b + 2a - 8b - 2a )i = -2 + 10i 
Two complex numbers are equal if their real parts and imaginary parts are equal. Group like terms. 
12a - 4b = -2 and - 4b = 10 
Solve the system of the unknown a and b to find: 
b = -5/2 and a = -1 
z = -1 - (5/2)i 

Since z = -2 + 7i is a root to the equation and all the coefficients in the terms of the equation are real numbers, then z' the complex conjugate of z is also a solution. Hence 
z3 + 6 z2 + 61 z + 106 = (z - (-2 + 7i))(z - (-2 - 7i)) q(z) 
= (z2 + 4z + 53) q(z) 
q(z) = [ z3 + 6 z2 + 61 z + 106 ] / [ z2 + 4z + 53 ] = z + 2 
Z + 2 is a factor of z3 + 6 z2 + 61 z + 106 and therefore z = -2 is the real root of the given equation. 

a) (2i)4 + (2i)3 + 2 (2i)2 + 4 (2i) - 8 
= 16 - 8i - 8 + 8i - 8 = 0 
b) 2i is a root -2i is also a root (complex conjugate because all coefficients are real). 
z4 + z3 + 2 z2 + 4 z - 8 = (z - 2i)(z + 2i) q(z) 
= (z2 + 4)q(z) 
q(z) = z2 + z - 2 
The other two roots of the equation are the roots of q(z): z = 1 and z = -2. 

Since all coefficients of polynomial P are real, the complex conjugate to the given zeros are also zeros of P. Hence 
P(z) = (z - (2 - i))(z - (2 + i))(z - (1 - i))(z - (1 + i)) = 
= z4 - 6 z3 + 15 z2 - 18 z + 10 
Hence: a = -6, b = 15, c = -18 and d = 10. 
%==========%
\end{document}
