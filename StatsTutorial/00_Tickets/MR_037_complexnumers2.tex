\documentclass[12pt, a4paper]{report}
\usepackage{epsfig}
\usepackage{subfigure}
%\usepackage{amscd}
\usepackage{amssymb}
\usepackage{graphics}
\usepackage{multicol}
\usepackage{enumerate}
\usepackage{graphicx}
%\usepackage{amscd}
\usepackage{amssymb}
\usepackage{subfiles}
\usepackage{framed}
\usepackage{subfiles}
\usepackage{amsthm, amsmath}
\usepackage{amsbsy}
\usepackage{framed}
\usepackage[usenames]{color}
\usepackage{listings}
\lstset{% general command to set parameter(s)
	basicstyle=\small, % print whole listing small
	keywordstyle=\color{red}\itshape,
	% underlined bold black keywords
	commentstyle=\color{blue}, % white comments
	stringstyle=\ttfamily, % typewriter type for strings
	showstringspaces=false,
	numbers=left, numberstyle=\tiny, stepnumber=1, numbersep=5pt, %
	frame=shadowbox,
	rulesepcolor=\color{black},
	,columns=fullflexible
} %
%\usepackage[dvips]{graphicx}
\usepackage{natbib}
\usepackage{epstopdf}
\bibliographystyle{chicago}
\usepackage{vmargin}
% left top textwidth textheight headheight
% headsep footheight footskip
\setmargins{3.0cm}{2.5cm}{15.5 cm}{22cm}{0.5cm}{0cm}{1cm}{1cm}
\renewcommand{\baselinestretch}{1.5}
\pagenumbering{arabic}
\theoremstyle{plain}
\newtheorem{theorem}{Theorem}[section]
\newtheorem{corollary}[theorem]{Corollary}
\newtheorem{ill}[theorem]{Example}
\newtheorem{lemma}[theorem]{Lemma}
\newtheorem{proposition}[theorem]{Proposition}
\newtheorem{conjecture}[theorem]{Conjecture}
\newtheorem{axiom}{Axiom}
\theoremstyle{definition}
\newtheorem{definition}{Definition}[section]
\newtheorem{notation}{Notation}
\theoremstyle{remark}
\newtheorem{remark}{Remark}[section]
\newtheorem{example}{Example}[section]
\renewcommand{\thenotation}{}
\renewcommand{\thetable}{\thesection.\arabic{table}}
%\renewcommand{\thefigure}{\thesection.\arabic{figure}}
\title{Research notes: linear mixed effects models}
\author{ } \date{ }


\begin{document}
\begin{enumerate}
\item Express each of the following in the form $a + $: 
\begin{multicols}{2}
\begin{enumerate}[(i)]
\item  (3 + 2i) -4 (5 i) 
\item  (6 - i) + (4 - 31) 
\item  (-2 t 3i) + (6 - 41) 
\item  (-2 - 1) -r (-1 + 7i) 
\item  61+ (3 + 
\item  (a 4- ib)+ (c + id) 
\end{enumerate}
\end{multicols}
\item Simplify each of the following: 
\begin{multicols}{2}
\begin{enumerate}[(i)]
\item (2 - 61) - (1 + 
\item (3 6i) - (2 + 4i) 
\item (2 - i) - ( I + 4i) 
\item 3 - (2 4i) 
\item (6 - 21) - 4
\end{enumerate}
\end{multicols}
%(a+ ib) - (2 - 31) 

\item  Express in the form $a + ib$: 
\begin{multicols}{2}
\begin{enumerate}[(i)]
\item (3 + i)(2 + 41) 
\item (1 - i)(2 + 3i) 
\item  (2 - i)(3 + 21) 
\item (2 + 3i)(2 - 3i) 
\item (1 - 402 (2 + 6. 
\end{enumerate}
\end{multicols}

\item  If z1 = 3 - i, z2 1 + 2i and z3 - -2i, express in the form a+ ib: \begin{multicols}{2}
\begin{enumerate}[(i)]
\item 3z1 
\item z1 - z3 
\item 2z1 + z2 
\item 2z2 + z3 \item -2z2 
\item iz2 
\item 2z1 + iz3 
\item i(z2.z3) 
\end{enumerate}
\end{multicols}

\item  If z = -3 + 5i, find (1) z (ii) 8. Express the following in the form x + iy: 1 2 +. i 3 + 2/  (i) 2 - 3i (ii) 1 2i (iii) 2 - 3i 
2i  3 - 2i (iv) 27--F (ir) - 9. If z = 1 + V3 , find z 2. 3 Hence verify that (-12- + i V3) 2 10. If z, = 2 - 5i and z2 = 1 - 4i, express in the form a - ib: 1 (i) z1 + z2 (ii) z,z2(iii) (iv) ii,z2 ,  
2 - 3i (vi) 2 + 
\item  If $z = 3 - i$, express z + 1-z in the form $a + ib$. 
\item  If z = x + iy, prove that Re(z) = 1 1 (z + I) and Im(z) = (z - i) \item  (i) Evaluate 1 + 3/2 + /7 (ii) Given that (2 + 3i)z = 4 - i, find the complex number z in the form a + ib. 

\item  Show that (cos 0 + i sin 0)2 = cos 20 + i sin 20. .. 1 - 15. Express (i) 1 + i TT-, and (ii) in the form a + ib. Hence, or otherwise, find k if 1+ i = , 1-i 1 + 1 
\item i Ar3- Express - -1 +  in the form a+ ib, where a and b are real numbers. -1 - /-0 
\end{enumerate}
%====================%
write down 

331 
\begin{enumerate}
\item 

2. Solve these equations, giving the roots in the form a + ib. (i) z2 —2z +17 = 0 (ii) z2 + 4z + 7 = 0. 3. If z = x + iy, verify that (i) z + z = 2 Re(z) 4. If z = x + iy, verify that (i) z, — z2 = — (ii) z112 = 2i Im(z,i2) - (iii) Re(z,12)= 2(z,z2 + z,z2) 
(ii) 142 = z. 
\item Show that 1 + i is a root of the equation z3 — 4z2 + 6z 4 =0 and find the other two roots. 
\item Form the quadratic equation whose roots are $—2 \pm i$. Now show that —2 + i is a root of the equation z3 + z2 — 7z — 15 = 0 and find the other roots. 
\item Form the quadratic equation whose roots are —3 ± 2i. Hence form the cubic equation whose roots are —3 ± 2i and 1. S. 
\begin{enumerate}[(a)]
\item Form the quadratic equation, one of whose roots is 3 —1.
\item Form the cubic equation, two of whose roots are 2 and —1 + i.  
\item Find the real root of the equation z3 + z + 10 = 0, given that one root is 
\end{enumerate}
1 2i 
1 i 2 . is 
\item . s a root of the quadratic equation axe bx + 5 = 0, where $a,b \in R$. 1 Find the values of a and b. 
\item Verify that 2i is a root of the equation x4 + 2x3 + 7x2 + 8x + 12 = 0 and find the other roots.
\item Factorise z3 • 1 and hence solve the equation z3 — 1 = 0, giving the complex roots in the form a ib. 
If the complex roots are a and /3, verify that a2 = 13. If 2 + i is a root of the equation z3 + az2 + bz + 10 = 0, find the values of a and b given that the product of the three roots is —10. 
\item 14.,. Explain why the roots of the equation z1 — (3+21)z +1+3i = 0 do not occur in conjugate pairs. 
Now use the formula z —  b — 4ac 2a equation in the form a I ib. 
to express the roots of the given 

\end{enumerate}
%====================%

%%- http://qmplus.qmul.ac.uk/pluginfile.php/990096/mod_resource/content/1/tutorial9.pdf
\begin{enumerate}
\item Simplify the following:
\begin{multicols}{2}
\begin{itemize}
	\item[(i)]  $i^2$
	\item[(ii)]  $i^5$
	\item[(iii)]  $3i$
	\item[(iv)]  $i^13$
\end{itemize}
\end{multicols}
\item Evaluate the following:
\begin{multicols}{2}
\begin{enumerate}[(i)]
	\item $(2 + 3i) + (5 + 7i)$,
	\item $(2 + 3i) − (5 + 7i)$,
	\item $(2 + 3i)(2 − 3i)$,
	\item $(3 − 5i)^3$
	,
	\item ${\displaystyle \frac{3 − 5i}{	−1 − 2i} }$


\end{enumerate}
\end{multicols}


\item For each part of question 2, draw the complex numbers on an Argand diagram, and express in
the form $re^{i\theta}$

\item Evaluate $(-2 - 5i)(3- 2i)$.
\item Evaluate $(6 - 2i)(1 - i)(2 - 2i)$.
\item Evaluate 
${\displaystyle \frac{1}{i} \times \frac{6-2i}{1-i} }$


.
\item Express $z = e^{2+i(\pi/4)}$ in the form a + bi.
\item Express $3 - 2i$ in polar form.
\item Find the square roots of \[z = 4(cos π
3 + isin π
3
)\], and draw these on an Argand diagram. Identify
the principal root.
\item For $z = 3(cos \pi/6 + isin \pi/6)$, calculate z
4
in polar form.
\item For $z = 3 - 2i$, find the five roots $z^{1/5}$
, and plot these on an Argand diagram indicating the
principal root.


\item Find the values of $x$, $y$ (both real) which satisfy the equation \[ x(x + y) + x − yi = −1 − 3i\].
\item Two competing “\textit{probability amplitudes}” A1 and A2 for a quantum mechanical transition from
some initial state |ii to some final state |fi are given by
$A1 = aei\psi1$,$ A2 = bei\psi2$
.
and the total probability amplitude (A) for the process is given by A1 + A2. Given that the probability
for a process is given by AA∗
, calculate the probability for the transition from |ii to |fi.
Using this, calculate the probability if $\psi1 = \psi2$.


\end{enumerate}

%%- http://tutorial.math.lamar.edu/Classes/Alg/ComplexNumbers.aspx

\begin{enumerate}

\item  Perform the indicated operation and write the answers in standard form.
\begin{itemize}
\item[(i)] 
\item[(ii)] 
\item[(iii)] 
\end{itemize}


\item   Multiply each of the following and write the answers in standard form.
\begin{itemize}
\item[(i)]    
\item[(ii)]    
\item[(iii)]    
\item[(iv)] 
\end{itemize}
   
\item  Write each of the following in standard form.
\begin{itemize}
\item[(i)]    
\item[(ii)]    
\item[(iii)]    
\item[(iv)]    
\end{itemize}

\end{enumerate}

%============================================================%

Exercises[edit]
$(7 + 2i) + (11 - 6i) =$
$(8 - 3i) - (6i) =$
$(9 + 4i)(3 - 16i) =$
$3i {\displaystyle \times } \times  9i =$

${\displaystyle {\frac {i}{2+i}}=} {\displaystyle {\frac {i}{2+i}}=}$

${\displaystyle {\frac {11+3i}{{\sqrt {3}}-4i}}=} {\displaystyle {\frac {11+3i}{{\sqrt {3}}-4i}}=}$

${\displaystyle {(x+yi)}^{-1}=} {\displaystyle {(x+yi)}^{-1}=}$
Answers[edit]
18 - 4i
8 - 3i - 6i = 8 - 9i
27 - 144i + 12i - 64i2 = 91 - 132i
-27. $( {\displaystyle i\times i=i^{2}=-1} {\displaystyle i\times i=i^{2}=-1})$

${\displaystyle {\frac {i}{2+i}}\times {\frac {2-i}{2-i}}={\frac {1+2i}{5}}={\frac {1}{5}}+{\frac {2}{5}}i} {\displaystyle {\frac {i}{2+i}}\times {\frac {2-i}{2-i}}={\frac {1+2i}{5}}={\frac {1}{5}}+{\frac {2}{5}}i}$

${\displaystyle {\frac {11+3i}{{\sqrt {3}}-4i}}\times {\frac {{\sqrt {3}}+4i}{{\sqrt {3}}+4i}}=} {\displaystyle {\frac {11+3i}{{\sqrt {3}}-4i}}\times {\frac {{\sqrt {3}}+4i}{{\sqrt {3}}+4i}}=}$

${\displaystyle {\frac {11{\sqrt {3}}+44i+3i{\sqrt {3}}+12i^{2}}{3+16}}=}$ ${\displaystyle {\frac {11{\sqrt {3}}+44i+3i{\sqrt {3}}+12i^{2}}{3+16}}=}$

${\displaystyle {\frac {(11{\sqrt {3}}-12)+(44i+3i{\sqrt {3}})}{19}}=}$ ${\displaystyle {\frac {(11{\sqrt {3}}-12)+(44i+3i{\sqrt {3}})}{19}}=}$

${\displaystyle {\frac {11{\sqrt {3}}-12}{19}}+{\frac {44+3{\sqrt {3}}}{19}}i}$ 
${\displaystyle {\frac {11{\sqrt {3}}-12}{19}}+{\frac {44+3{\sqrt {3}}}{19}}i}$


Recall that ${\displaystyle x^{-1}={\frac {1}{x}}} {\displaystyle x^{-1}={\frac {1}{x}}}$ can be seen as the division of two complex numbers:

${\displaystyle {(x+yi)}^{-1}={\frac {1}{x+yi}}\times {\frac {x-yi}{x-yi}}=} {\displaystyle {(x+yi)}^{-1}={\frac {1}{x+yi}}\times {\frac {x-yi}{x-yi}}=}$

${\displaystyle {\frac {x-yi}{x^{2}-y^{2}i^{2}}}={\frac {x}{x^{2}+y^{2}}}-{\frac {y}{x^{2}+y^{2}}}i} {\displaystyle {\frac {x-yi}{x^{2}-y^{2}i^{2}}}={\frac {x}{x^{2}+y^{2}}}-{\frac {y}{x^{2}+y^{2}}}i}$
%====================================================================%
\newpage
MA4604: Science Maths 4, Homework Week 1
\begin{enumerate}
\item Write each of the complex numbers below in the form a + ib, that is simplify each
expression to find the real numbers a and b.
\begin{multicols}{2}
\begin{itemize}
\item[(a)] (3 + 8i) + (2i - 6) - (-5 + i) \item[(b)] (3 - i)(5 + 6i) \item[(c)] (1 + i)(2 − 5i)(7 + 3i)
\item[(d)] $ {\displaystyle \frac{3 − 4i}{2 − i} } $

\item[(e)] $ {\displaystyle \frac{5 − i}{2 − 3i} } $ 

\item[(f)] $ {\displaystyle \frac{2 + 4i}{i(1 − i)} } $ 

\end{itemize}
\end{multicols}
\item Find the real number(s) t that makes each expression below real:
\item[(a)] (4 + 6i)(3i)(t + 6i) \item[(b)] i(t + 4i)
2
\item[(c)] 5 − 10i
4i + t
.
\item Solve the complex equation 2z + i$\bar{z}$ = 5 + 4i (hint: write z = x + iy).
\item Find all the roots of, and hence factorise fully, each given polynomial:

\begin{multicols}{2}
\begin{itemize}
\item[(a)] x
2 − 8x + 25 \item[(b)] z
2 + (4 + 3i)z + 14 + 6i \item[(c)] x
4 − 16 \item[(d)] x
3 + 8.
\end{itemize}
\end{multicols}
\item Given that x = 1 + 4i is a root of the quartic polynomial x
4 + 13x
2 + 34x, find the other
three roots and write it as a product of linear factors.
\item Plot each of the given complex numbers in an Argand diagram:

\begin{multicols}{3}
\item[(a)] 2 + i \item[(b)] −1 + 3i \item[(c)] −2i \item[(d)] −2 − 2i
\item[(e)] 26
%\pi 4
\item[(f)] 46
% \pi
% 2
\item[(g)] 36
% \pi
% 3
% \item[(h)] 56 \pi
% \item[(i)] 4 e− i\pi4 
% \item[(j)] 2e− i\pi
% 2 
% \item[(k)] 3 e4i\pi3 
% \item[(l)] √3 e5i\pi3 .
\end{multicols}
\item Write each of the complex numbers in Q.6 
\item[(e)] through (l) in the form x + iy, that is
convert them from polar or exponential form to standard form.
\end{enumerate}



\newpage
%============================================================================%
MA4604: Science Maths 4, Homework Week 2
\begin{enumerate}
\item Write each of the complex numbers below in polar form $r(\cos \theta +  i \sin \theta)$ and in exponential form $re^{i\theta}$
(hint: first find r = |z| and $\theta = arg(z)$):
\begin{multicols}{2}
\begin{enumerate}[(i)]
	\item z = -2 \item z = -5i \item z = 3 - 3i \item z = -3 - 3i.
\end{enumerate}
\end{multicols}

\item For each complex number given below in exponential form reiθ, find its absolute value |z|
and its argument Arg(z); express z in the form $x + iy$:
%\item[(a)] 3 e
%i\pi
%2 \item[(b)] 2 e
%i\pi
%3 \item[(c)] 4 e
%− i\pi
%4 \item[(d)] 6 e
%2i\pi \item[(e)] 5 e
%−i\pi
.
\item Use de Moivre’s theorem to simplify the given powers: \item[(a)] (1 + i)
20 \item[(b)] (1 + i
√
3)12
.
\item Use de Moivre’s theorem with n = 3 to write cos(3θ) in terms of cos θ and to write sin(3θ)
in terms of sin θ (hint: you’ll also need to use the fact that cos2
θ + sin2
θ = 1).
\item Find the twelve twelfth roots of unity in exponential and in Cartesian form.
\item By first writing the given number in complex exponential form, evaluate each of the
following in Cartesian form:

\begin{multicols}{2}
\begin{enumerate}[(i)]
\item[(a)] all cube roots of 27; \item[(b)] all cube roots of -64;
\item[(c)] all cube roots of -64i; \item[(d)] all fourth roots of 81; \item[(e)] all fourth roots of -4;
\item[(f)] all square roots of -9i.
\end{enumerate}
\end{multicols}

\end{enumerate}
\end{document}
