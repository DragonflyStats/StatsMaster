 \documentclass[a4paper,12pt]{article}
 %%%%%%%%%%%%%%%%%%%%%%%%%%%%%%%%%%%%%%%%%%%%%%%%%%%%%%%%%%%%%%%%%%%%%%%%%%%%%%%%%%%%%%%%%%%%%%%%%%%%%%%%%%%%%%%%%%%%%%%%%%%%%%%%%%%%%%%%%%%%%%%%%%%%%%%%%%%%%%%%%%%%%%%%%%%%%%%%%%%%%%%%%%%%%%%%%%%%%%%%%%%%%%%%%%%%%%%%%%%%%%%%%%%%%%%%%%%%%%%%%%%%%%%%%%%%
 \usepackage{eurosym}
 \usepackage{vmargin}
 \usepackage{amsmath}
 \usepackage{multicol}
 \usepackage{graphics}
 \usepackage{enumerate}
 \usepackage{epsfig}
 \usepackage{framed}
 \usepackage{subfigure}
 \usepackage{fancyhdr}
 
 \setcounter{MaxMatrixCols}{10}
 %TCIDATA{OutputFilter=LATEX.DLL}
 %TCIDATA{Version=5.00.0.2570}
 %TCIDATA{<META NAME="SaveForMode" CONTENT="1">}
 %TCIDATA{LastRevised=Wednesday, February 23, 2011 13:24:34}
 %TCIDATA{<META NAME="GraphicsSave" CONTENT="32">}
 %TCIDATA{Language=American English}
 
 %\pagestyle{fancy}
 \setmarginsrb{20mm}{0mm}{20mm}{25mm}{12mm}{11mm}{0mm}{11mm}
 %\lhead{MA4413 2013} \rhead{Mr. Kevin O'Brien}
 %\chead{Midterm Assessment 1 }
 %\input{tcilatex}
 
 \begin{document}

\subsection*{Question 1 - Short Questions (40 Marks)}

%========================================================%
% Short Questions
% 1 - Inverse
% 2 - Domain and Range
% 3 - Vertical Asymptotes
% 4 - Horizontal Asymptotes
% 5 - Limit ( Indeterminate Answer - Factorize)
%======================================================%
\begin{itemize}
	
	\item[(i)](4 Marks) Find $f^{-1}(x)$ the inverse of the function
	\[f(x) = e^{3x}. \]
		\smallskip
	\item[(ii)](4 Marks)	Solve the following limit:
	\[\lim_{x \to 5 } \frac{x^2- 7x +10}{x-5}.\]
	\smallskip
	\item[(iii)] (4 Marks) The domain and range of the function $f(x) = \cos(x)$ are $(-\infty, \infty)$ and $[-1,1]$ respectively. State the domain and the range of the following functions:
	\begin{multicols}{2}
		\begin{itemize}
		
		
	\item[(a)] $f(x) = \cos^2(x)  $,
		\item[(b)] $f(x) = \cos(2x)  $,
	\item[(c)] $f(x) = 2\cos(x)  $,
		\item[(d)] $f(x) = \cos(x) +2$.
	\end{itemize}
	\end{multicols}
	\smallskip

	
	\item[(iv)] (4 Marks) Find the x-intercepts and the y-intercept of the following function:
	
	\[ f(x) = x^2 + x - 20 . \]
	
	\smallskip
	\item[(v)] (4 Marks) Determine the vertical asymptote(s) of the following function:
	
	
	\[ y = f(x) = \frac{4x^2}{x^2-25}. \]	
	\smallskip
	\item[(vi)] (4 Marks) Determine the horizontal asymptote(s) of the following function:
	
	\[ y = f(x) = \frac{4x^2}{x^2-25}.\]
		
	% answer = 3
	\smallskip
	%======================================================%
	% Short Questions
	% 6 - Indefinite Integral
	% 7 - Definite Integral
	% 8 - Partial Derivatives
	% 9 - Convergence
	% 10 - Sum of a Series
	\newpage
	%======================================================%
	\item[(vii)] (4 Marks) Evaluate the following indefinite integrals:
	\begin{multicols}{2}
	\begin{itemize}
		\item[(a)] $\displaystyle{ \int  e^{3x} + x^2\; dx, }  $
		\item[(b)] $\displaystyle{ \int \;\sin(2x)+ \cosh(x)\; dx}$
		
	\end{itemize}
	\end{multicols}

	\smallskip
%	\item (4 Marks) Evaluate the following indefinite integral:
%	\[ \int \cos(5x)+e^{2x}\; dx   \]
	%======================================================%
	\item[(viii)] (4 Marks)Evaluate the following definite integral:
	
	\[  \int^{5}_{2}  2x+5\; dx.  \]
	\smallskip

	
%	%======================================================%
%	\item[(viii)] Determine both of the first order partial derivatives with respect to $z$ of the following expression.
%	
%	\[ z = \frac{x^2y^3}{3} +  y \cos(x)\]
	
	
	\item[(ix)] (4 Marks) Find $\displaystyle{ \frac{ \partial^2 z }{ \partial x^2 }}$ and $\displaystyle{ \frac{ \partial^2 z }{ \partial y^2 }}$  for both of the following functions:
%	\[ z = 3x^2y^3 + y\cosh(x) - 9y^2 +10x^3.\]	
\smallskip
	\begin{multicols}{2}
		\begin{itemize}
			\item[(a)] $\displaystyle{  z = 9y^2 +10x^3. }  $
				\item[(b)] $\displaystyle{  z = 3x^2y^3 + y\cosh(x).}$
				
			\end{itemize}
		\end{multicols}
		\medskip
	%========================================================%
	% READY
	\item[(x)] (4 Marks)  Compute the following summation:
	
	\[ \sum_{i=10}^{50} i . \]
\end{itemize}





	

	
%		
%
%		\item (4 Marks) Find the sum of the following geometric series: 
%		\[2 + 6 + 18 +  \ldots + 39366		\]
%
%		\item (4 Marks) Express the following repeating decimal number as a simple fraction. Show your workings.
%		
%		\[0.297297297....\]
%		\textit{(Simplify your answer such that both the numerator and the denominator are prime numbers.)}
%
%		\item (4 Marks) The three terms below are three successive terms in an aritmetic progression. Compute the value for $x$, and the common difference $d$
%		\[ \ldots,\; 3x-1,\; 2x+6,\;	5x-7,\; \ldots		  \]

\bigskip
\subsection*{Question 2 - Limits and Functions (15 Marks)}
\subsubsection*{Part A - Limits (5 Marks)}
\begin{enumerate}
	\item[(i)] (2 Marks)  Compute the limit of the following function.
		{
			\large
	\[\lim_{x \to 6 } \frac{x^2 + 2x-10}{x-4}.\]
}
	\bigskip
	
	\item[(iii)] (3 Marks) Compute the limit of the following function.
		{
			\large
	\[ \lim_{x \to \infty } \frac{3 + 2x^2 - 6x^3 }{4x^3 - 5x + 7}. \]	
}
\end{enumerate}
\medskip
\subsubsection*{Part B - Functions (6 Marks)}
\begin{enumerate}[(i)]
	\item (3 Marks) Determine if the function $f(x) = x^3\sin(x)$ is an even function, an odd function or neither.
	\bigskip
		\item (3 Marks) Determine if the function $f(x) = x^2+x$ is an even function, an odd function or neither.
		\bigskip
%	\item (2 Marks) Given the functions $g(x) = x^2+1$ and $f(x) = 1-3x$ determine expressions for $f \circ g(x)$ and $g \circ f(x)$.
\end{enumerate}
\medskip
%	Odd or Even functions %DONE IN CLASS LECTURE 6A%
%	\[f(x) = x^4\sin(x)\]
%	\[f(x) = x^3\cos(x)\]
\subsubsection*{Part C - Hyperbolic Functions (4 Marks)}


\begin{enumerate}[(i)]
	\item (4 Marks) Using their definition in terms of exponentials, prove the following hyperbolic identity   
	{
		\large
	\[\sinh^2 (x) = \frac{1}{2}\left[\cosh(2x)-1\right]\]
}
	%	\item Show that \[Cosh^2 x = Cosh2x + Sinh2x\]	
	%	\item Show that \[ cosh(x+y) = cosh(x)cosh(y) + sinh(x)sinh(y)\]
	
\end{enumerate}
%========================================================================================= %
%========================================================================================= %

\bigskip
\subsection*{Question 3 - Curve Sketching (15 Marks)}
Consider the function $f(x) = x^4 -  8x^2 + 9$

\begin{enumerate}[(i)]
	\item (2 Marks) Find the y intercept of the function $f(x)$.
	\item (3 Marks) Find the three turning points of the function $f(x)$ and classify them as local
	maxima or local minima.
	\item (3 Marks)  Find the two points of inflection of the function $f(x)$.
	
	\item (3 Marks)  Determine the behaviour of the function $f(x)$ as $x \rightarrow + \infty$ and as $x \rightarrow - \infty $.
	\item (4 Marks) Sketch the graph of the function $f(x)$ illustrating the features of the curve obtained
	in parts (i – iv). 
\end{enumerate}



\subsection*{Question 4 - Sequences and Series (15 Marks)}
% Sequences and Series

% 4 Marks - Geometric Series
% 3 Marks - Arithmetic Series
% 3 Marks - Telescoping Series
% 3 Marks - Ratio Test
% 2 Marks - Repeating Decimal 

\begin{enumerate}[(i)]
	\item (3 Marks)
	The third term $u_3$ of a geometric sequence is 24. The fourth term $u_4$ is -48. \\ \bigskip Answer the following questions. Both questions are worth 2 Marks each.
	\begin{itemize}
		\item[(a)] Find the common ratio $r$. 
		\item[(b)] Find the first and fifth term $u_1$ and $u_5$.
	\end{itemize}
	
	
	\bigskip
	
	\item (3 Marks)	Three consecutive terms of an arithmetic series are 
	\[4x+4,\;\;6x-6,\;\;7x-5.\]
	Find the values for $x$ and constant difference $d$.
	%answer is 460
	
		\bigskip
		
		\item (2 Marks) Express the following repeating decimal number as a simple fraction. Show your workings.
		
		\[0.2162162162162....\]
		
	\bigskip
	%------------------------%
	%%updated for 2016
	\item (4 Marks) Suppose that the following term is the general term for a series. Use the Ratio Test to test this series for convergence
	
	\[u_n=\frac{2^n}{(2n)!}\]
	
	\bigskip
	
	%------------------------%
	%%updated for 2016
	\item (3 Marks) Find the sum of the telescoping series  \[ \sum^{\infty}_{n=1} \frac{4
		}{(2n-1)(2n+1)}\]
	

	%	\item (2 Marks) Find the sum of the following telescoping series
	%	\[  \sum^{\infty}_{n=1}   \frac{6}{(3n+1)(3n+4)}  \]
	%	
	%	
	\newpage

\end{enumerate}
\subsection*{Question 5 - Integration (15 Marks)}


%Integration
%Substitution Technique
%Area Between Curves
%Definite Integrals

% - http://media.wix.com/ugd/b064dd_62ab7d7000c34b7eb4a8e2108269cf27.pdf

\begin{enumerate}[(i)]
	\item (3 Marks)  Evaluate the definite integral
	\[ \int^3_1 3x^2 + \sqrt{x} + \frac{e^x}{4} dx. \]
	% \[ \int cos x(1 + sin x)^3 dx \]
	%---------------------------%
	\item (3 Marks) By finding a good substitution, evaluate the indefinite integral
	
	\[ \int \frac{12x + 12}{3x^2 + 6x + 8} dx.\]
	
	\item (3 Marks) By finding a good substitution, evaluate the indefinite integral
	\[ \int 2x\sqrt{x^2 + 2} dx.\]
	% - \[ \int x(x^2 − 1)^5 dx\]
	% - \[ \int 2x cosh(3x^2+2) dx \]
	% - \[ \int \frac{12x + 14}{3x^2 + 7x + 3} dx\]
	%---------------------------%
	\item (3 Marks) Use integration by parts to evaluate the indefinite integral
	\[ \int  2xe^{2x} \;dx. \]
	% Use integration by parts to find \[ \int x cos(x) dx \]
	%---------------------------%
	\item (3 Marks) By first performing a partial fraction expansion (that is, by writing the integrand
	as follows)
	\[  \frac{A}{x - 2} + \frac{B}{x + 3},
	\]evaluate the definite integral
	
	\[ \int \frac{7x + 6}{(x -2)(x + 3)}  dx. \]
	
\end{enumerate}
\bigskip 
%========================================================================================= %
%========================================================================================= %
\newpage
\subsection*{Question 6 - Applications of Calculus (15 Marks)}

\subsubsection*{Part A - Applications of Integration}
\begin{enumerate}[(i)]
	\item (5 Marks) Find the area enclosed by the curve $y=1+2x-x^2$ and the line $y=x-1$.
	Clearly state the points of intersection in your answer.
	\smallskip
	%\item (5 Marks) Find the area enclosed by the curve $y = 3x^2 − 4x + 1$ and the x axis.
	\item (5 Marks) A current $i(t) = 4 + 5\cos(5t)$ passes through a capacitor at time $t$.
	The capacitor is uncharged at time $ t = 0 $. Find the charge $q(t)$ at all times $t$.
	%\item (5 Marks) A moving object has acceleration $a(t) = 3 + 5 sin t$ at time t. It starts from rest at time $t = 0$. Find its velocity at all time t.	Also, find its velocity at time $t = 4$.
\end{enumerate}



%\item Use Simpsons with 5 equal subintervals to find an approximation
%for
\bigskip
\subsubsection*{Part B - Partial Derivatives}
\begin{enumerate}[(i)]
	
	\item (5 Marks)
	Show that the function $z=e^{-3y}\cos(x)$ satisfies the partial differential equation
	\[ 9\frac{\partial^2 z}{\partial x^2} + \frac{\partial^2 z}{\partial y^2} = 0. \]
\end{enumerate}
%
%\subsubsection*{MacLaurin Series}
%\begin{enumerate}[(i)]
%	\item	Find the Maclaurin series of $e^x$ up to an including the term containing $x^4$
%	
%	%\item Evaluate all the first partial derivatives, i.e. $\frac{\partial z}{\partial x}$ and $\frac{\partial z}{\partial y} $, 
%	%of the following function:
%	%
%	%\[ z = 2xy^3 + cos(2x) - sin(y) \]
%
%
%
%	\item
%\end{enumerate}

\newpage
	\section*{Formula Sheet}
	
\subsection*{Logarithms}
If $a^b = c$ \, then \, $\mbox{log}_a c = b$.



\subsection*{Change of Base Formula}

\[ \log_A(B) = \frac{ \log_e(B) }{ \log_e(A) }  \]
	\subsection*{Sum and Difference of Two Cubes}
	\[ a^3 + b^3 = (a-b)(a^2 - ab + b^2)\]
	\[ a^3 - b^3 = (a-b)(a^2 + ab + b^2)\]
	
	%======================================== %
	
	\subsection*{Sequences and Series}
Arithmetic Series Summation:
\begin{multicols}{2}	
	\[ \sum_{i=1}^{n} i = \frac{n(n+1)}{2}\]
	

	\[ S_n = \frac{n}{2} \left(2a + (n-1) d \right)\]
\end{multicols}	
	Geometric Series Summation:
	\begin{multicols}{2}
	\[ S_n = a\left(\frac{1-r^n}{1-r}\right)\]
	
	\[ S_\infty = \frac{a}{1-r}\]
	\end{multicols}
	
	
	\subsection*{Ratio Test}
	
	For a series with general term $u_n$, if
	
	\[ \lim_{n \to \infty } \left| \frac{u_{n+1}}{u_n} \right| = r\]
	then the series converges (absolutely) if $r<1$
	
	
	%==========================================================================================%
	
	
	\subsection*{Curve Sketching}
	\begin{description}
		\item[Horizontal Asymptote:] The horizontal asymptote is computed as
		\[ \lim_{x \to \infty } f(x) \]
	\end{description}
	%==========================================================================================%
	
	\subsection*{Maclaurin Series}
	\[f(x) = f(0) + f^{\prime}(0) + \frac{f^{\prime \prime}(0)}{2!} + \frac{f^{\prime\prime \prime}(0)}{3!} + \ldots \]
	%==========================================================================================%
	
	\subsection*{Hyperbolic Functions }
	
	\begin{multicols}{2}
	\[ \cosh(x)  =  \frac{e^{x} + e^{-x}}{2} \]
	
	\[ \sinh(x)  = \frac{e^{x} - e^{-x}}{2} \]
	\end{multicols}

	
	%==========================================================================================%
	

	\subsection*{Integration}
	
	Integration by parts: 
	
	\[ \int u dv = uv - \int v du \]  
	
\noindent Further formulae and special cases on pages 25 \& 26 of the log tables provided.
	
	\subsection*{Dynamics}
	Where $s(t)$ denotes displacement at time $t$, $v(t)$ denotes the velocity at time $t$ and $a(t)$
	denotes the acceleration at time $t$, 
	\begin{multicols}{2}
	\[  \frac{ds(t)}{dt}  = v(t),\]
	\[  \frac{dv(t)}{dt}  = a(t).\]
	\end{multicols}
	\subsection*{Electrical Circuits}
	Where $q(t)$ denotes the charge at time $t$ and $i(t)$ denotes the current at time $t$,
	\[  \frac{dq(t)}{dt}  = i(t).\]
	
	

\end{document}

	\subsection*{Rules of Differentiation}
	%% - http://media.wix.com/ugd/b064dd_62ab7d7000c34b7eb4a8e2108269cf27.pdf
	\begin{description}
		\item[Product Rule:]  with $y = uv$
		
		
		\[ \frac{dy}{dx} = u \frac{dv}{dx} +  v \frac{du}{dx} \]
		
		\item[Quotient Rule:] \[ y = \frac{u}{v}\]
		\[ \frac{dy}{dx}  = \frac{v \frac{du}{dx} - u \frac{dv}{dx} }{v^2} \]
		
		
		
		\item[Chain Rule:]
		
		% y = f(u) and u = u(x), that is y = f(u(x)) ⇒ dy
		% dx
		\[ \frac{dy}{dx} = \frac{dy}{du} \times \frac{du}{dx}  \]
	\end{description}
	%==========================================================================================%
