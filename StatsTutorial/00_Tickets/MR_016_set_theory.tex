\documentclass[a4paper,12pt]{article}
%%%%%%%%%%%%%%%%%%%%%%%%%%%%%%%%%%%%%%%%%%%%%%%%%%%%%%%%%%%%%%%%%%%%%%%%%%%%%%%%%%%%%%%%%%%%%%%%%%%%%%%%%%%%%%%%%%%%%%%%%%%%%%%%%%%%%%%%%%%%%%%%%%%%%%%%%%%%%%%%%%%%%%%%%%%%%%%%%%%%%%%%%%%%%%%%%%%%%%%%%%%%%%%%%%%%%%%%%%%%%%%%%%%%%%%%%%%%%%%%%%%%%%%%%%%%
\usepackage{eurosym}
\usepackage{vmargin}
\usepackage{amsmath}
\usepackage{graphics}
\usepackage{epsfig}
\usepackage{subfigure}
\usepackage{multicol}
\usepackage{fancyhdr}
\usepackage{amssymb}

\setcounter{MaxMatrixCols}{10}
%TCIDATA{OutputFilter=LATEX.DLL}
%TCIDATA{Version=5.00.0.2570}
%TCIDATA{<META NAME="SaveForMode" CONTENT="1">}
%TCIDATA{LastRevised=Wednesday, February 23, 2011 13:24:34}
%TCIDATA{<META NAME="GraphicsSave" CONTENT="32">}
%TCIDATA{Language=American English}

\pagestyle{fancy}
\setmarginsrb{20mm}{0mm}{20mm}{25mm}{12mm}{11mm}{0mm}{11mm}
\lhead{MathsResource.com} \rhead{Tutorial Sheet 1}
\chead{Set Theory  }
%\input{tcilatex}

\begin{document}

\begin{enumerate}
\item Describe the following sets by the listing method. E :•C5} 
\begin{enumerate}
\item (h) {b E Z : -1 < b < 4}} 
    \item (c) {s : s is an odd integer and 2 C s < 10}     \item (2ni : Tr; E 7, and 5 < m < 10)     \item(2' t E 2, and 0 < 5). 
\end{enumerate}

\item Describe the following sets by giving a 
suitable universal set and rules of inclusion (note: there is more than one correct answer in each case). 
\begin{enumerate}[(a)]
\item $\{l2, 13, 14, 15, 16, 17\}$
\item  {0,5. 5,10,- )0,15, -.15, 
\item  $\{-2, - 3, -4, ... 101\}$ 
\item $\{6,8,10,12,14, 16,18\}$ 
\item $\{1,5,52,53,54-4 \}$
\item $\{0.1,0.01: 0.0014.0001,. .1. \}$
\end{enumerate}

\item Which of the following sets are equal? 
{1.Y. z} (z. x;!/) ; {x; Y; z, Y; z} ; Give the cardinality of each. l. 
\item For the set $V= \{a, e, t, 0, u\}$, give the 5-bit binary string that codes each of the following subsets: {e}; V; 0. Which subset 
is represented by the 5-bit string 10001? 
\item Let $S = \{0, 1, 2, 3,4,5\}$ and A = $\{2,4,5\}$. Put the correct sign, $\in$ or $\subset$ , between each of the following pairs: 
{y,z,x,x). 
A, S; 3, S: 0, 5; O, S; 5,2(S); A, 2(S) 
\item Draw a Venn diagram to represent the universal set $U = \{0, 1, 2, 3, 4, 5, 6\}$ with subsets A :IT (2 4, 5'! and $B= \{1,4,5, 0\}$. 

Find each of the following: 

\begin{enumerate}
\item 
A LAB; 
\ite A n 13; 
\ite (A U BY; 
\item A- B; B A:
\item A (D B. 
\end{enumerate}
\item Leto, B he subsets of a universal set U Construct a membership table for the sets A and B and add columns 
for .4 - B, (A B)', A' and .4' B. Hence prove that (A - B)' = A' U B. Illustrate this result on a Venn diagram. 

\item 
S Let 4, f3 he subsets of a universal qet U. (a) Use membership tab:es to prove De Morgan's Law that (A U = As n 13' . 
(I)) is this law to show that 
(A' Li = A n B. 
\item  (a) Draw a Venn diagram to show three subsets A, B, C of a universal set U intersecting in the mast general way. Shade the region corresponding 
to the subset X defined by the membership table below. 
\begin{enumerate}[(a)]
\item Repeat part f a) for each of the sul:sets Y and Z in place of X . 
\item How are the sets X and 2 related? -?,c1) Can you describe each of the subsets X, Y and Z in terms of the sets A, f3, C, using the operations union, 
intersection and set complement? 
\end{enumerate}
\begin{center}
\begin{tabular}{|c|c|c|c|}
A	&	B	&	C	&	Y  	\\ \hline
0	&	0	&	0	&		\\ \hline
0	&	0	&	1	&		\\ \hline
0	&	1	&	0	&		\\ \hline
0	&	1	&	1	&		\\ \hline
1	&	0	&	0	&		\\ \hline
1	&	0	&	1	&		\\ \hline
1	&	1	&	0	&		\\ \hline
1	&	1	&	1	&		\\ \hline
0	&	0	&	0	&		\\ \hline
0	&	0	&	1	&		\\ \hline
0	&	1	&	0	&		\\ \hline
0	&	1	&	1	&		\\ \hline
1	&	0	&	0	&		\\ \hline
1	&	0	&	1	&		\\ \hline
1	&	1	&	0	&		\\ \hline
1	&	1	&	1	&		\\ \hline
\end{tabular}
\end{center}
Let A, B, C be subsets of a. universal set U. 
\begin{enumerate}[(a)]
    \item  Construct membership tables For each of the sets (A B) — C and A -- — C).     \item  Which, if any, of the following statements are true for all subsets A B, C?     \item  (A— 11) — C = A — (B     \item  (A B) C C A — (B — C)     \item  A — (13 C)     \item  (A • B) 

\end{enumerate}
\end{enumerate}
\newpage
%--------------------------------- %
\section*{The Universal Set and the Empty Set}
\begin{itemize}
	\item The first is the \textbf{\textit{universal set}}, typically denoted $U$. This set is all of the elements that we may choose from. This set may be different from one setting to the next. 
	
	\item For example one universal set may be the set of all real numbers, denoted $\mathbb{R}$, whereas for another problem the universal set may be the whole numbers $\{0, 1, 2,\ldots\}$.
	
	\item The other set that requires consideration is called the \textit{\textbf{empty set}}. The empty set is the unique set is the set with no elements. We write this as $\{ \}$ and denote this set by $\emptyset$.
\end{itemize}
%---------------------------------%
\section*{Number Sets}
The font that the following symbols are written in (i.e. $\mathbb{N}$, $\mathbb{R}$) is known as \textit{\textbf{blackboard font}}.
\begin{itemize}
	\item $\mathbb{N}$ Natural Numbers ($1,2,3,\ldots$) 
%	\textit{(Not used in the CIS102 Syllabus)}
	\item $\mathbb{Z}$ Integers ($-3,-2,-1,0,1,2,3, \ldots$)
	\begin{itemize}
		\item[$\bullet$] $\mathbb{Z}^{+}$ Positive Integers
		\item[$\bullet$] $\mathbb{Z}^{-}$ Negative Integers
		\item[$\bullet$] 0 is not considered as either positive or negative.
	\end{itemize}
	\item $\mathbb{Q}$ Rational Numbers
	\item $\mathbb{R}$ Real Numbers
	\item $\mathbb{C}$ Complex Numbers
\end{itemize}
%========================================================================================= %
\newpage
\section*{Rules of Inclusion, Listing and Cardinality}
For each of the following sets, a set is specified by the rules of inclusion method and listing method respectively. Also stated is the cardinality of that data set.
\subsection*{Worked example 1}
\begin{itemize}
\item $\{ x : x $ is an odd integer $ 5 \leq x \leq 17 \}$
\item $x = \{5,7,9,11,13,15,17\}$
\item The cardinality of set $x$ is 7.
\end{itemize}

\subsection*{Worked example 2}
\begin{itemize}
\item $\{ y : y $ is an even integer $ 6 \leq y < 18 \}$
\item $y = \{6,8,10,12,14,16\}$
\item The cardinality of set $y$ is 6.
\end{itemize}

\subsection*{Worked example 3}
A perfect square is a number that has a integer value as a
square root. 4 and 9 are perfect squares ($\sqrt{4} = 2$,
$\sqrt{9} = 3$).
\begin{itemize}
\item $\{ z : z $ is an perfect square $ 1 < z < 100 \}$
\item $z = \{4,9,16,25,36,49,64,81\}$
\item The cardinality of set $z$ is 8.
\end{itemize}
\item Let A,B be subsets of the universal set U.

\newpage

\subsection*{Exercises}
For each of the following sets, write out the set using the listing method.
Also write down the cardinality of each set.

\begin{itemize} 
\item $\{ s : s $ is an negative integer $ -10 \leq s \leq 0 \}$
\item $\{ t : t $ is an even number $ 1 \leq t \leq 20 \}$
\item $\{ u : u $ is a prime number $ 1 \leq u \leq 20 \}$
\item $\{ v : v $ is a multiple of 3 $ 1 \leq v \leq 20 \}$
\end{itemize}
%-------------------------------------------------% 
\newpage
\section*{Power Sets}
\subsection*{Worked Example}
Consider the set $Z$:
\[ Z = \{ a,b,c\}  \]
\begin{itemize}
\item[(i)] How many sets are in the power set of $Z$? 
\item[(ii)] Write out the power set of $Z$. 
\item[(iii)] How many elements are in each element set?
\end{itemize}
%----------------------------------------------%
\subsection*{Solutions to Worked Example}

\begin{itemize}


\item[(i)] There are 3 elements in $Z$. So there is $2^3 = 8$ element sets contained in the power set.

\item[(ii)] Write out the power set of $Z$.
\[ \mathcal{P}(Z) = \{ \emptyset, \{a\}, \{b\}, \{c\}, \{a,b\}, \{a,c\}, \{b,c\}, \{a,b,c\} \} \]

\item[(iii)]
\begin{itemize}
\item[$\bullet$] One element set is the null set - i.e. containing no
elements \item[$\bullet$] Three element sets have only elements \item[$\bullet$]
Three element sets have two elements \item[$\bullet$] One element set
contains all three elements \item[$\bullet$] 1+3+3+1=8
\end{itemize}
\end{itemize}
\subsection*{Exercise}
For the set $Y = \{u,v,w,x\}$ , answer the questions from the
previous exercise


%------------------------------------------------------%

\section*{Complement of a Set}
%(2.3.1) 
Consider the universal set $U$ such that
\[U=\{2,4,6,8,10,12,15\} \]
For each of the sets $A$,$B$,$C$ and $D$, specify the complement sets.
{
	\LARGE
\begin{center}
\begin{tabular}{|c|c|}
  \hline
Set &\phantom{sp} Complement \phantom{sp}\\
\hline \phantom{sp} $A=\{4,6,12,15\}$ \phantom{sp} &
$A^{\prime}=\{2,8,10\}$ \\ \hline $B=\{4,8,10,15\}$ & \\ \hline
$C=\{2,6,12,15\}$ & \\ \hline $D=\{8,10,15\}$ & \\ \hline

\end{tabular}
\end{center}
}
%-------------------------------------%
 % Binary Operations on Sets (2.3.2)
 % Union , Intersection, Symmetric Difference
 % Set Difference

%======================================================================================= %
\newpage
\section*{Set Operations}
\begin{itemize}
	\item Union ($\cup$) - also known as the \textbf{OR} operator. A union signifies a bringing together. The union of the sets A and B consists of the elements that are in either A or B.
	\item Intersection ($\cap$) - also known as the \textbf{AND} operator. An intersection is where two things meet. The intersection of the sets A and B consists of the elements that in both A and B.
	\item Complement ($A^{\prime}$ or $A^{c}$) - The complement of the set A consists of all of the elements in the universal set that are not elements of A.
\end{itemize}

\subsection*{Exercise}
Consider the universal set $U$ such that
\[U=\{1,2,3,4,5,6,7,8,9\} \] 
and the sets
\[A=\{2,5,7,9\} \] 
\[B=\{2,4,6,8,9\} \]

\begin{multicols}{2}
\begin{itemize}
	\item[(a)] $A-B$
	\item[(b)] $A \otimes B$
	\item[(c)] $A \cap B$
	\item[(d)] $A \cup B$
	\item[(e)] $A^{\prime} \cap B^{\prime}$
	\item[(f)] $A^{\prime} \cup B^{\prime}$
\end{itemize}
\end{multicols}

%======================================================================================= %
\newpage

\section*{Venn Diagrams}




%------------------------------------------------------------------------%
\begin{itemize}
	\item[(i)] Describe the following set by the listing method
	\[ \{ 2r+1 : r \in Z^{+} and r \leq 5  \} \]
	\item[(ii)] Let A,B be subsets of the universal set U.
	
	
\end{itemize}

%================================================================ %
\subsection*{Question 8}
\begin{itemize}
	
	\item[(i)] 
	
	\item[(ii)]
	
	\item[(iii)]
	
\end{itemize}

\end{document}



   




%-------------------------------------% Ellipsis

When using Ellipsis, it should be clear what the pattern is

%-------------------------------------%


%-----------Reference to section 2.2.3 Power Sets




%-------------------------------------%

 %----(Reference to Section 2.2.2 Cardinality)



%-------------------------------------% % Complement of a set


%--------------------------------------%
\subsection*{ Three Sets }

%- Section 2.3.5 %- Associative Law %- Distribution Law





%-------------------------------------% %- Section 3
Propositional Logic A statement is a declarative sentence that
is either true or false.
\begin{itemize}
\item $\tilde q$ not q \item $p \vee q$ \item $p \wedge \tilde
q$
\end{itemize}



%======================================================================================= %

Question 5

-----------------------------------------------------


--------------------------------------------

Let A, B be subsets of the universal set \mathcal{U}.

Use membership tables to prove De Morgan's Laws.



Construct Membership tables for each of the sets
(A-B) - C
A-(B- C)

(A-B) -C = A-(B-C)
A



\begin{itemize}
\item[a.] (1 mark) Write out the sample space for the outcomes for both players A and B.
\item[b.] (1 mark) Write out the sample space for the outcomes of C, where C is the difference of the two scores (i.e. B-A)
\item[c.] (1 mark) Are the sample points for the sample space of C equally probable? Provide a brief justification for your answer.
\end{itemize}

%----------------------------------------------------------%
\newpage
\section*{Section B: Set Operations}
\begin{itemize}
\item[B.1] complement of A $A^{\prime}$
\item[B.2] Union $A \cup B$
\item[B.3] Intersection $A \cap B$
\item[B.4] Relative Difference $A \otimes B$
\item[A.5]
\item[A.6]
\item[A.7]
\item[A.8]
\end{itemize}
\newpage


\begin{itemize}
\item Specifying Sets
\item Listing Method
\item Rules of Inclusion method
\end{itemize}


\begin{itemize}
\item Subsets Notation of a subset
\item Cardinality of a set
\item Power of a set
\end{itemize}

\subsection*{Operation on Sets}

\begin{itemize}
\item The complement of Set
\item Binary Operations
\begin{itemize}
\item Union
\item Intersection
\end{itemize}
\item Membership tables
\item Laws for Combining Sets
\end{itemize}

\newpage


\subsection*{Associative Laws}
\[ (A \cup B) \cup C =  A \cup (B \cup C)  \]
\[ (A \cap B) \cap C =  A \cap (B \cap C)  \]

\subsection*{Distributive Laws}
\[ (A \cup B) \cap C =  (A \cup B) \cap (A \cup C)  \]
\[ (A \cap B) \cup C =  (A \cap B) \cup (A \cap C)  \]


\[ (A \cup B) \cap B^{\prime} \]
%----------------------------------------------------------%

\end{document} 
