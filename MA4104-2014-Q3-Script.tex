
\documentclass[]{article}

\usepackage{framed}
\usepackage{amsmath}
\begin{document}
	
	\Large
	
	\section*{MA4104 - Question 3 - Inference Procedures}
	%===============================================================%
	\begin{framed}
		\noindent \textbf{Types of Inference Procedure}
		\begin{enumerate}
			\item Confidence Intervals
			\item Hypothesis Testing
		\end{enumerate}
	\end{framed}
	\medskip
	\begin{framed}
		\noindent \textbf{Parameter}
		\begin{enumerate}
			\item Single-Sample Means
			\item Difference of Means
			\item Single-Sample Proportions
			\item Difference of Proportions
		\end{enumerate}
	\end{framed}
	\medskip
	\begin{framed}
		\noindent \textbf{Types of Sample Size}
		\begin{enumerate}
			\item Sample Size is usually denoted $n$
			\item Large Sample ($n>30$)
			\item Hypothesis Testing
		\end{enumerate}
	\end{framed}
	
%===============================================================%
\newpage
\subsection*{Question 3 - Part A}

\begin{itemize}
	\item A business would like to compare customer ratings (on a scale of 0-10, where \textbf{0 = poor} and 10 = excellent) of two new products (A and B). 
	\item A sample of 80 customers were randomly assigned to rate either Product A or Product B. 
	\item The sample data are summarised as:
\end{itemize}

\begin{array}{ccc}
& Product A & Product B \\

Sample size &  40 & 40 \\

Mean & 7.2 & 6.1 \\

Standard Deviation & 2.3 & 3.8 \\ 
\end{array} 

\begin{itemize}
\item[(i)] At the 5\% significance level, test whether product ratings differ between Product A and Product B. Interpret the results.

\item[(ii)] Compute and interpret a 95\% confidence interval for the difference in mean product ratings between product A and product B.
\end{itemize}
%- (13 marks)

\newpage
(b) An accountant is interested in investigating whether the number of loans defaulting within one year in a credit institution has significantly changed from 2013 to 2014. He selects a random sample of loans from the loan book and summarises the data in the following table.

Year Total 2013 2014

Default No 115 85 200 Yes 20 7 27

Total 135 92 227

\begin{itemize}
	

\item[(i)] What is the percentage of loans taken out in 2014 that have defaulted?

(2 marks)

\item[(ii)] Test whether there is an association between year and default, stating clearly the null and alternative hypotheses. What conclusions would you make following the analysis? (Note: ) (8 marks)

\item[(iii)] What condition must be satisfied for the Chi-square test to be valid? Do the data in this question satisfy this condition? Justify your answer.

(2 marks)
\end{itemize}

\end{document}