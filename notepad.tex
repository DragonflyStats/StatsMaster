
\documentclass{beamer}

\usepackage{default}

\begin{document}
The normal yield value quoted for a bond is the flat yield which is the fixed coupon divided by the current bond price as a percentage. The Gross Redemption Yield (GRY) takes into account not only the flat yield but also any capital gain received by holding the bond to maturity.

If for example a bond has a maturity value of £100 in 2 years time but is bought now for £95, there will be a capital gain of 5\% over two years (approxiamtely 2.47\% per annum) to add to the flat yield, if the bond is held to maturity. If the flat yield is 6\% per annum then the Gross Redemption Yield is 8.47\% per annum.


\end{document}

%----------------------------------------------------------------------------------------------%

RSS Grad Dip 3 - Question 3B2
 
 

SPECIMEN A

SPECIMEN A
6. Consider the AR(2) model
 
 
\[Y_t={1\over 3}Y_{t-1}+{1\over 12}Y_{t-2}+ \epsilon_t\]

for a process {Y_t}, where {\epsilon_t} is a white noise process.($-\infty \leq t \leq \infty$)

(i) Find the roots of the autoregressive characteristic equation and check that the stationarity condition is satisfied.(4)
 
(ii) Find the Yule-Walker equations that are satisfied by the autocorrelation function t.(4)
 
(iii) Obtain the value of $\rho_1$        (3)
 
(iv) Show that a general expression for the autocorrelation function is given by
\[\rho_t={35\over 44}{1\over 12}^t +{9\over 44}{-1\over 6}^t
\]

where $\tau \geq 0$     (9)


<hr>



Simple Spreadsheet commands that can used to convert decimal numbers to binary or hexadecimal, and vice versa.

Bin2Dec
Dec2bin
hex2dec
dec2hex

%----------------------------------------------------------------------------------------------%

%-----------------%
R Prop.test
%-----------------%
5
Pearson Correlation SS identities
For Loop
model goodness of fit metrics
transport NWC
transport LC
transport VAM shadow costs
10
%------------------%
MM3CB
15 
Normal Z score
Normal Using tables
Normal Complement Rule
Normal Interval Rule 
Normal Symmetry Rule  


robust regression
 anova
 interaction plots

Logistic regression
 SA heart data
 
this data set is available in the elemstatlearn package
 fit a logistic regression model where the chd us the binary outcome, and all other variables are used as predictor variables
 
use the AIC criterion 
http://aimotion.blogspot.ie/2011/11/machine-learning-with-python-logistic.html?m=1
 
http://www.datarobot.com/blog/classification-with-scikit-learn/

%%%%%%%%%%%%%%%%%%%%%%%%%%%%%%%%%%%%%%%%%%%%%%%%%%%%%%%%%%%%%%%%%5

Attribute charts
 
Variable charts
 
multivariate charts
 
np charts
>
 > Part D checklist
 >
 > use of colours in plot

 > scatterplot abline
 residual plots
 cooks distance

Part D checklist
 
use of colours in plot
 scatterplot
 title and subtitle in plot
 Grubbs Test For Outlier
 log transformation
Set theory
 Set operations
 complement
 union intersection
 set difference
 venn diagrams
 8 Disjoint Regions
 membership tables
 
proof by truth tables
 AND
 OR
 NOT
 Set difference
 symmetric difference



Normal Distribution 
normal probability plot 

( qqplot boxplots histograms density plots)
 %-----------------------------------------------%
 skew and kurtosis describe function in the psych package 
grubbs test for outliers (outliers package)
 shapiro wilk test
 
anderson darling test (package normtest)
 
kolmogorov smirnov test for a single sample
 
correlation. Cor.test
 log transformation
 %----------------------------------------------%
 The t-tests 
variance test boxplots
 two sample tests 
paired t tests
 %-----------------------------------------------------%
 non parametric distributions
 ks-test
 wilcixon test for medians
Install.packages("packagename")
 library(packagename)
 #########
 data(datasetname) 
help(datasetname)
 
summary(datasetname)
 names(datasetname)
 
##########
 summary(datasetname$variablename)
 
## or 
attach(datasetname)
 summary(variablename)
 detach(datasetname)
 
##########
 
class(objectname)
 names(objectname)
 str(objectname)
 mode(objectname)
 
##########
 plot(variablename)
 plot(variablename1,variablename2)
 hist(variablename)

%%%%%%%%%%%%%%%%%%%%%%%%%%%%%%%%%%%%%%%%%%%%%%%%%%%%%%%%%%%%%%%%%%%%%%%%%%%%%%%%%%%%%%%%%%%%%%%%%%%%%%5


ANOVA with R
 
The null hypothesis is that population mean is the same for each of the groups.
 
The alternative hypothesis is that there is at least  one group with a different population mean diffetent to the rest
 
the first tjing we do is to re-write the data in long form
 
Important :remember to construct tbe grouping variable as a factor. 


<hr>
Categorical variables are referred to by R (and in fact several branches of statistics as "factors". 

The categories for a factor are known as levels. 

Sex is an example of a factor variables, where the levels are Male and Female. 

Factors differ from character vectors. R woyld nit recognize a character variable as indicating membership of a category, rather it would just identify it as a piece of text..


 
-------------------------------------------

 
When computing confidence intervals or performing hypothesis tests, the most computationally complex component is the calculation of the standard error.
 
For each procedure, there is a standard error formula. These are usually given in an appendix to examination papers. It is strongly recommended that you familiarise yourself with these before your exam.



Scatterplots are useful for determining outliers. There is no single definition of an outlier, rather it depends on the type of analysis being used.

%%%%%%%%%%%%%%%%%%%%%%%%%%%%%%%%%%%%%%%%%%%%%%%%%%%%%%%%%%%%%%%%%%%%%%%%%%%%%%%%%%%%%%%%%%%%%%%

Medical statistic hypothesis testing
 
Type I error This describes a hypothesis test in which the null hypothesis was rejected, when it should not have been. 

Type II error This describes a hypothesis test in which the null hypothesis was rejected when it was in fact true.
 
