
\section{Weighted Regression}

\textbf{Homoscedasticity} - the standard deviations of
y-observations from the straight line are the same independently
of the underlying x-observations.

\textbf{Heteroscedasticity} - the standard deviations of
y-observations depend on the underlying x-observations.

In the first case, standard regression analysis should be
performed, while in the second the weighted regression is more
suitable.

\begin{verbatim}
>Conc=c(0,2,4,6,8,10)
>StDev=c(0.001,0.004,0.010,0.013,0.017,0.022)
>Abs=c(0.009,0.158,0.301,0.472,0.577,0.739)
>n=length(Conc)
>weights=StDevˆ(-2)/mean(StDevˆ(-2))
>wreg=lm(Abs˜Conc,weights=weights)
>reg=lm(Abs˜Conc)
>summary(wreg)
\end{verbatim}


It is often convienent to express the regression analysis using
ANOVA table. The following equation is the basis for such
representation

It is often shortened to SST = SSLR + SSR; where SST is referred

to as the total sum of squares, SSLR is the sum of squares due to
linear regression (within regression), SSR is the sum of squares
due to residuals (outside regression).
