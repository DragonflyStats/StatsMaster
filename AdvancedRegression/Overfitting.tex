\documentclass[MASTER.tex]{subfiles}

\begin{document}
\section{Overfitting} 
Overfitting describes the error which occurs when a fitted model is too 
closely fit to a limited set of observations. Overfitting the model generally 
takes the form of making an overly complex model (i.e. using an 
excessive amount of independent variables) to explain the behaviour in 
the data under study. 
 
In reality, the data being studied often has some degree of error or 
random noise within it. Thus attempting to make the model conform too 
closely to sample data can undermine the model and reduce its predictive 
power. 
 
% (Remark: This will be the basis for a lab exercise) 
 
\end{document}
