% MA4128

% https://www3.nd.edu/~rwilliam/stats1/OLS-SPSS.pdf
%http://psychweb.psy.umt.edu/denis/datadecision/front/stat_II_2011/psyx_521_multiple_regression_part_II.pdf

% ftp://public.dhe.ibm.com/software/analytics/spss/documentation/statistics/20.0/en/client/Manuals/IBM_SPSS_Regression.pdf

%http://www.education.umd.edu/EDMS/fac/Harring/Past-Classes/EDMS651/Notes/LRA-3.pdf

% http://www.psych.yorku.ca/lab/psy6140/lectures/ModelSelection2x2.pdf

%http://philosophy.wisc.edu/forster/220/simplicity.html

%http://statistics.uchicago.edu/~s220e/Lect/lec21.pdf

%http://publib.boulder.ibm.com/infocenter/spssstat/v20r0m0/index.jsp?topic=%2Fcom.ibm.spss.statistics.help%2Fsyn_regression_criteria_variable_selection.htm

\documentclass[a4paper,12pt]{article}
%%%%%%%%%%%%%%%%%%%%%%%%%%%%%%%%%%%%%%%%%%%%%%%%%%%%%%%%%%%%%%%%%%%%%%%%%%%%%%%%%%%%%%%%%%%%%%%%%%%%%%%%%%%%%%%%%%%%%%%%%%%%%%%%%%%%%%%%%%%%%%%%%%%%%%%%%%%%%%%%%%%%%%%%%%%%%%%%%%%%%%%%%%%%%%%%%%%%%%%%%%%%%%%%%%%%%%%%%%%%%%%%%%%%%%%%%%%%%%%%%%%%%%%%%%%%
\usepackage{eurosym}
\usepackage{vmargin}
\usepackage{amsmath}
\usepackage{graphics}
\usepackage{epsfig}
\usepackage{framed}
\usepackage{subfigure}
\usepackage{fancyhdr}

\setcounter{MaxMatrixCols}{10}
%TCIDATA{OutputFilter=LATEX.DLL}
%TCIDATA{Version=5.00.0.2570}
%TCIDATA{<META NAME="SaveForMode"CONTENT="1">}
%TCIDATA{LastRevised=Wednesday, February 23, 201113:24:34}
%TCIDATA{<META NAME="GraphicsSave" CONTENT="32">}
%TCIDATA{Language=American English}

\pagestyle{fancy}
\setmarginsrb{20mm}{0mm}{20mm}{25mm}{12mm}{11mm}{0mm}{11mm}
\lhead{MA4128} \rhead{Kevin O'Brien} \chead{Week 8} %\input{tcilatex}

%http://www.electronics.dit.ie/staff/ysemenova/Opto2/CO_IntroLab.pdf
\begin{document}

\section{Weighted Regression}

\textbf{Homoscedasticity} - the standard deviations of
y-observations from the straight line are the same independently
of the underlying x-observations.

\textbf{Heteroscedasticity} - the standard deviations of
y-observations depend on the underlying x-observations.

In the first case, standard regression analysis should be
performed, while in the second the weighted regression is more
suitable.

\begin{verbatim}
>Conc=c(0,2,4,6,8,10)
>StDev=c(0.001,0.004,0.010,0.013,0.017,0.022)
>Abs=c(0.009,0.158,0.301,0.472,0.577,0.739)
>n=length(Conc)
>weights=StDevˆ(-2)/mean(StDevˆ(-2))
>wreg=lm(Abs˜Conc,weights=weights)
>reg=lm(Abs˜Conc)
>summary(wreg)
\end{verbatim}


It is often convienent to express the regression analysis using
ANOVA table. The following equation is the basis for such
representation

It is often shortened to SST = SSLR + SSR; where SST is referred

to as the total sum of squares, SSLR is the sum of squares due to
linear regression (within regression), SSR is the sum of squares
due to residuals (outside regression).

\section{Weighted Regression}
Unweighted regression requires that the variability of the
residuals is constant over the measured range of values.
(This is called homoskedasticity).

Weighted regression does not have this requirement.
There may be differing variability over the range of values.
(This is called heteroskedasticity).

Weighted regression requires extra information on the standard deviations of the responses so as to compute the weights.

Unweighted regression doesn’t need or use any information on the response standard deviations.

Weighted regression is preferable if heteroskedasticity evident in the data

(If there is not constant variance for the residuals over the range of values)


\section{Weighted Regression}

\textbf{Homoscedasticity} - the standard deviations of
y-observations from the straight line are the same independently
of the underlying x-observations.

\textbf{Heteroscedasticity} - the standard deviations of
y-observations depend on the underlying x-observations.

In the first case, standard regression analysis should be
performed, while in the second the weighted regression is more
suitable.

\begin{verbatim}
>Conc=c(0,2,4,6,8,10)
>StDev=c(0.001,0.004,0.010,0.013,0.017,0.022)
>Abs=c(0.009,0.158,0.301,0.472,0.577,0.739)
>n=length(Conc)
>weights=StDevˆ(-2)/mean(StDevˆ(-2))
>wreg=lm(Abs˜Conc,weights=weights)
>reg=lm(Abs˜Conc)
>summary(wreg)
\end{verbatim}

\end{document}
