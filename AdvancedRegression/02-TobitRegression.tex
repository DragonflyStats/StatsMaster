
\documentclass[Master.tex]{subfiles}
\begin{document}


	
	%lm help file screen shot
	\section{Introduction}
%%-http://www.ats.ucla.edu/stat/r/dae/tobit.htm
%%-----------------------------------------------------------------------------------%%


\begin{itemize}
\item	
	Censored regression models commonly arise in econometrics in cases where the variable of interest is only observable under certain conditions. 

\item A common example is labor supply. Data are frequently available on the hours worked by employees, and a labor supply model estimates the relationship between hours worked and characteristics of employees such as age, education and family status. 
\item However, such estimates undertaken using linear regression will be biased by the fact that for people who are unemployed it is not possible to observe the number of hours they would have worked had they had employment. 

\item Still we know age, education and family status for those observations.
A model commonly used to deal with censored data is the Tobit model, including variations such as the Tobit Type II, Type III, and Type IV models.

\item Censored regression models are usually estimated using maximum likelihood estimation. The general validity of this approach has been shown by Schnedler in 2005, who also provides a method to find the likelihood for a broad class of applications.[1]
\end{itemize}

\end{document}

%===============================================================================================================================%
Tobit Regression

The Tobit model is a statistical model proposed by James Tobin (1958)[1] to describe the relationship between a non-negative dependent variable y_i and an independent variable (or vector) x_i. The term Tobit was derived from Tobin's name by truncating and adding -it by analogy with the probit model.[2]

The model supposes that there is a latent (i.e. unobservable) variable y_i^*. This variable linearly depends on x_i via a parameter (vector) \beta which determines the relationship between the independent variable (or vector) x_i and the latent variable y_i^* (just as in a linear model). In addition, there is a normally distributed error term u_i to capture random influences on this relationship. The observable variable y_i is defined to be equal to the latent variable whenever the latent variable is above zero and zero otherwise.

y_i = \begin{cases} 
    y_i^* & \textrm{if} \; y_i^* >0 \\ 
    0     & \textrm{if} \; y_i^* \leq 0
\end{cases}
where y_i^* is a latent variable:

 y_i^* = \beta x_i + u_i, u_i \sim N(0,\sigma^2) \, 


%===================================================================%

Tobit regression

%%- http://www.karlin.mff.cuni.cz/~pesta/NMFM404/tobit.html
%%- Really good
%====================%

The tobit model, also called a censored regression model, is designed to estimate 
linear relationships between variables when there is either left- or right-censoring in 
the dependent variable (also known as censoring from below and above, respectively). 

Censoring from above takes place when cases with a value at or above some threshold, all 
take on the value of that threshold, so that the true value might be equal to 
the threshold, but it might also be higher. In the case of censoring from 
below, values those that fall at or below some threshold are censored.

%===================================================================%
%- http://www.iasri.res.in/sscnars/socialsci/1-Tobit%20analysis.pdf
%- http://www.aae.wisc.edu/aae637/handouts/roncek_tobit_article.pdf
%======================================================================== %
% [fragile]
Description of the Data
For our data analysis below, we are going to expand on Example 3 from above. We have generated hypothetical data, which can be obtained from our website from within R. Note that R requires forward slashes, not back slashes when specifying a file location even if the file is on your hard drive.


\begin{verbatim}
dat <- read.csv("http://www.ats.ucla.edu/stat/data/tobit.csv")
\end{verbatim}

The dataset contains 200 observations. The academic aptitude variable is apt, the reading and math test scores are read and math respectively. The variable prog is the type of program the student is in, it is a categorical (nominal) variable that takes on three values, academic (prog = 1), general (prog = 2), and vocational (prog = 3). The variable id is an identification variable.

Now let's look at the data descriptively. Note that in this dataset, the lowest value of apt is 352. That is, no students received a score of 200 (the lowest score possible), meaning that even though censoring from below was possible, it does not occur in the dataset.
\begin{verbatim}
\begin{framed}
summary(dat)
##        id             read           math              prog    
##  Min.   :  1.0   Min.   :28.0   Min.   :33.0   academic  : 45  
##  1st Qu.: 50.8   1st Qu.:44.0   1st Qu.:45.0   general   :105  
##  Median :100.5   Median :50.0   Median :52.0   vocational: 50  
##  Mean   :100.5   Mean   :52.2   Mean   :52.6                   
##  3rd Qu.:150.2   3rd Qu.:60.0   3rd Qu.:59.0                   
##  Max.   :200.0   Max.   :76.0   Max.   :75.0                   
##       apt     
##  Min.   :352  
##  1st Qu.:576  
##  Median :633  
##  Mean   :640  
##  3rd Qu.:705  
##  Max.   :800
# function that gives the density of normal distribution
# for given mean and sd, scaled to be on a count metric
# for the histogram: count = density * sample size * bin width
f <- function(x, var, bw = 15) {
	dnorm(x, mean = mean(var), sd(var)) * length(var)  * bw
}
\end{verbatim}
\end{framed}

\begin{framed}
\begin{verbatim}	
# setup base plot
p <- ggplot(dat, aes(x = apt, fill=prog))

# histogram, coloured by proportion in different programs
# with a normal distribution overlayed
p + stat_bin(binwidth=15) +
stat_function(fun = f, size = 1,
args = list(var = dat$apt))
plot of chunk unnamed-chunk-4
\end{verbatim}
\end{framed}
\section{Tobit Regression}
Looking at the above histogram, we can see the censoring in the values of apt, that is, there are far more cases with scores of 750 to 800 than one would expect looking at the rest of the distribution. Below is an alternative histogram that further highlights the excess of cases where apt=800. In the histogram below, the breaks option produces a histogram where each unique value of apt has its own bar (by setting breaks equal to a vector containing values from the minimum of apt to the maximum of apt). 


Because apt is continuous, most values of apt are unique in the dataset, although close to the center of the distribution there are a few values of apt that have two or three cases. The spike on the far right of the histogram is the bar for cases where apt=800, the height of this bar relative to all the others clearly shows the excess number of cases with this value.

\begin{verbatim}
p + stat_bin(binwidth = 1) + stat_function(fun = f, size = 1, args = list(var = dat$apt, 
bw = 1))
\end{verbatim}
plot of chunk unnamed-chunk-5
Next we'll explore the bivariate relationships in our dataset.

\begin{verbatim}

cor(dat[, c("read", "math", "apt")])
##        read   math    apt
## read 1.0000 0.6623 0.6451
## math 0.6623 1.0000 0.7333
## apt  0.6451 0.7333 1.0000
# plot matrix
ggpairs(dat[, c("read", "math", "apt")])
plot of chunk unnamed-chunk-6
\end{verbatim}
\medskip
%======================================================================== %
% [fragile]
In the first row of the scatterplot matrix shown above, we see the scatterplots showing the relationship between read and apt, as well as math and apt. Note the collection of cases at the top these two scatterplots, this is due to the censoring in the distribution of apt.

\medskip

%======================================================================== %

In the output above, the first thing we see is the call, this is R reminding us what the model we ran was, what options we specified, etc.
The table labeled coefficients gives the coefficients, their standard errors, and the z-statistic. No p-values are included in the summary table, but we show how to calculate them below. Tobit regression coefficients are interpreted in the similar manner to OLS regression coefficients; however, the linear effect is on the uncensored latent variable, not the observed outcome. See McDonald and Moffitt (1980) for more details.
For a one unit increase in read, there is a 2.6981 point increase in the predicted value of apt.
A one unit increase in math is associated with a 5.9146 unit increase in the predicted value of apt.

\medskip

The terms for prog have a slightly different interpretation. The predicted value of apt is -46.1419 points lower for students in a vocational program than for students in an academic program.
The coefficient labeled "(Intercept):1" is the intercept or constant for the model.
The coefficient labeled "(Intercept):2" is an ancillary statistic. If we exponentiate this value, we get a statistic that is analogous to the square root of the residual variance in OLS regression. The value of 65.6773 can compared to the standard deviation of academic aptitude which was 99.21, a substantial reduction.


\medskip


The final log likelihood, -1041.0629, is shown toward the bottom of the output, it can be used in comparisons of nested models.
Below we calculate the p-values for each of the coefficients in the model. We calculate the p-value for each coefficient using the z values and then display in a table with the coefficients. The coefficients for read, math, and prog = 3 (vocational) are statistically significant.


\medskip

ctable <- coef(summary(m))
pvals <- 2 * pt(abs(ctable[, "z value"]), df.residual(m), lower.tail = FALSE)
cbind(ctable, pvals)
##                Estimate Std. Error z value      pvals
## (Intercept):1   209.555   32.45655   6.456  3.157e-10
## (Intercept):2     4.185    0.05268  79.432 1.408e-244
## read              2.698    0.61808   4.365  1.625e-05
## math              5.915    0.70480   8.392  8.673e-16
## proggeneral     -12.716   12.35467  -1.029  3.040e-01
## progvocational  -46.142   13.76971  -3.351  8.831e-04
We can test the significant of program type overall by fitting a model without program in it and using a likelihood ratio test.
\medskip
%======================================================================== %
% [fragile]
	% 
		\begin{verbatim}
m2 <- vglm(apt ~ read + math, tobit(Upper = 800), data = dat)

(p <- pchisq(2 * (logLik(m) - logLik(m2)), df = 2, lower.tail = FALSE))
## [1] 0.003155
\end{verbatim}
\end{framed}
The LRT with two degrees of freedom is associated with a p-value of 0.0032, indicating that the overall effect of prog is statistically significant.

Below we calculate the upper and lower 95% confidence intervals for the coefficients.
\medskip
%======================================================================== %
\begin{framed}
\begin{verbatim}
b <- coef(m)
se <- sqrt(diag(vcov(m)))

cbind(LL = b - qnorm(0.975) * se, UL = b + qnorm(0.975) * se)
##                     LL      UL
## (Intercept):1  145.941 273.169
## (Intercept):2    4.081   4.288
## read             1.487   3.909
## math             4.533   7.296
## proggeneral    -36.930  11.499
## progvocational -73.130 -19.154
\end{verbatim}
\end{framed}
\medskip
%======================================================================== %
% [fragile]
We may also wish to examine how well our model fits the data. One way to start is with plots of the residuals to assess their absolute as well as relative (pearson) values and assumptions such as normality and homogeneity of variance.
\medskip
%======================================================================== %
\begin{framed}
\begin{verbatim}
dat$yhat <- fitted(m)[,1]
dat$rr <- resid(m, type = "response")
dat$rp <- resid(m, type = "pearson")[,1]

par(mfcol = c(2, 3))
\end{verbatim}
\end{framed}
\medskip
%======================================================================== %
\begin{framed}
		\begin{verbatim}
with(dat, {
	plot(yhat, rr, main = "Fitted vs Residuals")
	qqnorm(rr)
	plot(yhat, rp, main = "Fitted vs Pearson Residuals")
	qqnorm(rp)
	plot(apt, rp, main = "Actual vs Pearson Residuals")
	plot(apt, yhat, main = "Actual vs Fitted")
})

\end{verbatim}
\end{framed}
\medskip
%======================================================================== %
% [fragile]
%plot of chunk unnamed-chunk-11
The graph in the bottom right was the predicted, or fitted, values plotted against the actual. This can be particularly useful when comparing competing models. We can calculate the correlation between these two as well as the squared correlation, to get a sense of how accurate our model predicts the data and how much of the variance in the outcome is accounted for by the model.
\medskip
%======================================================================== %
\begin{framed}
\begin{verbatim}
# correlation
(r <- with(dat, cor(yhat, apt)))
## [1] 0.7825
# variance accounted for
r^2
## [1] 0.6123
\end{verbatim}
\end{framed}
The correlation between the predicted and observed values of apt is 0.7825. If we square this value, we get the multiple squared correlation, this indicates predicted values share 61.23% of their variance with apt.
\medskip
%======================================================================== %
% [fragile]
References
Long, J. S. 1997. Regression Models for Categorical and Limited Dependent Variables. Thousand Oaks, CA: Sage Publications.

McDonald, J. F. and Moffitt, R. A. 1980. The Uses of Tobit Analysis. The Review of Economics and Statistics Vol 62(2): 318-321.

Tobin, J. 1958. Estimation of relationships for limited dependent variables. Econometrica 26: 24-36.
\medskip
\end{document}
