%%- http://www.stat.ncsu.edu/people/bloomfield/courses/st430-514/slides/MandS-ch09-sec03-03.pdf

Inverse prediction concerns predicting one set of measurements from another set when the 
direction of causation and possibly error operates in the opposite direction. 

An example is that of calibration in which it is desired to use a cheap and quick but error-prone measurement Y, to 
predict the true amount of a constituent x, which in itself can be measured accurately in laboratory conditions but 
at much greater effort or cost. After taking data on samples using both measurements (the training or learning data)
it is desired to use the cheap measurement in future.

%%-- http://www.jmp.com/support/help/Inverse_Prediction_with_Confidence_Limits.shtml
\subsection{Inverse Prediction with Confidence Limits}
Inverse prediction estimates a value of an independent variable from a response value. In bioassay problems, inverse prediction with confidence limits is especially useful. In JMP, you can request inverse prediction estimates for continuous and binary response models. If the response is continuous, you can request confidence limits for an individual response or an expected response.
The confidence limits are computed using Fieller’s theorem (Fieller, 1954), which is based on the following logic. The goal is predicting the value of a single regressor and its confidence limits given the values of the other regressors and the response.
•
Let b estimate the parameters β so that we have b distributed as N(β,V).
•
Let x be the regressor values of interest, with the ith value to be estimated.
•
Let y be the response value.
We desire a confidence region on the value of x[i] such that β’x = y with all other values of x given.
The inverse prediction is

where the parenthesized subscript (i) indicates that the ith component is omitted. A confidence interval can be formed from the relation

where t is the t value for the specified confidence level.
The equation

can be written as a quadratic in terms of z = x[i]:

where



Depending on the values of g, h, and f, the set of values satisfying the inequality, and hence the confidence interval for the inverse prediction, can have a number of forms:
•
an interval of the form (φ1, φ2), where φ1 < φ2
•
two disjoint intervals of the form , where φ1 < φ2
•
the entire real line,
•
only one of  or
In the case where the Fieller confidence interval is the entire real line, Wald intervals are presented.
Note: The Fit Y by X logistic platform and the Fit Model Nominal Logistic personalities use t values when computing confidence intervals for inverse prediction. The Fit Model Generalized Linear Model personality, as well as PROC PROBIT in SAS/STAT, use z values, which give different results.
