\documentclass[]{report}

\voffset=-1.5cm
\oddsidemargin=0.0cm
\textwidth = 480pt

\usepackage{framed}
\usepackage{subfiles}
\usepackage{graphics}
\usepackage{newlfont}
\usepackage{eurosym}
\usepackage{amsmath,amsthm,amsfonts}
\usepackage{amsmath}
\usepackage{color}
\usepackage{amssymb}
\usepackage{multicol}
\usepackage[dvipsnames]{xcolor}
\usepackage{graphicx}
\begin{document}

\section{Summary}
\begin{itemize}
\item 
Statistical inference is the process of making inferences about populations from information provided by samples.
\end{itemize}







Statistical inference is the process of making inferences about populations from information provided by samples.

\section{What is Statistical Inference?}
%---------------------------------------------------%

\begin{itemize}
\item Statistical inference is about inferring from the data about parameters that describe an assumed
model for the data.
\item 
Solution: In statistics, a model for the mechanism that has produced data is as-
sumed. The model is characterized by some parameters that are unknown. Having
data, statistics tries to infer from them some information about these parameters.
\end{itemize}











%----------------------------------------------------------------------------------------------------%
\subsection{Underlying rational of hypothesis testing}

\begin{itemize}\item 
If, under a given observed assumption, the probability of setting the sample is exceptionally small, we conclude that the underlying assumption is not correct.

\item When testing a clainm, we make an assumption (i.e. the null hypothesis) that contains, wholly or partially, a supposition of equality.

\item We then compare the assumption and the sample results and we form one of the following conclusions.

\item If the sample can easil occur when the assumption ( null hypothesis) is true, we attribute that relatively small discrepanncy between the assumption and the sample results to chance.

\item If the sample is an uncommon occurence when that assumption is true, we explain the relatively large discrepancy between the assumption and the sampke, by concluding that the assumption is not true.
\end{itemize}

%----------------------------------------------------%

\textbf{Statistical Inference : Estimation}
\begin{itemize}
\item When a parameter is being estimated, the estimate can be either a single number or it can be a range of numbers.
\item When the estimate is a single number, such as a sample mean, the estimate is called a \textbf{\emph{point estimate}}.
\item When the estimate is a range of values, the estimate is called an \textbf{\emph{interval estimate}}.
\item \textbf{\emph{Confidence intervals}} are used for interval estimation.
\item As we will soon see, point estimates are not usually as informative as confidence intervals.
\end{itemize}



%----------------------------------------------------%

\textbf{Statistical Inference : Confidence Intervals}
\begin{itemize}
\item Confidence intervals allow us to use sample data to estimate a parameter value, such as a population mean.
\item A confidence interval is a range of values for which we can be confident (at a specific level) that parameter value (such as the population mean)  lies within.
\item A confidence level will have a specified level of confidence, commonly $95\%$.
\item The $95\%$ confidence interval is a range of values which contains the parameter value of interest with a probability of 0.95.
\item We can expected that a $95\%$ confidence interval will not contain the parameter value of interest with a probability of 0.05.
\end{itemize}



\begin{itemize}

\item It is natural to interpret a $95\%$ confidence interval on the mean as an interval with a 0.95 probability of containing the population mean.
\item However, the proper interpretation is not that simple.
\item Consider the case in which 1,000 studies estimating the value of $\mu$  in a certain population all resulted
in estimates between 30 and 40.
\item Suppose one more study was conducted and the $95\%$ confidence interval on $\mu$ was computed
to be $40 \leq \mu \leq 50$ (based on that one study).

\item The probability that $\mu$ is between 40 and 50 is very low, the confidence interval not withstanding.

\end{itemize}

%--------------------------------------------------------------------------------------------------------------------------%

\subsection{Types of inference procedures}

\begin{itemize}
\item The two main types of inference procedures are confidence intervals and hypothesis tests .\item There are two ways of conducting a hypothesis test.\item  One method is to compute the test statistic, and compare to the critical values.\item The second method is to compute the probability value (p-value), and compare it to the significance level.\item Nearly all computer programs use the p-value approach. In this course we will focus more on the p-value approach.
\end{itemize}

\subsection{The probability value }
\begin{itemize}
\item The probability value (sometimes called the p value) is the probability of obtaining a statistic as different from or more different from the parameter specified in the null hypothesis as the statistic obtained in the experiment. 
\item The precise meaning of the p-value
\end{itemize}

%----------------------------------------------------------------------------------------------------%

\section{Assumptions for testing claims about population means}
{
\begin{itemize}
\item The sample is a simple random sample
\item The value of the population variance $\sigma$ is known.
\item Either one or both of these conditions is satisfied
\begin{itemize}
\item The Population is normally distributed.
\item The sample size $n$ is greater than 30.
\end{itemize}
\end{itemize}
}

%---------------------------------------------------------------------------------------------%

\section{Point Estimates}

Statisticians use sample statistics to estimate population parameters. 

Some commonly used point estimates are
\begin{itemize}
\item Sample means $\bar{x}$ are used to estimate population means $\mu$
\item Sample proportions $\hat{p}$ are used to estimate population proportions $\pi$.
\end{itemize}




%%- \subsubsection{Point Estimates}

In statistics, estimation refers to the process by which one makes inferences about a population, based on information obtained from a sample.
There are two types of estimations:

\begin{itemize}
\item Point Estimation
\item Interval Estimation
\end{itemize}


%%- \subsubsection{Point Estimates}

An estimate of a population parameter may be expressed in two ways:

Point estimate. A point estimate of a population parameter is a single value of a statistic. 


% http://www.math.uah.edu/stat/point/


Point estimation refers to the process of estimating a parameter from a probability 
distribution, based on observed data from the distribution. 



% http://www.cliffsnotes.com/math/statistics/principles-of-testing/point-estimates-and-confidence-intervals


You have seen that the sample mean $\bar{x}$ is an unbiased estimate of the population mean $\mu$. 
Another way to say this is that $\bar{x}$  is the best point estimate of the true value of $\mu$. 



Some error is associated with this estimate, however—the true population mean may be larger 
or smaller than the sample mean. Instead of a point estimate, you might want to identify a 
range of possible values $p$ might take, controlling the probability that $\mu$ is not lower 
than the lowest value in this 
range and not higher than the highest value. Such a range is called a confidence interval.





\section{Interval Estimates}
\begin{itemize}
\item Confidence Intervals
\item Prediction Intervals and Tolerance Intervals
\item The Student $t-$ distribution
\item Concluding Remarks
\end{itemize}







\section{Sampling Distribution of the Sample Means}


Instead of working with individual scores, statisticians often work with means. What happens is that several samples are taken, the mean is computed for each sample, and then the means are used as the data, rather than individual scores being used. The sample is a sampling distribution of the sample means.

When all of the possible sample means are computed, then the following properties are true:

\begin{itemize}
\item The mean of the sample means will be the mean of the population
\item The variance of the sample means will be the variance of the population divided by the sample size.
\item The standard deviation of the sample means (known as the standard error of the mean) will be smaller than the population mean and will be equal to the standard deviation of the population divided by the square root of the sample size.
\item If the population has a normal distribution, then the sample means will have a normal distribution.
\item If the population is not normally distributed, but the sample size is sufficiently large, then the sample means will have an approximately normal distribution. Some books define sufficiently large as at least 30 and others as at least 31.
\end{itemize}




\subsection{Sampling Distribution}
\begin{itemize}

\item A probability distribution of a statistic obtained through a large number of samples drawn from a specific population. 
\item The sampling distribution of a given population is the distribution of frequencies of a range of different outcomes 
that could possibly occur for a statistic of a population. 
\end{itemize}


\section{Confidence Intervals}

$\nu$ is the degrees of freedom. For large samples ( samples of a
size greater than thirty) $\nu = \infty$. For small sample (
samples of a size thirty or less)  $\nu = n-1$


\section{Hypothesis Testing and p-values}
\begin{itemize}
\item In hypothesis tests, the difference between the observed value and the parameter value specified by $H_0$ is computed and the probability of obtaining a difference this large or large is calculated.
\item The probability of obtaining data as extreme, or more extreme, than the expected value under the null hypothesis is called the \textbf{\emph{p-value}}.
\item There is often confusion about the precise meaning of the p-value probability computed in a significance test. It is not the probability of the null hypothesis itself.
\item Thus, if the probability value is $0.0175$, this does not mean that the probability that the null hypothesis is either true or false is $0.0175$.
\item It means that the probability of obtaining data as different or more different from the null hypothesis as those obtained in the experiment is $0.0175$.
\end{itemize}

\textbf{Significance (Die Throw Example)}
\begin{itemize}
\item Suppose that the outcome of the die throw experiment was a sum of 401. In previous lectures, a simulation study found that only in approximately $1.75\%$ of cases would a fair die yield this result.
\item However, in the case of a crooked die (i.e. one that favours high numbers) this result would not be unusual.
\item A reasonable interpretation of this experiment is that the die is crooked, but importantly the experiment doesn't prove it one way or the other.
\item We will discuss the costs of making a wrong decision later (Type I and Type II errors).
\end{itemize}







%-----------------------------------%


\section{Inference}
\numberwithin{equation}{section}

\subsection{Standard Errors}

\begin{eqnarray*}
H_{o}: p = p_{0} \\
H_{a}: p \neq p_{0}
\end{eqnarray*}

\begin{equation}
S.E.(\hat{p})=\sqrt{\frac{(p_{o}(1-p_{o})}{n}}
\end{equation}


%% \chapter{17. Inference : Worked Examples}

\begin{framed}
%------------------------------------------------------------- %

\textbf{Important:}

Formulae at back of Exam Paper
\begin{itemize}
\item Murdoch Barnes Table 3 (The Z distribution)
\item Murdoch Barnes Table 7 (The Student t distribution)

\end{itemize}

Important Considerations



\begin{itemize}
\item Significance  $\alpha$   /    confidence 1- $\alpha$
\item 

Number of tails   
\begin{itemize}
\item one tailed procedure or two tailed procedure
\item Confidence intervals are always two tailed
\end{itemize}


\item Sample size   
\begin{itemize}
\item degrees of freedom depends on sample size
\end{itemize} 
\end{itemize}
\end{framed}



%---------------------------------------------------------------------------------------------%
{
\subsection{Two tailed test}

$H_0:  = $
$H_1:  \neq $

$\alpha$ is divided equally between the two tail of the critical region.

$\neq$ i.e. \emph{``not equal to"} can also mean \emph{``less than or greater than"}.
}

%----------------------------------------------------------------------------------------------------%
\section{Assumptions for testing claims about population means, with unknown variance.}
{
\begin{itemize}
\item The sample is a simple random sample
\item The value of the population variance $\sigma$ is not known.
\item Either one or both of these conditions is satisfied
\begin{itemize}
\item The Population is normally distributed.
\item The sample size $n$ is greater than 30.
\end{itemize}
\end{itemize}
}

%----------------------------------------------------------------------------------------------------%
{
\[ t = \frac{\bar{x} - \mu}{\frac{s}{\sqrt{n}}} \]

degrees of freedom (df or $\nu$) = n-1
}












%----------------------------------------------------------------------------------------------------%
{
What is the standard error $\mbox{S.E.}(\hat{p})$ 

\[ \mbox{S.E.}(\hat{p}) = \sqrt{{\hat{p} \times (1- \hat{p}) \over n}} \]

When computing the standard error for a hypothesis test, we use this formula.

\[ \mbox{S.E.}(\hat{p}) = \sqrt{{\hat{p_0} \times (1- \hat{p_o}) \over n}} \]

Where $p_0$ is the population proportion, as proposed by the null hypothesis.
}




%----------------------------------------------------------------------------------------------------%
\section{Inferences around two proportions}

{
\subsection{Assumptions}

\begin{itemize}
\item We have proportions from two independent simple random samples
\item For both samples the conditions $np \geq 5$ $ n(1-p) \geq 5$ are met.
\end{itemize}
For population $1$, let
\begin{itemize}
\item $p_1$  population proportion
\item $n_1$ sample size
\item $x_1$ number of successes in sample 1
\item $hat{p}_1$ is the sample proportion, an estimate for $p_1$.
\end{itemize}





%-------------------------------------------------------%
\section{Confidence Intervals}

{
\subsection{Margin of Error}

\[ E = {\mbox{ upper conf. limit} - \mbox{ lower conf. limit} \over 2} \]

\subsection{Point Estimates}
\begin{itemize}
\item The sample proportion $\hat{p}$ is the best point estimate of the population proportion $p$.
\item The sampk
\end{itemize}

\subsection{The pooled estimate}

The pooled estimate of $p_1$ and $p_2$ is denoted by $\bar{p}$.

\[ \bar{p} = { x_1 +  x_2  \over n_1 + n_2 } \]

i.e. The overall proportion of successes in the aggregate sample, regardless of population.


\subsection{Inferences for matched pairs}
Assumptions
\begin{itemize}
\item[1] The sample data is comprised of matched pairs.
\item[2] The samples are simple random samples.
\end{itemize}


%-------------------------------------------------------%
{
\subsection{Standard Error of a sample mean}

The standard error of the sample mean is given as

\[  \mbox{S.E.}(\bar{x}) = { \sigma \over \sqrt{n} } \]

}


%--------------------------------------------------------------------%






\subsection{Confidence interval of a mean (small sample)}

If the data have a normal probability distribution and the sample
standard deviation $s$ is used to estimate the population
standard deviation $\sigma$, the interval estimate is given by:
\begin{equation}
\bar{X} \pm t_{1-\alpha/2,n-1}\frac{s}{\sqrt{n}}
\end{equation}
where $t_{1-\alpha/2,n-1}$ is the value providing an area of $\alpha/2$ in the upper tail of a Student’s t distribution with n - 1 degrees of freedom.





\section{Interval Estimates}
\begin{itemize}
\item Confidence Intervals
\item Prediction Intervals and Tolerance Intervals
\item The Student $t-$ distribution
\item Concluding Remarks
\end{itemize}



%--------------------------------------------------------%
\subsection{two sample t-test}
Problem: You have obtained the number of years of education from one random sample of 38 police 
officers from City A and the number of years of education from a second random sample of 30 police 
officers from City B.\ The average years of education for the sample from City A is 15 years with a 
standard deviation of 2 years. The average years of education for the sample from City B is 14 years with a standard deviation of 2.5 years. \\
Is there a statistically significant difference between the education levels of police officers in City A and City B?




\begin{itemize}
\item The schedule of formulae that will be at the back of your examination paper will be posted on the SULIS site shortly.
\item It is advisable to familiarise yourself with the contents before the examination.
\item Please let me know as soon as possible if there are any issues with it.
\end{itemize}




\textbf{Type I and Type II errors - Die Example}
\begin{itemize}
\item Recall our die throw experiment example. 
\item Suppose we perform the experiment twice with two different dice.
\item We dont not know for sure whether or not either of the dice is fair or crooked.
\item Suppose we get a sum of 401 from one die, and 360 from the other.
\end{itemize}

\textbf{Examples}

For the sake of brevity, in the following example the following values are always used or realised.

\begin{itemize}
\item Test statistic  = 2.7
\item Significance level $\alpha$ = 0.05. Tests are always two tailed.
\item Samples are either large or of size $n = 10$
\item In the case of large samples CV = 1.96. For small samples, CV = .
\item
\end{itemize}

\begin{itemize}
\item All of the following example will have the same significance level and test statistic.
\item Hence the overall outcomes are the same ( $\alpha$ = 0.05, k=2, TS = 2.7)
\item We will use both the p-value method and the critcal value approach.
\item 
\end{itemize}


\textbf{Hypothesis testing for Two Samples}

\begin{itemize}
\item Paired t test
\item Difference of two mean (large samples)
\item Difference of two mean (small samples)
\item Difference of two propotions (large sample only)
\end{itemize}


\textbf{Computing the Standard Error and Test Statistic }
\begin{itemize}
\item The point estimate is therefore $\bar{x}_1 -\bar{x}_2$
\item The standard error is computed from formulae.
\item The test statistic is therefore 
% \[ (\bar{x}_1 -\bar{x}_2) - (0) \over \mbox{S.E. }(\bar{x}_1 -\bar{x}_2  )} \]
\end{itemize}

\textbf{Difference of proportions}

\begin{itemize}
\item The point estimate is the difference 
\item Observed difference  = $7\%$. Under the null hypothesis, the expected difference is $0\%$
\item The standard error formulae
\end{itemize}





%================================================================================%
Example 1
A fair die is thrown. The number shown on the die is the random variable X. Tabulate the possible outcomes.
Solution
X takes the six possible outcomes 1, 2, 3, 4, 5, 6 which each have probability 1/6 (i.e. one sixth).

r 1 2 3 4 5 6
P(X = r)1/6 1/6 1/6 1/6 1/6 1/6
Example 2
Two unbiased spinners, one numbered 1, 3, 5, 7 and the other numbered 1, 2, 3 are spun. The random variable X is the sum of the two results.
Find the probability distribution for X.



Solution
Listing all the possible outcomes is best done in a table.




%================================================================================%



\subsection{Question 3.2 }


What is the probability of getting a number divisible by 3 in each of 3 throws of a dice?




Solution



Numbers divisible by 3 : 3 and 6            probability of throwing 3 or 6:   



Probability of throwing 3 or 6 three times in a row  ( Each throw of a dice is an independent event.)



P[3T] = P[T]P[T]P[T]=133=127

%================================================================================%






Question 1


%----------------------------------------------------------%
Question 2

%----------------------------------------------------------%
Question 3
$\sqrt(\hat{p} \times (1-\hat{p} ))$

Pen and Paper calculations are a bit easier when working in terms of percentages.

$\sqrt(\hat{p} \times (100-\hat{p} ))$
%----------------------------------------------------------%

Question 4
(Question 4 will be covered again in next weeks tutorial, and does not need to be completed)



%----------------------------------------------------------%

%\sqrt{ \frac{2275}{100}} 

%\sqrt{ \frac{2400}{100}} 


%\sqrt{ \frac{2500}{100}} 


%\sqrt{ \frac{2400}{100}} 

%\sqrt{ \frac{2275}{100}} 



%----------------------------------------------------------%
Question 7

Point Estimate : The sample mean $\bar{X}$

95\% confidence interval quantile 1.96 

(Apart from pointing out that the sample is large, the justification for using this number is not required)

%----------------------------------------------------------%
Question 8

Point Estimate : $\bar{X}{n}$

$x$ is the number of successes
$n$ is the sample size

$\hat{p} = 68.8\%$


$\sqrt(\hat{p} \times (100-\hat{p} ))$

%----------------------------------------------------------%

\section{Revision for Inference Procedures}

\begin{itemize}
\item Definitions
\item Computing Confidence Intervals 
\item Performing Hypothesis Testing
\begin{itemize}
\item by comparing test statistics to critical values
\item by considering the p-value
\item by using the confidence interval.
\end{itemize}
\end{itemize}





{
\subsection{Inference : Confidence Intervals }

Basic Structure

\[ \mbox{Observed value} \pm [\mbox{Quantile}\times \mbox{Standard Error}] \] 



}





\section{Inference}
{
\subsection{Inference}
\begin{itemize}
\item Inferential statistics uses sampple data to make inferences (or generalizations) about a population.
\end{itemize}
}


%---------------------------------------------------------------------------------%
{
\subsection{Example}
Given the population of men has normal distributed weights with a mean of 172 lbs and a standard deviation of 29 lbs,



\subsection{Example}
Let $X$ be the score from the throw of a die.

It can be shown that, no matter what distribution theunderlying values are from, the statistics will follow a normal distribution.


\subsection{Standard Error}

\begin{itemize}

\item As the sampling size increases, the sampling distribution approaches the normal distribution


\end{itemize}

\subsection{Central Limit Theorem}

\begin{itemize}

\item Consider a sample of size $n$.

\item The population variance $\sigma^2$ is unknown, but can be estimated by the sample variance $s^2$


\end{itemize}


}




%--------------------------%
% - Interpreting Confidence Interval. READY
% - Introducing Hypothesis testing. READY
% - The Null and Alternative Hypotheses.
% - Computing Test Statistics.
% - P-values
% - Critical values
% - Decision Rules
% - One Side ad Two Sided tests
% - Type I and Type 2 Error
% - The Paired T test.
%----------------------------------------------------------------------------------------------------%




%--------------------------------------------------------------------------------------------------------------------------%


\section{New Section}

\subsubsection{Formulae}
\begin{itemize}
\item The schedule of formulae that will be at the back of your examination paper will be posted on the SULIS site shortly.
\item It is advisable to familiarise yourself with the contents before the examination.
\item Please let me know as soon as possible if there are any issues with it.
\end{itemize}

%------------------------------------------------------------------------------------------------------------%












\subsubsection{Confidence Intervals (Revision) }

\begin{itemize}
\item The $95\%$ confidence interval is a range of values which contain the true population parameter (i.e. mean, proportion etc) with a probability of $95\%$.
\item We can expect that a $95\%$ confidence interval will not include the true parameter values $5\%$ of the time.
\item A confidence level of $95\%$ is commonly used for computing confidence interval, but we could also have confidence levels of $90\%$,$99\%$ and $99.9\%$.
\end{itemize}



%-----------------------------------------------------------%



\subsubsection{Confidence Level (Revision) }

\begin{itemize}
\item A confidence level for an interval is denoted to $1-\alpha$ (in percentages: $100(1-\alpha)\%$) for some value $\alpha$.
\item A confidence level of $95\%$ corresponds to $\alpha = 0.05$.
\item $100(1-\alpha)\%$ = $100(1-0.05)\%$  = $100(0.95)\%$ = $95\%$
\item For a confidence level of $99\%$, $\alpha = 0.01$.
\item Knowing the correct value for $\alpha$ is important when determining quantiles.
\end{itemize}








%------------------------------------------------------------------------------%

Although the sample mean is useful as an unbiased estimator of the population mean, there is no way of
expressing the degree of accuracy of a point estimator. In fact, mathematically speaking, the probability that the
sample mean is exactly correct as an estimator of the population mean is $P = 0$.

%------------------------------------------------------------------------------%

A confidence interval for the
mean is an estimate interval constructed with respect to the sample mean by which the likelihood that the interval
includes the value of the population mean can be specified.

The \emph{level of confidence} associated with a confidence interval indicates the long-run percentage
of such intervals which would include the parameter being estimated.

%------------------------------------------------------------------------------%

\begin{itemize}
\item Confidence intervals for the mean typically are constructed with the unbiased estimator $\bar{x}$ at the midpoint
of the interval.

\item The $\pm Z \sigma_x$ or $\pm Z s_x$ frequently is called the \textbf{\emph{margin of error}} for the confidence interval.
\end{itemize}

%------------------------------------------------------------------------------%

We indicated that use of the normal distribution in estimating a population mean is warranted
for any large sample ($n > 30$), \textbf{and} for a small sample ($n \leq 30$) only if the population is normally distributed
and $\sigma$ is known.

%------------------------------------------------------------------------------%

\begin{itemize}
\item Now we consider the situation in which the sample is small and the population is normally distributed,
but $\sigma$ is not known.
\item The distribution is a family of distributions, with
a somewhat different distribution associated with the degrees of freedom ($df$). For a confidence interval for the
population mean based on a sample of size n, $df = n - 1$.
\end{itemize}

%------------------------------------------------------------------------------%
{

\subsubsection{Computing the Standard Error}

\[
S.E. (\hat{p}) \;=\; \sqrt{ {\hat(p) \times (100 -\hat{p} )\over n}}
\]



\[
\hat{p} = {144/200}  \times 100\%  = 0.72 \times 100\%.  = 72%
\]

$100\% - \hat{p} = 100\% - 72\% = 28\% $

}


%------------------------------------------------------------------------------%
{
\textbf{Computing the Standard Error}

\[
S.E. (\hat{p}) \;=\; \sqrt{ {72 \times 28 \over 200 }}
\]


}
%------------------------------------------------------------------------------%



% Confidence Interval for a Proportion
% One Sample
%------------------------------------------------------------------------------%
{
\textbf{Computing the point estimate}

Sample percentage

\[
\hat{p} = \frac{x}{n} \times 100%
\]

\begin{itemize}
\item $\hat{p}$ - sample proportion.
\item $x$  - number of ``successes".
\item $n$  - the sample size.
\end{itemize}

}








\section{Inference}
\numberwithin{equation}{section}

\subsection{Standard Errors}

\begin{eqnarray*}
H_{o}: p = p_{0} \\
H_{a}: p \neq p_{0}
\end{eqnarray*}

\begin{equation}
S.E.(\hat{p})=\sqrt{\frac{(p_{o}(1-p_{o})}{n}}
\end{equation}


%% \chapter{17. Inference : Worked Examples}

\begin{framed}
%------------------------------------------------------------- %

\textbf{Important:}

Formulae at back of Exam Paper
\begin{itemize}
\item Murdoch Barnes Table 3 (The Z distribution)
\item Murdoch Barnes Table 7 (The Student t distribution)

\end{itemize}

Important Considerations



\begin{itemize}
\item Significance  $\alpha$   /    confidence 1- $\alpha$
\item 

Number of tails   
\begin{itemize}
\item one tailed procedure or two tailed procedure
\item Confidence intervals are always two tailed
\end{itemize}


\item Sample size   
\begin{itemize}
\item degrees of freedom depends on sample size
\end{itemize} 
\end{itemize}
\end{framed}



\end{document}
