

%------------------------------------------------------------------------------------------------------------%

%%%% Type I and Type II errors here
\frame{
\frametitle{The Paired t-test}
A paired t-test is used to compare two population means where you have two samples in
which observations in one sample can be \textbf{\emph{paired}} with observations in the other sample.\\
\bigskip
Examples of where this might occur are:
\begin{itemize}
\item Before-and-after observations on the same subjects (e.g. students� diagnostic test
results before and after a particular module or course).
\item A comparison of two different methods of measurement or two different treatments
where the measurements/treatments are applied to the \textbf{\emph{same}} subjects.
\end{itemize}
The difference between two paired measurements is known as a \textbf{\emph{case-wise}} difference.
}



%-------------------------------------------------------------------------------------------%
\begin{frame}
\frametitle{The Paired t-test}
\begin{itemize}
\item We will often be required to compute the case-wise differences, the average of those differences and the standard deviation of those difference.

\item The mean difference for a set of differences between paired observations is
\[ \bar{d} = {\sum d_i \over n }\]

\item The computational formula for the standard deviation of the differences
between paired observations is
\[s_d = \sqrt{ {\sum d_i^2 - n\bar{d}^2 \over n-1}}\]
\item It is nearly always a small sample test.
\end{itemize}
\end{frame}


%----------------------------------------------------------------------------------------------------%
\frame{
\frametitle{The Paired t-test}
\begin{itemize}
\item $\mu_d$ mean value for the population of case-wise differences.
\item The null hypothesis is that that $\mu_d = 0$
\item Given $\bar{d}$ mean value for the sample of differences, and $s_d$ standard deviation of the differences for the paired sample data, we can compute this test in the same manner as a one-sample test for the mean
\end{itemize}
}










%--------------------------------------------------------------------------------------------------------------------------%

%%--------------------------------------------------------------------------------------%
%\begin{frame}
%\frametitle{Computing the Test Statistic}
%
%The general structure of a test statistic is as follows:
%\[ TS = {\mbox{observed value} - \mbox{null value} \over \mbox{Std. Error}}   \]
%where ``null value" is shorthand for the expected value under the null hypothesis.
%
%Refer to the formulae for the appropriate standard error.
%\end{frame}

%%--------------------------------------------------------------------------------------%
%\begin{frame}
%\frametitle{The p-value}
%\begin{itemize}
%\item When using the p-value approach for determining the outcome of a test, you may be required to determine the appropriate p-value from R code. \item  When a Test Statistic is specified the p-value can be computed as \texttt{1-pnorm(TS)}.
%\item Suppose you compute a test statistic of 2.13.
%
%\item From the R code on the next slide, the p-value is 0.01658581
%\end{itemize}
%\end{frame}

%--------------------------------------------------------------------------------------%
%\begin{frame}[fragile]
%\frametitle{The p-value}
%\begin{verbatim}
%> TSs = 200:220/100
%> TSs
% [1] 2.00 2.01 2.02 2.03 2.04 2.05 2.06 2.07 2.08
%[10] 2.09 2.10 2.11 2.12 2.13 2.14 2.15 2.16 2.17
%[19] 2.18 2.19 2.20
%> 1-pnorm(TSs)
% [1] 0.02275013 0.02221559 0.02169169 0.02117827
% [5] 0.02067516 0.02018222 0.01969927 0.01922617
% [9] 0.01876277 0.01830890 0.01786442 0.01742918
%[13] 0.01700302 0.01658581 0.01617738 0.01577761
%[17] 0.01538633 0.01500342 0.01462873 0.01426212
%[21] 0.01390345
%\end{verbatim}
%\end{frame}
%\frame{
%\frametitle{Hypothesis Testing: Some Worked Examples}
%\large
%\begin{itemize}
%\item[1] Small sample - test of mean
%\item[2] Difference of two mean (large samples, using p-value approach)
%\item[3] Difference of two mean (large samples, using CV approach)
%\item[4] Difference of two mean (small samples)
%\item[5] Difference of two proportions
%\item[6] Paired t-test
%\end{itemize}
%}

%--------------------------------------------------------------------------------------%
\begin{frame}
\frametitle{Example 1 (a) Small Sample Hypothesis Test}
\large
\begin{itemize}
%\item The standard deviation of the life for a particular brand of ultraviolet tube is known to
%\item Also it is assumed, but not known, that the operating life of the tubes is normally distributed.

\item The manufacturer claims that average tube life for a particular brand of ultraviolet tube
is 9,000 hr. \item Test this claim at the 5 percent level of significance against the alternative hypothesis
that the mean life is not 9,000 hr \item We are given the following information:  a sample of $n = 10$ tubes the mean operating
life was $\bar{x} = 8,800$ hr. The sample standard deviation is be $s = 500$ hr.
%\item (Intuitively this would suggest a one-tailed test that the mean is less than 9000 hours)
\end{itemize}
\end{frame}


%--------------------------------------------------------------------------------------%
\begin{frame}
\frametitle{Example 1 (b) }
\large
\begin{itemize}
\item $H_0 \mbox{ : } $ $\mu = 9000$ (Average life span is 9000 hours.)
\item $H_1 \mbox{ : } $ $\mu \neq 9000$ (Average life span is not 9000 hours.)
\end{itemize}
\bigskip
\begin{itemize}
\item The observed difference is -200 hours. (i.e. 8,800 - 9,000 hours)
\item The standard error is determined from formulae.
\[ S.E. (\bar{x}) = {s \over \sqrt{n}} = {500 \over \sqrt{10}}  = 158.1139 \]
\end{itemize}
\end{frame}
%--------------------------------------------------------------------------------------%
\begin{frame}[fragile]
\frametitle{Example 1 (c) : Test Statistic and Critical Value }
\large
\[ TS = \frac{8800 - 9000}{158.11} \]
\begin{itemize}
\item The test statistic $TS = -1.265$
\item The CV is determined with $\alpha = 0.05$ and $k = 2$ (column = $\alpha/k=0.025$).
\item The sample is small n = 10 $df = n-1 = 9$ (i.e. row =9).
\item Therefore $CV = 2.262$

\item (Remark: If the sample was large, we could use $CV = 1.96$).

\end{itemize}
\end{frame}

%--------------------------------------------------------------------------------------%
\begin{frame}
\frametitle{Example 1 (d): Decision Rule }
\large
\begin{itemize}
\item \textbf{Decision:}Is $|TS| >CV$? Is $1.265 > 2.262$?
\item No. We fail to reject the null hypothesis. \item There is not enough evidence to say that the mean lifespan is not 9000 hours.
\end{itemize}
\end{frame}
%----------------------------------------------------------------------------------%
\begin{frame}
\frametitle{Example 2: Paired Difference (a)}
\begin{itemize}
\item An automobile manufacturer collects mileage data for a sample of $n = 10$ cars in various weight categories
using a standard grade of gasoline with and without a particular additive. \item Of course, the engines were tuned to the same
specifications before each run, and the same drivers were used for the two gasoline conditions (with the driver in fact being
unaware of which gasoline was being used on a particular run). \item Given the mileage data on the next slide,  test the hypothesis
that there is no difference between the mean mileage obtained with and without the additive, using the 5 percent level of
significance \end{itemize}
\end{frame}
%-------------------------------------------------------------------------------------------%
\begin{frame}
\frametitle{Example 2: Paired Difference (b)}
\small
\begin{center}
\begin{tabular}{|c|c|c|c|c|}\hline
car & with additive & without additive & $d_i$ & $d^2_i$\\\hline
1&36.7&36.2&0.5&0.25\\\hline
2&35.8&35.7&0.1&0.01\\\hline
3&31.9&32.3&-0.4&0.16\\\hline
4&29.3&29.6&-0.3&0.09\\\hline
5&28.4&28.1&0.3&0.09\\\hline
6&25.7&25.8&-0.1&0.01\\\hline
7&24.2&23.9&0.3&0.09\\\hline
8&22.6&22.0&0.6&0.36\\\hline
9&21.9&21.5&0.4&0.16\\\hline
10&20.3&20.0&0.3&0.09\\\hline
\end{tabular}
\end{center}
\end{frame}

%--------------------------------------------------------------------------------------------------------------------------%
%-------------------------------------------------------------------------------------------%
\begin{frame}
\frametitle{Example 2: Paired Difference (c)}
\begin{itemize}
\item The average of the case wise differences is computed as \[\bar{d} = {\sum d_i \over n}\]
\[ \bar{d} = { 0.05 + 0.1  - 0.4 + \ldots + 0.30 \over 10 }= 0.17 \]
\item Also, using last column, $\sum d^2_i = (0.25 + 0.01 + 0.16 + \ldots + 0.09) = 1.31$
\end{itemize}

\end{frame}


\begin{frame}
\frametitle{Example 2: Paired Difference (d)}
\textbf{Sample standard deviation of the case-wise differences}:
\large
\[s_d = \sqrt{ {\sum d_i^2 - n\bar{d}^2 \over n-1}}\]
We know the following:
\begin{itemize}
\item The sample size $n$ which is 10.
\item The average of the case-wise differences. $\bar{d} = 0.17$
\item  $\sum d^2_i = 1.31$
\end{itemize}
\end{frame}



\begin{frame}
\frametitle{Example 2: Paired Difference (e)}
\textbf{Sample standard deviation  of the case-wise differences}://
\[s_d = \sqrt{ {\sum d_i^2 - n\bar{d}^2 \over n-1}}\]

\[s_d = \sqrt{ { 1.31 - 10(0.17)^2 \over 9}} = 0.337\]

\textbf{The standard error:} \[ S.E.(\bar{d}) = s_d / \sqrt{n} = {0.0337 \over 3.16} = 0.107\]
\end{frame}

\begin{frame}
\frametitle{Example 2: Paired Difference (f)}
\textbf{Null and Alternative Hypotheses}:
\begin{itemize}
\item That is, the null hypothesis is:\\
$H_0: \mu_d = 0$ Additive makes no difference to performance\\
$H_1: \mu_d \neq 0$ Additive makes a significant difference to performance \\
\end{itemize}
\textbf{Test Statistic}:
\[ TS = \frac{0.17 - 0}{0.107} = 1.59\]

\end{frame}
\end{document}
