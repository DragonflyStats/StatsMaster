\documentclass[a4]{beamer}
\usepackage{amssymb}
\usepackage{graphicx}
\usepackage{subfigure}
\usepackage{framed}
\usepackage{newlfont}
\usepackage{amsmath,amsthm,amsfonts}
%\usepackage{beamerthemesplit}
\usepackage{pgf,pgfarrows,pgfnodes,pgfautomata,pgfheaps,pgfshade}
\usepackage{mathptmx}  % Font Family
\usepackage{helvet}   % Font Family
\usepackage{color}

\mode<presentation> {
 \usetheme{Frankfurt} % was
 \useinnertheme{rounded}
 \useoutertheme{infolines}
 \usefonttheme{serif}
 %\usecolortheme{wolverine}
% \usecolortheme{rose}
\usefonttheme{structurebold}
}

\setbeamercovered{dynamic}

\title[MA4603]{Science Maths 3 \\ {\normalsize MA4603 Lecture 11A}}
\author[Kevin O'Brien]{Kevin O'Brien \\ {\scriptsize Kevin.obrien@ul.ie}}
\date{Autumn Semester 2017}
\institute[Maths \& Stats]{Dept. of Mathematics \& Statistics, \\ University \textit{of} Limerick}

\renewcommand{\arraystretch}{1.5}

\begin{document}
%--------------------------------------------------------%

\begin{frame}
\frametitle{Mean Difference Between Matched Data Pairs}


The approach described in this lesson is valid whenever the following conditions are met:

\begin{itemize}
\item The data set is a simple random sample of observations from the population of interest.
\item Each element of the population includes measurements on two paired variables (e.g., x and y) such that the paired difference between x and y is: d = x - y.
\item The sampling distribution of the mean difference between data pairs (d) is approximately normally distributed.
\end{itemize}



The observed data are from the same subject or from a matched subject and are drawn from a population with a normal distribution
does not assume that the variance of both populations are equal



\end{frame}

%---------------------------------------------------------------------------------------------------------------%
\begin{frame}
\frametitle{Computing the Case Wise Differences}
\begin{center}
\small
\begin{tabular}{|c||c|c|c|c|} \hline
Student & Before & After & Difference $(d_i)$ & $ (d_i -\bar{d})^2$ \\\hline
1 &90& 95& 5& 16 \\\hline
2 &85& 89& 4& 9 \\\hline
3 &76 &73 &-3 &4 \\\hline
4 &90& 92& 2& 1 \\\hline
5 &91 &92 &1 &0 \\\hline
6 &53 &53& 0& 1 \\\hline
7 &67 &68 &1 &4 \\\hline
8 &88 &90 &2 &9 \\\hline
9 &75 &78 &3 &16\\\hline
10 &85& 89 &4& 25 \\\hline
\end{tabular}
\end{center}

\end{frame}

%---------------------------------------------------------------------------------------------------------------%


\begin{frame}
\frametitle{Computing the Case Wise Differences}
Compute the mean difference

\[ \bar{d}  = {\sum{d_i} \over n } = { 3+6 \over 8} \]

Compute the variance of the differences.

\[ s^2_{d}  ={\sum(d_i -\bar{d})^2 \over n-1 } =  { 3+6 \over 9} \]

\end{frame}

%--------------------------------------------------------%

\begin{frame}
\frametitle{Difference of Two Means}
\begin{itemize}
\item
\item
\end{itemize}
\end{frame}
%---------------------------------------------------------%

\begin{frame}
\frametitle{How a paired t test works}
\begin{itemize}
\item The paired t test compares two paired groups.
\item It calculates the difference between each set of pairs, and analyzes that list of differences based on the assumption that the differences in the entire population follow a Gaussian distribution.
\item First we calculate the difference between each set of pairs, keeping track of sign.
\item If the value in column B is larger, then the difference is positive.
If the value in column A is larger, then the difference is negative.
\item The t ratio for a paired t test is the mean of these differences divided by the standard error of the differences. If the t ratio is large (or is a large negative number), the P value will be small. The number of degrees of freedom equals the number of pairs minus 1. Prism calculates the P value from the t ratio and the number of degrees of freedom.
\end{itemize}
\end{frame}
%---------------------------------------------------------%
\begin{frame}
\[ ( \bar{X} - \bar{Y} ) \pm \left[ \mbox{Quantile } \times S.E(\bar{X}-\bar{Y}) \right] \]
\begin{itemize}
\item If the combined sample size of X and Y is greater than 30, even if the individual sample sizes are less than 30, then we consider it to be a large sample.
\item The quantile is calculated according to the procedure we met in the previous class.
\end{itemize}
\end{frame}
%---------------------------------------------------------%
\begin{frame}\begin{itemize}
\item Assume that the mean ($\mu$) and the variance ($\sigma$) of the distribution
of people taking the drug are 50 and 25 respectively and that the mean ($\mu$)
and the variance ($\sigma$) of the distribution of people not taking the drug are
40 and 24 respectively.
\end{itemize}
\end{frame}





%---------------------------------------------------------%
\begin{frame}
\frametitle{Difference in Two means}
For this calculation, we will assume that the variances in each of the two populations are equal. This assumption is called the assumption of homogeneity of variance.

The first step is to compute the estimate of the standard error of the difference between means ().

\[ S.E.(\bar{X}-\bar{Y}) = \sqrt{\frac{s^2_x}{n_x} + \frac{s^2_y}{n_y}} \]

\begin{itemize}
\item $s^2_x$ and $s^2_x$ is the variance of both samples.
\item $n_x$ and $n_y$ is the sample size of both samples.
\end{itemize}
The degrees of freedom is $n_x + n_y -2$.
\end{frame}



\end{document}
