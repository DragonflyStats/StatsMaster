
\documentclass[a4paper,12pt]{article}
%%%%%%%%%%%%%%%%%%%%%%%%%%%%%%%%%%%%%%%%%%%%%%%%%%%%%%%%%%%%%%%%%%%%%%%%%%%%%%%%%%%%%%%%%%%%%%%%%%%%%%%%%%%%%%%%%%%%%%%%%%%%%%%%%%%%%%%%%%%%%%%%%%%%%%%%%%%%%%%%%%%%%%%%%%%%%%%%%%%%%%%%%%%%%%%%%%%%%%%%%%%%%%%%%%%%%%%%%%%%%%%%%%%%%%%%%%%%%%%%%%%%%%%%%%%%
\usepackage{eurosym}
\usepackage{vmargin}
\usepackage{amsmath}
\usepackage{graphics}
\usepackage{epsfig}
\usepackage{subfigure}
\usepackage{fancyhdr}

\setcounter{MaxMatrixCols}{10}
%TCIDATA{OutputFilter=LATEX.DLL}
%TCIDATA{Version=5.00.0.2570}
%TCIDATA{<META NAME="SaveForMode"CONTENT="1">}
%TCIDATA{LastRevised=Wednesday, February 23, 201113:24:34}
%TCIDATA{<META NAME="GraphicsSave" CONTENT="32">}
%TCIDATA{Language=American English}

\pagestyle{fancy}
\setmarginsrb{20mm}{0mm}{20mm}{25mm}{12mm}{11mm}{0mm}{11mm}
\lhead{MA4128} \rhead{Kevin O'Brien} \chead{Assumptions for Linear Models} %\input{tcilatex}

\begin{document}
	
	\tableofcontents
	\newpage
	
\section{Confidence Intervals}

\subsection{Large sample distribution of sample mean}
%Chemo 2009 2B

\subsection{Confidence limits of the mean for large samples}
Now that we know the form of the sampling distribution of the mean we can return to the problem of using a sample to define a range which we may reasonably assume includes the true value. (Remember that in doing this we are assuming systematic errors to be absent.) Such a range is known as a confidence interval and the extreme values of the range are called the confidence limits.

The term 'confidence' implies that we can assert with a given degree of confidence, i.e. a ' certain probability, that the confidence interval does include the true value.

The size of the confidence interval will obviously depend on how certain we want to be that it includes the true value: the greater the certainty, the greater the interval required.

Figure 2.6 shows the sampling distribution of the mean for samples of size $n$. If we assume that this distribution is normal then $95\%$  of the sample means will lie in the range given by:
\[ \mu - 1.96(\sigma/\sqrt{n}) < \mu < \mu + 1.96(\sigma/\sqrt{n}) (2.6)\]

(The exact value 1.96 has been used in this equation rather than the approximate
value, 2, quoted in Section 2.2. The reader can use Table A.1 to check that the pro-
portion of values between z= -1.96 and z = 1.96 is indeed 0.95,)

\subsection{Large Sample Confidence Intervals based on the sample mean}

In practice, however, we usually have one sample, of known mean, and we require a range for $\mu$, the true value.

Equation (2.6) can be rearranged to give this:
\[ \bar{X} - 1.96(\sigma/\sqrt{n}) < \mu < \bar{X} + 1.96(\sigma/\sqrt{n}) (2.7)\]

Equation (2.7) gives the $95\%$  confidence interval of the mean. The $95\%$  confidence
limits are $\bar{X} \pm 19.6 \sigma/\sqrt{n}$
In practice we are unlikely to know $\sigma$ exactly. However, provided that the sample is
large, $\sigma$ can be replaced by its estimate, $s$.


\subsection{Example of computations using R}
Finding confidence intervals for the mean for the nitrate ion
concentrations in Table 2.1.
\begin{verbatim}
#reading data
x=scan("Table2_1.txt")
#setting the confidence level
CL=0.95
#computing confidence interval
n=length(x)
pm=sd(x)*c(qnorm(0.025),qnorm(0.975))/sqrt(n)
CI=mean(x)+pm
\end{verbatim}

\subsection{Small-Sample Case (n < 30)}
If the data have a normal probability distribution and the sample
standard deviation s is used to estimate the population
standard deviation $s$, the interval estimate is given by:
\[ \bar{X} \pm t_{\alpha/2}s /\sqrt{n} \]
where $\alpha/2$ is the value providing an area of $\alpha/2$  in the upper tail
of a Student's $t-$distribution with $n - 1$ degrees of freedom.


\subsection{Confidence limits of the mean for small samples}

As the sample size gets smaller, s becomes less reliable as an estimate of $\sigma$. This can
be seen by again treating each column of the results in Table 2.2 as a sample of size
five. The standard deviations of the 10 columns are \[\{0.009, 0.015, 0.026, 0.021
0.013, 0.019, 0.013, 0.017, 0.010 and 0.018\}.\] We see that the largest value of $s$ is
nearly three times the size of the smallest. To allow for this, equation (2.8) must be
modified.

For small samples, the confidence limits of the mean are given by
\[ \bar{X} \pm t_{n-1}s /\sqrt{n} \]


\newpage


 

 

Confidence Intervals

Calculation of upper and lower limits 

We are initially asked to use a particular Confidence level (1-$\alpha$)\%  

From it we deduce the Significance level required   ($\alpha$)\%  

 is then determined from the tables

 

Example letting the confidence level be 95\%, we immediately deduce that the

Significance level is 5\%

 

(1-$\alpha$/2) is therefore 97.5\%

From the tables we see the associated Z value is 1.96

 

 

 

 

The (1-$\alpha$)\%  [95\%]confidence interval means the true value of the parameter (i.e the mean) lies within our confidence interval with a probability of 95\%.

 

The remaining $\alpha$\%  probability is distributed on either side of this confidence interval. In other words, the (1-$\alpha$)\%  confidence interval occurs within between the ($\alpha$/2)\% quantile and the (1-$\alpha$/2)\% quantile.

 

A 95\% confidence interval is located between the 2.5\% and 97.5\% quantiles.

 

Formally the lower limit 

and the Upper limit is 

 

Condensing the calculations

 

We can use the symmetry property , the fact that , to simplify into this form 

 

e.g. 

 

\end{document}
