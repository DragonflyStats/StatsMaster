\documentclass[a4]{beamer}
\usepackage{amssymb}
\usepackage{graphicx}
\usepackage{subfigure}
\usepackage{newlfont}
\usepackage{amsmath,amsthm,amsfonts}
%\usepackage{beamerthemesplit}
\usepackage{pgf,pgfarrows,pgfnodes,pgfautomata,pgfheaps,pgfshade}
\usepackage{mathptmx}  % Font Family
\usepackage{helvet}   % Font Family
\usepackage{color}

\mode<presentation> {
 \usetheme{Default} % was
 \useinnertheme{rounded}
 \useoutertheme{infolines}
 \usefonttheme{serif}
 %\usecolortheme{wolverine}
% \usecolortheme{rose}
\usefonttheme{structurebold}
}

\setbeamercovered{dynamic}

\title[MA4704]{Technological Mathematics 4 \\ {\normalsize MA4704 Lecture 10A}}
\author[Kevin O'Brien]{Kevin O'Brien \\ {\scriptsize Kevin.obrien@ul.ie}}
\date{Spring Semester 2013}
\institute[Maths \& Stats]{Dept. of Mathematics \& Statistics, \\ University \textit{of} Limerick}

\renewcommand{\arraystretch}{1.5}

\begin{document}


\begin{frame}
\titlepage
\end{frame}

%---------%
\begin{frame}
\frametitle{Using \texttt{R} for Inference Procedures}
\begin{itemize}
\item[1] Review of the Paired t-test.
\item[2] Paired t-test using \texttt{R}
%\item[3] Two Sample Test for Proportions
\item[3] Test for the equality of variances for two samples
\item[4] Shapiro-Wilk Test for Normality
\item[5] Graphical procedures for assessing normality
\item[6] Grubb's Procedure for Determinin an Outlier
\end{itemize}
\end{frame}
%---------%
\begin{frame}
\frametitle{p-values using \texttt{R}}
\begin{itemize}
\item In every inference procedure performed using \texttt{R}, a p-value is presented to the screen for the user to interpret.

\item If the p-value is larger than a specified threshold $\alpha/k$ then the appropriate conclusion is a
failure to reject the null hypothesis.

\item Conversely, if the p-value is less than threshold, the appropriate conclusion is to reject the null hypothesis.

\item In this module, we will use a significance level$\alpha=0.05$ and almost always the procedures will be two tailed ($k=2$). Therefore the threshold $\alpha/k$ will be $0.025$.
\end{itemize}
\end{frame}

%---------%
\begin{frame}
\frametitle{Using Confidence Limits}
\begin{itemize}
\item Alternatively, we can use the confidence interval to make a decision on whether or not we should reject or fail to reject the null hypothesis.
\item If the null value is within the range of the confidence limits, we fail to reject the null hypothesis.
\item If the null value is outside the range of the confidence limits, we reject the null hypothesis.
\item Occasionally a conclusion based on this approach may differ from a conclusion based on the p-value. In such a case, remark upon this discrepancy.
\end{itemize}
\end{frame}
%------------------------------------------%

\begin{frame}
\frametitle{The paired t-test (a)}

\begin{itemize}
\item Previously we have seen the paired t-test. It is used to determine whether or
not there is a significant difference between paired measurements. Equivalently whether or not
the case-wise differences are zero.
\item The mean and standard deviation of the case-wise differences are used to determine the test statistic.
\item Under the null hypothesis, the expected value of the case-wise differences is zero (i.e $H_0 : \mu_d = 0$).
\item The test statistic is computed as
\[ TS = \frac{\bar{d} - \mu_d}{\frac{s_d}{\sqrt{n}}} \]
\end{itemize}
\end{frame}


%------------------------------------------%
\begin{frame}[fragile]
\frametitle{The Paired t-test (b)}
\begin{itemize}
\item The calculation is dependent on the case-wise differences.
\item Here the case-wise differences between paired measurements (e.g. ``before" and ``after").
\item Under the null hypothesis, the mean of case-wise differences is zero.
\item As a quick example, the mean, standard deviation and sample size are presented in the next slide.
\end{itemize}
\end{frame}

%------------------------------------------%
\begin{frame}[fragile]
\frametitle{The paired t-test (c)}
\begin{itemize}
\item Observed Mean of Case-wise differences $\bar{d}$ = 8.21,
\item Expected Mean of Case-wise differences under $H_0$ : $\mu_d = 0$,
\item Standard Deviation of Case-wise differences $S_d$ = 7.90,
\item Sample Size $n=14$.
\end{itemize}

\[ TS = \frac{\bar{d} - \mu_d}{\frac{s_d}{\sqrt{n}}} = \frac{8.21 - 0}{\frac{7.90}{\sqrt{14}}} = 3.881 \]
\end{frame}
%------------------------------------------%
\begin{frame}[fragile]
\frametitle{The paired t-test (e)}

\begin{itemize}
\item Procedure is two-tailed, and you can assume a significance level of 5\%.
\item It is also a small sample procedure (n=14, hence df = 13).
\item The Critical value is determined from statistical tables (2.1603).
\item Decision Rule: Reject Null Hypothesis ($|TS|>CV$). Significant difference in measurements before and after.
\end{itemize}

\end{frame}
%------------------------------------------%
\begin{frame}[fragile]
\frametitle{The paired t-test (f)}
Alternative Approach : using p-values.
\begin{itemize}
\item The p-values are determined from computer code. (We will use a software called \texttt{R}. Other types of software inlcude \texttt{SAS} and \texttt{SPSS}.)
\item The null and alternative are as before.
\item The computer software automatically generates the appropriate test statistic, and hence the corresponding p-value.
\item The user then interprets the p-values. If p-value is small, reject the null hypothesis. If the p-value is large, the appropriate conclusion is a failure to reject $H_0$.
\item The threshold for being considered small is less than $\alpha/k$, (usually 0.0250). (This is a very arbitrary choice of threshold, suitable for some subject areas, not for others)
\end{itemize}
\end{frame}
%------------------------------------------%
\begin{frame}[fragile]
\frametitle{The paired t-test (g)}
Implementing the paired t-test using \texttt{R} for the example previously discussed.
\begin{verbatim}
> t.test(Before,After,paired=TRUE)

        Paired t-test

data:  Before and After
t = 3.8881, df = 13, p-value = 0.001868
alternative hypothesis: true difference in means is not 0
95 percent confidence interval:
  3.650075 12.778496
sample estimates:
mean of the differences
               8.214286

\end{verbatim}
\end{frame}

%-------------------------------------------------%
\begin{frame}[fragile]
\frametitle{The paired t-test (h)}
\begin{itemize}
\item The p-value ($0.001868$) is less than the threshold is less than the threshold $0.0250$.
\item We reject the null hypothesis (mean of case-wise differences being zero, i.e. expect no difference between ``before" and ``after").
\item We conclude that there is a difference between `before' and `after'.
\item That is to say, we can expected a difference between two paired measurements.
\end{itemize}
\end{frame}
%-------------------------------------------------%
\begin{frame}[fragile]
\frametitle{The paired t-test (i)}
\begin{itemize}
\item We could also consider the confidence interval. We are $95\%$ confident that the expected value of the case-wise difference is at least 3.65.
\item Here the null value (i.e. 0) is not within the range of the confidence limits.
\item Therefore we reject the null hypothesis.
\end{itemize}
\begin{verbatim}
> t.test(Before,After,paired=TRUE)
...
...
95 percent confidence interval:
  3.650075 12.778496
...
\end{verbatim}
\end{frame}




%----------------------------------------------------------%
\begin{frame}[fragile]
\frametitle{Grubbs Test for Determining an Outlier}

The Grubbs test is used to determine if there are any outliers in a data set.\\ \bigskip
There is no agreed formal definition for an outlier. The definition of outlier used for this procedure is a value that unusually distance from the rest of the values ( For the sake of clarity , we shall call this type of outlier a \textbf{Grubbs Outlier}). Consider the following data set: is the lowest value 4.01 an outlier?
\begin{verbatim}
6.98 8.49 7.97 6.64
8.80 8.48 5.94 6.94
6.89 7.47 7.32 4.01
\end{verbatim}

Under the null hypothesis, there are no outliers present in the data set. 
We reject this hypothesis if the p-value is sufficiently small.
\end{frame}

%-------------------------------------------------%
\begin{frame}[fragile]
\frametitle{Grubbs Test for Determining an Outlier}
\begin{verbatim}
> grubbs.test(x, two.sided=T)
Grubbs test for one outlier
data: x
G = 2.4093, U = 0.4243, p-value = 0.05069
alternative hypothesis: lowest value 4.01 is an outlier
\end{verbatim}
We conclude that while small by comparison to the other values, the lowest value 4.01 is not an outlier.
\end{frame}

\end{document}


Kolmogorov Smirnov Tests
Non-parametric procedures
Compute the test statistic
Same distribution
