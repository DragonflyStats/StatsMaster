\documentclass[a4paper,12pt]{article}
%%%%%%%%%%%%%%%%%%%%%%%%%%%%%%%%%%%%%%%%%%%%%%%%%%%%%%%%%%%%%%%%%%%%%%%%%%%%%%%%%%%%%%%%%%%%%%%%%%%%%%%%%%%%%%%%%%%%%%%%%%%%%%%%%%%%%%%%%%%%%%%%%%%%%%%%%%%%%%%%%%%%%%%%%%%%%%%%%%%%%%%%%%%%%%%%%%%%%%%%%%%%%%%%%%%%%%%%%%%%%%%%%%%%%%%%%%%%%%%%%%%%%%%%%%%%
\usepackage{eurosym}
\usepackage{vmargin}
\usepackage{amsmath}
\usepackage{graphics}
\usepackage{epsfig}
\usepackage{framed}
\usepackage{subfigure}
\usepackage{fancyhdr}

\setcounter{MaxMatrixCols}{10}
%TCIDATA{OutputFilter=LATEX.DLL}
%TCIDATA{Version=5.00.0.2570}
%TCIDATA{<META NAME="SaveForMode"CONTENT="1">}
%TCIDATA{LastRevised=Wednesday, February 23, 201113:24:34}
%TCIDATA{<META NAME="GraphicsSave" CONTENT="32">}
%TCIDATA{Language=American English}

\pagestyle{fancy}
\setmarginsrb{20mm}{0mm}{20mm}{25mm}{12mm}{11mm}{0mm}{11mm}
\lhead{MA4128} \rhead{Kevin O'Brien} \chead{Week 8} %\input{tcilatex}

%http://www.electronics.dit.ie/staff/ysemenova/Opto2/CO_IntroLab.pdf
\begin{document}
\subsection{The $t$ distribution}
TESTING A HYPOTHESIS CONCERNING THE MEAN BY USE OF THE t
DISTRIBUTION:

The $t$ distribution is the appropriate basis for
determining the standardized test statistic when the sampling
distribution of the mean is normally distributed but $s$ is not
known. The sampling distribution can be assumed to be normal
either because the population is normal or because the sample is
large enough to invoke the central limit theorem. The $t$
distribution is required when the sample is small ($n < 30$). For
larger samples, normal approximation can be used. For the critical
value approach, the procedure is identical to that described in
Section 10.3 for the normal distribution, except for the use of $t$
instead of z as the test statistic.

\section{Confidence Interval} CONFIDENCE INTERVALS FOR THE MEAN\\
suppose that you wish to estimate the mean sales amount per
retail outlet for a particular consumer product during the past
year. The number of retail outlets is large. Determine the
95 percent confidence interval given that the sales amounts are
assumed to be normally distributed, $\bar{X} = $3,425, s = $200$ ,
and $n = 25.$\\ Ans. $3;346:60 to $3;503:40
\\
8.24. Referring to Problem 8.23, determine the 95 percent
confidence interval given that the population is assumed to be
normally distributed, $\bar{X} = $3,425, s = $200$ , and $n = 25.$
\\Ans. $3;342:44 to $3;507:56
\section{Two sample test}
Suppose one has two independent samples, x1, ..., xm and y1, ...,
yn, and wishes to test the hypothesis that the mean of the x
population is equal to the mean of the y population:

$H0 : \mu_{x} = \mu_{y}.$

Alternatively this can be formulated as $H0 : \mu_{x} - \mu_{y} =
0$.

Let $\bar{X}$ and $\bar{Y}$ denote the sample means of the xs and
ys and let $S_{x}$ and $S_{y}$ denote the respective standard
deviations. The standard test of this hypothesis $H_{0}$ is based
on the t statistic
\begin{equation}T = \frac{\bar{X} - \bar{Y} }{S_{p} \sqrt{1/m + 1/n} }
\end{equation}

where $S_{p}$ is the pooled standard deviation.

\begin{equation}
S_{p} = \sqrt{ \frac{(m-1)S^{2}_{x} +  (n-1)S^{2}_{y}}{m + n - 2}}
\end{equation}

Under the hypothesis $H_{0}$, the test statistic T has a t
distribution with $m + n - 2$ degrees of freedom when
\begin{itemize} \item both the xs and ys are independent random samples
from normal distributions \item the standard deviations of the x
and y populations, $\sigma_{x}$ and $\sigma_{y}$, are equal
\end{itemize}.

Suppose the level of significance of the test is set at $\alpha$.
Then one will reject H when $|T| < tn+m.2,\alpha/2$, where
$tdf,\alpha$ is the $(1 - \alpha)$ quantile of a t random variable
with df degrees of freedom.

If the underlying assumptions of

\end{document}