\documentclass[a4]{beamer}
\usepackage{amssymb}
\usepackage{graphicx}
\usepackage{subfigure}
\usepackage{newlfont}
\usepackage{amsmath,amsthm,amsfonts}
%\usepackage{beamerthemesplit}
\usepackage{pgf,pgfarrows,pgfnodes,pgfautomata,pgfheaps,pgfshade}
\usepackage{mathptmx}  % Font Family
\usepackage{helvet}   % Font Family
\usepackage{color}

\mode<presentation> {
 \usetheme{Default} % was Frankfurt
 \useinnertheme{rounded}
 \useoutertheme{infolines}
 \usefonttheme{serif}
 %\usecolortheme{wolverine}
% \usecolortheme{rose}
\usefonttheme{structurebold}
}

\setbeamercovered{dynamic}

\title[MA4413t]{Statistics for Computing \\ {\normalsize Lecture 10A}}
\author[Kevin O'Brien]{Kevin O'Brien \\ {\scriptsize kevin.obrien@ul.ie}}
\date{Summer 2011}
\institute[Maths \& Stats]{Dept. of Mathematics \& Statistics, \\ University \textit{of} Limerick}

\renewcommand{\arraystretch}{1.5}


%----------------------------------------------------------------------------------------------------------%
\begin{document}

\begin{frame}
\frametitle{This Class}
\begin{itemize}
\item Look at some worked examples of hypothesis testing
\item The paired t -test
\item Type I and Type II errors
\item Formulae
\end{itemize}
\end{frame}
%------------------------------------------------------------------------------------------------------------%
\begin{frame}
\begin{itemize}
\item The schedule of formulae that will be at the back of your examination paper will be posted on the SULIS site shortly.
\item It is advisable to familiarise yourself with the contents before the examination.
\item Please let me know as soon as possible if there are any issues with it.
\end{itemize}
\end{frame}
%----------------------------------------------------------------------------------------------------%
\frame{
\frametitle{Paired t test}
The paired t test was devloped by Guinness Brewery employee, Gossett, in1908.
\begin{itemize}
\item First we have to compute the case-wise differences.
\item Then we compute the variance of those differences.
\end{itemize}

}

%--------------------------------------------------------------------------------------------------------------------------%

% -  Type I and Type II error
% - Paired T test
% - Examples
% - Test for the mean
% - Test for p-valuies

%--------------------------------------------------------------------------------------------------------------------------%
\begin{frame}
\frametitle{Relationship between significance and Type I error}
\begin{itemize}
\item
\item
\end{itemize}
\end{frame}
\frame{
\frametitle{Paired T test}
\large
\begin{itemize}
\item Firstly we have to compute each of the case-wise differences.
\item Then we have to compute the mean value of these differences.
\item Lastly we also have to compute the standard deviation of the differences.
\end{itemize}
}

%----------------------------------------------------------------------------------------------------%
\frame{
\frametitle{Paired T test}
\begin{itemize}
\item $\mu_d$ mean value for the population of differences.
\item $\bar{d}$ mean value for the sample of differences,
\item $s_d$ standrd deviation of the differences for the paired sample data.
\item $n$ number of pairs
\end{itemize}


}
%----------------------------------------------------------------------------------------------------%
\begin{frame}
\frametitle{Type I and Type II errors - Die Example}
\begin{itemize}
\item Recall our die throw experiment example. 
\item Suppose we perform the experiment twice with two different dice.
\item We dont not know for sure whether or not either of the dice is fair or crooked.
\item Suppose we get a sum of 401 from one die, and 360 from the other.
\end{itemize}
\end{frame}




%-----------------------------------------------------------------------------------------------------%

\begin{frame}
\frametitle{Examples}
\large
For the sake of brevity, in the following example the following values are always used or realised.

\begin{itemize}
\item Test statistic  = 2.7
\item Significance level $\alpha$ = 0.05. Tests are always two tailed.
\item Samples are either large or of size $n = 10$
\item In the case of large samples CV = 1.96. For small samples, CV = .
\item
\end{itemize}
\end{frame}

%----------------------------------------------------------------------------------------------------%

\begin{frame}
\large
\begin{itemize}
\item All of the following example will have the same significance level and test statistic.
\item Hence the overall outcomes are the same ( $\alpha$ = 0.05, k=2, TS = 2.7)
\item We will use both the p-value method and the critcal value approach.
\item 
\end{itemize}
\end{frame}

%----------------------------------------------------------------------------------------------------%
\frame{
\frametitle{Hypothesis testing for Two Samples}
\large
\begin{itemize}
\item Paired t test
\item Difference of two mean (large samples)
\item Difference of two mean (small samples)
\item Difference of two propotions (large sample only)
\end{itemize}
}

%----------------------------------------------------------------------------------------------------%
\begin{frame}
\frametitle{Computing the Standard Error and Test Statistic }
\begin{itemize}
\item The point estimate is therefore $\bar{x}_1 -\bar{x}_2$
\item The standard error is computed from formulae.
\item The test statistic is therefore 
% \[ (\bar{x}_1 -\bar{x}_2) - (0) \over \mbox{S.E. }(\bar{x}_1 -\bar{x}_2  )} \]
\end{itemize}
\end{frame}

%-----------------------------------------------------------------------------------------------------%

\begin{frame}
\frametitle{Difference of proportions}
\large
\begin{itemize}
\item The point estimate is the difference 
\item Observed difference  = $7\%$. Under the null hypothesis, the expected difference is $0\%$
\item The standard error formulae
\end{itemize}
\end{frame}
%--------------------------------------------------------------------------------------%
\begin{frame}
\frametitle{Example}
\large
\begin{itemize}
\item The standard deviation of the life for a particular brand of ultraviolet tube is known to be $s = 500$ hr,
\item Also it is assumed, but not known, that the operating life of the tubes is normally distributed. \item The manufacturer claims that average tube life
is at least 9,000hr. \item Test this claim at the 5 percent level of significance against the alternative hypothesis
that the mean life is less than 9,000 hr, and given that for a sample of $n = 10$ tubes the mean operating
life was $\bar{x}  =  8,800$ hr.
\end{itemize}
\end{frame}

%--------------------------------------------------------------------------------------%
\begin{frame}
\frametitle{Example}
\large
\begin{itemize}
\item $H_0 \mbox{ : } $ $\mu \geq 9000$	Average life span is not less than 9000 hours.
\item $H_1 \mbox{ : } $ $\mu < 9000$    Average life span is  less than 9000 hours.
\end{itemize}
\bigskip
\begin{itemize}
\item The observed difference is -200 hours. (i.e. 8,800 - 9,000 hours)
\item The standard error is determined from formulae.
\[ S.E. \bar{x}  = {s \over \sqrt{n}} = {500 \over \sqrt{10}} \]
\end{itemize}
\end{frame}

%--------------------------------------------------------------------------------------%
\begin{frame}
\frametitle{Example}
\large
\begin{itemize}
\item The CV is determined from Murdoch Barnes Table 7, with $\alpha = 0.05$ and $k = 1$.
\item The sample is small n= 10  $df = n-1 = 9$.Therefore CV = 
\item (Remark: If the distribution was known to be normal, we could use $df = \infty$, i.e $CV = 1.645$).
\item Decision rule : Is $|TS| >CV$? Yes. 
\end{itemize}
\end{frame}

%--------------------------------------------------------------------------------------%
\end{document}