\documentclass[]{report}

\voffset=-1.5cm
\oddsidemargin=0.0cm
\textwidth = 480pt

\usepackage{framed}
\usepackage{subfiles}
\usepackage{graphics}
\usepackage{newlfont}
\usepackage{eurosym}
\usepackage{amsmath,amsthm,amsfonts}
\usepackage{amsmath}
\usepackage{color}
\usepackage{amssymb}
\usepackage{multicol}
\usepackage[dvipsnames]{xcolor}
\usepackage{graphicx}
\begin{document}









\subsection*{Die Throw Example}
\begin{itemize}
\item Suppose that the outcome of the die throw experiment was a sum of 401. In previous lectures, a simulation study found that only in approximately $1.75\%$ of cases would a fair die yield this result.
\item However, in the case of a crooked die (i.e. one that favours high numbers) this result would not be unusual.
\item A reasonable interpretation of this experiment is that the die is crooked, but importantly the experiment doesn't prove it one way or the other.
\item We will discuss the costs of making a wrong decision later (Type I and Type II errors).
\end{itemize}

%--------------------------------------------------------------------------------------------------------------------------%

\subsection*{Significance Level ($\alpha$}


\begin{itemize}
\item Traditionally, experimenters have used either the 0.05 level (sometimes called the 5\% level) or the 0.01 level (1\% level), although the choice of levels is largely subjective.  \item The lower the significance level, the more the data must diverge from the null hypothesis to be significant. \item Therefore, the 0.01 level is more conservative than the 0.05 level. \item The Greek letter alpha ($\alpha$) is sometimes used to indicate the significance level. \item We will use a significance level of $\alpha =0.05$ only in this module. You may assume this level unless clearly stated otherwise
\end{itemize}
\subsection{Significance Level}

\begin{itemize}
\item The significance level of a statistical hypothesis test is a fixed probability of wrongly rejecting the null hypothesis $H_0$, if it is in fact true.

\item Equivalently, the significance level (denoted by $\alpha$) is the probability that the test statistics will fall into the \textbf{\emph{critical region}}, when the null hypothesis is actually true. ( We will discuss the critical region shortly).

\item Common choices for $\alpha$ are $0.05$ and $0.01$
\end{itemize}


\section{ What is Statistical Inference?}
%---------------------------------------------------%

\begin{itemize}
\item Statistical inference is about inferring from the data about parameters that describe an assumed
model for the data.
\item 
Solution: In statistics, a model for the mechanism that has produced data is as-
sumed. The model is characterized by some parameters that are unknown. Having
data, statistics tries to infer from them some information about these parameters.
\end{itemize}
\section{What is Statistical Inference?}
\begin{itemize}
\item Hyptothesis testing 
\item Confidence Intervals 
\item Sample size estimation.
\end{itemize}






\textbf{Significance (Dice Example)}
\begin{itemize}
\item Suppose that the outcome of the die throw experiment was a sum of 401. In previous lectures, a simulation study found that only in approximately $1.75\%$ of cases would a fair die yield this result.
\item However, in the case of a crooked die (i.e. one that favours high numbers) this result would not be unusual.
\item A reasonable interpretation of this experiment is that the die is crooked, but importantly the experiment doesn't prove it one way or the other.
\item We will discuss the costs of making a wrong decision later (Type I and Type II errors).
\end{itemize}

%--------------------------------------------------------------------------------------------------------------------------%




\end{document}




