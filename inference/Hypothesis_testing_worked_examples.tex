




\frame{
\frametitle{Hypothesis Testing: Some Worked Examples}
\large
\begin{itemize}
\item[1] Small test of mean
\item[2] Difference of two mean (large samples)
\item[3] Difference of two mean (large samples, using p-value approach)
\item[4] Difference of two mean (small samples)
\item[5] Difference of two proportions
\item[6] Paired t test
\end{itemize}
}



%--------------------------------------------------------------------------------------------------------------------------%

%--------------------------------------------------------------------------------------%
%--------------------------------------------------------------------------------------%
\begin{frame}
\frametitle{Example 1 (a)}
\large
\begin{itemize}
\item The standard deviation of the life for a particular brand of ultraviolet tube is known to be $s = 500$ hr,
\item Also it is assumed, but not known, that the operating life of the tubes is normally distributed. \item The manufacturer claims that average tube life
is at least 9,000hr. \item Test this claim at the 5 percent level of significance against the alternative hypothesis
that the mean life is 9,000 hr, and given that for a sample of $n = 10$ tubes the mean operating
life was $\bar{x} = 8,800$ hr.
\item (Intuitively this would suggest a one-tailed test that the mean is less than 9000 hours)
\end{itemize}
\end{frame}


%--------------------------------------------------------------------------------------%
\begin{frame}
\frametitle{Example 1 (b) }
\large
\begin{itemize}
\item $H_0 \mbox{ : } $ $\mu = 9000$ Average life span is 9000 hours.
\item $H_1 \mbox{ : } $ $\mu \neq 9000$ Average life span is not 9000 hours.
\end{itemize}
\bigskip
\begin{itemize}
\item The observed difference is -200 hours. (i.e. 8,800 - 9,000 hours)
\item The standard error is determined from formulae.
\[ S.E.(\bar{x}) = {s \over \sqrt{n}} = {500 \over \sqrt{10}}  = 158.1139 \]
\end{itemize}
\end{frame}
%--------------------------------------------------------------------------------------%
\begin{frame}
\frametitle{Example 1 (c) : Test Statistic }
\large
\begin{itemize}
\item The test statistic is ${(8800-9000) -0 \over  158.11} = -1.265$
\item The CV is determined from Murdoch Barnes Table 7, with $\alpha = 0.05$ and $k = 2$.
\item The sample is small n= 10 $df = n-1 = 9$.Therefore $CV = 2.262$
\item (Remark: If the distribution was known to be normal, we could use $df = \infty$, i.e $CV = 1.96$).
\end{itemize}
\end{frame}
%--------------------------------------------------------------------------------------%
\begin{frame}
\frametitle{Example 1 (d) }
\large
\begin{itemize}


\item \textbf{Decision:}Is $|TS| >CV$? Is $1.265 > 2.262$?
\item No. We fail to reject the null hypothesis. Not enough evidence to say that the mean lifespan is not 9000 hours.
\end{itemize}
\end{frame}

\begin{frame}
\frametitle{Example 2: Difference in Means (a) }
Two sets of patients are given courses of treatment under two different drugs. The benefits
derived from each drug can be stated numerically in terms of the recovery times; the readings are given below.

\begin{itemize}
\item Drug 1:  $n_1$ = 40 , $\bar{x}_1$ = 3.3 days and $s_1 = 1.524$
\item Drug 2:  $n_2$ = 45 , $\bar{x}_2$ = 4.3 days and $s_2 = 1.951 $
\end{itemize}
\end{frame}

%-------------------------------------------------------------------------------------------%
\begin{frame}
\frametitle{Example 2: Difference in Means (b) }
\begin{itemize}
\item
The first step in hypothesis testing is to specify the null hypothesis and an alternative hypothesis.
\item When testing differences between mean recovery times, the null hypothesis is that the two population means are equal.
\item That is, the null hypothesis is:\\
$H_0: \mu_1 = \mu_2$\\
$H_1: \mu_1 \neq \mu_2$\\
\end{itemize}
(Remark: Two Tailed Test k = 2, and $\alpha = 0.05$)
\end{frame}

%-------------------------------------------------------------------------------------------%
\begin{frame}
\frametitle{Example 2: Difference in Means (c) }
\begin{itemize}
\item The observed difference in means is 1 day.
\item The relevant formula for the standard error is \[ S.E.(x_1 - x_2) = \sqrt{{s^2_1\over n_1}+{s^2_2 \over n_2}} \]
\item  \[ S.E(x_1 - x_2) = \sqrt{{(1.524)^2 \over 40}+{(1.951)^2 \over 45}}  = 0.377\mbox{ days} \]
\end{itemize}
\end{frame}

%-------------------------------------------------------------------------------------------%
\begin{frame}
\frametitle{Example 2: Difference in Means (d) }
\begin{itemize}
\item The Test Statistic is therefore
\[ TS = {\mbox{observed} - \mbox{null} \over \mbox{Std. Error}}  = {1 - 0 \over 0.377 } = 2.65 \]
\item The critical value $CV = 1.96$.
\item Is the TS greater than the CV? Is $2.65 > 1.96$?

\item \textbf{Conclusion:} we reject the null hypothesis. There is a significant different between both drugs, in terms of recovery times.

\end{itemize}
\end{frame}


%-------------------------------------------------------------------------------------------%
\begin{frame}
\frametitle{Example 3: Difference in Means (a) }
\begin{itemize}
\item We will approach the same problem in example 1, but this time using the p-value approach.
\item The first two steps i.e. formally stating the null and alternative hypothesis, and computing the test statistic are the same, are the same as example 1.
\item The third step is to compute the p-value:  $P(Z \geq |TS|)$.
\item From Murdoch Barnes table 3: $P(Z \geq 2.65) = 0.00402$ (i.e. less than half a percent).

\end{itemize}
\end{frame}

%-------------------------------------------------------------------------------------------%
\begin{frame}
\frametitle{Example 3: Difference in Means (b) }
\begin{itemize}
\item The p-value is  $0.00402$
\item The critical region has size $\alpha/k = 0.05/2 = 0.0250$.
\item We reject the null hypothesis because the computed p-value is less than $0.0250$.
\item \textbf{Conclusion:} we reject the null hypothesis. There is a significant different between both drugs, in terms of recovery times.
\end{itemize}
\end{frame}



%-------------------------------------------------------------------------------------------%
\begin{frame}
\frametitle{Example 4: Difference in Means (a) }
\begin{itemize}
\item For a random sample of 10 light bulbs, the mean bulb life is 4,000 hr with a standard deviation of 200 hours.
\item For another brand of bulbs whose useful life is also assumed to be normally distributed, a random sample of 8 has a sample mean of 4,300 hours
and a sample standard deviation of 250 hours. \item Test the hypothesis that there is no difference between the
mean operating life of the two brands of bulbs, using the 5 percent level of significance
\end{itemize}
\end{frame}
%-------------------------------------------------------------------------------------------%

\begin{frame}
\frametitle{Example 4: Difference in Means (b) }
\begin{itemize}\item $n_1 = 10$ and $n_2 = 8$.
\item $\bar{x}_1 = 4000$, $\bar{x}_2 = 4,300 $ , therefore  $\bar{x}_2 - \bar{x}_1 = 300$ hours
\item $s_1  = 200$, $s_2 = 250$ hours.
\item Degrees of freedom $n_1 + n_2 - 2 = 10 + 8 - 2 = 16$
\end{itemize}\end{frame}
%-------------------------------------------------------------------------------------------%
\begin{frame}
\frametitle{Example 4: Difference in Means (c) }
\textbf{Pooled variance estimate}
\[ s^2_p = {(n_1 - 1)s^2_1  + (n_2 - 1)s^2_ 2\over n_1 + n_2 - 2 } = {(9 \times 200^2 ) +( 7 \times 250^2) \over 16 } = 49843.75 \]

\end{frame}
%-------------------------------------------------------------------------------------------%
\begin{frame}
\frametitle{Example 4: Difference in Means (d) }
\textbf{Computing the Standard Error}
\[ S.E(\bar{x}_1 - \bar{x}_2) = \sqrt{s^2_p \left({1\over n_1}+{1\over n_2} \right)}\]

\[ S.E(\bar{x}_1 - \bar{x}_2) = \sqrt{49843.75 \left({1\over 10}+{1\over 9} \right)}\]

\[ S.E(\bar{x}_1 - \bar{x}_2) = \sqrt{11214.84} = 105.9\]

\end{frame}

%-------------------------------------------------------------------------------------------%
\begin{frame}
\frametitle{Example 4: Difference in Means (e) }
\textbf{Test Statistic and Critical Value}\\
\begin{itemize}
\item The Test Statistic is \[ TS  = {(4000-4300) - 0 \over 105.9}  = -2.83 \]
\item The Critical Value is determined from Murdoch Barnes table 7 with $\alpha = 0.05$, $k=2$, $df = 16 $
\item $CV = 2.120$
\item We can now apply the decision rule : Is the absolute value of the Test Statistic greater than the Critical Value?
\item Is $2.83 > 2.12$? Yes We reject $H_0$. There is evidence of a difference in means.
\end{itemize}
\end{frame}



%-------------------------------------------------------------------------------------------%

\begin{frame}
\frametitle{Example 5: Difference in Proportions (a)}
\begin{itemize}
\item An experiment is conducted investigating the long-term effects of early childhood intervention programs (such as head start).
\item In one experiment, the high-school drop out rate of the experimental group (which attended the early childhood program)
 and the control group (which did not) were compared.
\item In the experimental group, 73 of 85 students graduated from high school. \item In the control group, only 43 of 82 students graduated.
Is this difference statistically significant? (Assume that the 0.05 level is chosen.) \end{itemize}
\end{frame}

%-------------------------------------------------------------------------------------------%
\begin{frame}
\frametitle{Example 5: Difference in Proportions (b)}
\begin{itemize}
\item
The first step in hypothesis testing is to specify the null hypothesis and an alternative hypothesis.
\item When testing differences between proportions, the null hypothesis is that the two population proportions are equal.
\item That is, the null hypothesis is:\\
$H_0: \pi_1 = \pi_2$\\
$H_1: \pi_1 \neq \pi_2$\\
\end{itemize}
(Remark: Two Tailed Test k = 2, and $\alpha = 0.05$)
\end{frame}
%-------------------------------------------------------------------------------------------%
\begin{frame}
\frametitle{Example 5: Difference in Proportions (c)}
\begin{itemize}
\item The next step is to compute the difference between the sample proportions.
\item In this example, $\hat{p}_1 - \hat{p}_2$ = $73/85 - 43/82$ = $0.8588 - 0.5244$ = 0.3344.
\item Difference is $33.44\%$
\end{itemize}
\end{frame}



%-------------------------------------------------------------------------------------------%
\begin{frame}
\frametitle{Example 5: Difference in Proportions (d)}
The formula for the estimated standard error is:

\[ S.E (\hat{p}_1 - \hat{p}_2)  = \sqrt{\bar{p}(100- \bar{p} \left( {1 \over n_1} + {1 \over n_2}  \right)} \]


where $\bar{p}$ is a aggregate proportion ( proportion of successes from overall sample, regardless of which group they are in).
\end{frame}

%-------------------------------------------------------------------------------------------%


\begin{frame}
\frametitle{Example 5: Difference in Proportions (d)}
\textbf{Aggregate Proportion}:\\
\[ \bar{p}  = {x_1  + x_2 \over n_1 + n_2} \times 100\% = {73+43 \over 85 + 82} \times 100\% = { 116 \over 167}\times 100\% = 69.5\% \]
\textbf{Standard Error}:\\
\[ S.E (\hat{p}_1 - \hat{p}_2)  =  \sqrt{69.5 \times 30.5 \left( {1 \over 85} + {1 \over 82}  \right)}  = 7.13\% \]
\end{frame}



%-------------------------------------------------------------------------------------------%
\begin{frame}
\frametitle{Example 5: Difference in Proportions (e)}
\textbf{Test Statistic}:
\begin{itemize} \item Observed difference :\\
85.88\% - 52.44\%  = 33.44\% [ i.e (73/85) - (43 /82) ]
\item Test Statistic is therefore \[T.S. = {33.44\% \over 7.13\%} = 4.69\]
\end{itemize}

\end{frame}
%-------------------------------------------------------------------------------------------%
\begin{frame}
\frametitle{Example 5: Difference in Proportions (e)}
\begin{itemize}
\item The Critical value is 1.96 ( Large sample , $\alpha = 0.05$, k=2).

\item The test statistic TS = 4.69, is greater than the critical value CV = 1.96, so we reject the null hypothesis.
\item The conclusion is that the probability of graduating from high school is greater for students who have participated in the early childhood intervention program than for students who have not.
\end{itemize}

\end{frame}




%----------------------------------------------------------------------------------%
\begin{frame}
\frametitle{Example 6: Paired Difference (a)}
\begin{itemize}
\item An automobile manufacturer collects mileage data for a sample of $n = 10$ cars in various weight categories
using a standard grade of gasoline with and without a particular additive. \item Of course, the engines were tuned to the same
specifications before each run, and the same drivers were used for the two gasoline conditions (with the driver in fact being
unaware of which gasoline was being used on a particular run). \item Given the mileage data on the next slide,  test the hypothesis
that there is no difference between the mean mileage obtained with and without the additive, using the 5 percent level of
significance \end{itemize}
\end{frame}
%-------------------------------------------------------------------------------------------%
\begin{frame}
\frametitle{Example 6: Paired Difference (b)}
\small
\begin{center}
\begin{tabular}{|c|c|c|c|c|}\hline
car & with additive & without additive & $d_i$ & $d^2_i$\\\hline
1&36.7&36.2&0.5&0.25\\\hline
2&35.8&35.7&0.1&0.01\\\hline
3&31.9&32.3&-0.4&0.16\\\hline
4&29.3&29.6&-0.3&0.09\\\hline
5&28.4&28.1&0.3&0.09\\\hline
6&25.7&25.8&-0.1&0.01\\\hline
7&24.2&23.9&0.3&0.09\\\hline
8&22.6&22.0&0.6&0.36\\\hline
9&21.9&21.5&0.4&0.16\\\hline
10&20.3&20.0&0.3&0.09\\\hline
\end{tabular}
\end{center}
\end{frame}

%--------------------------------------------------------------------------------------------------------------------------%
%-------------------------------------------------------------------------------------------%
\begin{frame}
\frametitle{Example 6: Paired Difference (c)}
\begin{itemize}
\item The average of the case wise differences is computed as \[\bar{d} = {\sum d_i \over n}\]
\[ \bar{d} = { 0.5 + 0.1  - 0.4 + \ldots + 0.3 \over 10 }= 0.17 \]
\item Also, using last column, $\sum d^2_i = (0.25 + 0.01 + 0.16 + \ldots + 0.09) = 1.31$
\end{itemize}

\end{frame}


\begin{frame}
\frametitle{Example 6: Paired Difference (d)}
\textbf{Sample variance of the case-wise differences}:\\
\large
\[s_d = \sqrt{ {\sum d_i^2 - n\bar{d}^2 \over n-1}}\]
We know the following:
\begin{itemize}
\item The sample size $n$ which is 10.
\item The average of the case-wise differences. $\bar{d} = 0.17$
\item  $\sum d^2_i = 1.31$
\end{itemize}
\end{frame}



\begin{frame}
\frametitle{Example 6: Paired Difference (e)}
\textbf{Sample variance of the case-wise differences}:\\
\[s_d = \sqrt{ {\sum d_i^2 - n\bar{d}^2 \over n-1}}\]

\[s_d = \sqrt{ {\sum 1.31 - 10(0.17)^2 \over 9}} = 0.337\]

\textbf{The standard error:}// \[ S.E.(\bar{d}) = s_d / \sqrt{n} = {0.0337 \over 3.16} = 0.107\]
\end{frame}

\begin{frame}
\frametitle{Example 6: Paired Difference (f)}
\textbf{Null and Alternative Hypotheses}:\\
\begin{itemize}
\item That is, the null hypothesis is:\\
$H_0: \mu_d = 0$ No Difference\\
$H_1: \mu_d \neq 0$ Difference \\
\end{itemize}
\textbf{Test Statistic}:\\
\begin{itemize}
\item TS = 0.17 / 0.107 = 1.59
\end{itemize}
\end{frame}

\begin{frame}
\frametitle{Example 6: Paired Difference (g)}
\textbf{Critical value}:
\begin{itemize}
\item $\alpha = 0.05, k = 2$ \item small sample , so $df = n-1 = 9$
\item From Murdoch Barnes Table 7: CV = 2.262.
\end{itemize}
\bigskip
\textbf{Decision Rule}:\\
Is $|TS| > CV$? No, we fail to reject the null hypothesis.
\end{frame}

\end{document}


