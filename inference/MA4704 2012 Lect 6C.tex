\documentclass[a4]{beamer}
\usepackage{amssymb}
\usepackage{graphicx}
\usepackage{subfigure}
\usepackage{newlfont}
\usepackage{amsmath,amsthm,amsfonts}
%\usepackage{beamerthemesplit}
\usepackage{pgf,pgfarrows,pgfnodes,pgfautomata,pgfheaps,pgfshade}
\usepackage{mathptmx}  % Font Family
\usepackage{helvet}   % Font Family
\usepackage{color}

\mode<presentation> {
 \usetheme{Default} % was
 \useinnertheme{rounded}
 \useoutertheme{infolines}
 \usefonttheme{serif}
 %\usecolortheme{wolverine}
% \usecolortheme{rose}
\usefonttheme{structurebold}
}

\setbeamercovered{dynamic}

\title[MA4704]{Technological Mathematics 4 \\ {\normalsize MA4704 Lecture 6C}}
\author[Kevin O'Brien]{Kevin O'Brien \\ {\scriptsize Kevin.obrien@ul.ie}}
\date{Spring Semester 2013}
\institute[Maths \& Stats]{Dept. of Mathematics \& Statistics, \\ University \textit{of} Limerick}

\renewcommand{\arraystretch}{1.5}

\begin{document}

\begin{frame}
\titlepage
\end{frame}

% - Last lecture
% - symmetric Intervals
% - computing Quantiles
% - student's t distribution
% - CI for means
% - CI for means ( small samples)
% - CI for Props
% - CI for differences.


%-----------------------------------------------------------%

\begin{frame}
\frametitle{Computing Confidence Intervals}
Confidence limits are the lower and upper boundaries / values of a confidence interval, that is, the values which define the range of a confidence interval. The general structure of a confidence interval is as follows:

\[ \mbox{Point Estimate}  \pm \left[ \mbox{Quantile} \times \mbox{Standard Error} \right] \]


\begin{itemize}
\item Point Estimate: estimate for population parameter of interest, i.e. sample mean, sample proportion.
\item Quantile: a value from a probability distribution that scales the intervals according to the specified confidence level.
\item Standard Error: measures the dispersion of the sampling distribution for a given sample size $n$.
\end{itemize}
\end{frame}



%-----------------------------------------------------------%


\begin{frame}
\frametitle{Confidence Intervals (Revision) }

\begin{itemize}
\item The $95\%$ confidence interval is a range of values which contain the true population parameter (i.e. mean, proportion etc) with a probability of $95\%$.
\item We can expect that a $95\%$ confidence interval will not include the true parameter values $5\%$ of the time.
\item A confidence level of $95\%$ is commonly used for computing confidence interval, but we could also have confidence levels of $90\%$, $99\%$ and $99.9\%$.
\end{itemize}

\end{frame}

%-----------------------------------------------------------%


\begin{frame}
\frametitle{Confidence Level (Revision) }

\begin{itemize}
\item A confidence level for an interval is denoted to $1-\alpha$ (in percentages: $100(1-\alpha)\%$) for some value $\alpha$.
\item A confidence level of $95\%$ corresponds to $\alpha = 0.05$.
\item $100(1-\alpha)\%$ = $100(1-0.05)\%$  = $100(0.95)\%$ = $95\%$
\item For a confidence level of $99\%$, $\alpha = 0.01$.
\item Knowing the correct value for $\alpha$ is important when determining quantiles.
\end{itemize}

\end{frame}
\begin{frame}
\frametitle{Quantiles }

\begin{itemize}
\item The quantile is a value from a probability distribution that scales the intervals according to the specified confidence level.
\item For practical purposes, the quantile can be taken from the standard normal distribution, if the sample is larger than 30, further to the central limit theorem.
\item For a specified confidence level $1-\alpha $, the corresponding quantile is the value $a$ that satisfies the following identity (when $n > 30$):

    \[ p( -a \leq Z \leq a) = 1- \alpha \]

\end{itemize}

\end{frame}
%-----------------------------------------------------------%

\begin{frame}
\frametitle{Quantiles}

\begin{itemize} \item When the sample size $n$ is greater than 30, we can compute the quantile using Murdoch Barnes table 3.

\item $95\%$ of Z random variables are between -1.96 ( quantile for $2.5\%$)and 1.96 ( quantile for $97.5\%$)
\end{itemize}

\begin{itemize}
\item If the confidence level is $95\%$, then the quantile is 1.96. Recall
\[ p( -1.96 \leq Z \leq 1.96) = 0.95 \]

\item If the confidence level is $90\%$, then the quantile is 1.645.
\[ p( -1.645 \leq Z \leq 1.645) = 0.90 \]

\item If the confidence level is $99\%$, then the quantile is 2.576.
\[ p( -2.576 \leq Z \leq 2.576) = 0.99 \]

\end{itemize}



\end{frame}
\begin{frame}
\frametitle{Confidence Intervals for Sample Means}
Broadly speaking, there are three different types of confidence interval
\begin{description}
\item[Type 1] Sample with \textbf{known} population variance
\begin{itemize}
\item The size of the sample doesn't matter.
\end{itemize}
\item[Type 2] Large sample with \textbf{unknown} population variance
\begin{itemize}
\item The size of the sample is more than 30 ($n > 30$)
\end{itemize}
\item[Type 3] Small sample with \textbf{unknown} population variance
\begin{itemize}
\item The size of the sample is 30 or less ($n\leq 30$)
\end{itemize}
\end{description}
\end{frame}

%--------------------------------------------------%
\begin{frame}
\frametitle{Confidence Intervals for Sample Means}
\textbf{Type 1 : Known Population Variance}\\
\begin{itemize}
\item This type of confidence interval is very rare in practive, but it is very simple to implement and used as introductory
teaching material with those studying confidence intervals for the first time.
\item This type of confidence interval is computed using this formula:
\[ \bar{x} \pm z_{(\alpha/2)}{\sigma \over \sqrt{n}} \]
\item A description of each item is on the next slide.
\end{itemize}
\end{frame}


%--------------------------------------------------%
\begin{frame}
\frametitle{Confidence Intervals for Sample Means}
\textbf{Type 1 : Known Population Variance}\\
\begin{itemize}

\item The point estimate is the sample mean : $\bar{x}$

\item We use a \textbf{\textit{quantile}} from the standard normal (Z) distribution to ``scale"
the confidence interval to the specified confidence level (usually 95\%).

\item Let the confidence level be denoted in the for $(1-\alpha)\times 100\%$, and hence determine $\alpha$.
For example, if the confidence level is 95\%, then $\alpha$ is 0.05 (or 5\%).

\item The Quantile ($z_{(\alpha/2)}$) is the value for the Standard Normal Tables (for example Murdoch Barnes Table 3) such that
\[ P(Z \geq z_{(\alpha/2)}) = {\alpha \over 2}\]

\item For a 95\% confidence interval, the quantile is 1.96. For a 99\% confidence interval, the quantile is 2.576.
\end{itemize}
\end{frame}
%--------------------------------------------------%
\begin{frame}
\frametitle{Confidence Intervals for Sample Means}
\textbf{Type 1 : Known Population Variance}\\
\begin{itemize}

\item The population standard deviation is $\sigma$. The sample size is $n$.
\item The standard error to be used in this confidence interval is
\[ \mbox{S.E}(\bar{x}) = \frac{\sigma}{\sqrt{n}}\]

\end{itemize}
\end{frame}


%--------------------------------------------------%
\begin{frame}
\frametitle{Confidence Intervals for Sample Means}
\textbf{Type 2 : Large Sample, Unknown Population Variance}\\
\begin{itemize}

\item The point estimate is the sample mean : $\bar{x}$

\item For a type 2 confidence interval, we can determine a \textbf{\textit{quantile}} for the confidence interval in the same way that for the Type 1 confidence interval.

\item Recall: For a 95\% confidence interval, the quantile is 1.96. For a 99\% confidence interval, the quantile is approximately 2.58.

\end{itemize}
\end{frame}

%--------------------------------------------------%
\begin{frame}
\frametitle{Confidence Intervals for Sample Means}
\textbf{Type 2 : Large Sample, Unknown Population Variance}\\
\begin{itemize}
\item The population standard deviation which we denote $\sigma^2$ is unknown.
Instead we are given the sample variance $s^2$. We use the sample standard deviation (the square root of the variance) as an estimate for the population standard deviation $\sigma$.
\[ s \mbox{ is an estimate for } \sigma \]

\item The sample size is $n$.
\item The standard error to be used in a Type 2 confidence interval is
\[ \mbox{S.E}(\bar{x}) = \frac{s}{\sqrt{n}}\]

\end{itemize}
\end{frame}

%--------------------------------------------------%
\begin{frame}
\frametitle{Confidence Intervals for Sample Means}
\textbf{Type 3 : Small Sample, Unknown Population Variance}\\
\begin{itemize}
\item Firstly, we will clarify the similarities of the Type 2 and Type 3 confidence interval.
\item The point estimate is the sample mean $\bar{x}$.
\item The sample standard deviation $s$ is used to estimate the population $\sigma$.
\item The standard error is
\[ \mbox{S.E}(\bar{x}) = \frac{s}{\sqrt{n}}\]
\end{itemize}
\end{frame}

%----------------------------------------------------%
%--------------------------------------------------%
\begin{frame}
\frametitle{Confidence Intervals for Sample Means}
\textbf{Type 3 : Small Sample, Unknown Population Variance}\\
\begin{itemize}
\item The key difference is in determining the quantile. Rather than use the standard normal distribution, we must use the student $t-$ distribution. Quantiles for this distribution are also tabulated in statistical tables (for Example, Murdoch Barnes Table 7).
\item Recall that we must determine a value for $\alpha$ ( and hence $\alpha/2$). For a 95\% confidence interval, $\alpha= 0.05$  and $\alpha/2 = 0.025$.
\item Computing a quantile from the $t-$ distribution additionally requires the specification of the \textit{\textbf{degrees of freedom}}. Degrees of Freedom are often denote as $df$ or by the greek letter $\nu$ (``nu").
\item For small sample confidence intervals (i.e. $n \leq 30$), the degrees of freedom are
\[ df = n-1 \]
\end{itemize}
\end{frame}




%--------------------------------------------------%
\begin{frame}
\frametitle{Using the $t-$distribution for large samples}

\begin{itemize}
\item The $t-$distribution is used for computing quantiles in the case of small samples (i.e. when sample size $n \leq 30$).
\item A key value in the $t-$distribution is the degrees of freedom, denoted $df$ (or sometimes $\nu$). For small samples \[ df= n-1\].
\item The $t-$distribution is used for computing quantiles in the case of large samples too, as an alternative to using the $Z$ distribution.
\item In this case , use the value $\infty$ as the degrees of freedom (see bottom row of table 7).
\[ df= \infty\]
\item This means that we can use the $t-$ distribution for finding the quantiles of all types of confidence intervals.

\end{itemize}
\end{frame}

%--------------------------------------------------%

%--------------------------------------------------%

%-----------------------------------------------------------%
\begin{frame}
\frametitle{The Central Limit Theorem }
\begin{itemize}
\item This theorem states that as sample size $n$ is increased, the sampling distribution of the mean (and for other sample statistics as well) approaches the normal distribution in form, regardless of the form of the population distribution from
which the sample was taken.

\item For practical purposes, the sampling distribution of the mean can be assumed to be
approximately normally distributed, even for the most non-normal populations or processes, whenever the
sample size is $n > 30$.

\item (For populations that are only somewhat non-normal, even a smaller sample size will
suffice. A variation of the normal distribution can be used for such circumstances.)
\end{itemize}


\end{frame}
%-----------------------------------------------------------%









%-----------------------------------------------------------%

\begin{frame}
\frametitle{Standard Error}

\begin{itemize}
\item The standard error measures the dispersion of the sampling distribution.
\item For each type of point estimate, there is a corresponding standard error.
\item A full list of standard error formulae will be attached in your examination paper.
\item The standard error for a  mean is
\[ S.E( \bar{x} )  = {\sigma \over \sqrt{n}} \]
However, we often do not know the value for $\sigma$. For practical purposes, we use the sample standard deviation $s$ as an estimate for $\sigma$ instead.
\[ S.E( \bar{x} )  = {s \over \sqrt{n}} \]
\end{itemize}

\end{frame}









%------------------------------------------------------------------------------%
\begin{frame}
\frametitle{Confidence Interval for a mean (1) }
Finally, an example to finish the class:
\begin{itemize}
\item For a given week, a random sample of 100 hourly employees selected from a very large number of
employees in a manufacturing firm has a sample mean wage of $\bar{x} = 280$ dollars, with a sample standard deviation of
$s = 40$ dollars.
\item Estimate the mean wage for all hourly employees in the firm with an interval estimate such that we can be 95
percent confident that the interval includes the value of the population mean.
\end{itemize}

\end{frame}
%------------------------------------------------------------------------------%
\begin{frame}
\frametitle{Confidence Interval for a mean (2) }

\begin{itemize}
\item The point estimate in this case is the sample mean $\bar{x} = 280$ dollars.
\item We have a large sample (n=100) and the confidence level is $95\%$. Therefore the quantile  is 1.96.
\item The standard error is computed as follows:

\[ S.E( \bar{x} )  = {s \over \sqrt{n}}  =  {40 \over \sqrt{100}} = 4  \]
\item \textbf{Confidence Interval for mean}

\[
280 \pm (1.96 \times 4)  = (280 \pm 7.84) = (\;272.16\;,\;287.84\;)
\]

\end{itemize}
\end{frame}

\end{document}

%-----------------------------------------------------------%

% -- Lecture 8B
% -- Revise the Tables
% -- Sample Size Estimation for mean
% -- Example SSE for mean
% -- SSE for Proportion
% -- Example SSE for proportion
% -- Paired Test



\end{document}













%--------------------------------------------------------%

\begin{frame}
\begin{itemize}
\item SE = $\sqrt{ [p_1 \times (1 - p_1) / n_1] + [p_2 \times (1 - p_2) / n_2] } $
\item SE = $\sqrt{ [0.40 \times 0.60 / 400] + [0.30 \times 0.70 / 300] } $
\item SE  = $\sqrt{[ (0.24 / 400) + (0.21 / 300) ]}$ = $\sqrt{(0.0006 + 0.0007)}$ = \sqrt{0.0013} = 0.036
\end{itemize}
\end{frame}

\end{document}













%-----------------------------------------------------------%

% Confidence Interval for a Proportion
% One Sample
%------------------------------------------------------------------------------%



