\documentclass[a4]{beamer}
\usepackage{amssymb}
\usepackage{graphicx}
\usepackage{subfigure}
\usepackage{newlfont}
\usepackage{amsmath,amsthm,amsfonts}
%\usepackage{beamerthemesplit}
\usepackage{pgf,pgfarrows,pgfnodes,pgfautomata,pgfheaps,pgfshade}
\usepackage{mathptmx}  % Font Family
\usepackage{helvet}   % Font Family
\usepackage{color}
\mode<presentation> {
 \usetheme{Default} % was
 \useinnertheme{rounded}
 \useoutertheme{infolines}
 \usefonttheme{serif}
 %\usecolortheme{wolverine}
% \usecolortheme{rose}
\usefonttheme{structurebold}
}
\setbeamercovered{dynamic}
\title[MA4413]{Statistics for Computing \\ {\normalsize Lecture 8A}}
\author[Kevin O'Brien]{Kevin O'Brien \\ {\scriptsize Kevin.obrien@ul.ie}}
\date{Autumn Semester 2011}
\institute[Maths \& Stats]{Dept. of Mathematics \& Statistics, \\ University \textit{of} Limerick}
\renewcommand{\arraystretch}{1.5}
\begin{document}

\begin{frame}
\titlepage
\end{frame}


% - Last lecture
% - symmetric Intervals
% - computing Quantiles
% - student's t distribution
% - CI for means
% - CI for means ( small samples)
% - CI for Props 
% - CI for differences.

\begin{frame}
\frametitle{Last Lecture}
In the last lecture, we discussed
\begin{itemize}
\item Sampling distributions and the central limit theorem
\item Standard error ( standard deviation )
\item Introduction to confidence intervals
\end{itemize}

\end{frame}
%-----------------------------------------------------------%


\begin{frame}
\frametitle{Symmetric Intervals}

\begin{itemize}
\item Suppose we want to determine a \textbf{\emph{symmetric interva}}l of values such that the tails (the areas on either side of the interval) are of equal dimension. 
\item The tails are the probability areas on either side of this interval
\item If the area of the interval corresponds to $1-\alpha$ (in percentages: $100(1-\alpha)\%$), then the area of each tail is $\alpha/2$.
\item For example, suppose we wanted a symmetric interval for $95\%$. In this case the value of $\alpha$ is $0.05$ ( or $5\%$).
\item This means that the area in each tail is $0.025$  ( or $2.5\%$).
\end{itemize}
\end{frame}
%-----------------------------------------------------------%
\frame{
IMAGE HERE\\
SYMMETRIC INTERVAL AND TWO TAILS
}
%-----------------------------------------------------------%
\begin{frame}
\frametitle{Symmetric Intervals}
\begin{itemize}
\item Supposed we are asked to compute a $95 \% $ symmetric interval.
\item What is the quantile of the standard normal distribution that corresponds to this interval
\item Find $a$ such that $P(-a \leq Z \leq a) = 0.95$
\item Necessarily $P(Z \geq a) = 0.0250$. 
\item Using the Murdoch Barnes tables we can compute $a$ to be 1.96.
\end{itemize}
\end{frame}
%-----------------------------------------------------------%
\begin{frame}
It is known that a normal distribution has the following characterisitcs
\begin{itemize}
\item mean $\pm$ 1.96$\sigma$ includes 95\% of the population.
\item mean $\pm$ 2.58$\sigma$ includes 99\% of the population.
\end{itemize}
\end{frame}
%-----------------------------------------------------------%
\begin{frame}
\frametitle{Computing Confidence Intervals}
\begin{itemize}
\item Point Estimate : Sample mean, Sample Proportion
\item Quantile 
\item Standard Error
\end{itemize}
\[ \mbox{ Point Estimate } \pm \mbox{Quantile } \times \mbox{Std. Error} \]
\end{frame}
%-----------------------------------------------------------%
\begin{frame}
\frametitle{Confidence Limits}
\begin{itemize}
\item
\item 
\end{itemize}
\end{frame}
%-----------------------------------------------------------%
\begin{frame}
\frametitle{Confidence Intervals : Computing the point estimate}
\begin{itemize}
\item The point estimate is derived from the sample.
\item Two examples of point estimates are  the sample mean $\bar{x}$ and the sample proportion $\hat{p}$.
\end{itemize}
\end{frame}
%-----------------------------------------------------------%
\begin{frame}
\frametitle{Quantile}
\begin{itemize}
\item If the variance of the population is known, or if it is larger than 30. then we can use a quantile from the standard normal distribution.
\item If the sample size is 30 or less, we must use a variation of the standard normal distribution, called the \textbf{\emph{Student's t- Distribution}}.  
\item For the time being, we will assume that the variance is known. 
\item Later we will consider the case of small samples.
\end{itemize}
\end{frame}
%------------------------------------------------------------------------------%
\begin{frame}
\frametitle{The Central Limit Theorem}
\begin{itemize}
\item Recall the central limit theorem
\item The central limit theorem gives the normal distribution its key place in statistical inference.
\end{itemize}
\end{frame}
%------------------------------------------------------------------------------%
\frame{
\frametitle{Computing the Standard Error}
Standard Error for a mean
\[
S.E. (\hat{x}) \;=\; {\sigma \over \sqrt{n }}
\]
Standard Error for a proportion
\[
S.E. (\hat{p}) \;=\; \sqrt{ {p \times 1-p \over n }}
\]
}
%------------------------------------------------------------------------------%
\frame{

\frametitle{Example 1 CI for Mean (1)}

A company purchases a very large quantity of hardware components, and wishes to know the average weight of a component. A random sample of 625 components is weighed, and it is found that the mean sample weight is 150 gram.
The sample standard deviation is 30 grams.
\begin{itemize}
\item What is the estimate for the population mean weight?
\item What is the standard error?
\end{itemize}


}

%------------------------------------------------------------------------------%
\frame{

\frametitle{Example 1 CI for Mean (2)}

The estimate for the population mean weight is the sample mean. $\bar{x} = 150$ grams.
Standard Error for a mean
\[
S.E. (\hat{x}) \;=\; {30 \over \sqrt{625 } = {30 \over 25 } =1.2}
\]

}
%-----------------------------------------------------------%

\begin{frame}
\frametitle{Example 2: CI for Proportion (1)}

\begin{itemize}
\item $\hat{p}  = 0.04$
\item Sample Size $n=100$
\item Confidence level $1-\alpha$ is $95\%$
\end{itemize}

\end{frame}

%------------------------------------------------------------------------------%
\frame{

\frametitle{Example 2: CI for Proportion (2)}

Standard Error for a proportion
\[
S.E. (\hat{p}) \;=\; \sqrt{ {0.40 \times 0.60 \over 1000 }}
\]

}

%-----------------------------------------------------------%

\begin{frame}\frametitle{Student T distribution}

\begin{itemize}
\item 
\item Student's T distribution is a variation of the normal distribution, designed to facotr in the increased uncertainty resutling from smlaler samples.
\item  It is characterized by its degrees of freedom.
\end{itemize}

\end{frame}
%-----------------------------------------------------------%

\begin{frame}
\frametitle{CI for Proportion: Example (1)}

\begin{itemize}
\item $\hat{p}  = 0.04$
\item Sample Size $n=100$
\item Confidence level $1-\alpha$ is $95\%$
\end{itemize}

\end{frame}
%-----------------------------------------------------------%

\begin{frame}\frametitle{CI for Proportion: Example (2)}

\begin{itemize}
\item First, lets determine the quantile.
\item The sample size is large, so we will use the Z distribution.
\item (Alternatively we can uses the $t-$ distribution with $\infty$ degrees of freedom.
\end{itemize}

\end{frame}
%-----------------------------------------------------------%

\begin{frame}
\frametitle{Confidence Interval for a proportion}

\begin{itemize}
\item
\item
\item
\end{itemize}

\end{frame}
%-----------------------------------------------------------%

% Confidence Interval for a Proportion
% One Sample
%------------------------------------------------------------------------------%
\frame{
\textbf{Computing the point estimate}

Sample percentage

\[
\hat{p} = \frac{x}{n} \times 100\%
\]

\begin{itemize}
\item $\hat{p}$ - sample proportion.
\item $x$  - number of ``successes".
\item $n$  - the sample size.
\end{itemize}

}



%------------------------------------------------------------------------------%
\frame{
\frametitle{Example (2)}
\begin{itemize}
\item Let's look at this example again, but this time we will express our values in terms of percentages, rather than proportions. \item
We have to adjust the formula for standard error accordingly.
\end{itemize}


\[
S.E. (\hat{p}) \;=\; \sqrt{ {\hat{p} \times (100 -\hat{p} )\over n}} \hspace{1cm} [\%]
\]


\begin{itemize}
\item $ \hat{p} = {144/200}  \times 100\%  = 0.72 \times 100\%.  = 72\%$

\item $100 - \hat{p} = 100\% - 72\% = 28\% $
\end{itemize}
}


%------------------------------------------------------------------------------%
\frame{
\frametitle{Computing the Standard Error}

\[
S.E. (\hat{p}) \;=\; \sqrt{ {72 \times 28 \over 200 }}
\]

\[
S.E. (\hat{p}) \;=\; \sqrt{ {2016 \over 200 }} = \sqrt{10.08} = 3.175 \%
\]

So the standard error is $3.175 \%$.

}

%------------------------------------------------------------------------------%
\frame{
\textbf{Finite Population Correction Factor}

}
\end{document}
