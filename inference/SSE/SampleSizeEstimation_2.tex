
%-------------------------------------------------------%
\begin{frame}
\frametitle{Sample Size Estimation}

\begin{itemize} \item Recall the formula for margin of error, which we shall denote $E$.
\[  E = Q_{(1-\alpha)} \times \mbox{Std. Error}\]

\item $Q_{(1-\alpha)}$ denotes the quantile that corresponds to a $1-\alpha$ confidence level. (There is quite a bit of variation in notation in this respect.)
\item Also recall that the only way to influence the margin of error is to set the sample size accordingly.

\item Sample size estimation describes the selection of a sample size $n$ such that the margin of error does not exceed a pre-determined level $E$.
\end{itemize}
\end{frame}

%--------------------------------------------------------%
\begin{frame}
\frametitle{Sample Size Estimation for the Mean}

\begin{itemize}

\item The margin of error does not exceed a certain threshold $E$.
\[ E \geq Q_{(1-\alpha)} \times S.E.(\bar{x}), \]

\item which can be re-expressed as
\[E \geq Q_{(1-\alpha)} \times {\sigma \over \sqrt{n} }.\]

\item Divide both sides by $\sigma \times Q_{(1-\alpha)}$.
\[ \frac{E}{\sigma Q_{(1-\alpha)}} \geq {1 \over \sqrt{n} } \]

\item Square both sides

\[ \frac{E^2}{\sigma^2 Q_{(1-\alpha)}^2} \geq {1 \over n } \]


\end{itemize}
\end{frame}
%--------------------------------------------------------%
\begin{frame}
\frametitle{Sample Size Estimation for the Mean}

\begin{itemize}
\item Square both sides

\[ \frac{E^2}{\sigma^2 Q^2_{(1-\alpha)}} \geq {1 \over n } \]

\item Invert both sides, changing the direction of the relational operator.

\[ \frac{\sigma^2 Q^2_{(1-\alpha)}}{E^2} \leq n \]


\item The sample size we require is the smallest value for $n$ which satisfies this identity.
\item The sample standard deviation $s$ may be used as an estimate for $\sigma$.
\item (This formula would be provided on the exam paper).
\end{itemize}


\end{frame}

%--------------------------------------------------------%
\begin{frame}
\frametitle{SSE for the Mean: Example}

\begin{itemize}
\item An IT training company has developed a new certification program. The company wishes to estimate the average score of those who complete the program by self-study.  \item The standard deviation of the self study group is assumed to be the same as the overall population of candidates, ie. 21.2 points.
    \item How many people must be tested if the sample mean is to be in error by no more than 3 points, with 95\% confidence.
\end{itemize}
\end{frame}
%--------------------------------------------------------%
\begin{frame}
\frametitle{SSE for the Mean: Example}

\begin{itemize}
\item The sample size we require is the smallest value for $n$ which satisfies this identity.
\[ n \geq \frac{\sigma^2 Q^2_{(1-\alpha)}}{E^2}  \]
\item Remark: $1-\alpha$ = 0.95, therefore $Q_{(1-\alpha)}$ = 1.96. Also $E=3$ and $\sigma =21.2$.
\[ n \geq \frac{(21.2)^2 \times (1.96)^2}{3^2} \]
\item Solving, the required sample size is the smallest value of $n$ that satisfies
\[ n \geq 191.8410 \]
\item Therefore, the company needs to test 192 self-study candidates.
\end{itemize}
\end{frame}

%--------------------------------------------------------%
\begin{frame}
\frametitle{Sample Size Estimation for proportions}
We can also compute appropriate sample sizes for studies based on proportions.
\begin{itemize}
\item From before; \[ E \geq Q \times S.E.(\hat{p}). \]
(For the sake of brevity, we will just use the notation $Q$ for quantile.)

\item Divide both sides by Q.

\[ E \geq Q \times \sqrt{ {\pi(1-\pi)  \over n} }. \]

\end{itemize}
\end{frame}

%--------------------------------------------------------%
\begin{frame}
\frametitle{Sample Size Estimation for proportions}
\begin{itemize}
\item Remark: $E$ must be expressed in the same form as $\pi$, either as a proportion or as a percentage.
\item Remark : The standard error is maximized at $\pi = 0.50$,which is to say $\pi(1-\pi)$ can never exceed 0.25 ( or 25\%). Therefore the standard error is maximized at $\pi = 0.50$. To make the procedure as conservative as possible, we will use $0.25$ as our value for $\hat{p}_1 \times (1 - \hat{p}_1)$.
\item If we use percentages, $\pi \times (100-\pi)$ can not exceed 2500 (i.e $ 50 \times (100-50)=2500)$.

\[ E \geq Q \times \sqrt{{2500 \over n}}. \]


\end{itemize}

\end{frame}
%--------------------------------------------------------%
\begin{frame}
\frametitle{Sample Size Estimation for proportions}

\begin{itemize}

\item Dive both sides by $Q$, the square both sides:

\[ \left({E\over Q}\right)^2 \geq {2500 \over n}. \]

\item Invert both sides, changing the direction of the relational operator, and multiply both sides by $2500$.

\[ \left({Q\over E}\right)^2 \times 2500 \leq n. \]

\item The sample size we require is the smallest value for $n$ which satisfies this identity. (This formula would be provided on the exam paper, but without the maximized standard error).
\end{itemize}
\end{frame}
%--------------------------------------------------------%
\begin{frame}
\frametitle{SSE for proportions: Example}
\begin{itemize}
\item An IT journal wants to conduct a survey to estimate the true proportion of university students that own laptops.
\item The journal has decided to uses a confidence level of $95\%$, with a margin of error of $2\%$.
\item How many university students must be surveyed?
\end{itemize}
\end{frame}

%--------------------------------------------------------%
\begin{frame}
\frametitle{Sample Size estimation for proportions}

\begin{itemize}
\item Confidence level = 0.95. Therefore the quantile is $Q_{(1-\alpha)} = 1.96$
\item Using the formula: \[ n \geq \left({1.96 \over 2 }\right)^2 \times 2500  \]
\item The required sample size is the smallest value for $n$ which satisfies this identity: \[ n \geq 2401  \]
\item The required sample size is therefore 2401.
\end{itemize}
\end{frame}

\end{document}