\documentclass[]{report}

\voffset=-1.5cm
\oddsidemargin=0.0cm
\textwidth = 480pt

\usepackage{framed}
\usepackage{subfiles}
\usepackage{graphics}
\usepackage{newlfont}
\usepackage{eurosym}
\usepackage{amsmath,amsthm,amsfonts}
\usepackage{amsmath}
\usepackage{color}
\usepackage{amssymb}
\usepackage{multicol}
\usepackage[dvipsnames]{xcolor}
\usepackage{graphicx}
\begin{document}

\section*{Confidence Intervals}
\subsection{Interpreting Confidence Intervals}
\begin{itemize}
\item The upper bound ($+75$) suggests that those in group $B$ could have, on average, 75 more points than those in group $A$.
\item But the lower bound ($-15$) suggests  that those in group $A$ could have, on average, 15 more points than those in group $B$.
\item Also, the confidence interval allows for the possibility of both groups having equal means  (i.e. $\bar{x}_B-\bar{x}_A$ = 0)
\item Essentially we can not be $95\%$ confident that group $B$ has a higher mark than group $A$.
\end{itemize}

%------------------------------------------------------------------------------%

Although the sample mean is useful as an unbiased estimator of the population mean, there is no way of
expressing the degree of accuracy of a point estimator. In fact, mathematically speaking, the probability that the
sample mean is exactly correct as an estimator of the population mean is $P = 0$.

%------------------------------------------------------------------------------%

A confidence interval for the
mean is an estimate interval constructed with respect to the sample mean by which the likelihood that the interval
includes the value of the population mean can be specified.

The \emph{level of confidence} associated with a confidence interval indicates the long-run percentage
 of such intervals which would include the parameter being estimated.

%------------------------------------------------------------------------------%

\begin{itemize}
\item Confidence intervals for the mean typically are constructed with the unbiased estimator $\bar{x}$ at the midpoint
of the interval.

\item The $\pm Z \sigma_x$ or $\pm Z s_x$ frequently is called the \textbf{\emph{margin of error}} for the confidence interval.
\end{itemize}

%------------------------------------------------------------------------------%

We indicated that use of the normal distribution in estimating a population mean is warranted
for any large sample ($n > 30$), \textbf{and} for a small sample ($n \leq 30$) only if the population is normally distributed
and $\sigma$ is known.

%------------------------------------------------------------------------------%

\begin{itemize}
\item Now we consider the situation in which the sample is small and the population is normally distributed,
but $\sigma$ is not known.
\item The distribution is a family of distributions, with
a somewhat different distribution associated with the degrees of freedom ($df$). For a confidence interval for the
population mean based on a sample of size n, $df = n - 1$.
\end{itemize}

%--------------------------------------------------------------------------------------------------------------------------%



%--------------------------%

\textbf{Interpreting Confidence Intervals}
\begin{itemize}
\item In the previous lectures, we looked at confidence intervals, noting that these intervals are a pair of limits defining an interval.
\item Often, we can use confidence intervals to make inferences on a population parameter.
\item Consider the following example: Suppose that, when considering the leaving cert points of two groups of students $A$ and $B$, the difference of the sample means was found to be $\bar{x}_B-\bar{x}_A$ = 30 points.
\item We would surmise than the average points level for group $B$ is higher.
\item Lets suppose that the $95\%$ confidence interval was $(-15,75)$ points. Consider what each of the two numbers mean,
\end{itemize}

\subsection{Confidence Intervals}

\begin{itemize}
\item The length of life of a type of battery is estimated from a sample of 100 test items taken from a large population. 
\item Sample results show that the mean length of life is 57.4 hours with a standard deviation of 15.1 hours. \\ \bigskip
\item  Construct a 
95\% confidence interval for the mean length of life of all of these batteries.
\end{itemize}

%--------------------------%


\textbf{Interpreting Confidence Intervals}
\begin{itemize}
\item The upper bound ($+75$) inferences that those in group $B$ could have, on average, 75 more points than those in group $A$.
\item But the lower bound ($-15$) inferences that those in group $A$ could have, on average, 15 more points than those in group $B$.
\item Also, the confidence interval allows for the possibility of both groups having equal means  (i.e. $\bar{x}_B-\bar{x}_A$ = 0)
\item Essentially we can not be $95\%$ confident that group $B$ has a higher mark than group $A$.
\end{itemize}


%--------------------------------------------------------------------------------------------------------------------------%
\end{document}
