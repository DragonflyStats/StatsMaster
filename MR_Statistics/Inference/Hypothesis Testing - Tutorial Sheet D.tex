\documentclass[a4paper,12pt]{article}
%%%%%%%%%%%%%%%%%%%%%%%%%%%%%%%%%%%%%%%%%%%%%%%%%%%%%%%%%%%%%%%%%%%%%%%%%%%%%%%%%%%%%%%%%%%%%%%%%%%%%%%%%%%%%%%%%%%%%%%%%%%%%%%%%%%%%%%%%%%%%%%%%%%%%%%%%%%%%%%%%%%%%%%%%%%%%%%%%%%%%%%%%%%%%%%%%%%%%%%%%%%%%%%%%%%%%%%%%%%%%%%%%%%%%%%%%%%%%%%%%%%%%%%%%%%%
\usepackage{eurosym}
\usepackage{vmargin}
\usepackage{amsmath}
\usepackage{enumerate}
\usepackage{fancyhdr}
\usepackage{enumerate}
\usepackage{listings}
\usepackage{framed}
\usepackage{graphics}
\usepackage{epsfig}
\usepackage{subfigure}
\usepackage{fancyhdr}

\setcounter{MaxMatrixCols}{10}
%TCIDATA{OutputFilter=LATEX.DLL}
%TCIDATA{Version=5.00.0.2570}
%TCIDATA{<META NAME="SaveForMode" CONTENT="1">}
%TCIDATA{LastRevised=Wednesday, February 23, 2011 13:24:34}
%TCIDATA{<META NAME="GraphicsSave" CONTENT="32">}
%TCIDATA{Language=American English}

\pagestyle{fancy}
\setmarginsrb{20mm}{0mm}{20mm}{25mm}{12mm}{11mm}{0mm}{11mm}
\lhead{Maths Resource} \rhead{Hypothesis Testing}
\chead{Tutorial Sheet}
%\input{tcilatex}

\begin{document}
\begin{enumerate}
\item The mean and the standard deviation of the number of marks obtained in the biology leaving certificate exam by randomly selected male and female pupils are described below:\\
\begin{center}
\begin{tabular}{|c|c|c|c|}

  \hline
  % after \\: \hline or \cline{col1-col2} \cline{col3-col4} ...
	&Number&	Mean&	Std. Dev.\\ \hline
Male	&10	&57	&12\\
Female	&12	&61	&11\\
  \hline
\end{tabular}
\end{center}

Calculate a 95\% confidence interval for the difference between the mean number of marks obtained by males and females in the population of school leavers as a whole.
%(7 marks)

Test the hypothesis that males and females on average obtain the same mark in the biology leaving certificate exam. Use a significance level of $5\%$. You may assume that all required assumptions have been validated.\\% State your hypotheses clearly. What is the significance level of this test?
\bigskip
\begin{itemize}
\item[i.] Formally state the null and alternative hypotheses.
\item[ii.] Compute the Test Statistic.
\item[iii.] State the appropriate Critical Value for this hypothesis test.
\item[iv.] Discuss your conclusion to this test, supporting your statement with reference to appropriate values.
\end{itemize}

    \item 
    

An environmental group states that fewer than 60\% of industrial plants comply with air pollution standards. An independent researcher takes a sample of 400 plants and finds that 270 are complying with air pollution standards. 
\begin{enumerate}[(a)]
	\item  Carry out a hypothesis test to investigate the claim made by the environmental group. Clearly state your null and alternative hypotheses and your conclusion.
	\item Compute the 95\% confidence interval.
	\item Compute the Test Statistic.
	\item[(ii)] By interpreting this confidence interval, state your conclusion about the environmental group's claim? Explain how you made this decision.
\end{enumerate}

\item The following questions are concerned with important topic in hypothesis testing
\begin{itemize}
\item[i.] In the context of hypothesis testing, explain what a p-value is, and how it is used. Support your answer with a simple example.
\item[ii.]What is meant by Type I error and Type II error?
\end{itemize}

\item 
The standard deviations of data sets \texttt{X} and \texttt{Y} are 10 and 9 respectively. An inference procedure was carried out to assess whether or not \texttt{X} and \texttt{Y} can be assumed to have equal variance.
\begin{itemize}
\item[i.] Formally state the null and alternative hypothesis.
\item[ii.] The Test Statistic has been omitted from the computer code output. Compute the value of the Test Statistic.
\item[iii.] What is your conclusion for this procedure? Justify your answer.
\item[iv.] Explain how a conclusion for this procedure can be based on the $95\%$ confidence interval.
\end{itemize}

%---- R code for Variance Test ----%
%---- Dummy Code Included                   ----%
\begin{framed}
\begin{verbatim}
        F test to compare two variances

data:  X and Y
F = ......, num df = 13, denom df = 11, p-value = 0.7349
alternative hypothesis: true ratio of variances is not equal to 1
95 percent confidence interval:
 0.3639938 3.9475262
sample estimates:
ratio of variances
          .......
\end{verbatim}
\end{framed}

\item 
 During the early development of a medicated skin patch to help smokers break the habit, a test was conducted with 112 volunteers.  Because of the so-called
 "placebo effect", of people tending to respond positively just because attention is paid to them, about half the volunteers were given an unmedicated skin patch.  At the end of the study, the number of persons in each group who were abstinent and who were smoking are as follows:
 
 \begin{center}
 \begin{tabular}{|c|c|c|c|}
 \hline
 	&     Abstinent  &        Smoking   &     \\ \hline
 	
 	Medicated patch    &      21         &            36   &            57\\ \hline 
 	
 	Unmedicated patch  &       11       &              44   &            55\\ \hline 
 	
 	&       32       &              80    &          112\\ \hline
 \end{tabular} 
 \end{center}
 
 Let $\pi_1$ and $\pi_2$ denote the probabilities of quitting smoking with the medicated and unmedicated patches, respectively.
 
 \begin{itemize}
 	\item[(i)] Calculate point estimates for $\pi_1$ and $\pi_2$ 
 	
 	\item[(ii)] Compute the confidence interval of the difference in proportions $\pi_1 - \pi_2$ 
 	
 	\item[(iii)]Find a 95\% confidence interval for  .
 	\item[(iv)]  Based on the confidence interval found, can you conclude that (a) the success rate with the medicated patch is higher than for the control group that received the untreated patches?  (b) the medicated patch is not very effective?  Explain your answers.
 	\item  Carry out a testing procedure to investigate the claim that the medicated patch helps smokers break the habit.  State clearly the null hypothesis, alternative hypothesis, and conclusions drawn.
 \end{itemize}

%-------------------------------------------------------------------------------------------------- %

\item 
 During the early development of a medicated skin patch to help smokers break the habit, a test was conducted with 112 volunteers.  Because of the so-called
 "placebo effect", of people tending to respond positively just because attention is paid to them, about half the volunteers were given an unmedicated skin patch.  At the end of the study, the number of persons in each group who were abstinent and who were smoking are as follows:
 
 \begin{center}
 \begin{tabular}{|c|c|c|c|}
 \hline
 	&     Abstinent  &        Smoking   &     \\ \hline
 	
 	Medicated patch    &      21         &            36   &            57\\ \hline 
 	
 	Unmedicated patch  &       11       &              44   &            55\\ \hline 
 	
 	&       32       &              80    &          112\\ \hline
 \end{tabular} 
 \end{center}
 
 Let $\pi_1$ and $\pi_2$ denote the probabilities of quitting smoking with the medicated and unmedicated patches, respectively.
 
 \begin{itemize}
 	\item[(i)] Calculate point estimates for $\pi_1$ and $\pi_2$ 
 	
 	\item[(ii)] Compute the confidence interval of the difference in proportions $\pi_1 - \pi_2$ 
 	
 	\item[(iii)]Find a 95\% confidence interval for  .
 	\item[(iv)]  Based on the confidence interval found, can you conclude that (a) the success rate with the medicated patch is higher than for the control group that received the untreated patches?  (b) the medicated patch is not very effective?  Explain your answers.
 \end{itemize}

 
 Carry out a testing procedure to investigate the claim that the medicated patch helps smokers break the habit.  State clearly the null hypothesis, alternative hypothesis, and conclusions drawn.




\medskip
\item The strength of dosage of a plant growth enhancement chemical is often measured by the proportion of plants that grow faster. A particular dosage of the chemical is fed to 107 plants of these plants, 92 actually show faster growth.

\begin{itemize}
	\item[(i)] (1 Mark) Calculate a point estimate $\hat{p}$ for the proportion of plants that grow faster due to the dosage. 									 
		
	\item[(ii)]  Find a 95\% confidence interval for the proportion. 					
\end{itemize}



\end{enumerate}

\end{document}
