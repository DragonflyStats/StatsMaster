
\documentclass[]{article}

\voffset=-1.5cm
\oddsidemargin=0.0cm
\textwidth = 480pt

\usepackage{framed}
\usepackage{subfiles}
\usepackage{graphics}
\usepackage{newlfont}
\usepackage{eurosym}
\usepackage{amsmath,amsthm,amsfonts}
\usepackage{amsmath}
\usepackage{enumerate}
\usepackage{framed}
\usepackage{color}
\usepackage{multicol}
\usepackage{amssymb}
\usepackage{multicol}
\usepackage[dvipsnames]{xcolor}
\usepackage{graphicx}
\begin{document}
\section*{Hypothesis Testing - Tutorial Sheet B  }
\begin{enumerate}




\item 


Six athletes run 400 metres both at sea level and 1000m above sea level. The times they take are given below. 

\begin{center}
\begin{tabular}{|c|c|c|c|c|c|c|} \hline
Runner & 	1&	2 & 	3	& 4	& 5&  	6\\ \hline
Time at sea level&	45.8&	47.1&	45.4&	46.1 & 	48.4& 	44.9\\ \hline
Time at altitude&	44.8&	45.1&	45.4&	45.1 &	45.4& 	45.9\\ \hline
\end{tabular}
\end{center}
\begin{enumerate}[(a)]
    \item Is there any evidence that their speed depends on altitude? 
\item Calculate a 95\% confidence interval for the mean difference between the time at sea level and the time at altitude. Using this confidence interval, test the null hypothesis that the average time at altitude is 3 seconds faster than the average time at sea level (state the appropriate significance level).
\item What assumption is used in both parts a) and b)? 
\end{enumerate}
%===================================%
\item  A claim has been made that the mean body temperature of healthy adults is equal to 98.6 degrees Fahrenheit. A sample of 106 people who are taking medication for a chronic illness has produced a mean body temperature of 100.3 degrees Fahrenheit and a standard deviation of 0.65 degrees. Test the claim that this population of people taking medication has a different mean body temperature to the general population.  Clearly state your null and alternative hypotheses and your conclusion. Use a 5\% level of significance.




%================================================%
\item
A new process has been developed to reduce the level of corrosion of car bodies. Experiments were carried out on 11 cars using the new process and 11 cars using the old process. The average level of corrosion using the new process was 3.4 with a standard deviation of 0.5. The average level of corrosion using the old process was 4.2 with a standard deviation of 0.8. 
\begin{enumerate}[(a)]
\item Test the hypothesis that the mean of the level of corrosion does not depend on the process used. \\ \textit{For the time being, you may assume that assumption of equal variance is valid. }
\item Is there any evidence that the new process is better at a significance level of 5\%?
\item Calculate a 95\% confidence interval for the difference between the mean levels of corrosion under the two processes. Can it be stated that the mean level of corrosion is reduced by 1.5 at a significance level of 5\%? 
%\item What assumptions were used in ii) and iii)? 
\end{enumerate}

%================================================%

\item 
 A researcher was investigating computer usage among students at a particular university. Three hundred undergraduates and one hundred postgraduates were chosen at random and asked if they owned a laptop. It was found that 150 of the undergraduates and 80 of the postgraduates owned a laptop. 
\begin{itemize}
    \item[(a)]  Find a 95\% confidence interval for the difference in the proportion of undergraduates and postgraduates who own a laptop. 
    \item[(b)] On the basis of this interval, do you believe that postgraduates and undergraduates are equally likely to own a laptop? 
\end{itemize}
%================================================%

 

%%%%%%%%%%%%%%%%%%%%%%%%%%%%%%%%%%%%%%%%%%%%%%%%%%%%%%%%%%%%%%%%%%%%%%%%%%%%%%%%%%%%%%%%%%%%%%%

\item 
A test of a specific blood factor has been devised such that, for adults in Western Europe, the test score is normally distributed with mean 100 and standard deviation 10. A clinical research organization is carrying out research on the blood factor levels for sufferers of a particular disease.  

\begin{itemize}
	\item A study has obtained the following test scores for 14 randomly selected patients suffering from the disease in Ireland 
	\[ \{115, 113, 111, 109, 115, 108, 121, 114, 104, 113, 122, 90, 
	103, 116\}\]
	
	\item A similar study has obtained the following test scores for 15 randomly selected patients suffering from the disease in Germany .
	\[\{120, 137, 114, 120, 116, 118, 101, 110, 125, 113, 111, 128, 
	119, 117, 121\}\]
	
	\end{itemize}
	
		\begin{center}	
	\begin{tabular}{|c|c|c|c|} \hline 
		Sample & Mean & Std. Deviation & Sample Size \\  \hline 
		Ireland & 111 & 8.143 & 14 \\ \hline 
		Gemany & 118 & 8.350 & 15 \\ \hline
		\end{tabular} 
				\end{center}
You believe that there may be a difference in blood factor levels between the two countries. Test this hypothesis using a 5\% level of significance. \textit{For the time being, you may assume that assumption of equal variance is valid. }
\begin{enumerate}[(a)]
	\item Clearly state your null and alternative hypotheses.

	\item Compute the pooled standard deviation and/or pooled variance.
	\item Compute the test statistic.
	\item Discuss your conclusion to this test, supporting your statement with reference to appropriate values.
\end{enumerate}	



\item 

%(b) 
A researcher was investigating computer usage among students at a particular university. Three hundred undergraduates and one hundred postgraduates were chosen at random and asked if they owned a laptop. It was found that 150 of the undergraduates and 80 of the postgraduates owned a laptop. 
\begin{enumerate}[(a)]
    \item Find a 95\% confidence interval for the difference in the proportion of undergraduates and postgraduates who own a laptop. 
    \item On the basis of this interval, do you believe that postgraduates and undergraduates are equally likely to own a laptop? 
\end{enumerate}


\item 

The weight of 10 students was observed before commencement of their studies and after graduation (in kgs).

\begin{center}
    \begin{tabular}{|c||c|c|c|c|c|c|c|c|c|c|} \hline 
Student &	1&	2	&3	&4	&5	&6	&7	&8	&9	&10\\ \hline 
Weight before &	68	& 74	& 59	& 65	& 82&67&57&90&74&77\\ \hline 
Weight after&71&73&61&67&85&66&61&89&77&83\\ \hline 
\end{tabular}
\end{center}


\begin{enumerate}[(i)]
\item Compute the mean and the standard deviation of the case-wise differences.
	\item  Calculate a 95\% confidence interval for the amount of weight that students put on during their studies. 
	\item   Test the hypothesis that the mean weight of students increases during their studies at a significance level of  5\%. 
\end{enumerate}
% Using this confidence interval, test the hypotheses that on average 
% i) students put on 3 kilos during their studies
% ii) students lose 3 kilos during their studies.

% c) What assumption was made in order to both carry out the test and calculate the confidence interval?

%=====================================================%
\item A microbiologist measures the total growth in 24 hours of two strains of a germ culture  in the same petri dish. Nine identical specimens are prepared. The growth rate for the eight samples for each strain are tabulated below:

\begin{center}
	\begin{tabular}{|c|c|c|} \hline 
		Specimen &	Strain 1	&	Strain 2	\\ \hline \hline
		1 & 212 & 224 \\ \hline
		2 & 234 & 231 \\ \hline
		3 & 214 & 209 \\ \hline
		4 & 236 & 243 \\ \hline
		5 & 221 & 231 \\ \hline 
		6 & 212 & 216 \\ \hline
		7 & 202 & 213 \\ \hline 
		8 & 210 & 216 \\ \hline
		9 & 248 & 242 \\ \hline
	\end{tabular} 
\end{center}
\noindent At a significance level of 5\%, is there sufficient evidence to state that there is any difference in growth rates between the two strains.


% State your hypotheses clearly. What is the significance level of this test?


\begin{itemize}
	\item[(i)] Formally state the null and alternative hypotheses.
	\item[(ii)]  Compute the mean and standard deviation of the case-wise differences.
	\item[(iii)] Compute the test statistic.
	\item[(iv)] State the appropriate critical value for this hypothesis test.
	\item[(v)] Discuss your conclusion to this test, supporting your statement with reference to appropriate values.
\end{itemize}




\item 
A web-based software company claims that the average amount of time it takes for
online queries to be dealt with is less than 2 hours. Out of a sample of 15 queries, the
sample mean J? = 3.5 hours and the standard deviation is 30 minutes.
\begin{itemize}
	\item[(i)] Construct the null and alternative hypothesis statements.
	\item[(ii)] Test this claim using a significance level of 0.05.
	\item[(iii)] Describe the two types of errors associated with hypothesis testing and how
	they relate to this question?
\end{itemize}


\item 
A manufacturer of a common cold cure claims that the product provides
relief for 70\% of people who use it. ln a test of 400 people, it was found
that 300 people said the treatment provided relief.

\begin{enumerate}[(i)]
	\item Calculate a 95\% confidence interval for the true proportion of
	people who would get relief from the product.
	
	\item Suppose the manufacturer wishes to be 95\% confident that the
	prediction is correct to within 2\% of the true proportion. What
	sample size is needed?
	
	\item Using a significance level of 5\%, test the hypothesis that more than
	70\% of people who use the product find relief.\\ Clearly state your
	null and alternative hypotheses and your conclusion.
\end{enumerate}

\item 
A study was carried out to compare two treatments for the flu. A total of 500
newly diagnosed flu patients were randomly assigned to one of the two treatments.
\begin{itemize}
	\item Of the 280 assigned to the first treatment, 168 still had the flu after 2 days after
	diagnosis. \item Of the 220 assigned to the second treatment, 176 still had the flu after 2
	days after diagnosis. \end{itemize} Let $p_l$ denote the probability that a flu patient assigned to the
first treatment will still have the flu after 2 days after diagnosis; let $p_2$ denote the
corresponding probability for the second treatment.

\item 
A company organizes two evening courses, one on stock market trading and one on
spread-betting. From a sample of 100 clients, 64 of them choose the stock market course.

\begin{enumerate}[(i)]
	\item Calculate a 95\% confidence interval for the percentage of clients who
	prefer the stock market course. (5 marks)
	\item lt is known that in past years the percentage of clients who preferred the
	stock market trading classes was 75\%. Should the company conclude that
	the interest in the stock market trading courses has decreased significantly in
	comparison to past years? Clearly state $H_0$ and $H_1$.
\end{enumerate}

\begin{enumerate}[(i)]
	\item Provide an estimate of the difference between the population
	proportions (i.e. $\pi_l -\pi_2$).
	\item Calculate a 95\% confidence interval for the difference between the
	population proportions.
	\item Use a 0.05 significance level to determine if there is a difference
	between the two proportions. [Clearly state the null and alternative hypotheses
	and your conclusion].
\end{enumerate}

\item 
The height of 100 Americans and 50 Spaniards was observed. The mean and
standard deviation of the height of the Americans was 172cm and 13cm,
respectively. The mean and standard deviation of the height of the Spaniards
was 167cm and 12cm, respectively.

\begin{itemize}
	\item[(i)]Calculate a 95\% confidence interval for the difference between the mean height
	of all Americans and the mean height of all Spaniards.
	
	
	\item[(ii)] Without doing any further calculations, test the hypothesis that the mean
	height of all Americans is equal to the mean height of all Spaniards. Give a brief
	justification of your conclusion. What is the significance level of this test'?
\end{itemize}

\item 
The mean and standard deviation of the salaries of 16 Irish full-time workers are 5000 and 3000 Euros, respectively.
\begin{itemize}
	\item[(i)] Test the hypothesis that the mean salary of all Irish full-time workers is E4000 at a significance level of 5\%.
	\item[(ii)] What assumption is made in this testing procedure? Is this assumption reasonable?
\end{itemize}

\item (Not using for MS4222 2018).\\
A survey of 1000 Irish indicates that 750 have access to the Internet. A survey of 2000 Spaniards
indicates that 1400 have access to the Internet.
\begin{itemize}
	\item[(i)]  By calculating the appropriate p-value, test the hypothesis that the proportion of all Irish
	having access to the Internet is equal to the proportion of all Spaniards having access to the
	internet at a significance level of 5%.
	
	\item[(ii)] Calculate a 99\% confidence interval for the difference between the proportion of all Irish
	having access to the Internet and the proportion of all Spaniards having access to the
	internet.
\end{itemize}

\item (Not using for MS4222 2018).\\
Let $\pi$ be the proportion of workers in Ireland who spend at least one hour
per day in front of a computer terminal. Suppose that a researcher is going to take a
sample of n workers and estimate $\pi$ using $\hat{p}$, the proportion of workers in the sample
who spend at least one hour per day in front of a computer terminal.

\begin{itemize}
	\item[a.] (1 mark) How large
	should $n$ be if the researcher wants to be 90\% certain that his error is less than 0.01?
\end{itemize}
\end{enumerate}
\end{document}
