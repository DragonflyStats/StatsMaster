
\smallskip

% Confidence Intervals for Means and Proportions
% Dixon Q-Test



\section*{Question 1}
\begin{enumerate}[(a)]
    \item An IT competency test, used for staff recruitment, is devised so as to give a normal distribution of
scores with a mean of 100. A random sample of 49 experienced IT users who are given the test
achieve a mean score of 121 with a standard deviation of 14.
Compute a 95\% confidence interval for the group.
\item A random sample of 64 data scientists was selected and it was found that their average income was \${50,000} with a standard deviation of \${3,200}. 
Compute a 95\% confidence interval for the mean salary.
\item 
A manufacturer of computer monitors has, for many years, used a process giving a mean working life of 4700 hours for components. A new process is tried to see if it will increase the life significantly. A sample of 100 monitors gave a mean life of 5000 hours, with a standard deviation of 1400 hours.
Compute a 95\% confidence interval for the mean life of components built using the new process.

\end{enumerate}


	





\section*{Question 6}

A manufacturer of computer monitors has, for many years, used a process giving a mean working life of 4700 hours for components.
A new process is tried to see if it will increase the life significantly. A sample of 100 monitors gave a mean life of 5000 hours, with a standard deviation of 1400 hours.

Compute a 95\% confidence interval for the mean life of components built using the new process.


\section*{Question 7}
A research company is comparing computers from two different companies, X-Cel and Yellow, on the basis of energy consumption per hour. Given the following data, compute a $95\%$ confidence interval for the difference in energy consumption.
\begin{center}
	\begin{tabular}{|c|c|c|c|}
		\hline
		Type & sample size & mean & variance \\ \hline
		X-cel & 17 & 5.353 & 2.743 \\ \hline
		Yellow & 17 & 3.882 & 2.985 \\ \hline
	\end{tabular}
\end{center}
Remark: It is reasonable to believe that the variances of both groups is the same. Be mindful of this.


\end{document}
