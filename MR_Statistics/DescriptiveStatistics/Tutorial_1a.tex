
\documentclass[]{article}
\voffset=-1.2cm
\oddsidemargin=0.0cm
\textwidth = 480pt
\usepackage[utf8]{inputenc}
\usepackage[english]{babel}
\usepackage{framed}
\usepackage{graphicx}
\usepackage{enumerate}% http://ctan.org/pkg/enumerate
\usepackage{multicol}
\usepackage{amsmath}
\usepackage{amssymb}

\usepackage{eurosym}
\usepackage{vmargin}
\usepackage{amsmath}
\usepackage{graphics}
\usepackage{epsfig}
\usepackage{subfigure}
\usepackage{fancyhdr}
\usepackage{listings}
\usepackage{framed}







\begin{document}

	
\section{Inference Procedure: Worked Examples}



b)The mean and variance of height in a sample of 25 Irish students are 174cm and 100cm2, respectively.

i)Test the hypothesis that the mean height of all Irish students is 170cm at a significance level of 5%. 




	\subsection*{Question 1}
	
	\subsection*{Question 3}
	

	\noindent For the following sample of numbers, calculate the following:\\[0.2cm]
	{\bf(a)} Mean. \quad {\bf(b)} Median. \quad {\bf(c)} Standard deviation. \quad {\bf(d)} Inter-quartile range.
	
		\begin{center}
			\begin{tabular}{|cccccccccc|}
				\hline
				&&&&&&&&&\\[-0.4cm]
				2 & 4 & 2 & 1 & 5 & 3 & 0 & 4 & 1 & 8 \\
				\hline
			\end{tabular}
		\end{center}


\section{KB Tutorial 1}




\subsection*{Question 3}

\begin{center}
	\begin{tabular}{|cccccccccc|}
		\hline
		&&&&&&&&&\\[-0.4cm]
		2 & 4 & 2 & 1 & 5 & 3 & 0 & 4 & 1 & 8 \\
		\hline
	\end{tabular}
\end{center}

For the above sample of numbers, calculate the following:\\[0.2cm]
{\bf(a)} Mean. \quad {\bf(b)} Median. \quad {\bf(c)} Standard deviation. \quad {\bf(d)} Inter-quartile range.






\end{document}
