
\documentclass[a4paper,12pt]{article}
%%%%%%%%%%%%%%%%%%%%%%%%%%%%%%%%%%%%%%%%%%%%%%%%%%%%%%%%%%%%%%%%%%%%%%%%%%%%%%%%%%%%%%%%%%%%%%%%%%%%%%%%%%%%%%%%%%%%%%%%%%%%%%%%%%%%%%%%%%%%%%%%%%%%%%%%%%%%%%%%%%%%%%%%%%%%%%%%%%%%%%%%%%%%%%%%%%%%%%%%%%%%%%%%%%%%%%%%%%%%%%%%%%%%%%%%%%%%%%%%%%%%%%%%%%%%
\usepackage{eurosym}
\usepackage{vmargin}
\usepackage{amsmath}
\usepackage{graphics}
\usepackage{framed}
\usepackage{epsfig}
\usepackage{subfigure}
\usepackage{fancyhdr}

\setcounter{MaxMatrixCols}{10}
%TCIDATA{OutputFilter=LATEX.DLL}
%TCIDATA{Version=5.00.0.2570}
%TCIDATA{<META NAME="SaveForMode"CONTENT="1">}
%TCIDATA{LastRevised=Wednesday, February 23, 201113:24:34}
%TCIDATA{<META NAME="GraphicsSave" CONTENT="32">}
%TCIDATA{Language=American English}

\pagestyle{fancy}
\setmarginsrb{20mm}{0mm}{20mm}{25mm}{12mm}{11mm}{0mm}{11mm}
\lhead{MA4128} \rhead{Kevin O'Brien} \chead{Assumptions for Linear Models} %\input{tcilatex}

\begin{document}
\section{Information Criterions}
We define two types of information criterion: the Akaike Information Criterion (AIC) and the
Schwarz?s Bayesian Information Criterion (BIC). The Akaike information criterion is a measure
of the relative goodness of fit of a statistical model.
\[AIC = 2p - 2 ln(L)\]


%=================================================%
\begin{itemize}
	\item p is the number of predictor variables in the model.
	\item L is the value of the Likelihood function for the model in question.
	\item For AIC to be optimal, n must be large compared to p.
\end{itemize}

An alternative to the AIC is the Schwarz BIC, which additionally takes into account the
sample size n.
\[BIC = p ln n - 2 ln(L)\]
When using the AIC (or BIC) for selecting the optimal model, we choose the model for which
the AIC (or BIC) value is lowest.


Akaike Information Criterion

\begin{itemize}
	\item Akaike?s information criterionis a measure of the goodness of fit of an estimated statistical
	model. The AIC was developed by Hirotsugu Akaike under the name of ?an information
	criterion? in 1971.
	\item The AIC is a model selection tool i.e. a method of comparing two or more candidate
	regression models. The AIC methodology attempts to find the model that best explains
	the data with a minimum of parameters. (i.e. in keeping with the law of parsimony)
	\item The AIC is calculated using the ?likelihood function? and the number of parameters .
	The likelihood value is generally given in code output, as a complement to the AIC.
	(Likelihood function is not on our course)
	\item Given a data set, several competing models may be ranked according to their AIC, with
	the one having the lowest AIC being the best. (Although, a difference in AIC values of
	less than two is considered negligible).
	
\end{itemize}

\end{document}
