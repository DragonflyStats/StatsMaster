\documentclass[a4paper,12pt]{article}
%%%%%%%%%%%%%%%%%%%%%%%%%%%%%%%%%%%%%%%%%%%%%%%%%%%%%%%%%%%%%%%%%%%%%%%%%%%%%%%%%%%%%%%%%%%%%%%%%%%%%%%%%%%%%%%%%%%%%%%%%%%%%%%%%%%%%%%%%%%%%%%%%%%%%%%%%%%%%%%%%%%%%%%%%%%%%%%%%%%%%%%%%%%%%%%%%%%%%%%%%%%%%%%%%%%%%%%%%%%%%%%%%%%%%%%%%%%%%%%%%%%%%%%%%%%%
\usepackage{eurosym}
\usepackage{vmargin}
\usepackage{amsmath}
\usepackage{graphics}
\usepackage{epsfig}
\usepackage{subfigure}
\usepackage{fancyhdr}
%\usepackage{listings}
\usepackage{framed}
\usepackage{graphicx}

\setcounter{MaxMatrixCols}{10}
%TCIDATA{OutputFilter=LATEX.DLL}
%TCIDATA{Version=5.00.0.2570}
%TCIDATA{<META NAME="SaveForMode" CONTENT="1">}
%TCIDATA{LastRevised=Wednesday, February 23, 2011 13:24:34}
%TCIDATA{<META NAME="GraphicsSave" CONTENT="32">}
%TCIDATA{Language=American English}

\pagestyle{fancy}
\setmarginsrb{20mm}{0mm}{20mm}{25mm}{12mm}{11mm}{0mm}{11mm}
\lhead{Statistics with \texttt{R}} \rhead{Dublin \texttt{R}}
\chead{Regression Models}
\begin{document}
	
	\Large
	\section{Prediction Interval for Linear Regression}
	
	
	
	Assume that the error term $\epsilon$ in the simple linear regression model is independent of x, and is normally distributed, with zero mean and constant variance. For a given value of x, the interval estimate of the dependent variable y is called the prediction interval.
	
	\subsection{Problem}
	In the data set \textbf{\textit{mtcars}}, develop a 95\% prediction interval of the miles per gallon (\texttt{mpg})  for the weight (\texttt{wt}) of 4.4, 4.8 and 4.8 tonnes.\\
	\bigskip
	
	\noindent \textbf{Solution}\\
	We apply the \texttt{lm} function to a formula that describes the variable mpg by the variable wt, and save the linear regression model in a new variable \texttt{myModel}.
	
	\begin{framed}
		\begin{verbatim}
		myModel <- lm(mpgs ~ wt, data = mtcars)
		\end{verbatim}
	\end{framed}
	\noindent Then we create a new data frame called \texttt{myNewData}. This dataframe must have the same variable names of the explanatory variables used in the model fitting process. It is not necesary to have the response variable.
	\begin{framed}
		\begin{verbatim}
		myNewData <- data.frame(wt=c(4.4,4.6,4.8))
		\end{verbatim}
	\end{framed}
	\noindent We now apply the \texttt{predict} function and set the predictor variable in the \texttt{newdata} argument. We also set the interval type as "\texttt{predict}", and use the default 0.95 confidence level.
	
	\begin{framed}
		\begin{verbatim}
		> predict(myModel, newdata =myNewData, 
		    interval="predict") 
		  fit      lwr      upr
		1 13.76945 7.309732 20.22917
		2 12.70056 6.189263 19.21185
		3 11.63166 5.061258 18.20207
		> 
		
		\end{verbatim}
	\end{framed}
	\noindent \textbf{Answer}: The 95\% prediction interval of the mpg (miles per gallon) for the wt time of 4.4 tonnes is between 7.31 and 20.22 tonnes.\\ 
	\bigskip
	
	\noindent Further detail of the \texttt{predict} function for linear regression model can be found in the \texttt{R} documentation (i.e. \texttt{help(predict.lm)} )
\end{document}

