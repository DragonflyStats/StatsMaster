
\documentclass[a4]{beamer}
\usepackage{amssymb}
\usepackage{graphicx}
\usepackage{subfigure}
\usepackage{newlfont}
\usepackage{amsmath,amsthm,amsfonts}
%\usepackage{beamerthemesplit}
\usepackage{pgf,pgfarrows,pgfnodes,pgfautomata,pgfheaps,pgfshade}
\usepackage{mathptmx} % Font Family
\usepackage{helvet} % Font Family
\usepackage{color}
\mode<presentation> {
\usetheme{Default} % was Frankfurt
\useinnertheme{rounded}
\useoutertheme{infolines}
\usefonttheme{serif}
%\usecolortheme{wolverine}
% \usecolortheme{rose}
\usefonttheme{structurebold}
}
\setbeamercovered{dynamic}
\title[MA4704]{Technology Maths 4 \\ {\normalsize Lecture 11C}}
\author[Kevin O'Brien]{Kevin O'Brien \\ {\scriptsize kevin.obrien@ul.ie}}
\date{Spring 2013}
\institute[Maths \& Stats]{Dept. of Mathematics \& Statistics, \\ University \textit{of} Limerick}
\renewcommand{\arraystretch}{1.5}
%----------------------------------------------------------------------------------------------------------%
\begin{document}

\begin{frame}
\titlepage
\end{frame}

\begin{frame}
\frametitle{Regression}
The steps in determining the relationship between two
quantitative variables are
\begin{itemize} \item Draw a scatter plot.
\item If the scatter plot indicates an approximately linear
relationship, then measure the strength and direction of the
relationship by calculating the correlation coefficient r.
\item Calculate the equation of a line which best describes the
relationship between the variables.
\end{itemize}
This line is called the Regression line, or fitted line.

\end{frame}
%----------------------------------------------------------- %
\begin{frame}
\frametitle{Regression}
We consider the simplest type of regression line where there are only two
variables
\begin{itemize}
\item  one response variable (Y)
\item one predictor variable (X)
\end{itemize}
This is called Simple Linear Regression and the mathematical model is:
\[ Y = \beta_0 + \beta_1X + \epsilon \]

The true values of $\beta_0$ and  $\beta_1$ are almost always unknown, but are instead estimated from sample data.
\end{frame}
%---------------------------------------------------------------------%
\begin{frame}
\frametitle{Regression Equation}
The equation that best describes the relationship between X and Y is called
the regression equation. The regression equation used in simple linear
regression is as follows:
\[ y = \beta_0 + \beta_1x + \epsilon \]
\begin{itemize}
\item y = value of the response variable Y
\item x = value of the predictor variable X
\item $\beta_0$ = the intercept (where the line cuts the Y-axis)
\item $\beta_1$ = slope of the regression line
\item $\epsilon$ = error term (a.k.a. the residual)
\end{itemize}

\end{frame}

\begin{frame}
\frametitle{Regression Coefficients} Meaning of the Regression Coefficients:
\begin{itemize} \item $\beta_1$ is the slope of the regression line.
\item It gives the average change in the response variable Y for each unit
change in X.
\item The slope can either be negative or positive.
\item A positive slope of 5, for example, means that for every 1 unit increase
in X we can expect an average 5 units increase in Y.
\item $\beta_0$ indicates the value of Y when X is zero (only holds if the population could
have X values of 0 - otherwise $\beta_0$ does not have a meaningful interpretation
in the regression model).
\end{itemize}
\end{frame}
%------------------------------------------------------- %
\begin{frame}
\frametitle{Regression Least Square Estimation}
\begin{itemize}
\item There are many straight lines that could be drawn to represent the
relationship between X and Y.
\item The \textbf{least squares} method is one of the methods that can be used to find
the straight line that provides the best approximation for the relationship
between the independent and dependent variables.
\item This method consists of choosing estimate values of $b_0$ and $b_1$ which minimize the sum
of the squared vertical distances measured from the data to the line.
\end{itemize}
\end{frame}
%------------------------------------------------------- %
\begin{frame}
\frametitle{Regression Least Square Estimation}
The
equation of the line obtained using the least squares method is of the form:
\[ \hat{y} = b_0 + b_1x \]
where
\begin{itemize}
\item $b_0$ = estimated intercept of the line
\item $b_1$ = estimated slope of the line
\item $\hat{y}$ = estimated value of the response variable
\end{itemize}
\end{frame}
%--------------------------------------------------------------------- %
\begin{frame}
\frametitle{Regression : Least Square Estimation}
It can be shown, using differential calculus that the values of $b_0$
and $b_1$ that best estimate the true parameters $\beta_0$ and $\beta_1$ are:

\[ b_1 = \frac{S_{XY}}{S_{XX}}\]
and
\[ b_0 = \bar{y} - b_1 \bar{x} \]

$\bar{x}$ and $\bar{x}$ are the sample means of X and Y respectively.
\end{frame}


%---------------------------------------------------------------------%
\begin{frame}
\frametitle{Regression Line}
\begin{itemize}
\item A regression line is a line drawn through the points on a scatter-plot to summarize the relationship between the variables being studied.\item When it slopes down (from top left to bottom right), this indicates a negative or inverse relationship between the variables; when it slopes up (from bottom right to top left), a positive or direct relationship is indicated.

\item The regression line often represents the regression equation on a scatter-plot.
\end{itemize}
\end{frame}

%---------------------------------------------------------------------%
\begin{frame}
\frametitle{Example}

The US Olympic committee are interested in determining if there is a relationship between the experience of a coach and the number of medals won in the Olympics. A random selection of 8 coaches were analyzed and the results are given in the Table below.
\begin{itemize}
\item Predictor: Average Years Coaching Experience (x)	
\item Response : Average Number of Medals Won (y)
\end{itemize}
\begin{center}

\begin{tabular}{|c|c|c|c|c|c|c|c|c|}
  \hline
  % after \\: \hline or \cline{col1-col2} \cline{col3-col4} ...
  Coach & A & B & C & D & E & F & G & H \\
  Experience & 5 & 8 & 4 & 9 & 3 & 6 & 4 & 7\\
  Average Medals &4.0 & 6.9&3.9&7.8&2.7&6.1&4.4&6.2\\
  \hline
\end{tabular}
\end{center}
\end{frame}


\begin{frame}
\frametitle{Example : Summations}
\begin{center}
\begin{tabular}{|c|c|c|c|c|c|}
  \hline
  % after \\: \hline or \cline{col1-col2} \cline{col3-col4} ...
	Case &	X	&	Y	&	XY	&	$X^2$	&	$Y^2$	\\  \hline
A	&	5	&	4	&	20	&	25	&	16	\\
B	&	8	&	6.9	&	55.2	&	64	&	47.61	\\
C	&	4	&	3.9	&	15.6	&	16	&	15.21	\\
D	&	9	&	7.8	&	70.2	&	81	&	60.84	\\
E	&	3	&	2.7	&	8.1	&	9	&	7.29	\\
F	&	6	&	6.1	&	36.6	&	36	&	37.21	\\
G	&	4	&	4.4	&	17.6	&	16	&	19.36	\\
H	&	7	&	6.2	&	43.4	&	49	&	38.44	\\ \hline
(sum)	&	46	&	42	&	266.7	&	296	&	241.96	\\ \hline
\end{tabular}
\end{center}
\end{frame}

\begin{frame}
\frametitle{Example}
Compute the Regression Equation, given that
\begin{itemize}
\item $\sum xy = 266.7 $
\item $\sum x^2 = 296 $
\item $\sum x = 46$	
\item $\sum y =  42$	
\item $\sum y^2 = 241.96$
\end{itemize}
\end{frame}
\begin{frame}
\frametitle{Calculations}
\begin{itemize}
\item\item $S_{XY}$
\[ S_{XY} = \sum xy - \frac{(\sum x)\times (\sum y)}{n} = 266.7 - \frac{46 \times 42}{8} \]
\[ S_{XY} = 266.7-241.5  = 25.2 \]
\item $S_{XX}$
\[ S_{XX} = \sum x^2 - \frac{(\sum x)^2}{n} = 296 - \frac{46 ^2}{8}\]
\[ S_{XX} = 296 - 264.5 = 31.5 \]

\item $S_{YY}$
\[ S_{YY} = \sum y^2 - \frac{(\sum y)^2}{n} = 241.96- \frac{42^2}{8} \]
\[ S_{YY} = 241.96 - 220.5 =  21.46 \]
\end{itemize}
\end{frame}
%------------------------------------------%

\begin{frame}
\frametitle{Calculations}
\begin{itemize}
\item The Slope Estimate is therefore
\[ b_1 = \frac{S_{XY}}{S_XX} = \frac{25.2}{31.5} =  0.8 \]
\item The mean values of X and Y are computed as follows
\[ \bar{X}  = \frac{\sum x}{n} = \frac{46}{8} = 5.75 \]
\[ \bar{Y}  = \frac{\sum y}{n} = \frac{42}{8} = 5.25 \]
\item The Intercept Estimate 
\[ b_0 = \bar{y} - b_1 \bar{x} = 5.25 - (0.8\times 5.75) =0.65 \] 
\end{itemize}
\end{frame}
%------------------------------------------%

\begin{frame}
\frametitle{Calculations}
\begin{itemize}
\item $y$ is an observed value at some value of x,
\item $\hat{y}$ is a predicted value for some value of x.
\end{itemize}

\[ \hat{y} = b_0 + b_1x \]
\[ \hat{y} = 0.65 + 0.8x \]
\end{frame}

\begin{frame}
\frametitle{Calculations}
Suppose we wish to make a prediction as to how many medals, on average, a coach with five years experience would win. 
Here let $x=5$
\[ \hat{y} = 0.65 + 0.8(5) = 4.65\]
\end{frame}
\end{document}
%------------------------------------------%

%----------------------------------------------------------------------%
\end{document}

%---------------------------------------------------------------------%




\end{document} 