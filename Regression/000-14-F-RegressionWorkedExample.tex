
\documentclass[]{report}

\voffset=-1.5cm
\oddsidemargin=0.0cm
\textwidth = 480pt

\usepackage{framed}
\usepackage{subfiles}
\usepackage{graphics}
\usepackage{newlfont}
\usepackage{eurosym}
\usepackage{amsmath,amsthm,amsfonts}
\usepackage{amsmath}
\usepackage{color}
\usepackage{amssymb}
\usepackage{multicol}
\usepackage[dvipsnames]{xcolor}
\usepackage{graphicx}
\begin{document}
	\tableofcontents{3}

	
\section*{May 2013 Question 6b Correlation and Regression }
Calculate the correlation coefficient and interpret the value.
\begin{tabular}{|c|c|c|}
	\hline Residence	& X	  & Y \\ 
	\hline  &  &  \\ 
	\hline  &  &  \\ 
	\hline 
\end{tabular} 



\section*{May 2013 Question 5 Regression and Correlation}
\begin{center}
	\begin{tabular}{|c|c|c|}
		\hline  & Experience & No. of Rejects \\ 
		\hline A & 4 & 22 \\ 
		\hline B & 5 & 20 \\ 
		\hline C & 7 & 18 \\ 
		\hline D & 9 & 15 \\ 
		\hline E & 9 & 16 \\ 
		\hline F & 10 & 11 \\ 
		\hline G & 14 & 10 \\ 
		\hline 
	\end{tabular} 
\end{center}

% X = c(4, 5, 7, 9, 9, 10, 14)
% Y = c(22, 20, 18, 15, 16, 11, 10)
% plot(X,Y,pch=16,col="red",cex=2,xlim=c(0,16),ylim=c(0,25),font.lab=2) 
% abline(coef(lm(Y~X)),col="blue",lty=2)
\begin{itemize}
	\item $\sum x$ = 58
	\item $\sum y$ = 118
	\item $\sum x^2$ = 548
	\item $\sum y^2$ = 1910
	\item $\sum xy$ = 843
\end{itemize}

\begin{itemize}
	\item $s_{xx}$ = 67.428
	\item $s_{yy}$ = 118
	\item $s_{xy}$ = -85
\end{itemize}
\subsection*{The correlation coefficient}
The correlation coefficient is $r = -0.9529$.

\[ r = \frac{s_{xy}}{\sqrt{(s_xx \times s_yy)}}= \frac{-85}{\sqrt{(67.428 \times 118)}}\]

\[r= \frac{-85}{sqrt{(7956.504)}} =  {-85 \over 89.199}  \]

\[r = -0.9529\]

\subsection*{The coefficient of determination}
The coefficient of determination $r^2$ is computed as the square of the correlation coefficient.
\[r^2 = (-0.9529)^2 = 0.90801\]
%-------------------------------------------------%



%-------------------------------------------------%

\subsection{Other Correlation Coefficients}
Pearson's Correlation Coefficient is one approach to estimating the strength of relation between two variables.
Other approaches are as follows:
\begin{itemize}
	\item Spearman's Rank Correlation
	\item Kendall Tau Correlation
\end{itemize}
These are not part of the course.

%-------------------------------------------------%

\subsection{Example 1}
The height of a boy was observed at 7 different ages.
Comment on the relationship between height and age over this
period of time and calculate the Pearson correlation coefficient for
this data.

\begin{center}
	\begin{tabular}{|c|c|c|c|c|c|c|c|}
		Age  & 6 & 7  & 8 & 9 & 10 & 11 & 12 \\ 
		Height (cm)& 108 115& 120 &126& 132& 139 & 145\\
	\end{tabular} 
	
\end{center}

\begin{itemize}
	\item In order to investigate the nature of the relationship, we draw a
	scatter plot.
	\item X (the independent variable) is defined to be age and Y is defined
	to be height (the dependent variable).
\end{itemize}



\subsection{Identities}
\begin{itemize}
	\item $S_{XY} = -283.8$
	\item $S_{XX} = 613.6$
	\item $S_{YY} = 148.9$
	\item $\sum(X_i)  = 318 $
	\item $\sum(Y_i)  = 61$
\end{itemize}



\subsection{Example 2 Part 1}

\begin{itemize}
	\item Calculate the correlation coefficient and interpret its value.
	\item The correlation coefficient is computed using the following formula:
	\[ r_{X,Y} = \frac{\S_{XY}}{\sqrt{\S_{XX}\S_{YY}}} \]
	\item From the values given
	\[ r_{X,Y} = \frac{-283.8}{\sqrt{(613.6)(148.9)}} = -0.9389 \]
	\item Very strong negative linear relationship
\end{itemize}




\newpage
\begin{itemize}
	\item Correlation is a measure of the relation between two or more variables. 
	%\item The measurement scales used should be at least interval scales, but other correlation coefficients are available to handle other types of data.
	\item Correlation coefficients can range from -1.00 to +1.00. The value of -1.00 represents a perfect negative correlation while a value of +1.00 represents a perfect positive correlation. A value of 0.00 represents a lack of correlation.
	
	
	\item 
	The most widely-used type of correlation coefficient is Pearson r, also called linear or product- moment correlation.
	The pearson correlation coefficient is a metric.
	
	\item  Two variables that have no linear relationship have a correlation close to zero. 
	
	\item Scatter plots are a useful way of determing the likely relationship between two variables. 
	
	\item The Pearson correlation coefficient is most commonly used estimate for correlation. 
	
	\item Other types of correlation are tbe \textit{\textbf{Spearman Rho}} and the \textit{\textbf{Kendal Tau}} correlation coefficients. 
	
	%\item  These are not not part of this course,but it is important to know that they exist. 
\end{itemize}
%--------------------------------------------- %



\begin{itemize}
	\item Correlation is a measure of strength of \textbf{Linear Relationship} between two variables.
	\item The Pearson correlation coefficient (denoted $r$) is the most comonly used statistical estimate for correlation. 
	\item Correlation estimates are defined to be between -1 and 1. It is not possible to have a correlation value outside this range of values
	\[ -1 \leq r \leq 1\]
	\item 
	Additionally correlation estimates are not denominated in any units. (Contrast this to standard deviation, which is denominated in the same units as the mean).
\end{itemize}

%--------------------------------------------- %



\begin{itemize}
	\item A strong positive linear relationship describes a relationship between two variables whereby an increase in one variable will closely coincide with an increase in the other variable. 
	\item Conversely a strong negative linear relationship describes a relationship whereby an increase in one variable closely coincides with a decrease in the other. 
\end{itemize}
\begin{itemize}
	\item The Pearson correlation estimate, which us based on sample data, is denoted r (although related metrics use capital R).
	\item This measure is used as an estimate for the Population correlation, denoted by the greek letter $\rho$ ( pronounced ``Rho"). 
	The estimate is computed using summation identities.
	% (See the formulae ).
	
	%\item 
	%Equivalently it can be computed using the  Sums of Squares Identities that are used to compute covariance and standard deviation <INSERT FORMULA> .
	%Example
	%determine the correlation estimate for the Spend V Impressions data.  
\end{itemize}
%-------------------------------------------------------- %


\subsection{Outliers}
\begin{itemize}
	\item Outliers can greatly influence the computed value of an estimate.
	\item  Correlation is closely related to Simple linear regression models, in that both are concerned with the linear relationship between variables. However Linear Regression has a different emphasis.
	\item  Simple Linear Regression describes one independent variable (IV) and the response of the dependent variable (DV). 
\end{itemize}





\subsection{Correlation and Causality }
Implicit is simple linear regression is the notion of causality. The dependent variable changes as the independent variable changes. The converse is not true.
<some examples : hot temperature / ice cream example> .
Correlation is not concerned with causality at all, hence the often used expression "causation does not imply causality ".

%-----------------------------------------------------%
\section*{May 2013 Question 5 Regression and Correlation}
\begin{center}
	\begin{tabular}{|c|c|c|}
		\hline  & Experience & No. of Rejects \\ 
		\hline A & 4 & 22 \\ 
		\hline B & 5 & 20 \\ 
		\hline C & 7 & 18 \\ 
		\hline D & 9 & 15 \\ 
		\hline E & 9 & 16 \\ 
		\hline F & 10 & 11 \\ 
		\hline G & 14 & 10 \\ 
		\hline 
	\end{tabular} 
\end{center}

% X = c(4, 5, 7, 9, 9, 10, 14)
% Y = c(22, 20, 18, 15, 16, 11, 10)
% plot(X,Y,pch=16,col="red",cex=2,xlim=c(0,16),ylim=c(0,25),font.lab=2) 
% abline(coef(lm(Y~X)),col="blue",lty=2)
\begin{itemize}
	\item $\sum x$ = 58
	\item $\sum y$ = 118
	\item $\sum x^2$ = 548
	\item $\sum y^2$ = 1910
	\item $\sum xy$ = 843
\end{itemize}

\begin{itemize}
	\item $s_{xx}$ = 67.428
	\item $s_{yy}$ = 118
	\item $s_{xy}$ = -85
\end{itemize}
\subsection*{The correlation coefficient}
The correlation coefficient is $r = -0.9529$.

\[ r = \frac{s_{xy}}{\sqrt{(s_xx \times s_yy)}}= \frac{-85}{\sqrt{(67.428 \times 118)}}\]

\[r= \frac{-85}{sqrt{(7956.504)}} =  {-85 \over 89.199}  \]

\[r = -0.9529\]

\subsection*{The coefficient of determination}
The coefficient of determination $r^2$ is computed as the square of the correlation coefficient.
\[r^2 = (-0.9529)^2 = 0.90801\]
%-------------------------------------------------%


%-------------------------------------------------%
\newpage
\section*{May 2012 Question 2 Correlation and Regression}
\begin{itemize}
	\item The sample size $n$ = 10.
	\item The \textbf{\textit{independent}} variable, usually denoted $x$, is the "cause variable" or "predictor variable".
	\item The \textbf{\textit{dependent}} variable, usually denoted $y$, is the "effect variable".
	\item Here the Maths achievement test score is the independent variable and the final grade in statistics is the dependent variable.
	\item A big hint is given in the notation of the question.
\end{itemize}

% X = c(39,43,21,64,57,47,28,75,34,52)
% Y = c(65,78,52,82,92,89,73,98,56,75)

%---------------------------------------------------------- %
\subsection*{Sums of Squares Identities}
Before we do anything, We need to compute the following sums of squares identities
\begin{itemize}
	\item $s_{xx}$
	\item $s_{yy}$
	\item $s_{xy}$
\end{itemize}
\subsection*{Calculation 1}
\[ s_{xx}  = \sum(x^2) - \frac{\sum(x)^2}{n} \]
\[ s_{xx}  = 23.634 - \frac{\sum(x)^2}{10} \]

\subsection*{Calculation 2}
\[ s_{yy}  = \sum(y^2) - \frac{\sum(y)^2}{n} \]
%\[ s_{xx}  = 23.634 - \frac{\sum(x)^2}{10} \]


\subsection*{Calculation 3}
\[ s_{xy}  = \sum(xy) - \frac{\sum(x)\times \sum{y}}{n} \]
%\[ s_{xx}  = 23.634 - \frac{\sum(x)^2}{10} \]

%------------------------------------------------------------%

\subsection*{Part iv - Prediction}
\begin{itemize}
	\item Suppose the regression equation is as follows
	\[ \hat{y} = 40.78424 + 0.76556 x \]
	\item If a student scored 5 marks on the achievement test (i.e. $x=5$), predict the students statistics grade.
	
	\[ \hat{y}_{(x=5)} = 40.78424 + (0.76556 \times 5) \]
	
	\item Solving using a calculator we get
	\[ \hat{y}_{(x=5)} = 44.61204 \]
	
	\item The score should be approximately 44.61.
\end{itemize}

\begin{itemize}
	\item 
	The first part of the question will require the drawing of a scatter plot. 
	When doing do, remember to label the axes, and to put in the relevant units. (I.e. Metres, Degrees, Hours etc)
	\item 
	The Explanatory variable is on the X-axis and the Response variable is on the Y Axis.
	\item 
	A Trend line will be useful in demonstrating what type of relationship exists between the response variable and the explanatory variable.
	\item There are five possible plot types
	
	
	
	\begin{enumerate}
		
		\item Strong positive linear relationship 
		\item Weak positive linear relationship
		\item Strong negative linear relationship
		\item Weak negative linear relationship
		\item No Relationship
	\end{enumerate}
	\item  In the strong case – the points of the graph correspond to the trend line quite closely, whereas in the weak case they don’t.
	\item In the positive case the response values Y increase as the explanatory values X increases. In the negative case the response values Y decrease as the explanatory values X increases. 
\end{itemize}

%----------------------------------------------------------- %



\begin{itemize}
	\item The coefficient of determination can take any value between 0 and 1.
	\item The closer the value is to 1, the better the explanatory power of the independent variable.
\end{itemize}

\begin{tabular}{|c|c|c|}
	\hline Variables & $(X_1,Y_1)$ & $(X_2,Y_2)$ \\ 
	\hline Correlation ($r$) & 0.9 & 0.7 \\ 
	\hline r-square ($r^2$) &  &  \\ 
	\hline 
\end{tabular} 

%-------------------------------------------------%
\newpage

\section*{May 2012 Question 2 Correlation and Regression}
\begin{itemize}
	\item The sample size $n$ = 10.
	\item The \textbf{\textit{independent}} variable, usually denoted $x$, is the "cause variable" or "predictor variable".
	\item The \textbf{\textit{dependent}} variable, usually denoted $y$, is the "effect variable".
	\item Here the Maths achievement test score is the independent variable and the final grade in statistics is the dependent variable.
	\item A big hint is given in the notation of the question.
\end{itemize}

% X = c(39,43,21,64,57,47,28,75,34,52)
% Y = c(65,78,52,82,92,89,73,98,56,75)

%---------------------------------------------------------- %
\subsection*{Sums of Squares Identities}
Before we do anything, We need to compute the following sums of squares identities
\begin{itemize}
	\item $s_{xx}$
	\item $s_{yy}$
	\item $s_{xy}$
\end{itemize}
\subsection*{Calculation 1}
\[ s_{xx}  = \sum(x^2) - \frac{\sum(x)^2}{n} \]
\[ s_{xx}  = 23.634 - \frac{\sum(x)^2}{10} \]

\subsection*{Calculation 2}
\[ s_{yy}  = \sum(y^2) - \frac{\sum(y)^2}{n} \]
%\[ s_{xx}  = 23.634 - \frac{\sum(x)^2}{10} \]


\subsection*{Calculation 3}
\[ s_{xy}  = \sum(xy) - \frac{\sum(x)\times \sum{y}}{n} \]
%\[ s_{xx}  = 23.634 - \frac{\sum(x)^2}{10} \]

%------------------------------------------------------------%

\subsection*{Part iv - Prediction}
\begin{itemize}
	\item Suppose the regression equation is as follows
	\[ \hat{y} = 40.78424 + 0.76556 x \]
	\item If a student scored 5 marks on the achievement test (i.e. $x=5$), predict the students statistics grade.
	
	\[ \hat{y}_{(x=5)} = 40.78424 + (0.76556 \times 5) \]
	
	\item Solving using a calculator we get
	\[ \hat{y}_{(x=5)} = 44.61204 \]
	
	\item The score should be approximately 44.61.
\end{itemize}

%-------------------------------------------------%

\section{May 2012 Question 2 Correlation and Regression}
\begin{itemize}
	\item The sample size $n$ = 10.
	\item The \textbf{\textit{independent}} variable, usually denoted $x$, is the "cause variable" or "predictor variable".
	\item The \textbf{\textit{dependent}} variable, usually denoted $y$, is the "effect variable".
	\item Here the Maths achievement test score is the independent variable and the final grade in statistics is the dependent variable.
	\item A big hint is given in the notation of the question.
\end{itemize}

% X = c(39,43,21,64,57,47,28,75,34,52)
% Y = c(65,78,52,82,92,89,73,98,56,75)

%---------------------------------------------------------- %
\subsection*{Sums of Squares Identities}
Before we do anything, We need to compute the following sums of squares identities
\begin{itemize}
	\item $s_{xx}$
	\item $s_{yy}$
	\item $s_{xy}$
\end{itemize}
\subsection*{Calculation 1}
\[ s_{xx}  = \sum(x^2) - \frac{\sum(x)^2}{n} \]
\[ s_{xx}  = 23.634 - \frac{\sum(x)^2}{10} \]

\subsection*{Calculation 2}
\[ s_{yy}  = \sum(y^2) - \frac{\sum(y)^2}{n} \]
%\[ s_{xx}  = 23.634 - \frac{\sum(x)^2}{10} \]


\subsection*{Calculation 3}
\[ s_{xy}  = \sum(xy) - \frac{\sum(x)\times \sum{y}}{n} \]
%\[ s_{xx}  = 23.634 - \frac{\sum(x)^2}{10} \]

%------------------------------------------------------------%

\subsection*{Part iv - Prediction}
\begin{itemize}
	\item Suppose the regression equation is as follows
	\[ \hat{y} = 40.78424 + 0.76556 x \]
	\item If a student scored 5 marks on the achievement test (i.e. $x=5$), predict the students statistics grade.
	
	\[ \hat{y}_{(x=5)} = 40.78424 + (0.76556 \times 5) \]
	
	\item Solving using a calculator we get
	\[ \hat{y}_{(x=5)} = 44.61204 \]
	
	\item The score should be approximately 44.61.
\end{itemize}

\begin{itemize}
	\item 
	The first part of the question will require the drawing of a scatter plot. 
	When doing do, remember to label the axes, and to put in the relevant units. (I.e. Metres, Degrees, Hours etc)
	\item 
	The Explanatory variable is on the X-axis and the Response variable is on the Y Axis.
	\item 
	A Trend line will be useful in demonstrating what type of relationship exists between the response variable and the explanatory variable.
	\item There are five possible plot types
	
	
	
	\begin{enumerate}
		
		\item Strong positive linear relationship 
		\item Weak positive linear relationship
		\item Strong negative linear relationship
		\item Weak negative linear relationship
		\item No Relationship
	\end{enumerate}
	\item  In the strong case – the points of the graph correspond to the trend line quite closely, whereas in the weak case they don’t.
	\item In the positive case the response values Y increase as the explanatory values X increases. In the negative case the response values Y decrease as the explanatory values X increases. 
\end{itemize}

%----------------------------------------------------------- %



\begin{itemize}
	\item The coefficient of determination can take any value between 0 and 1.
	\item The closer the value is to 1, the better the explanatory power of the independent variable.
\end{itemize}

\begin{tabular}{|c|c|c|}
	\hline Variables & $(X_1,Y_1)$ & $(X_2,Y_2)$ \\ 
	\hline Correlation ($r$) & 0.9 & 0.7 \\ 
	\hline r-square ($r^2$) &  &  \\ 
	\hline 
\end{tabular} 

%-------------------------------------------------%
\newpage

\section*{May 2012 Question 2 Correlation and Regression}
\begin{itemize}
	\item The sample size $n$ = 10.
	\item The \textbf{\textit{independent}} variable, usually denoted $x$, is the "cause variable" or "predictor variable".
	\item The \textbf{\textit{dependent}} variable, usually denoted $y$, is the "effect variable".
	\item Here the Maths achievement test score is the independent variable and the final grade in statistics is the dependent variable.
	\item A big hint is given in the notation of the question.
\end{itemize}

% X = c(39,43,21,64,57,47,28,75,34,52)
% Y = c(65,78,52,82,92,89,73,98,56,75)

%---------------------------------------------------------- %
\subsection*{Sums of Squares Identities}
Before we do anything, We need to compute the following sums of squares identities
\begin{itemize}
	\item $s_{xx}$
	\item $s_{yy}$
	\item $s_{xy}$
\end{itemize}
\subsection*{Calculation 1}
\[ s_{xx}  = \sum(x^2) - \frac{\sum(x)^2}{n} \]
\[ s_{xx}  = 23.634 - \frac{\sum(x)^2}{10} \]

\subsection*{Calculation 2}
\[ s_{yy}  = \sum(y^2) - \frac{\sum(y)^2}{n} \]
%\[ s_{xx}  = 23.634 - \frac{\sum(x)^2}{10} \]


\subsection*{Calculation 3}
\[ s_{xy}  = \sum(xy) - \frac{\sum(x)\times \sum{y}}{n} \]
%\[ s_{xx}  = 23.634 - \frac{\sum(x)^2}{10} \]

%------------------------------------------------------------%

\subsection*{Part iv - Prediction}
\begin{itemize}
	\item Suppose the regression equation is as follows
	\[ \hat{y} = 40.78424 + 0.76556 x \]
	\item If a student scored 5 marks on the achievement test (i.e. $x=5$), predict the students statistics grade.
	
	\[ \hat{y}_{(x=5)} = 40.78424 + (0.76556 \times 5) \]
	
	\item Solving using a calculator we get
	\[ \hat{y}_{(x=5)} = 44.61204 \]
	
	\item The score should be approximately 44.61.
\end{itemize}




%------------------------------------------------------------------------------------------------%

\section*{May 2013 Question 6b Correlation and Regression }
Calculate the correlation coefficient and interpret the value.
\begin{tabular}{|c|c|c|}
	\hline Residence	& X	  & Y \\ 
	\hline  &  &  \\ 
	\hline  &  &  \\ 
	\hline 
\end{tabular} 

\end{document}