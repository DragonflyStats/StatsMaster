% - http://en.wikiversity.org/wiki/Eta-squaredEta-squared
% - http://coeweb.gsu.edu/coshima/EPRS8540/uploadf07/Questions%20on%20ES.pdf
Bellcurve.svg	Subject classification: this is a statistics resource .
Sciences humaines.svg	Educational level: this is a tertiary (university) resource.
Eta-squared (\eta^2) is a measure of effect size for use in ANOVA.

\eta^2 is analagous to R2 from multiple linear regression.

\eta^2

= SSbetween / SStotal = SSB / SST
= proportion of variance in Y explained by X
= Non-linear correlation coefficient
\eta^2 ranges between 0 and 1.

Interpret \eta^2 as for r2 or R2; a rule of thumb (Cohen):

.02 ~ small
.13 ~ medium
.26 ~ large
In SAS, eta-squared statistics can be found in semi-partial eta-squared statistics in SAS 9.2.

The eta-squared column in SPSS F-table output is actually partial eta-squared (\eta^2_p) in versions of SPSS prior to V 11.0. [1]

\eta^2 was not previously provided by SPSS, however, it is available in V 18.0. It can also be calculated manually: \eta^2 = Between-Groups Sum of Squares / Total Sum of Squares.

R2 is provided at the bottom of SPSS F-tables is the linear effect as per MLR – however, if an IV has 3 or more non-interval levels, this won’t equate with \eta^2.
