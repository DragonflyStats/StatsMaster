
Akaike's information criterion 

Akaike's information criterion is a measure of the goodness of fit of an estimated statistical model. The AIC was developed by Hirotsugu Akaike under the name of "an information criterion" in 1971.

The AIC is a "model selection" tool i.e. a method of comparing two or more candidate models.

The AIC is calculated using the "likelihood function" and the number of parameters. (Not on course). The likelihood value is generally given in code output, as a complement to the AIC.


The AIC methodology attempts to find the model that best explains the data with a minimum of parameters. (i.e. in keeping with the Law of parsimony)

Given a data set, several competing models may be ranked according to their AIC, with the one having the lowest AIC being the best.

(A difference in AIC values of less than two is considered negligible)
