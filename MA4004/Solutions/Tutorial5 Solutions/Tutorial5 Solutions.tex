\documentclass[12pt]{article}
\usepackage{amsmath}
%\usepackage[paperwidth=21cm, paperheight=29.8cm]{geometry}
\usepackage[angle=0,scale=1,color=black,hshift=-0.3cm,vshift=15cm]{background}
\usepackage{multirow}
\usepackage{enumerate}
\usepackage[gen]{eurosym}

%\SetBgScale{1}
%\SetBgAngle{0}
%\SetBgColor{black}
%\SetBgContents{\rule{1pt}{30cm}}
%\SetBgHshift{-8.4cm}
%
%\backgroundsetup{contents={
%\begin{tabular}{c|c}
%\hspace{2cm} & \\[0.7cm]
%& {\bf Statistics for Computing ------ Lecture 1 ------ Solutions} \\[0.3cm]
%%\hline
%\hspace{2cm} & \hspace{18.5cm} \\ [28cm]
%\end{tabular}}}

\backgroundsetup{contents={
{\bf \centering Statistics for Computing ------------------------ Tutorial 5 ------------------------------------------ Solutions} }}


\setlength{\voffset}{-3cm}
\setlength{\hoffset}{-3.45cm}
\setlength{\parindent}{0cm}
\setlength{\textheight}{27cm}
\setlength{\textwidth}{19.7cm}

\pagestyle{empty}



\begin{document}


\framebox[1.02\textwidth]{
\begin{minipage}[t]{0.98\textwidth}
\begin{minipage}[t]{0.47\textwidth}
\subsection*{Question 1}
\begin{enumerate}[a)]
\item ``The number of heads in 35 flips'' is\\ $X \sim \text{Binomial}(n=4,p=0.5)$ where \\
    $x \in \{0,1,2,3,4\}$.
\begin{align*}
\bullet\,\Pr(X = x) &= \binom{n}{x} \, p^x \, (1-p)^{n-x}\\ &= \binom{4}{x} \, 0.5^x \, 0.5^{4-x} \\[0.2cm]
\bullet\,E(X) &= n\,p = 4(0.5) = 2 \text{ heads}\\[0.2cm]
\bullet\,Sd(X) &= \sqrt{n\,p\,(1-p)} \\&= \sqrt{4(0.5)(0.5)} = 1 \text{ head}
\end{align*}
\item ``The number of individuals with the disease per square mile'' is $X \sim \text{Poisson}(\lambda=3)$ where \\
    $x \in \{0,1,2,\ldots,\infty\}$.
\begin{align*}
\bullet\,\Pr(X = x) &= \frac{\lambda^x}{x\,!}e^{-\lambda} = \frac{3^x}{x\,!}e^{-3} \\[0.2cm]
\bullet\,E(X) &= \lambda = 3 \text{ individuals}\\[0.2cm]
\bullet\,Sd(X) &= \sqrt{\lambda} = \sqrt{3} = 1.73 \text{ individuals}
\end{align*}
\item ``The number of defective bulbs in a group of 100'' is $X \sim \text{Binomial}(n=100,p=0.03)$ where \\
    $x \in \{0,1,2,\ldots,100\}$.
\begin{align*}
\bullet\,\Pr(X = x) &= \binom{100}{x} \, 0.03^x \, 0.97^{100-x} \\[0.2cm]
\bullet\,E(X) &= 100(0.03) = 3 \text{ bulbs}\\[0.2cm]
\bullet\,Sd(X) &= \sqrt{100(0.03)(0.97)} = 1.71 \text{ bulbs}
\end{align*}
\item We are given $E(T) = \frac{1}{\lambda} = 15$ minutes $\Rightarrow$ $\lambda = \frac{1}{E(T)} = \frac{1}{15}$ customers per minute. We may prefer to work in hours $\Rightarrow$ $\lambda = \frac{1}{15}\times60 = 4$ customers per hour.\\[0.3cm]
    ``The time (in hours) between customers'' is \\$T \sim \text{Exponential}(\lambda=4)$ where $t \in [0,\infty)$.
\begin{align*}
\bullet\,\Pr(T > t) &= e^{-\lambda \,t} = e^{-4 \,t}\\
\bullet\,E(T) &= \frac{1}{\lambda} = \frac{1}{4} =0.25 \text{ hours}\\
\text{(i.e., 15}& \text{\,minutes)} \\[0.2cm]
\bullet\,Sd(X) &= \sqrt{\frac{1}{\lambda^2}} = \frac{1}{\lambda} = \frac{1}{4} = 0.25 \text{ hours}
\end{align*}
\end{enumerate}
\end{minipage}\hspace{0.04\textwidth}
\begin{minipage}[t]{0.47\textwidth}
\begin{enumerate}
\item[e)] ``The number of individuals with the disease in a group of 35'' is\\
 $X \sim \text{Binomial}(n=35,p=0.05)$ where \\
    $x \in \{0,1,2,\ldots,35\}$.
\begin{align*}
\bullet\,\Pr(X = x) &= \binom{35}{x} \, 0.05^x \, 0.95^{35-x} \\[0.2cm]
\bullet\,E(X) &= 35(0.05) = 1.75 \text{ individuals}\\[0.2cm]
\bullet\,Sd(X) &= \sqrt{35(0.05)(0.95)} = 1.29\\&\hspace{4cm}\text{ individuals}
\end{align*}
\item[f)] We are given the average time between cars passing. This is the exponential average, i.e., $E(T) = \frac{1}{\lambda} = 2$ minutes $\Rightarrow$ $\lambda = \frac{1}{E(T)} = \frac{1}{2} = 0.5$ cars per minute.\\[0.2cm]
    Thus, for one hour $\lambda = 0.5(60) = 30$ cars per hour.\\[0.2cm]
 ``The number of cars per hour'' is\\ $X \sim \text{Poisson}(\lambda=30)$ where\\
    $x \in \{0,1,2,\ldots,\infty\}$.
\begin{align*}
\bullet\,\Pr(X = x) &= \frac{30^x}{x\,!}e^{-30} \\[0.2cm]
\bullet\,E(X) &= 30 \text{ cars}\\[0.2cm]
\bullet\,Sd(X) &= \sqrt{30} = 5.48 \text{ cars}
\end{align*}
\item[g)] Flaws occur at a rate of $\lambda = 0.1$ per square metre $\Rightarrow$ $\lambda = 0.1(20) = 2$ flaws per 20 square metres.\\[0.2cm]
 ``The number of flaws per 20 square metres'' is $X \sim \text{Poisson}(\lambda=2)$ where \\
    $x \in \{0,1,2,\ldots,\infty\}$.
\begin{align*}
\bullet\,\Pr(X = x) &= \frac{2^x}{x\,!}e^{-2} \\[0.2cm]
\bullet\,E(X) &= 2 \text{ flaws}\\[0.2cm]
\bullet\,Sd(X) &= \sqrt{2} = 1.41 \text{ flaws}
\end{align*}
\item[h)] ``The time (in hours) between texts'' is \\$T \sim \text{Exponential}(\lambda=4)$ where $t \in [0,\infty)$.
\begin{align*}
\bullet\,\Pr(T > t) &= e^{-4 \,t}\\[0.2cm]
\bullet\,E(T) &= \frac{1}{4} = 0.25 \text{ hours}\\
\bullet\,Sd(X) &= \frac{1}{4} = 0.25 \text{ hours}
\end{align*}
\end{enumerate}
\end{minipage}
\end{minipage}}\vspace{0.03\textwidth}



\framebox[1.02\textwidth]{
\begin{minipage}[t]{0.98\textwidth}
\begin{minipage}[t]{0.47\textwidth}
\subsection*{Question 1 continued}
\begin{enumerate}[a)]
\item[i)] ``The number of correctly guessed answers in 15 questions'' is $X \sim \text{Binomial}(n=15,p=0.25)$ where
    $x \in \{0,1,2,\ldots,15\}$.
\begin{align*}
\bullet\,\Pr(X = x) &= \binom{15}{x} \, 0.25^x \, 0.75^{15-x} \\[0.2cm]
\bullet\,E(X) &= 15(0.25) = 3.75 \text{ answers}\\[0.2cm]
\bullet\,Sd(X) &= \sqrt{15(0.25)(0.75)} = 1.68\\&\hspace{4cm}\text{ answers}
\end{align*}
\item[j)] ``The time (in years) between failures'' is \\$T \sim \text{Exponential}(\lambda=6)$ where $t \in [0,\infty)$.
\begin{align*}
\bullet\,\Pr(T > t) &= e^{-6 \,t}\\[0.2cm]
\bullet\,E(T) &= \frac{1}{6} \text{ years}\\
\bullet\,Sd(X) &= \frac{1}{6} \text{ years}
\end{align*}
\end{enumerate}
\end{minipage}\hspace{0.04\textwidth}
\begin{minipage}[t]{0.47\textwidth}
\begin{enumerate}
\item[k)] Failures occur at a rate of $\lambda = 6$ per year $\Rightarrow$ $\lambda = \frac{6}{12} = 0.5$ failures per month.\\[0.2cm]
 ``The number of failures per month'' is\\ $X \sim \text{Poisson}(\lambda=0.5)$ where \\
    $x \in \{0,1,2,\ldots,\infty\}$.
\begin{align*}
\bullet\,\Pr(X = x) &= \frac{0.5^x}{x\,!}e^{-0.5} \\[0.2cm]
\bullet\,E(X) &= 0.5 \text{ failures}\\[0.2cm]
\bullet\,Sd(X) &= \sqrt{0.5} = 0.71 \text{ failures}
\end{align*}
\end{enumerate}
\end{minipage}
\end{minipage}}\vspace{0.03\textwidth}




\framebox[1.02\textwidth]{
\begin{minipage}[t]{0.98\textwidth}
\begin{minipage}[t]{0.47\textwidth}
\subsection*{Question 2}
\begin{enumerate}[a)]
\item $\lambda = \frac{1}{300}$ per metre $\Rightarrow$  $\lambda = \frac{1000}{300} = \frac{10}{3}$ per 1km.
    \begin{align*}
    \Pr(X=0) = \frac{(\frac{10}{3})^0}{0\,!} e^{-\frac{10}{3}} = 0.0357.
    \end{align*}
\item $\lambda = \frac{6000}{300} = 20$ per 6km.
    \begin{align*}
    \Pr(X\ge15) = 0.8951 \text{ (using tables)}.
    \end{align*}
\item $\lambda = \frac{3000}{300} = 10$ per 3km.
    \begin{align*}
    \Pr(10\ge&\, X \ge12) = p(10)+p(11)+p(12)\\
    &= \frac{10^{10}}{10\,!} e^{-10} + \frac{10^{11}}{11\,!} e^{-10} + \frac{12^{10}}{12\,!} e^{-10} \\
    &= 0.1251 + 0.1137 + 0.0948\\
    &= 0.3336.
    \end{align*}
This can be done using tables also:
\begin{align*}
\Pr(10\ge X \ge12) &= \Pr(X \ge 10) - \Pr(X \ge 13)\\
&= 0.5421 - 0.2084 = 0.3337.
\end{align*}
     \end{enumerate}
\end{minipage}\hspace{0.04\textwidth}
\begin{minipage}[t]{0.47\textwidth}
\begin{enumerate}
\item[d)] We are given the probability and have to work out the $x$ value, i.e., using the tables in reverse. We find that:
    \begin{align*}
    \bullet\,\Pr(X \ge 15) &= 0.0835
    \end{align*}
    is the closest probability to $0.1$ $\Rightarrow$ $x = 15$, i.e., there is an 8.35\% chance of seeing 15 or more potholes on a 3km stretch.
\item[e)] $T$ represents the distance between potholes. \\[-0.45cm]
\begin{align*}
\Pr(T < 100) &= 1 - \Pr(T > 100) \\
&= 1 - e^{-\frac{1}{300}(100)} \\
&= 1 - 0.7165 = 0.2835.
\end{align*}
\item[f)] \quad \\[-1.45cm]
\begin{align*}
\Pr(T > 1000) &= e^{-\frac{1}{300}(1000)} \\
&= 0.0357.
\end{align*}
\item[g)] \quad \\[-1.45cm]
\begin{align*}
\Pr(300 < &\,T < 1200)\\ &= \Pr(T >300) - \Pr(T > 1200)\\
&=  e^{-\frac{1}{300}(300)} - e^{-\frac{1}{300}(1200)} \\
&= 0.3679 - 0.0183 = 0.3496.
\end{align*}
\item[h)] \quad \\[-1.45cm]
\begin{align*}
E(T) &= \frac{1}{\lambda} = \frac{1}{\frac{1}{300}} = 300 \text{ \,metres}\\
Sd(T) &= \sqrt{\frac{1}{\lambda^2}} = E(T) = 300 \text{ \,metres}
\end{align*}
\end{enumerate}
\end{minipage}
\end{minipage}}\vspace{0.03\textwidth}




\framebox[1.02\textwidth]{
\begin{minipage}[t]{0.98\textwidth}
\begin{minipage}[t]{0.47\textwidth}
\subsection*{Question 3}
\begin{enumerate}[a)]
\item $E(T) = 2$ years $\Rightarrow$ $\lambda = \frac{1}{2} = 0.5$ failures\,/\,year.
\item $Sd(T) = \sqrt{\frac{1}{\lambda^2}} = E(T) = 2$ years.
\item \quad \\[-1.45cm]
\begin{align*}
\Pr(T > 1) &= e^{-0.5(1)} = 0.6065.
\end{align*}
\item \quad \\[-1.45cm]
\begin{align*}
\Pr(T < 5) &= 1 - \Pr(T > 5) \\
&= 1- e^{-0.5(5)}\\
&= 1- 0.0821 \\
&= 0.9179.
\end{align*}
\end{enumerate}
\end{minipage}\hspace{0.04\textwidth}
\begin{minipage}[t]{0.47\textwidth}
\begin{enumerate}
\item[e)] \quad \\[-1.45cm]
\begin{align*}
\Pr(2 < T < 5) &= \Pr(T >2) - \Pr(T>5)\\
&= e^{-0.5(2)} - e^{-0.5(5)} \\
&= 0.3679 - 0.0821\\
&= 0.2858.
\end{align*}
\item[f)] \quad \\[-1.45cm]
\begin{align*}
\Pr(T >t) &= 0.2\\
e^{-0.5\,t} &= 0.2\\
\ln e^{-0.5\,t} &= \ln 0.2\\
-0.5\,t &= \ln 0.2\\
t &= \frac{1}{-0.5}\,\ln 0.2\\
&= 3.22 \text{ years},
\end{align*}
i.e., 20 \% of hard disks last longer than 3.22 years or, similarly, 80\% fail before this time.
\end{enumerate}
\end{minipage}
\end{minipage}}\vspace{0.03\textwidth}




\framebox[1.02\textwidth]{
\begin{minipage}[t]{0.98\textwidth}
\begin{minipage}[t]{0.47\textwidth}
\subsection*{Question 4}
\begin{enumerate}[a)]
\item \quad \\[-1.45cm]
\begin{align*}
\Pr(H) = \Pr(T > 1) = e^{-0.5(1)} = 0.6065.\\[0.2cm]
\Pr(H^c) = \Pr(T < 1) = 1 - 0.6065 = 0.3935.
\end{align*}
{\bf Note:} As $H_1$ and $H_2$ are \emph{independent} we can calculate the joint probabilities via\\[0.1cm]
$\Pr(H_1 \cap H_2) = \Pr(H_1) \Pr(H_2)$ and\\[0.1cm]
$\Pr(H_1^c \cap H_2^c) = \Pr(H_1^c) \Pr(H_2^c)$.
\item \quad \\[-1.45cm]
\begin{align*}
\Pr(\text{R-0 fails within 1yr}) &= \Pr(\text{at least one fails}) \\
&= \Pr(H_1^c \cup H_2^c) \\
&= 1 - \Pr(H_1 \cap H_2) \\
&= 1 - \Pr(H_1) \Pr(H_2) \\
&= 1 - (0.6065)^2 \\
&= 0.6322.
\end{align*}
\item \quad \\[-1.45cm]
\begin{align*}
\Pr(\text{R-1 fails within 1yr}) &= \Pr(\text{both fail}) \\
&= \Pr(H_1^c \cap H_2^c) \\
&= \Pr(H_1^c) \Pr(H_2^c) \\
&= (0.3935)^2 \\
&= 0.1548.
\end{align*}
\item We want $\Pr(\text{R-1 fails within 1yr})=0.05$.\\Note that:
\begin{align*}
\Pr(\text{R-1 fails within 1yr}) &= \Pr(H_1^c) \Pr(H_2^c) \\
&= (1 - e^{-\lambda\,(1)})^2 \\
&= (1 - e^{-\lambda})^2
\end{align*}
Thus, we set the above equal to 0.05 and solve for $\lambda$ (from which we can calculate $E(T)$).
\end{enumerate}
\end{minipage}\hspace{0.04\textwidth}
\begin{minipage}[t]{0.47\textwidth}
\begin{enumerate}
\item[] \quad\\[-1.3cm]
\begin{align*}
\Rightarrow (1 - e^{-\lambda})^2 &= 0.05 \\
1- e^{-\lambda} &= \sqrt{0.05} \\
 - e^{-\lambda} &= -1+\sqrt{0.05} \\
  e^{-\lambda} &= 1-\sqrt{0.05} \\
  \ln e^{-\lambda} &= \ln ( 1-\sqrt{0.05} ) \\
-\lambda &= \ln ( 1-\sqrt{0.05} ) \\
\lambda &= -\ln ( 1-\sqrt{0.05} ) \\
&= 0.253.
\end{align*}
$\Rightarrow$ $E(T) = \frac{1}{0.253} = 3.95$ years, i.e., if the two hard disks have an average life of 3.95 years then the RAID-1 system has a 5\% chance of failing within 1 year.
\item[e)] Another option is to use $k$ of the original hard disks where $\Pr(H^c) = 0.3935$:
    \begin{align*}
\Pr(\text{R-1 fails}&\text{\, within 1yr}) \\
&= \Pr(H_1^c)\Pr(H_2^c)\cdots\Pr(H_k^c)\\
&=(0.3935)^k.\\[-1cm]
\end{align*}
\begin{align*}
\Rightarrow (0.3935)^k &= 0.05 \\
\ln (0.3935)^k &= \ln 0.05 \\
k \, \ln (0.3935) &= \ln 0.05 \\
k  &= \frac{\ln 0.05}{\ln (0.3935)} \\
&= 3.21 \text{ hard disks}.
\end{align*}
$\Rightarrow$ In practice we can use 3 or 4 hard disks: 3 gives a probability above 0.05 and 4 gives a probability below 0.05.
\end{enumerate}
\end{minipage}
\end{minipage}}\vspace{0.03\textwidth}


\framebox[1.02\textwidth]{
\begin{minipage}[t]{0.98\textwidth}
\subsection*{Question 5}
The solution to this question is in Lecture9 solutions (i.e., Q1 of Lecture9).
\end{minipage}}\vspace{0.03\textwidth}


\framebox[1.02\textwidth]{
\begin{minipage}[t]{0.98\textwidth}
\begin{minipage}[t]{0.47\textwidth}
\subsection*{Question 6}
\begin{enumerate}[a)]
\item $E(T) = 5$ minutes and $\lambda_a = 2$ per minute.
\begin{align*}
\Pr(N) &= \lambda_a \, E(T) \\
&= 2(5)\\
&= 10 \text{ cars on the road.}
\end{align*}
\item $\lambda_a = 4$ per hour and $E(T) = 30$ minutes, i.e., $ E(T) = 0.5$ hours.
\begin{align*}
\Pr(N) &= \lambda_a \, E(T) \\
&= 4(0.5)\\
&= 2 \text{ jobs in the system.}
\end{align*}
\item $\lambda_a = 20$ per hour and $E(N) = 10$ people.
\begin{align*}
\Pr(N) &= \lambda_a \, E(T) \\
10 &= 20\,E(T) \\
\frac{10}{20} &= E(T) \\[0.2cm]
\Rightarrow E(T) &= 0.5 \text{ hours} \\
&= 30 \text{ minutes}.
\end{align*}
\end{enumerate}
\end{minipage}\hspace{0.04\textwidth}
\begin{minipage}[t]{0.47\textwidth}
\subsection*{Question 7}
\begin{enumerate}[a)]
\item $\lambda_a = 40$ per hour and $E(T) = 0.5$ hours.
\begin{align*}
\Pr(N) &= \lambda_a \, E(T) \\
&= 40(0.5)\\
&= 20 \text{ people.}
\end{align*}
\item $\lambda_a = 60$ per hour.
\begin{align*}
\Pr(N) &= \lambda_a \, E(T) \\
&= 60(0.5)\\
&= 30 \text{ people.}
\end{align*}
\item We want $E(N) = 20$ people while still maintaining $\lambda_a = 60$ per hour.
\begin{align*}
 \Pr(N) &= \lambda_a \, E(T) \\
20 &= 60\,E(T) \\
\frac{20}{60} &= E(T) \\[0.2cm]
\Rightarrow E(T) &= \frac{1}{3} \text{ hours} \\
&= 20 \text{ minutes}.
\end{align*}
\end{enumerate}
\end{minipage}
\end{minipage}}\vspace{0.03\textwidth}












\end{document} 