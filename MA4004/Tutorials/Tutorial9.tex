\documentclass[12pt]{article}
\usepackage{amsmath}
%\usepackage[paperwidth=21cm, paperheight=29.8cm]{geometry}
\usepackage[angle=0,scale=1,color=black,hshift=-0.4cm,vshift=15cm]{background}
\usepackage{multirow}
\usepackage{enumerate}
\usepackage[gen]{eurosym}

%\SetBgScale{1}
%\SetBgAngle{0}
%\SetBgColor{black}
%\SetBgContents{\rule{1pt}{30cm}}
%\SetBgHshift{-8.4cm}
%
%\backgroundsetup{contents={
%\begin{tabular}{c|c}
%\hspace{2cm} & \\[0.7cm]
%& {\bf Statistics for Computing ------ Lecture 1 ------ Solutions} \\[0.3cm]
%%\hline
%\hspace{2cm} & \hspace{18.5cm} \\ [28cm]
%\end{tabular}}}

\backgroundsetup{contents={
{\bf \centering Statistics for Computing ------------------ Tutorial 9 --------------------------- Questions} }}


\setlength{\voffset}{-3cm}
\setlength{\hoffset}{-2cm}
\setlength{\parindent}{0cm}
\setlength{\textheight}{27cm}
\setlength{\textwidth}{17cm}

\pagestyle{empty}



\begin{document}


\subsection*{Question 1}
A random sample of 18 software engineers was selected and it was found that their average income was \euro{40,000} with a standard deviation of \euro{3,125}. \\[-0.2cm]

Calculate the following:\\[0.2cm]
{\bf(a)} An 80\% confidence interval for $\mu$. \quad {\bf(b)} A 95\% confidence interval for $\mu$. \quad {\bf(c)} A 99\% confidence interval for $\mu$.



\subsection*{Question 2}
Guinness brewery wish to compare the quality of stout made using two different varieties of barley. Samples of the drink were prepared and subsequently tested. Taking various factors into consideration, each one was then given an overall quality score (where a higher score indicates better quality). The results are as follows:\\[-0.3cm]
\begin{center}
\begin{tabular}{|c|cccccc|}
\cline{1-7}
&&&&&&\\[-0.2cm]
Variety1 & 10 & 8 & 7 & 8 & 6 & \\[0.2cm]
\cline{1-7}
&&&&&&\\[-0.2cm]
Variety2 & 5  & 6 & 8 & 6 & 7 & 7 \\[0.2cm]
\cline{1-7}
\end{tabular}
\end{center}

The aim is to compare the mean scores in the two groups.\\[0.3cm]
{\bf(a)} If we wish to make the equal variance assumption in our calculation - what test must we carry out? \quad {\bf(b)} By carrying out this test, show that the equal variance assumption is reasonable here. \quad {\bf(c)} Calculate a 95\% confidence interval for the difference between the two means (using the equal variance approach). State your conclusion. \quad {\bf(d)} What is the advantage of the \emph{unequal} variance approach? Calculate a 95\% confidence interval using this approach.



\subsection*{Question 3}
Seven athletes were asked to run 100m without warming up prior to running. On another day they warmed up first and then ran. On both occasions they were timed and the results (in seconds) are as follows:\\[-0.3cm]
\begin{center}
\begin{tabular}{|c|ccccccc|}
\hline
&&&&&&&\\[-0.3cm]
Individual & 1 & 2 & 3 & 4 & 5 & 6 & 7 \\[0.1cm]
\hline
&&&&&&&\\[-0.3cm]
No Warm Up    & 13.6 & 12.8 & 12.3 & 11.7 & 12.0 & 13.3 & 10.5 \\[0.1cm]
\hline
&&&&&&&\\[-0.3cm]
Warm Up       & 13.9 & 12.4 & 12.2 & 11.6 & 11.9 & 12.7 & 10.4 \\[0.1cm]
\hline
\end{tabular}
\end{center}

{\bf(a)} Calculate a 95\% confidence interval for the \emph{average difference} in times and hence comment on the usefulness of warming up (hint: the data is paired).






\end{document} 