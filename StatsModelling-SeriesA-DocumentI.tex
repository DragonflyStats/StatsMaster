\documentclass[12pt, a4paper]{report}
\usepackage{epsfig}
\usepackage{subfigure}
%\usepackage{amscd}
\usepackage{amssymb}
\usepackage{amsbsy}
\usepackage{amsthm}
%\usepackage[dvips]{graphicx}
\usepackage{natbib}
\bibliographystyle{chicago}
\usepackage{vmargin}
% left top textwidth textheight headheight
% headsep footheight footskip
\setmargins{3.0cm}{2.5cm}{15.5 cm}{22cm}{0.5cm}{0cm}{1cm}{1cm}
\renewcommand{\baselinestretch}{1.5}
\pagenumbering{arabic}
\theoremstyle{plain}
\newtheorem{theorem}{Theorem}[section]
\newtheorem{corollary}[theorem]{Corollary}
\newtheorem{ill}[theorem]{Example}
\newtheorem{lemma}[theorem]{Lemma}
\newtheorem{proposition}[theorem]{Proposition}
\newtheorem{conjecture}[theorem]{Conjecture}
\newtheorem{axiom}{Axiom}
\theoremstyle{definition}
\newtheorem{definition}{Definition}[section]
\newtheorem{notation}{Notation}
\theoremstyle{remark}
\newtheorem{remark}{Remark}[section]
\newtheorem{example}{Example}[section]
\renewcommand{\thenotation}{}
\renewcommand{\thetable}{\thesection.\arabic{table}}
\renewcommand{\thefigure}{\thesection.\arabic{figure}}
\title{MA4605}
\author{ } \date{ }


\begin{document}
\author{Kevin O'Brien}
\title{MA4605}

\tableofcontents \setcounter{tocdepth}{2}
%-------------------------------------------------

\chapter{Chemometrics}






\section{MA4605: ANOVA}
We compute the test statistics $F = 62/3 \sim 20.7$ while the
$95\%$ quantile of F distribution with 3 and 8 degrees of freedom
is given as
\begin{verbatim}
>qf(0.95,3,8)
4.066181
\end{verbatim}

We clearly see that the test informs us about a significant
difference between the means. But which means are different?

The least significant difference method described in Section 3.9.

We compute the least significant difference $s \sqrt{2/n} \times
t$, where $s^{2}$ is within sample estimate of variance and $t$ is
the $97.5\%$ quantile of Student-$t$ distribution with $h(n-1)$
degrees of freedom.

\begin{verbatim}
>sqrt(mean(s))*sqrt(2/3)*qt(0.975,8)
# 3.261182
>m=apply(x,1,mean)
>m
#[1] 101 102 97 92
\end{verbatim}

The associated degrees of freedom: for within-sample $h(n - 1)$
(in our example $4 \times 2 = 8$), for between-sample $h - 1$ (in
our example 3). Total number of degrees freedom $hn-1$ and we see
$hn - 1 = h(n-1) + h - 1$.

But there is more then the relation between degrees of freedom.
Namely SST = SSM + SSR; where

WRONG
\begin{eqnarray}
SST = \sum_{j}\sum_{j}(x-\bar{x})^2\\
SSM= \sum_{j}\sum_{j}(x-\bar{x})^2\\
SSE = \sum_{j}\sum_{j}(x-\bar{x})^2\\
\end{eqnarray}


\begin{verbatim}
x=c(102,100,101,101,101,104,97,95,99,90,92,94)
factors=c(rep("A",3),rep("B",3),rep("C",3),rep("D",3))
res=aov(x˜factors) anova(res)

Analysis of Variance Table Response:

x   Df Sum Sq Mean Sq   F value Pr(>F) factors 3 186 62
20.6670.0004002 *** Residuals 8 24 3
---
Signif. codes: 0 *** 0.001 ** 0.01 * 0.05 . 0.1 1
\end{verbatim}
\newpage


It is often convienent to express the regression analysis using
ANOVA table. The following equation is the basis for such
representation

It is often shortened to SST = SSLR + SSR; where SST is referred

to as the total sum of squares, SSLR is the sum of squares due to
linear regression (within regression), SSR is the sum of squares
due to residuals (outside regression).




\section{Testing Normality}
An assessment of the normality of data is a prerequisite for many statistical tests as normal data is an underlying assumption in parametric testing. There are two main methods of assessing normality - graphically and numerically.


%------------------------------------------------------------------------%
\chapter{Linear Models}
\section{Multiple Linear Regression}
\section{Variable Selection Procedures}

\begin{verbatim}
           Coefficients  Std Error    t Stat      P-value
Intercept  0.002107      0.004787     0.440144    0.678209
Conc       0.025164      0.000266     94.76047    2.48E-09
\end{verbatim}

Intercept and Slope estimates are the coefficient.

\begin{itemize}
\item Akaike Information Criterion
\item Multicollinearity
\end{itemize}

$95\%$ confidence interval for slope
\begin{itemize}
\item Estimate of Slope     = 	0.025164
\item Std. Error for slope 	= 	0.000266 from \texttt{R} output
\item Quantiles (given) 	=	-2.57 	for Lower bound
				            =   2.57  	for Upper bound

\item Lower bound		=	0.025164 + (-2.57)( 0.000266)
				=	0.0243
`
\item Upper bound		=	0.025164 + (2.57)( 0.000266)
				=	0.0257

\item Confidence Interval = [0.0243, 0.0257]
\end{itemize}
%------------------------------------------------------------------------%











%------------------------------------------------------------------------------------------------%

\section{Example: Poisson}

A computer server breaks down on average once every three months.

\begin{itemize}
\item What is the probability that the server breaks down three times in a quarter?
\item What is the probability that a server breaks down exactly five times in one year?
\end{itemize}




\section{Sample size Estimation}
For a certain variable, the standard deviation in a large population is equal to 12.5.
How big a sample is needed to be 95\% sure that the sample mean is within 1.5 units of the population mean?


For a certain variable, the standard deviation in a large population is equal to 8.5.
How big a sample is needed to be 90\% sure that the sample mean is within 1.5
units of the population mean?



%---------------------------------------------------------------------------------------------------Inference%
\chapter{Statistical Inference}

\section{Confidence Interval examples}

\subsection{Example}
A random sample of 15 observations is taken from a normally distributed population
of values. The sample mean is 94.2 and the sample variance is 24.86.
Calculate a 99\% confidence interval for the population mean.


\subsubsection{Solution}
$t_(14,0.005) = 2.977$
99\% CI is $94.2 \pm 2.977 \sqrt{24.86/15}$ \\i.e. $94.2 \pm 3.83$ \\i.e. $(90.37,98.03)$


\subsection{Example 1: paired T test}


\begin{tabular}{|c|c|c|c|c|c|c|}
  \hline
X & 5.20 & 5.15 & 5.17 & 5.16 & 5.19 & 5.15\\
Y & 5.20 & 5.15 & 5.17 & 5.16 & 5.19 & 5.15\\
  \hline
\end{tabular}


\subsection{Example 2}

Seven measurements of the pH of a buffer solution gave the
following results:

\begin{tabular}{|c|c|c|c|c|c|c|}
  \hline
5.12 & 5.20 & 5.15 & 5.17 & 5.16 & 5.19 & 5.15\\
  \hline
\end{tabular}

Task 1: Calculate the 95\% confidence limits for the true pH
utilizing $R$.


Solution. We are using Student t distribution with six degrees of
freedom and the following code gives us the confidence interval
for this problem.
\begin{verbatim}
>x <- c(5.12, 5.20, 5.15, 5.17, 5.16, 5.19, 5.15)
>n =length(x)
>alpha =0.05
>stderr =sd(x)/sqrt(n)
>LB=mean(x)+qt(alpha/2,6)* stderr
>UB=mean(x)+qt(1-alpha/2,6)* stderr
>LB
#[1] 5.137975
>UB
#[1]5.187739
\end{verbatim}


\subsubsection{example}
A survey of study habits wishes to determine whether the mean
study hours completed by women at a particular college is higher
than for men at the same college. A sample of $n_1$ = 10 women and
$n_2$ = 12 men were taken, with mean hours of study $\bar{x}_1$ =
120 and $\bar{x}_2$ = 105 respectively. The standard deviations
were known to be $\sigma_1$ = 28 and $\sigma_2$ = 35.

The hypothesis being tested is:

\begin{eqnarray}
H_{0}: \mu_1 = \mu_2\qquad \qquad (\mu_1 - \mu_2= 0)\\
H_{a}: \mu_1 \neq \mu_2 \qquad \qquad (\mu_1 - \mu_2 \neq 0)
\end{eqnarray}

In $R$, the test statistic is calculated using:

\begin{verbatim}
xbar1 <- 120
xbar2 <- 105
sd1 <- 28
sd2 <- 35
n1 <- 10
n2 <-12
TS <- ( (xbar1 - xbar2) - (0) )/sqrt( (sd1^2/n1) + (sd2^2/n2) )
TS
[1] 1.116536
\end{verbatim}
Now need to calculate the critical value or the p-value.


The critical value can be looked up using qnorm. Since this is a
one-tailed test and there is a > sign in $H_1$:

\begin{verbatim}
qnorm(0.95)
[1] 1.644854
\end{verbatim}

Since the test statistic is less than the critical value ( 1.116536 < 1:645 )there is not enough evidence to reject $H_0$
and conclude that the population mean hours study for women is
not higher than the population mean hours study for men.


The p-value is determined using pnorm.

Careful! Remember pnorm
gives the probability of getting a value LESS than the value specified. We want the probability of getting a value greater than
the test statistic.
\begin{verbatim}
1-pnorm(1.116536) # OR pnorm(1.116536, lower.tail=FALSE)
[1] 0.1320964
\end{verbatim}
\newpage




\subsection{Example}
Suppose that 9 bags of salt granules are selected from the supermarket
shelf at random and weighed. The weights in grams are 812.0, 786.7, 794.1,
791.6, 811.1, 797.4, 797.8, 800.8 and 793.2. Give a 95\% confidence interval for the
mean of all the bags on the shelf. Assume the population is normal.


Here we have a random sample of size n = 9. The mean is 798.30. The sample
variance is $s^2 = 72.76$, which gives a sample standard deviation $s = 8.53$.

The upper 2.5\% point of the Student's $t$ distribution with n-1 (= 9-1 = 8) degrees of freedom is 2.306.

The 95\% confidence interval is therefore from \\
$(798.30 - 2.306 \times (8.53/\sqrt{9}), 798.30 + 2.306 \times (8.53/\sqrt{9})$\\
which is\\
$(798.30 - 6.56, 798.30 + 6.56) = (791.74, 804.86)$\\
It is sometimes more useful to write this as $798.30 \pm 6.56$.

Note that even if we do not assume the population is normal (that assumption is
never really true) the Central Limit Theorem might suggest that the confidence interval
is nearly right. A larger confidence would increase the length of the interval, so we
trade off increased certainty of coverage against a longer interval.

\subsection{Example}
Ten soldiers visit the rifle range on two different weeks. The first
week their scores are:
67 24 57 55 63 54 56 68 33 43
The second week they score
70 38 58 58 56 67 68 77 42 38
Give a 95\% confidence interval for the improvement in scores from week one to
week two.


\subsubsection{Answer}


This is a case of paired samples, for the scores are repeated observations for each
soldier, and there is good reason to think that the soldiers will differ from each other
in their shooting skill. So we work with the individual differences between the scores.
We shall have to assume that the pairwise differences are a random sample from a
normal distribution.

The differences are:

3 14 1 3 -7 13 12 9 9 -5


Effectively we now have a single sample of size 10, and want a 95\% confidence
interval for the mean of the population from which these differences are drawn. For
this we use a Student's $t$ interval. The sample mean of the differences is 5.2, and
$s^2$ = 54.84. So $s = 7.41$, and the 95\% $t$ interval for the difference in the means is
$5.2 - 2.26(7.41)/\sqrt{10},  5.2 + 2.26(7.41)/\sqrt{10} = (.0.1, 10.5)$.

\subsection{Example} A sample of 50 households in one community
shows that 10 of them are watching a TV special on the national
economy. In a second community, 15 of a random sample of 50
households are watching the TV special. We test the hypothesis
that the overall proportion of viewers in the two communities does
not differ, using the 1 percent level of significance, as follows:

\subsection{2 sided test}
A two-sided test is used when we are concerned about a possible
deviation in either direction from the hypothesized value of the
mean. The formula used to establish the critical values of the
sample mean is similar to the formula for determining confidence
limits for estimating the population mean, except that the
hypothesized value of the population mean m0 is the reference
point rather than the sample mean.







%------------------------------------------------%
%\newpage
%\addcontentsline{toc}{section}{Bibliography}
%\bibliography{MA4125bib}
\end{document} 