
\section{Summation}
The summation sign $\sum$ is commonly used in most areas of statistics.
Given $x_1 = 3, x_2= 1, x_3 = 4, x_4 = 6, x_5= 8 $ find:

\[
(i) \displaystyle\sum_{i=1}^{i=n} x_{i}  \hspace{3cm}
(ii) \displaystyle\sum_{i=3}^{i=4} x_{i}^2
\]
\begin{eqnarray*}(i) \displaystyle\sum_{i=1}^{i=n} x_{i} &=& x_1 + x_2 +  x_3 +  x_4 + x_5 \\  &=& 3 +1 +4 +6 + 8 \\ &=& \textbf{22} \end{eqnarray*}

\[ (ii) \displaystyle\sum_{i=1}^{i=n} x_{i}^2 = x_3^2 + x_4^2  = 9 + 16 = \textbf{25} \]

\noindent When all elements of a data set are used, a simple version of the summation notation can be used.
$\displaystyle\sum_{i=1}^{i=n} x_{i}$  can simply be written as $\sum x$


\subsection*{Example}
Given that $p_1= 1/4, p_2= 1/8, p_3= 1/8,p_4= 1/3, p_5 = 1/6$ find:

\begin{itemize}
\item $\displaystyle\sum_{i=1}^{i=n} p_{i} \times x_{i}$
\item $\displaystyle\sum_{i=1}^{i=n} p_{1} \times x_{i}^2$
\end{itemize}