\subsection{Regression example}

A study was made by a retailer to determine the relation between weekly advertising
expenditure and sales (in thousands of pounds). Find the equation of a regression line
to predict weekly sales from advertising. Estimate weekly sales when advertising
costs are £35,000.

Adv. Costs(in £‘000) 40 20 25 20 30 50 40 20 50 40 25 50

Sales (in £‘000) 385 400 395 365 475 440 490 420 560 525 480 510



\subsection{example}
Concentration (ng/ml) 0 5 10 15 20 25 30
Absorbance 0.003 0.127 0.251 0.390 0.498 0.625 0.763

\begin{verbatim}
# DO A FULL LINEAR REGRESSION ANALYSIS ON THE DATA

>concentration=c(0,5,10,15,20,25,30)
>absorbance=c(0.03,0.127,0.251,0.390,0.498,0.625,0.763)
>regr=lm(absorbance~concentration)
# READ AS; ABSORBANCE DEPENDENT ON CONCENTRATION
>summary(regr)
\end{verbatim}

This output from this code is as follows:
\begin{verbatim}
Call:
lm(formula = absorbance ~ concentration)

Residuals:
        1         2         3         4         5         6         7
 0.015357 -0.010571 -0.009500  0.006571 -0.008357 -0.004286  0.010786

Coefficients:
               Estimate Std. Error t value Pr(>|t|)
(Intercept)   0.0146429  0.0079787   1.835    0.126
concentration 0.0245857  0.0004426  55.551 3.58e-08 ***
---
Signif. codes:  0 ‘***’ 0.001 ‘**’ 0.01 ‘*’ 0.05 ‘.’ 0.1 ‘ ’ 1

Residual standard error: 0.01171 on 5 degrees of freedom
Multiple R-squared: 0.9984,     Adjusted R-squared: 0.9981
F-statistic:  3086 on 1 and 5 DF,  p-value: 3.576e-08

\end{verbatim}


\begin{itemize}
\item Estimation of Slope = 0.0251643 \item  Standard error of the
estimation of the slope is 0.0002656 \item Estimation of Intercept
= 0.0021071 \item Standard error of the estimation of the
intercept is 0.0047874 \item Degrees-of-Freedom1 = 7-2 = 5 • The
critical values for testing is are -2.57 and 2.57 since the area
under the Student’s t distribution
curve with 5 degrees-of-freedom outside this range is 5%
\item p-value for intercept is 67.8\% implying that it is not
significantly different from zero \item The p-value for the
intercept is the area under the Student’s t-distribution curve
with 5 degrees-of-freedom outside the range of [-0.44,0.44]. 

\item 
p-value for slope is less than 5\% implying that it is
significantly different from zero 
\item The p-value for the slope
is the area under the Student’s t-distribution curve with 5 
degrees of- freedom outside the range of [-94.76,94.76].
\end{itemize}

\section{Regression}

The argument to lm is a model formula in which the tilde symbol
(~) should be read as ``described by”.


This was seen several times earlier, both in connection with
boxplots and stripcharts and with the t and Wilcoxon tests.



\subsection{Multiple Linear Regression}
The \texttt{lm()} function handles much more
complicated models than simple linear regression. There can be many other things besides a dependent and a descriptive variable in a model formula.

A multiple linear regression analysis (which we discuss in Chapter
11) of, for example, y on x1, x2, and x3 is specified as $y ~ x1 +
x2 + x3$.


This is an F test for the hypothesis that the regression coefficient is zero. This test is not really interesting in a
simple linear regression analysis since it just duplicates information already given—it becomes more interesting when there is more than one explanatory variable.

\subsection{Regression}

\begin{verbatim}

> lm(short.velocity~blood.glucose)
\end{verbatim}

\section{Inference for Regression}
To determine the confidence interval for the slope we use the
following equation:
\begin{equation}
b \pm t_{1-\alpha/2,n-2} S.E.(b)
\end{equation}

\begin{itemize}
\item b = Estimation of Slope (0.0251643) \item S.E.(b) = Standard
Error of Slope(0.0002656) \item n = Sample Size (7) \item $\alpha$
= Alpha Value (5\%) \item $t_{1-\alpha/2,n-2}$ = Quantile Value
from Student’s t-distribution (2.570582)
\end{itemize}

\begin{equation}
(0.0251643) \pm (0.0002656)(2.570582) = [ 0.0245,0.0258 ]
\end{equation}








\end{document}