\documentclass[a4paper,12pt]{article}
%%%%%%%%%%%%%%%%%%%%%%%%%%%%%%%%%%%%%%%%%%%%%%%%%%%%%%%%%%%%%%%%%%%%%%%%%%%%%%%%%%%%%%%%%%%%%%%%%%%%%%%%%%%%%%%%%%%%%%%%%%%%%%%%%%%%%%%%%%%%%%%%%%%%%%%%%%%%%%%%%%%%%%%%%%%%%%%%%%%%%%%%%%%%%%%%%%%%%%%%%%%%%%%%%%%%%%%%%%%%%%%%%%%%%%%%%%%%%%%%%%%%%%%%%%%%
\usepackage{eurosym}
\usepackage{vmargin}
\usepackage{amsmath}
\usepackage{enumerate}
\usepackage{fancyhdr}
\usepackage{enumerate}
\usepackage{listings}
\usepackage{framed}
\usepackage{graphics}
\usepackage{epsfig}
\usepackage{subfigure}
\usepackage{fancyhdr}

\setcounter{MaxMatrixCols}{10}
%TCIDATA{OutputFilter=LATEX.DLL}
%TCIDATA{Version=5.00.0.2570}
%TCIDATA{<META NAME="SaveForMode" CONTENT="1">}
%TCIDATA{LastRevised=Wednesday, February 23, 2011 13:24:34}
%TCIDATA{<META NAME="GraphicsSave" CONTENT="32">}
%TCIDATA{Language=American English}

\pagestyle{fancy}
\setmarginsrb{20mm}{0mm}{20mm}{25mm}{12mm}{11mm}{0mm}{11mm}
\lhead{Maths Resource } \chead{Probability}
\rhead{Tutorial Sheet C}
%\input{tcilatex}

\begin{document}
\section*{Probability Tutorial Sheet C}
\begin{enumerate}

\item A new test has been developed to diagnose a particular disease. 
If a person has the disease, the test has a 98\% chance of identifying them as having the 
disease. If a person does not have the disease, the test has a 4\% chance of identifying them as having the disease. 3\% of the population have this disease. 
Suppose we select a person at random from the population.

\begin{enumerate}[(i)]
\item  What is the probability that the test will identify them as having the disease?
\item  What is the probability that the person has the disease given that the test identifies 
them as having the disease?
\end{enumerate}

    \item 
An electronics assembly subcontractor receives its entire supply of resistors from two suppliers. Company A provides 70\% of the subcontractor's resistors, while company B supplies the remainder. The additional information has also been made available.
\begin{itemize}
\item[$\ast$] 2\% of the resistors provided by company A failed the final test,
\item[$\ast$] 3\% of company B's resistors also fail final test.
\end{itemize}
\noindent Answer the following questions:
\begin{enumerate}[(i)]
\item  What is the probability that a resistor fails the final test?
\item   What is the probability that a resistor fails the final test given that the resistor in question came from company A?
\item What is the probability that a resistor that fails final test was supplied by company A?
\end{enumerate}

\item 
The probability distribute of discrete random variable $X$ is tabulated below. There are 5 possible outcome of $X$, i.e. 1, 2, 4, 6 and 8.
\begin{center}
\begin{tabular}{|c||c|c|c|c|c|}
\hline
$x_i$  & 1 & 2 & 4 & 6 & 8  \\\hline
$p(x_i)$ & 0.50 & 0.15 & 0.20 & 0.05 & 0.10 \\
\hline
\end{tabular}
\end{center}

\begin{enumerate}[(i)]
\item  Compute the value of $k$.
\item  What is the expected value of X?
\item  Compute the value of $E(X^2)$.
\item  Given that $E(X^2) = 12.5$, compute the variance of $X$.
\end{enumerate}
\item 
 The probability distribution of discrete random variable $X$ is tabulated below. There are 5 possible outcomes of $X$, i.e. 1, 2, 3, 5 ,10 and 20.
\begin{center}
\begin{tabular}{|c||c|c|c|c|c|}
\hline
$x_i$  & 1 & 2 & 5 & 10 & 20 \\\hline
$P(x_i)$ &  0.10 & 0.25 & 0.30& 0.20 &0.15\\

\hline
\end{tabular}
\end{center}

\begin{enumerate}[(i)]
\item  Determine the expected value $E(X)$.
\item  Evaluate $E(X^2)$.
\item  Compute the variance of random variable $X$.
\end{enumerate}






    
\end{enumerate}
\end{document}
